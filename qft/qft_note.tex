\documentclass[a4paper,10pt]{amsart}

\usepackage[protrusion=true,expansion=true]{microtype} 
\usepackage{fancyhdr}
\usepackage[utf8]{inputenc}
\usepackage{graphicx} 
\usepackage{wrapfig} 

\usepackage{mathpazo}
\usepackage[T1]{fontenc}
\usepackage{amsmath}
\usepackage{amssymb}
\usepackage{hyperref}
\usepackage{cleveref}
\usepackage{comment}
\usepackage{color}


\newtheorem{example}{Example}[section]
\newtheorem{theorem}{Theorem}[section]
\newtheorem{proposition}{Proposition}[section]
\newtheorem{corollary}{Corollary}[section]
\newtheorem{definition}{Definition}[section]
\newtheorem{lemma}{Lemma}[section]
\newtheorem{remark}{Remark}[section]
\newtheorem{question}{Question}[section]

\crefname{lemma}{Lemma}{lemmas}
\crefname{remark}{Remark}{remark}
\crefname{corollary}{Corollary}{corollary}
\crefname{theorem}{Theorem}{theorem}
\crefname{example}{Example}{example}
\crefname{definition}{Definition}{definition}

\newcommand{\AAA}{\mathfrak A}
\newcommand{\BBB}{\mathcal B}
\newcommand{\CCC}{\mathcal C}
\newcommand{\HHH}{\mathcal H} %for Hilbert space
\newcommand{\MMM}{\mathcal M}
\newcommand{\SSS}{\mathcal S}
\newcommand{\FFF}{\mathfrak F}
\newcommand{\DDD}{\mathfrak D}
\newcommand{\PPP}{\widetilde{P}}
\newcommand{\LLL}{\widetilde{L}}

\newcommand{\tr}{\tau}
\newcommand{\C}{\mathbb C} %for complex number
\newcommand{\R}{\mathbb R}  %for real number
\newcommand{\Z}{\mathbb Z} %for integer
\newcommand{\N}{\mathbb N} % for nature number
% self defined vars
\newcommand{\titleinfo}{Note On Quantum Field Theory}
\newcommand{\authorinfo}{Wei Yuan} 

\linespread{1.05}
\pagestyle{fancyplain}
\fancyhf{}
\fancyhf[HLE,HRO]{\titleinfo}
\fancyhf[HRE,HLO]{\authorinfo}
\fancyhf[FC]{\thepage}

\begin{document}

\title{\LARGE\textbf{\titleinfo}} 
\author{\large\textsc{\authorinfo}} 
\address{AMSS}  
\email{}

\date{}

%\renewcommand{\abstractname}{Summary} 
\begin{abstract}
Enter abstract here
\end{abstract}

% Keywords
\subjclass[2010]{Primary 47L75; Secondary 15A30}
\keywords{Mobius function}
\thanks{}
\maketitle


\section{Field Operators}

\subsection{Distributions}

\begin{definition}
    Let $\SSS(\R^{n})$ be the Schwartz space of rapidly decreasing smooth
    functions, that is the complex vector space of all functions 
    $f: \R^n \to \C$ with continuous partial derivatives of any order for
    which
    \begin{align} \label{sch_norm}
        |f|_{p,k} = \sup_{|\alpha | \leq q} \sup_{x \in \R^n}
    | \partial^{\alpha} f(x)|(1+|x|^2)^k < \infty,
    \end{align}
    for all $p$, $k \in \N$ and $\alpha = (\alpha_1, \alpha_2, \ldots,
    \alpha_n) \in \N^{n}$.
\end{definition}

\begin{definition}
   A tempered distribution $T$ is a linear functional $T: \SSS \to \C$
   which is continuous with respect to all the seminorms $|.|_{p,k}$ 
   defined in \cref{sch_norm}, $p, k \in \N$.
\end{definition}

\begin{example}
   \begin{enumerate}
       \item Let $g \in L^{\infty}(\R^n)$. 
           \begin{align*}
               T_{g}(f) = \int_{\R^{n}} g(x)f(x) dx, \qquad f \in \SSS. 
           \end{align*}
        \item The delta distribution given by
            \begin{align*}
                \delta_{y}: f \to f(y), \qquad f \in \SSS.
            \end{align*}
   \end{enumerate} 
\end{example}

If $T$ is a tempered distribution, then
\begin{align*}
    \langle \partial^{\alpha}T, f \rangle = 
    \langle T, (-1)^{|\alpha|}\partial^{\alpha}f \rangle, 
\end{align*}
i.e., 
\begin{align*}
\partial^{\alpha}T(f) = (-1)^{|\alpha|}T(\partial^{\alpha}f), \qquad
    f \in \SSS.
\end{align*}

\begin{example}
    Let $\varkappa$ be the characteristic function of $[0, \infty)$. 
    Then
    \begin{align*}
        \frac{d}{dt}T_{\varkappa}(f) = -\int_{0}^{\infty}f'(x)dx = f(0)
        = \delta_{0}(f).
    \end{align*}
\end{example}

\begin{proposition}
   Every tempered distribution $T$ has a representation as a finite 
   sum of derivatives of continuous functions of polynomial growth, that
   is there exist $g_{alpha}: \R^{n} \to \C$ such that
   \begin{align*}
       T = \sum_{0 \leq |\alpha| \leq k}\partial^{\alpha}T_{g_{\alpha}}. 
   \end{align*}
\end{proposition}

For a polynomial $P(x) = c_{\alpha}X^{\alpha} \in \C[x_1, \ldots, x_n]$
in $n$ variables with complex coefficients $c_{\alpha} \in \C$ on obtains
the partial differential operator
\begin{align*}
   P(i\partial) = c_{\alpha}(i\partial)^{\alpha} 
   =\sum (i)^{|\alpha|}c_{\alpha_1, \ldots, \alpha_2}
   \partial_{1}^{\alpha_1} \ldots
\partial_{n}^{\alpha_n}.
\end{align*}

\begin{example}
   \begin{enumerate}
       \item $P(x) = -(x_{1}^{2} + \cdots + x^{2}_{n})$ gives the Laplace
           operator
           \begin{align*}
               \Delta = \partial^{2}_1 + \cdots + \partial^{2}_n. 
           \end{align*}
        \item $P(x) = -x^{2}_{0} + x^{2}_1 + \cdots + x^{2}_n$ gives the
            Laplace-Beltrami operator
            \begin{align*}
                \square = \partial_{0}^{2} 
                - (\partial^{2}_1 + \cdots + \partial^{2}_n).
            \end{align*}
   \end{enumerate} 
\end{example}

For any $f \in \SSS$, let 
\begin{align*}
    (\FFF f)(x) 
    = \Hat{f}(p) = \int_{\R^{n}} f(x)e^{-i \langle x, p \rangle} dx 
\end{align*}
be the Fourier transform of $f$. The inverse Fourier transform of $f$ is
\begin{align*}
    (\FFF^{-1} \Hat{f})(x) = f(x) = \frac{1}{(2\pi)^n}
    \int_{\R^n} \Hat{f}(p)e^{i \langle p, x \rangle} dp. 
\end{align*}

Note that
\begin{align*}
    \FFF(\partial_k f)(p) = 
\int_{\R^{n}} \partial_k f(x)e^{-i \langle x, p \rangle } dx= 
-\int_{\R^{n}} f(x)\partial_{k} e^{-i \langle x, p \rangle } dx =
i p_{k} \FFF(f)(p).
\end{align*}
Similarly, we have
\begin{align*}
    \FFF^{-1}(\partial_k \Hat{f})(x) = -ix_{k} \FFF^{-1}(\Hat{f})(x). 
\end{align*}

\begin{definition}
   For any tampered distribution $T$, let
   \begin{align*}
       \FFF(T)(f) = (T \circ \FFF)(f) = T(\FFF(f)).
   \end{align*}
   If $P(p)$ is a polynomial, then
   \begin{align*}
       PT(f) = T(Pf). 
   \end{align*}
\end{definition}

\begin{example}
   \begin{enumerate}
       \item $\FFF(\delta_0)(f) = \FFF(f)(0) = 
           \int_{\R^{n}}f(x) dx$, therefore 
           \begin{align*}
           \FFF(\delta_0) = 1.
           \end{align*}
        \item
            \begin{align*}
                \FFF^{-1}(e^{i \langle p, y \rangle}) = \delta_{x-y}. 
            \end{align*}
   \end{enumerate} 
\end{example}

\begin{remark}
   Let $T$ be a tempered distribution and $f \in \SSS$. 
   \begin{align*}
       \partial_{k} \circ \FFF(T)(f) = T(\FFF (\partial_{k} f)) = 
       T(ip_{k}\FFF(f)) = \FFF (ip_{k} T)(f). 
   \end{align*}
   And
   \begin{align*}
       \FFF(i \partial_{k}T)(f) = T(i\partial_{k}\FFF(f)) 
       = T(\FFF(x_k f)) = (x_k \FFF(T)) (f)
   \end{align*}
\end{remark}

\subsection{Klein-Gordon Equation}

Consider the Klein-Gordon Equation with mass $m > 0$:
\begin{align} \label{k_g_eq}
    (\square + m^2)T_1 = T_2, 
\end{align}
where $T_i$, $i = 1, 2$ are tempered distribution.

If $T_2 = \delta_0$, then by applying the Fourier transform on both side
of \cref{k_g_eq}, we only need to solve the division problem
\begin{align*}
    (-\mathbf{p}^2 + m^2)T = 1, 
\end{align*}
where $\mathbf{p}^2 = p_{0}^{2} - (p_{1}^{2} + \cdots + p_{n-1}^{2})$. 

Consider the homogeneous equation
\begin{align*}
    (\square - m^2)f = 0 \Longleftrightarrow 
    (\mathbf{p}^{2} - m^2)\Hat{f} = 0. 
\end{align*}


\subsection{Fields Operators}

\begin{definition}
    Let $A$ be a closed operator. The spectrum of $A$ is 
    \begin{align*}
        \sigma(A) = \{z \in \C : (zI - A)^{-1} \mbox{ is not a bounded
        operator.} \}
    \end{align*}
\end{definition}

\begin{definition}
   Let $\mathfrak{O}(\HHH)$ be the set of all densely defined 
   operators in $\HHH$. 
   A field operator or quantum field is an operator-valued distribution
   (on $\R^{n}$), this is a map
   \begin{align*}
       \Phi: \SSS(\R^{n}) \to \mathfrak{O}(\HHH)
   \end{align*}
   such that there exists a dense subspace $\DDD \subset \HHH$ satisfying
   \begin{enumerate}
       \item For each $f \in \SSS$ the domain of $\Phi(f)$ contains $\DDD$.
       \item The induced map $f \to \Phi(f)|_{\DDD}$ is linear.
       \item For each $\xi \in \DDD$ and $\beta \in \HHH$ the 
           assignment $f \to \langle \Phi(f)\xi, \beta \rangle$ is a
           tempered distribution.
   \end{enumerate}
\end{definition}

\subsection{Wightman Axioms}

Let $M = \R^{1, D-1}$ be the $D$-dimensional Minkowski space with the 
(Lorentz) metric
\begin{align*}
    \langle x, x \rangle = (x^{0})^{2} - \sum_{j=1}^{D-1}(x^{j})^{2},
    \qquad x = (x^{0}, \ldots, x^{D-1}) \in M.
\end{align*}

Two subsets $X, Y \subset M$ are called to be space-like separated if for
any $x \in X$ and $y \in Y$ we have
\begin{align*}
    (x^{0} - y^{0}) - \sum^{D-1}_{j=1}(x^{j} - y^{j})^2 < 0. 
\end{align*}

The forward cone is 
\begin{align*}
    C_{+} = \{x \in M : \langle x, x \rangle \geq 0, x^{0} \geq 0\}
\end{align*}
and the \textbf{causal order} is given by $x \geq y$ iff $x-y \in C_{+}$.

Let $P = P(1, D-1)$ be the \textit{Poincar\'{e}} group and
$L = SO_{0}(1, D-1) \subset GL(D, \R)$ be the identity component of 
the orthogonal group $O(1, D-1)$ preserving the metric.
It is know that $P = \R^{D} \rtimes L$, where $\R^{D}$ represents 
the translation group.

The \textit{Poincar\'{e}} group acts on $\SSS$ by
\begin{align*}
    g \cdot  f(x) = f(g^{-1}x), \qquad g \in P, \quad f \in \SSS.
\end{align*}

If $g = (q, \Lambda) \in \R^{D} \rtimes L$, then
\begin{align*}
    g \cdot f (x) = (q, \Lambda)f(x) = f(\Lambda^{-1}(x-q)). 
\end{align*}

Let $\PPP = \R^{D} \rtimes Spin(1, D-1)$ for $D > 2$ where
$Spin(1,D-1)=\LLL$ is the spin group, 
the universal covering group of the Lorentz group $L = SO(1, D-1)$.

\begin{remark}
   The Lorentz group is six-dimensional. The following gives a basis of
   the Lie algebra:
   \begin{align*}
       &J_{01} = \begin{pmatrix}
           0 & 1 & 0 & 0\\
           1 & 0 & 0 & 0\\
           0 & 0 & 0 & 0\\
           0 & 0 & 0 & 0 \\
       \end{pmatrix},
        J_{02} = \begin{pmatrix}
           0 & 0 & 1 & 0\\
           0 & 0 & 0 & 0\\
           1 & 0 & 0 & 0\\
           0 & 0 & 0 & 0 \\
       \end{pmatrix},
        J_{03} = \begin{pmatrix}
           0 & 0 & 0 & 1\\
           0 & 0 & 0 & 0\\
           0 & 0 & 0 & 0\\
           1 & 0 & 0 & 0 \\
       \end{pmatrix} \\
        &J_{12} = \begin{pmatrix}
           0 & 0 & 0 & 0\\
           0 & 0 & -1 & 0\\
           0 & 1 & 0 & 0\\
           0 & 0 & 0 & 0 \\
       \end{pmatrix},
        J_{13} = \begin{pmatrix}
           0 & 0 & 0 & 0\\
           0 & 0 & 0 & 1\\
           0 & 0 & 0 & 0\\
           0 & -1 & 0 & 0 \\
       \end{pmatrix},
        J_{23} = \begin{pmatrix}
           0 & 0 & 0 & 0\\
           0 & 0 & 0 & 0\\
           0 & 0 & 0 & -1\\
           0 & 0 & 1 & 0 \\
       \end{pmatrix}.
   \end{align*}
   The subgroup generated by $J_{01}$ is 
   \begin{align*}
      \begin{pmatrix}
          \cosh \theta & \sinh \theta & 0 & 0\\
          \sinh \theta & \cosh \theta & 0 & 0\\
          0 & 0 & 1 & 0\\
          0 & 0 & 0 & 1
      \end{pmatrix}.
   \end{align*}
   The subgroup generated by $J_{12}$ is 
   \begin{align*}
      \begin{pmatrix}
          1 & 0 & 0 & 0\\
          0 & \cos \theta & -\sin \theta & 0\\
          0 & \sin \theta & \cos \theta & 0 \\
          0 & 0 & 0 & 1
      \end{pmatrix}.
   \end{align*}

   We can identify $\R^4$ with the space of $2$ by $2$ complex self-adjoint
   matrices by
   \begin{align*}
       (p_0, p_1, p_2, p_3) \leftrightarrow 
       \begin{pmatrix}
           p_0 + p_3 & p_1 - ip_2\\
           p_1 + ip_2 & p_0 -p_3
       \end{pmatrix}
   \end{align*}
   and observe that
   \begin{align*}
       \det \begin{pmatrix}
           p_0 + p_3 & p_1 - ip_2\\
           p_1 + ip_2 & p_0 -p_3
       \end{pmatrix}  = p_{0}^2 - p_{1}^2 - p_{2}^2 - p_{3}^{2}.
   \end{align*}
   The for any $\Lambda \in SL_{2}(\C)$, it is obviously that
   the following map preserves the determinant and self-adjointness:
   \begin{align*}
       \begin{pmatrix}
           p_0 + p_3 & p_1 - ip_2\\
           p_1 + ip_2 & p_0 -p_3
       \end{pmatrix}
       \to
        \Lambda
        \begin{pmatrix}
           p_0 + p_3 & p_1 - ip_2\\
           p_1 + ip_2 & p_0 -p_3
        \end{pmatrix} \Lambda^{*}
   \end{align*}
   Note that both $\Lambda$ and $-\Lambda$ give the same linear 
   transformation. This implies that $SL_{2}(\C)$ is a double covering 
   of $SO(1,3)$. Thus $spin(1,3) \simeq SL_2(\C)$. 
\end{remark}

Now assume that we have a unitary representation of $\PPP$ which will be 
denoted by
\begin{align*}
    U : \PPP \to U(\HHH), \qquad (q, \Lambda) \to U(q, \Lambda), 
    \qquad (q, \Lambda) \in \R^{D} \rtimes \LLL.
\end{align*}

By Stone's Theorem, there exist $D$ self-adjoint closed operator
$P_0, P_1, \ldots, P_{D-1}$ such that
\begin{align*}
    U(q, 1) = e^{i(q^{0}P_{0} - q^{1}P_1 - \cdots - q^{D-1}P_{D-1})},
\end{align*}
where $q_i \in \R$. $P_0$ is interpreted as the energy operators 
and $P_j$, $j > 0$ as the components of th momentum.

\textbf{Wightmen Axioms.} A Wightman quantum field theory in dimension
D consists of the following data:
\begin{itemize}
    \item the space of states, which is the projective space $P(\HHH)$ of 
        a separable complex Hilbert space $\HHH$;
    \item the vacuum vector $\Omega \in \HHH$ of norm $1$;
    \item a unitary representation $U: \PPP \to U(\HHH)$ of $\PPP$;
    \item a collection of field operator $\Phi_{a}$, $a \in I$,
        \begin{align*}
            \Phi_{a}: \SSS \to \mathfrak{O}(\HHH), 
        \end{align*}
        with a dense subspace $D \subset \HHH$ as their common domain 
        ( that is the domain $\DDD(\Phi_{a}(f))$ contains $\DDD$ 
        for all $a \in I$ and $f \in S$) such that $\Omega \in \DDD$.
\end{itemize}

This data satisfy the following three axioms:

\textbf{Axiom 1(Covariance)}
\begin{enumerate}
    \item $\Omega$ is $\PPP$-invariant, i.e., 
        \begin{align*}
            U(q, \Lambda) \Omega = \Omega, \qquad \forall (q, \Lambda) \in
            \PPP,
        \end{align*}
        and $\DDD$ is $\PPP$-invariant, i.e.,
        \begin{align*}
            U(q, \Lambda) \DDD \subset \DDD, 
            \qquad \forall (q, \Lambda) \in
            \PPP;
        \end{align*}
      \item 
          \begin{align*}
              \Phi_{a}(f) \DDD \subset \DDD, \qquad \forall f \in \SSS
              \mbox{, and } a \in I;
          \end{align*}
      \item 
          \begin{align*}
              U(q, \Lambda)\Phi_{a}(f)U(q, \Lambda)^{*} = 
              \Phi_{a}((q,\Lambda)f), \qquad \forall f \in \SSS
              \mbox{ and } (q, \Lambda) \in \PPP.
          \end{align*}
\end{enumerate}

\textbf{Axiom 2(Locality)} $\Phi_{a}(f)$ and $\Phi_{b}(g)$ commute on 
$\DDD$ if the supports of $f, g \in \SSS$ are space-like separated, that
is on $\DDD$
\begin{align*}
    [\Phi_a(f), \Phi_b(f)] = 0. 
\end{align*}

\textbf{Axiom 3(Spectrum Condition)} The joint spectrum of the operators
$P_j$ is contained in the forward cone $C_{+}$, i.e.,
the eigenvalues rector $(p_0, p_1, \ldots, p_{D-1}) \in C_{+}$.

\begin{remark}
   It is customary to require that the vacuum is cyclic in the sense that
   the subspace $\DDD_{0} \subset \DDD$ spanned by all vectors
   \begin{align*}
       \Phi_{a_1}(f_1) \Phi_{a_2}(f_2)\cdots \Phi_{a_m}(f_m)\Omega  
   \end{align*}
   is dense in $\DDD$.
\end{remark}

As an additional axiom, one can require the vacuum $\Omega$ to be unique:

\textbf{Axiom 4(Uniqueness of the Vacuum)} The only vectors in $\HHH$ left
invariant by the translations $U(q,1)$, $q \in \R^{D}$, are the scalar
multiples of the vacuum $\Omega$.

\begin{example}[Free Bosonic QFT]
    Let
    \begin{align*}
        \Gamma_m = \{(p_0, p_1, p_2, p_3) \in \R^{(1,3)} : 
            \langle p, p \rangle = p_{0}^2 - p_{1}^{2} - p_{2}^{2}
        -p_{3}^{2} = m^2, p_{0} > 0 \}.
    \end{align*}
    Then
    \begin{align*}
        \rho: \R^{3} \to \Gamma_{m}, (p_1, p_2, p_3) \to
        (\sqrt{p_{1}^{2} + p_{2}^{2}+ p_{3}^{2} + m^{2}}, p_1, p_2, p_3), 
    \end{align*}
    is an isomorphism.
    In the sequel, denote $\sqrt{p_{1}^{2} + p_{2}^{2}+ p_{3}^{2} + m^{2}}$
    with $\omega(p)$ for any $p = (p_1, p_2, p_3) \in \R^{3}$.
      
    Let $\lambda_m$ be the invariant measure on $\Gamma_m$ given by
    \begin{align*}
        \int_{\Gamma_m} f(\rho(\mathbf{p})) d\lambda_m(\rho(\mathbf{p})) 
        = \int_{\R^{3}}\frac{h(\mathbf{p})}
        {2\sqrt{\mathbf{p}^2 + m^2}}d\mathbf{p},
    \end{align*}
    where $\mathbf{p}^2 = p_{1}^{2} + p^{2}_{2} + p^{2}_{3}$.

    Let 
    \begin{align*}
        \Lambda = \begin{pmatrix}
            \cosh(x) & \sinh(x) & 0 & 0\\
            \sinh(x) & \cosh(x) & 0 & 0 \\
            0 & 0 & 1 & 0\\
            0 & 0 & 0 & 1
        \end{pmatrix}. 
    \end{align*}
    The isometry 
    \begin{align*}
        \begin{pmatrix}
           p_0 \\
           p_1 \\
           p_2 \\
           p_3
        \end{pmatrix}
        \to
        \begin{pmatrix}
            \cosh(x) & \sinh(x) & 0 & 0\\
            \sinh(x) & \cosh(x) & 0 & 0 \\
            0 & 0 & 1 & 0\\
            0 & 0 & 0 & 1
        \end{pmatrix}
        \begin{pmatrix}
           p_0 \\
           p_1 \\
           p_2 \\
           p_3
        \end{pmatrix},
    \end{align*}
    induced the map form $\R^{3}$ onto $\R^{3}$
    \begin{align*}
        \begin{pmatrix}
           p_1 \\
           p_2 \\
           p_3
        \end{pmatrix}
        \to 
        \begin{pmatrix}
            \sinh(x)\sqrt{m^2+\mathbf{p}^2} + \cosh(x)p_1 \\
           p_2 \\
           p_3
        \end{pmatrix}.
    \end{align*}
    The Jacobi of this transformation is
    \begin{align*}
        \frac{\sinh(x)p_1 + \cosh(x)p_0}{\sqrt{m^2+\mathbf{p}^2}}. 
    \end{align*}
    Therefore it is easy to see that the measure $d\lambda_m$ is invariant.
     

Let $\HHH_{1} = L^{2}(\Gamma_m, d\lambda_m)$ and 
$\mathit{H}_n$ be the space of rapidly decreasing functions on 
the n-fold product of the upper hyperboloid $\Gamma_m$ which are
symmetric in the variables ($\mathbf{p}_1, \mathbf{p}_2, \ldots, 
\mathbf{p}_n) \in \Gamma_{m}^{n}$ ($\mathit{H} = \SSS(\Gamma_m)$).
$\mathit{H}_n$ has the inner product
\begin{align*}
    \langle u, v \rangle = \int_{\Gamma_m^{n}}u(\mathbf{p}_1, \ldots,
        \mathbf{p}_n) \Bar{v}(\mathbf{p}_1, \ldots, \mathbf{p}_n)
        d\lambda_{m}(\mathbf{p}_1)\ldots d\lambda_{m}(\mathbf{p}_n).
\end{align*}

The Hilbert space completion of $\mathit{H}_n$ well be denoted by
$\HHH_n$. Let
\begin{align*}
    \DDD = \bigoplus_{n=0}^{\infty}\mathit{H}_n 
\end{align*}
($\mathit{H}_0 = \C$ with the vacuum $\Omega = 1 \in \mathit{H}_0$) has
a natural inner product given by
\begin{align*}
    \langle f, g \rangle = f_0 \Bar{g}_{0} + \sum_{n \geq 1}
    \frac{1}{n!} \langle f_n, g_n \rangle,
\end{align*}
where $f = (f_0, f_1, \ldots)$, $(g_0, g_1, \ldots) \in \DDD$. The 
completion of $\DDD$ w.r.t this inner product is the Fock space $\HHH$.

Let $f \in \SSS(\R^{4})$ and $g = (g_0, g_1, \ldots) \in \DDD$.
Define $\Phi(f)$ by
\begin{align*}
    (\Phi(f)g)_n(\mathbf{p_1}, \ldots, \mathbf{p_n}) &= 
    \int_{\Gamma_{m}} \Hat{f}(\mathbf{p})g_{N+1}(\mathbf{p},
    \mathbf{p}_1, \ldots, \mathbf{p}_n) d\lambda_{n}(\mathbf{p}) \\
    & + \sum_{j=1}^{n} \Hat{f}(-\mathbf{p}_j) g_{n-1}(\mathbf{p}_1,
    \ldots \Hat{\mathbf{p}}_{j}, \ldots, \mathbf{p}_n).
\end{align*}

Note that for any $f \in \SSS(\R^4)$, we have $\Phi(\square f - m^2 f) = 0$
since
\begin{align*}
    \FFF(\square f - m^2 f) = (-\mathbf{p}^2 + m^2)\Hat{f} 
\end{align*}
vanishes on $\Gamma_m$.

\end{example}
%-----------------------------------------------------------------------------------
%BIBLIOGRAPHY
%-----------------------------------------------------------------------------------
\begin{thebibliography}{9}
\bibitem{Davenport37}
  H.Davenport
  \emph{On some infinite series involving arithmetical functions(II)}.
  Quarterly Journal of Mathematics, 8,
  313-320, 1937.

\end{thebibliography}
%-----------------------------------------------------------------------------------

\end{document}
