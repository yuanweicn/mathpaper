\documentclass[a4paper,10pt]{amsart}

\usepackage[protrusion=true,expansion=true]{microtype} 
\usepackage{fancyhdr}
\usepackage[utf8]{inputenc}
\usepackage{graphicx} 
\usepackage{wrapfig} 
\usepackage{mathrsfs}

\usepackage{mathpazo}
\usepackage[T1]{fontenc}
\usepackage{amsmath}
\usepackage{amssymb}
\usepackage{hyperref}
\usepackage{cleveref}
\usepackage{comment}
\usepackage{color}


\newtheorem{example}{Example}[section]
\newtheorem{theorem}{Theorem}[section]
\newtheorem{proposition}{Proposition}[section]
\newtheorem{corollary}{Corollary}[section]
\newtheorem{definition}{Definition}[section]
\newtheorem{lemma}{Lemma}[section]
\newtheorem{remark}{Remark}[section]
\newtheorem{question}{Question}[section]

\crefname{lemma}{Lemma}{lemmas}
\crefname{remark}{Remark}{remark}
\crefname{corollary}{Corollary}{corollary}
\crefname{theorem}{Theorem}{theorem}
\crefname{example}{Example}{example}
\crefname{definition}{Definition}{definition}

\newcommand{\AAA}{\mathfrak A}
\newcommand{\BBB}{\mathcal B}
\newcommand{\CCC}{\mathcal C}
\newcommand{\HHH}{\mathscr H} %for Hilbert space
\newcommand{\LLL}{\mathcal L} % for lattice
\newcommand{\MMM}{\mathcal M}
\newcommand{\PP}{\mathscr P}
\newcommand{\FF}{\mathscr F}
\newcommand{\M}{\mathscr M}
\newcommand{\RR}{\mathcal R}

\newcommand{\Lat}{\mathcal Lat}
\newcommand{\Alg}{\mathcal Alg}
\newcommand{\tr}{\tau}
\newcommand{\C}{\mathbb C} %for complex number
\newcommand{\R}{\mathbb R}  %for real number
\newcommand{\Z}{\mathbb Z} %for integer
\newcommand{\N}{\mathbb N} % for nature number
% self defined vars
\newcommand{\titleinfo}{Entropy}
\newcommand{\authorinfo}{Junhao Shen and Wei Yuan} 

\linespread{1.05}
\pagestyle{fancyplain}
\fancyhf{}
\fancyhf[HLE,HRO]{\titleinfo}
\fancyhf[HRE,HLO]{\authorinfo}
\fancyhf[FC]{\thepage}

\begin{document}

\title{\LARGE\textbf{\titleinfo}} 
\author{\large\textsc{\authorinfo}} 
\address{UNH and AMSS}  
\email{}

\date{}

%\renewcommand{\abstractname}{Summary} 
\begin{abstract}
Enter abstract here
\end{abstract}

% Keywords
\subjclass[2010]{Primary 47L75; Secondary 15A30}
\keywords{Entropy}
\thanks{}
\maketitle


\section{Introduction}
Introduction here!


\section{Entropy}
By $\HHH$ we shall denote a complex separable Hilbert space of 
infinite dimension, by $\PP f(\HHH)$ the finite subsets of $\HHH$ and 
by $\FF(\HHH)$ the finite-dimensional subspaces of $\HHH$.
If $\omega \in \PP f(\HHH)$ and $A \subset \HHH$ we shall write
$\omega \subset_{\delta} A$ if for every $h \in \omega$ we can find
$h' \in A$ such that $\| h - h' \| < \delta$.

\begin{definition}
    If $\omega \in \PP f(\HHH)$ and $\delta > 0$, we define
    \begin{align*}
        d(\omega; \delta) = \inf \{dim \chi : \chi \in \FF (\HHH), \omega
        \subset_{\delta} \chi \}.
    \end{align*}
\end{definition}

Suppose that $G$ be a group with a finite set of generators $\Sigma$. 
Let $c_n = c_n(\Sigma)$ be 
number of elements of $G$ whose shortest representative in 
$\Sigma \cup \Sigma^{-1}$ has exactly length $n$. The growth function
$C(z)$ of $(G, \Sigma)$ is the formal power series 
$C(z) = \sum c_n(\Sigma)z^n$.
For $F_m$ the free group on $m$ generators, the growth function
with respect to a free basis is
\begin{align*}
    C(z) = \frac{1+z}{1-(2m-1)z}. 
\end{align*}
Thus $c_0 = 1$ and $c_n = 2m(2m-1)^{n-1}$ for $n > 1$.

Let $b_n = b_n(G, \Sigma) = \sum^{n}_{i=0}c_i$ be the number of elements 
of $G$ that
can be expressed in terms of words of length at most $n$ in the 
generating set $\Sigma \cup \Sigma^{-1}$.
The growth rate of $G$ with respect to $\sigma$ is defined to be
\begin{align*}
    r(G, \Sigma) = \lim_{n \rightarrow \infty} \sqrt[n]{b_n (G, \Sigma)}. 
\end{align*}
If $r(G, \Sigma)=1$, i.e., $G$ has subexponential growth rate, 
then $G$ is amenable. For $F_m$, we have
\begin{align*}
    b_n(F_m, \Sigma) = 
   \begin{cases}
       2m  & \mbox{ if $m =1$},\\
       1+ \frac{m[(2m-1)^{n} -1]}{m-1} &\mbox{ if $m>1$}.
    \end{cases}
    \qquad m > 1.
\end{align*}
Thus, $r(F_{m}, \Sigma) = 2m-1$. 
In general, there's no particular connection between rate of growth and amenability 
between these two extremes. In \cite{AGG} is showed that for each $m>1$,
there is a sequence of nonamenable groups on $m$ generators whose growth rates 
approach 1. On the other hand, in \cite{Gl} is exhibited for each $m>1$ 
a sequence of amenable groups on $m$ generators whose growth rates approach $2m-1$.


In the rest of this note, we will use $\kappa_m(n)$ to denote 
$b_n(F_m, \Sigma)$.


On the one hand, Let $\AAA$ be a von Neumann algebra and $\FF U(\AAA)$ be the finite 
subsets of unitaries of $\AAA$. Suppose that $\Sigma = 
 \{U_1, \ldots, U_m \}\in \FF U(\AAA)$, let
\begin{align*}
    (\Sigma \cup \Sigma^{-1})^n= \{V_{i_1}V_{i_2}\cdots V_{i_n}: V_{i_k} \in \Sigma 
    \cup \Sigma^{-1} \cup \{I\} \}.
\end{align*}

\begin{definition}
    If $\delta > 0$, $\omega \in \PP f(\HHH)$ and 
    $\Sigma = \{U_1, \ldots, U_m \} \in 
    \FF U(\AAA)$, we define
    \begin{align*}
        fh(\Sigma,\omega; \delta) &=
           \limsup\limits_{n \rightarrow \infty}
           \frac{1}{\kappa_{m}(n)}d\left 
           ((\Sigma \cup \Sigma^{-1})^{n}\omega
            ; \delta \right ) \\
        fh(\Sigma, \omega) &= \sup_{\delta > 0}h(\Sigma, \omega; \delta),\\
        fh(\Sigma, \HHH) &= \sup\{h(\Sigma, \omega): \omega \in \PP f(\HHH)\}.
    \end{align*}
\end{definition}

\begin{definition}
    If $\delta > 0$, $\omega \in \PP f(\HHH)$ and 
    $\Sigma = \{U_1, \ldots, U_m \} \in 
    \FF U(\AAA)$. Let $G$ be the group generated by
    $\Sigma$. We define
    \begin{align*}
        h(\Sigma, \omega; \delta) &=
           \limsup\limits_{n \rightarrow \infty}
           \frac{1}{b_{n}(G, \Sigma)}d\left 
           ((\Sigma \cup \Sigma^{-1})^{n}\omega
            ; \delta \right ) \\
        h(\Sigma, \omega) &= \sup_{\delta > 0}h(\Sigma, \omega; \delta),\\
        h(\Sigma, \HHH) &= \sup\{h(\Sigma, \omega): \omega \in \PP f(\HHH)\}.
    \end{align*}
\end{definition}

Since $\kappa_{m}(n) \geq b_{n}(G, \Sigma)$, we have the following lemma.

\begin{lemma} \label{e_l1}
    $fh(\Sigma, \HHH) \leq h(\Sigma, \HHH)$.
\end{lemma}

\begin{lemma} \label{e_l2}
    Let $\Sigma \in \FF U(\AAA)$ and 
    $\omega_j \in \PP f(\HHH)$, $j \in \N$, 
    $\omega_1 \subset \omega_2
    \subset \ldots$, such that $
    \bigcup_{j \in \N}\omega_{j}$ 
    is a dense subset of 
    $\HHH$. Then
    \begin{align*}
        h(\Sigma, \HHH) = \sup_{j \in \N}h(\Sigma, \omega_{j}).
    \end{align*}
\end{lemma}

\begin{lemma} \label{e_l3}
    Let $\Sigma \in \FF U(\AAA)$ and 
    $\omega_j \in \PP f(\HHH) \cap (\HHH)_{1}$, $j \in \N$, 
    $\omega_1 \subset \omega_2
    \subset \ldots$, such that $
    \bigcup_{j \in \N}\omega_{j}$
    is a dense subset of 
    $(\HHH)_{1}$. Then
    \begin{align*}
        h(\Sigma, \HHH) = \sup_{j \in \N}h(\Sigma, \omega_{j}).
    \end{align*}
\end{lemma}

\begin{proof}
    Let $C = \max \{\|\xi\|: \xi \in \omega\}$.
    By the assumptions, there is $j$ such that 
    \begin{align*}
        \{\frac{\xi}{\|\xi\|}: \xi \in \omega\}  
        \subset_{\frac{\delta}{2C}} \omega_{j}
    \end{align*}
    It is easy to see that
    \begin{align*}
        (\Sigma \cup \Sigma^{-1})^{n}\omega_{j} \subset_{
        \frac{\delta}{2C}} B
    \end{align*}
    implies
    \begin{align*}
        (\Sigma \cup \Sigma^{-1})^{n} \omega
        \subset_{\delta} B.
    \end{align*}
    Hence
    \begin{align*}
        h(\Sigma, \omega; \delta) &=
           \limsup\limits_{n \rightarrow \infty}
           \frac{1}{b_{n}(G, \Sigma)}d\left 
           ((\Sigma \cup \Sigma^{-1})^{n}\omega
            ; \delta \right ) \\
           & \leq 
           \limsup\limits_{n \rightarrow \infty}
           \frac{1}{b_{n}(G, \Sigma)}d\left 
           ((\Sigma \cup \Sigma^{-1})^{n}\omega_j
           ; \frac{\delta}{2C} \right ) =  
           h(\Sigma, \omega_{j}; \frac{\delta}{2C}).
    \end{align*}
\end{proof}

\begin{remark}
    \cref{e_l2} and \cref{e_l3} are also true for $fh(\Sigma)$.
\end{remark}


\begin{lemma}
    Let $\Sigma \in \FF U(\AAA)$ and 
    $\omega_j \in \PP f(\HHH)$, $j \in \N$, 
    $\omega_1 \subset \omega_2
    \subset \ldots$, be such that $
    \bigcup_{j \in \N} \bigcup_{n \in \N}(
    (\Sigma \cup \Sigma^{-1})^{n}(\omega_{j})$ 
    spans a dense subspace of 
    $\HHH$. If the group generated by $\Sigma$ is amenable, then
    \begin{align*}
        h(\Sigma) = \sup_{j \in \N}h(\Sigma, \omega_{j}).
    \end{align*}
\end{lemma}

\begin{proof}
    It suffices to show that given $\omega \in \PP f(\HHH)$ and 
    $\delta > 0$ there is $\delta_{1} > 0$ and $\omega_{j}$ such that
    \begin{align*}
        h(\Sigma, \omega; \delta) \leq h(\Sigma, \omega_{j}; \delta_1). 
    \end{align*}
    By the assumptions, there is $N \in \N$ so that
    \begin{align*}
        \omega \subset_{\frac{\delta}{2}}N 
        co\left ( \mathbb{T} \left ( 
(\Sigma \cup \Sigma^{-1})^{N} \omega_{j}) \right) \right),
    \end{align*}
    where $\mathbb{T} = \{z \in \C: |z|=1\}$ and $co$ denotes
    the convex hull.

    Let $\delta_{1} = \frac{\delta}{2Nn_{j}^{\kappa_m(N)}}$ where
    $n_j = \#\omega_{j}$. Then for $B \in \FF(\HHH)$ and 
    \begin{align*}
        (\Sigma \cup \Sigma^{-1})^{N+n}\omega_{j} \subset_{\delta_1} B
    \end{align*}
    implies
    \begin{align*}
        (\Sigma \cup \Sigma^{-1})^{n}
        N co\left ( \mathbb{T} \left ( 
        (\Sigma \cup \Sigma^{-1})^{N} \omega_{j}) \right) \right)
        \subset_{\frac{\delta}{2}} B.
    \end{align*}
    Therefore
    \begin{align*}
        (\Sigma \cup \Sigma^{-1})^{n} \omega
        \subset_{\delta} B.
    \end{align*}
    Hence
    \begin{align*}
        h(\Sigma, \omega; \delta) &=
           \limsup\limits_{n \rightarrow \infty}
           \frac{1}{b_{n}(G, \Sigma)}d\left 
           ((\Sigma \cup \Sigma^{-1})^{n}\omega
            ; \delta \right ) \\
           & \leq 
           \limsup\limits_{n \rightarrow \infty}
       \frac{b_{n+N}(G, \Sigma)}{b_{n}(G, \Sigma)} 
           \frac{1}{b_{n+N}(G, \Sigma)}d\left 
           ((\Sigma \cup \Sigma^{-1})^{N+n}\omega_j
            ; \delta \right ) =  
        h(\Sigma, \omega_{j}; \delta_1)
    \end{align*}
\end{proof}

\begin{lemma}[Proposition 8.6 in \cite{V}]
   Suppose $\AAA$ is a von Neumann algebra acting on $\HHH$ and 
   $\Sigma = \{U_1, \ldots, U_m \} \in \FF U(\AAA)$, let 
   $\Sigma \otimes I_n = \{U_i \otimes I_n: U_i \in \Sigma\} \in
   \FF (\AAA \otimes I_n)$, where $\AAA \otimes I_n$ acting on 
   $\HHH \otimes l^{2}(\Z_n)$. We have
   \begin{align*}
       nh(\Sigma, \HHH) = h(\Sigma \otimes I, \HHH\otimes l^2(\Z_n)). 
   \end{align*}
\end{lemma}

\begin{lemma}[Proposition 8.4 in \cite{V}]
    Let $\AAA$ be a von Neumann algebra acting on $\HHH$. Suppose that 
    $\HHH_{1} \subset \HHH_2 \subset \ldots  \subset \HHH$ are invariant 
    subspaces of $\AAA$ and $\bigcup_{j}\HHH_{j} = \HHH$. Then
    \begin{align*}
        h(\Sigma, \HHH) = \sup_{j \in \N}h(\Sigma, \HHH_{j}). 
    \end{align*}
\end{lemma}

\begin{proposition}[Proposition 8.8 in \cite{V}]
   Let $\AAA$ be a finite factor acting on $\HHH$. Then
   \begin{align*}
       h(\Sigma, \HHH) = dim_{l^{2}(\AAA, \mu)}\HHH \times h(\Sigma, l^{2}(\AAA, \mu)),  
   \end{align*}
   where $dim_{l^{2}(\AAA, \mu)}\HHH$ is the von Neumann dimension of $\HHH$.
\end{proposition}

\section{Folner Entropy}
\begin{definition}
   Let $O\PP f(\HHH)$ be the set contains the finite orthonormal subsets of $\HHH$.
\end{definition}


\begin{remark}
    By Lemma 7.8 in \cite{V}, we have
    \begin{align*}
        d(\omega; \delta) \geq n(1-\delta^2), 
    \end{align*}
    where $\omega = \{e_1, \ldots, e_n \} \in O\PP(\HHH)$. Therefore we will use the
    following definition.
\end{remark}

\begin{definition}
    Let $\AAA$ be a von Neumann algebra.
    If $\delta > 0$, $\omega \in O\PP f(\HHH)$ and 
    $\Sigma = \{U_1, \ldots, U_m \} \in 
    \FF U(\AAA)$, we define
    \begin{align*}
        Foh(\Sigma; \delta) &= \inf_{\omega}
           \frac{1}{dim(\omega)}d\left 
           ((\Sigma \cup \Sigma^{-1})\omega
            ; \delta \right ) \\
        Foh(\Sigma, \HHH) &= \limsup\limits_{\delta \rightarrow 0}Foh(\Sigma; \delta),\\
        Foh(\AAA, \HHH) &= \inf_{\Sigma}\{Foh(\Sigma, \HHH): 
            \omega \in \PP f(\HHH), \mbox{ and $\Sigma$ generates
        $\AAA$ } \}.
    \end{align*}
\end{definition}

\begin{remark}
    It is easy to see that $Foh(\AAA, \HHH) \geq 1$. 
\end{remark}

\begin{lemma}
    If $\HHH = \HHH_1 \oplus \HHH_2$ and $\AAA = \AAA_{1} \oplus \AAA_2$, then
    \begin{align*}
        Foh(\AAA, \HHH) \leq \min(Foh(\AAA_1, \HHH_1), Foin h(\AAA_2, \HHH_2)). 
    \end{align*}
\end{lemma}

\begin{lemma}
    Let $\RR$ be the hyperfinite II$_1$ factor. $\HHH = L^{2}(\RR, \tau)$ where
    $\tau$ is the faithful normal trace on $\RR$. For any subset of unitaries 
    $\Sigma = \{U_1, \ldots, U_n\}$ of $\RR$,
    we have
    \begin{align*}
        Foh(\Sigma, \HHH) = 1. 
    \end{align*}
\end{lemma}

\begin{proof}
    Let $\delta > 0$.
    Since $\RR$ is hyperfinite, there exist a type I$_n$ subfactor $\mathcal{N}$ 
    of $\RR$ such that there are $n$ unitaries $V_1 \ldots V_n$ in $\mathcal{N}$
    satisfying
    \begin{align*}
       \|V_i - U_i\|_2 \leq \delta. 
    \end{align*}
    Let
    \begin{align*}
       W = \begin{pmatrix}
           1 & 0 & \ldots & 0 & 0\\
           0 & \gamma & \ldots & 0 & 0\\
           \vdots & \vdots & \ddots &\vdots & \vdots\\
           0 & 0& \ldots & \gamma^{n-2} & 0\\
           0 & 0 & \ldots & 0 &\gamma^{n-1}\\
       \end{pmatrix} \mbox{ and }  
        S = \begin{pmatrix}
           0 & 0 & \ldots & 0 & 1\\
           1 & 0 & \ldots & 0 & 0\\
           0 & 1 & \ldots & 0 & 0\\
           \vdots & \vdots & \ddots & \vdots & \vdots\\
           0 & 0 & \ldots & 1 & 0\\
       \end{pmatrix}
    \end{align*}
    be two unitaries that generates $\mathcal{N}$, where $\gamma = e^{\frac{2\pi i}{n}}$.
    Then $\omega = \{W^{i}S^{j}\Omega : i,j \in \{0,\ldots, n-1 \}\}$ is 
    an orthonormal subsets in $F = \{A \Omega : A \in \mathcal{N}\}$, where $\Omega$ is a
    trace vector. It is not hard to see that
    \begin{align*}
           \frac{1}{dim(\omega)}d\left 
           ((\Sigma \cup \Sigma^{-1})\omega
            ; \delta \right ) = 1.
    \end{align*}
    This clearly implies that $Foh(\Sigma, \HHH) = 1$. 
\end{proof}

\begin{lemma}
    Let $\AAA$ be a II$_1$ factor. If there is a subset of unitaries 
    $\Sigma = \{U_1, \ldots, U_n\}$ in $\AAA$ such that $\Sigma$ generates 
    $\AAA$ and $Foh(\AAA, L^{2}(\AAA, \tau)) = 0$, then $\AAA$ is hyperfinite.
\end{lemma}

%-----------------------------------------------------------------------------------
%BIBLIOGRAPHY
%-----------------------------------------------------------------------------------
\begin{thebibliography}{9}
\bibitem{V}
    Dan Voiculescu,
    \emph{Dynamical approximation entropies and topological entropy 
    in operator algebras}, 
    Comm. Math. Phys. Volume 170, Number 2 (1995)

\bibitem{AGG}
    G.N. Arzhantseva, V.S. Guba and L. Guyot. 
    \emph{Growth rates of amenable groups}, 
    Journal of Group Theory, 8 (2005), no.3, 389-394.

\bibitem{Gl}
    R. Grigorchuk and P. de la Harpe,
    \emph{Limit behaviour of exponential growth rates for finitely generated groups}, 
    Monographie de L’Enseignement Mathematique 38 (2001), 351-370.
\end{thebibliography}
%-----------------------------------------------------------------------------------

\end{document}
