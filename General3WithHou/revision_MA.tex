\documentclass[12pt]{article}

%Setup Begin-------------------------------------------------------------------------------------
\usepackage{amssymb}
\usepackage{latexsym}
\usepackage{amsmath}
\usepackage{amsfonts}

\textwidth=145truemm \textheight=240truemm
%\headsep=4truemm
\topmargin= 0pt
%\oddsidemargin=0pt \evensidemargin=0pt
\parindent=18pt

\newtheorem{theorem}{Theorem}[section]
\newtheorem{corollary}{Corollary}[section]
\newtheorem{main}{Main Theorem}[section]
\newtheorem{lemma}{Lemma}[section]
\newtheorem{prop}{Proposition}[section]
\newtheorem{df}{Definition}[section]
\newtheorem{remark}{Remark}[section]
\newtheorem{example}{Example}[section]
\newtheorem{question}{Question}[section]

\newcommand{\AAA}{\mathfrak A} \newcommand{\TTT}{\mathfrak T}
\newcommand{\BBB}{\mathcal B}
\newcommand{\CCC}{\mathcal C}
\newcommand{\DDD}{\mathcal D}
\newcommand{\GGG}{\mathcal G}
\newcommand{\HHH}{\mathcal H} %for Hilbert space
\newcommand{\LLL}{\mathcal L} % for lattice
\newcommand{\MMM}{\mathcal M}
\newcommand{\NNN}{\mathcal N} %for nest
\newcommand{\RRR}{\mathcal R}
\newcommand{\SSS}{\mathcal S}
\newcommand{\WWW}{\mathcal W}
\newcommand{\FFF}{\mathcal F}



\newcommand{\Lat}{\mathcal Lat}
\newcommand{\Alg}{\mathcal Alg}
\newcommand{\tensor}{\mathop{\bar \otimes}}
\newcommand{\tr}{\tau}

\newcommand{\C}{\mathbb C} %for complex number
\newcommand{\R}{\mathbb R}  %for real number
\newcommand{\Z}{\mathbb Z} %for integer
\newcommand{\N}{\mathbb N} % for nature number
\newcommand{\Q}{\mathbb Q} %for rational number


%The following is defined by hou

\def\L{{\mathcal{L}}}
\def\H{{\mathcal{H} }}
\def\D{{\mathscr{D} }}
\def\M{{\mathscr{ M}}}
\def\F{{\mathscr{F}}}
%\def\P{{\mathscr{P}}}
\def\P{{\mathcal{P}}}
\def\N{{\mathbb{N}}}
\def\I{{\mathbb{I}}}
\def\Z{{\mathbb{Z}}}\def\C{{\mathbb{C}}}
\def\Lat{{\mathcal Lat}}

\def\l{{\mathcal{L}}}

%%%%%%%%%%%%%%%%%%
%short cuts
%%%%%%%%%%%%%%%%%%

%Setup End----------------------------------------------------------------------------------------
\begin{document}

\begin{center}
{\Large \bf Minimal Generating Reflexive Lattices of Projections in
Finite von Neumann Algebras}
\end{center}

\begin{center}

{\bf Chengjun Hou}\\

Department of Mathematics, Qufu Normal University, Qufu 273165,
China\\
e-mail: cjhou@mail.qfnu.edu.cn\\
\vspace{2mm}

{\bf Wei Yuan}\\
L. K. Hua Key Laboratory of Mathematics, Chinese Academy of
Sciences,
Beijing 100190, China\\
e-mail: wyuan@math.ac.cn

\end{center}

\noindent{\small {\small\bf Abstract} \ \ We show that the reflexive
lattice generated by a double triangle lattice of projections in a
finite von Neumann algebra is topologically homeomorphic to the
two-dimensional sphere $S^2$ (plus two distinct points corresponding
to zero and $I$). Furthermore, such a reflexive lattice is in
general minimally generating for the von Neumann algebra it
generates. As an application, we show that if a reflexive lattice
$\FFF$ generates the algebra $M_n(\C)$ of all $n\times n$ complex
matrices, for some $n\geq 3$, then $\FFF\setminus\{0,I\}$ is
connected if and only if it is homeomorphic to $S^2$.



\vspace{2mm}\baselineskip 12pt

\noindent{\small\bf Keywords} \ \ Kadison-Singer algebra,
Kadison-Singer lattice, von Neumann algebra, unbounded operator.

\vspace{2mm}\baselineskip 12pt

\noindent{\small\bf MSC(2010)} 46L10, 47L75, 47L60



\date{}
\title{{\bf Minimal Generating Reflexive Lattices of Projections in
Finite von Neumann Algebras}}
\author{\bf Chengjun Hou$^{\ast}$, Wei Yuan}\thanks{
Corresponding author: Chengjun Hou\\
Department of Mathematics, Qufu Normal University, Qufu 273165,
China\\
e-mail: cjhou@mail.qfnu.edu.cn\\
Wei Yuan\\
L. K. Hua Key Laboratory of Mathematics, Chinese Academy of
Sciences,
Beijing 100190, China\\
e-mail: wyuan@math.ac.cn} \maketitle

\baselineskip 14pt

 \noindent{\small {\small\bf Abstract} \ \ We show that the reflexive
lattice generated by a double triangle lattice of projections in a
finite von Neumann algebra is topologically homeomorphic to the
two-dimensional sphere $S^2$ (plus two distinct points corresponding
to zero and $I$). Furthermore, such a reflexive lattice is in
general minimally generating for the von Neumann algebra it
generates. As an application, we show that if a reflexive lattice
$\FFF$ generates the algebra $M_n(\C)$ of all $n\times n$ complex
matrices, for some $n\geq 3$, then $\FFF\setminus\{0,I\}$ is
connected if and only if it is homeomorphic to $S^2$. }



\vspace{2mm}
\baselineskip 12pt

\noindent{\small\bf Keywords} \ \ Kadison-Singer algebra,
Kadison-Singer lattice, von Neumann algebra, unbounded operator.

\vspace{2mm}\baselineskip 12pt

 \noindent{\small\bf MSC(2010)}
46L10, 47L75, 47L60

\baselineskip 16pt

%\newpage
\section{Introduction}

In their seminal article \cite{KS}, Kadison and Singer initiated the
study of non selfadjoint algebras of bounded linear operators on
Hilbert spaces.  They introduced and studied  a class of operator
algebras which they called triangular (operator) algebras. An
algebra $\mathcal{T}$ is said to be triangular  in a factor
$\mathcal{M}$ ( a von Neumann algebra with a trivial center) when
its diagonal subalgebra $\mathcal{T}\cap \mathcal{T}^*$ is a maximal
abelian selfadjoint subalgebra of $\mathcal{M}$. Nest algebras were
introduced by Ringrose (\cite{Ri}) as generalizations of certain
triangular algebras. Reflexive (operator) algebras are more general
than nest algebras (\cite{Ha}).  Let $\HHH$ be a complex Hilbert
space and $B(\HHH)$ the algebra of all bounded linear operators on
$\HHH$. By a nest $\NNN$ we mean a totally ordered family of
(orthogonal) projections on a Hilbert space $\HHH$ containing $0$
and the identity operator $I$ which is closed in the strong operator
topology. The nest algebra associated with $\NNN$ is the set of all
operators in $B(\HHH)$ that leave the range of each projection in
$\NNN$ invariant. A subalgebra $\AAA$ of $B(\HHH)$ is called
reflexive if $\AAA$ is equal to the algebra of all operators in
$B(\HHH)$ which leave invariant the range of each projection in a
family of projections in $B(\HHH)$. Similarly, a lattice of
projections in $B(\HHH)$ is called reflexive if it is equal to the
set of all projections (with their ranges) invariant under each
operator in a family of operators in $B(\HHH)$. Hence reflexive
algebras are completely determined by their lattices of invariant
projections. In general, non selfadjoint operator algebras are
closely related to operator theory and invariant subspaces of
operators. Parallel to the selfadjoint theory ($C^*$- and von
Neumann algebras), non selfadjoint theory has undergone a vigorous
development in the past 50 years. Many important results were
obtained by many authors, e.g., Arveson (\cite{Ar}), Larson
(\cite{La}), Davidson (\cite{Da}) and Lance (\cite{Lan}).


Recently, motivated by the work of Kadison and Singer on maximal
triangular algebras, Ge and Yuan (\cite{GY1}) introduced a new class
of non selfadjoint algebras which they called Kadison-Singer
algebras, or KS-algebras for simplicity. KS-algebras combine
triangularity (\cite{KS}), reflexivity (\cite{Ha,Da}) and von
Neumann algebra properties into one package. Recall that a
subalgebra $\AAA$ of $B(\HHH)$ is called a KS-algebra if $\AAA$ is
reflexive and maximal with respect to the diagonal subalgebra $\AAA
\cap \AAA^*$ of $\AAA$, in the sense that if there is another
reflexive subalgebra $\mathcal{B}$ of $B(\HHH)$ such that $\AAA
\subseteq\mathcal{B}$ and $\mathcal{B}\cap\mathcal{B}^*=\AAA
\cap\AAA^*$, then $\AAA=\mathcal{B}$. The lattice of all invariant
projections of a KS-algebra is called a KS-lattice. Equivalently, a
reflexive lattice $\LLL$ of projections in $B(\HHH)$ is called a
KS-lattice if $\LLL$ is a minimally generating reflexive lattice of
the von Neumann algebra it generates, which means that $\LLL$ and
any proper reflexive sublattice of $\LLL$ do not generate the same
von Neumann subalgebra of $B(\HHH)$. Hence the study of KS-lattices
is closely related to the generator problem for von Neumann algebras
(\cite{GS}).




In \cite{GY1, GY2}, Ge and Yuan constructed KS-algebras with
hyperfinite factors as their diagonals. They also proved that the
reflexive lattice determined by three free projections with trace
$\frac12$ is homeomorphic to the two-dimensional sphere $S^2$ (plus
two distinct points corresponding to zero and $I$). It is easy to
see that the lattice generated by three free projections with trace
$\frac12$ is a double triangle lattice, which means that it is a
five-element lattice containing three nontrivial elements such that
the intersection of any two is zero and the union of any two is $I$.
In \cite{GY2}, the authors claimed that if a double triangle lattice
of projections in a type $II_1$ factor satisfies some specified
conditions, then the reflexive lattice generated by it is also
homeomorphic to $S^2$ (plus zero and $I$). Hence, for these cases,
Halmos's problem, which asks whether any realization of a double
triangle lattice is reflexive, was settled. These results also
indicate that many type $II_1$ factors can be minimally generated by
reflexive lattices of projections which are topologically
homeomorphic to $S^2$ (plus zero and $I$). Some examples of
KS-lattices with different topological structures were also given in
\cite{Hou, WY}.

In this paper, we study the reflexive lattice generated by a double
triangle lattice of projections in a finite von Neumann algebra. Our
methods are different from those in \cite{GY2} where heavy free
probability techniques were involved. In the section following the
introduction, by using the theory of unbounded operators affiliated
with a von Neumann algebra, we show that the reflexive lattice
generated by any double triangle lattice of projections in a finite
von Neumann algebra is topologically homeomorphic to $S^2$ (plus
zero and $I$). In particular, we prove that each nontrivial
projection in such a reflexive lattice has trace $\frac12$, and that
such a reflexive lattice is a KS-lattice if the von Neumann algebra
generated by the double triangle lattice cannot be generated by any
two nontrivial projections. In Section 3, we apply our results in
Section 2 to describe the ``connected" reflexive lattices in matrix
algebras. We show that if a reflexive lattice $\FFF$ generates the
matrix algebra $M_n(\C)$ for $n\geq 3$, and $\FFF\setminus\{0,I\}$
is connected, then $\FFF\setminus\{0,I\}$ is topologically
homeomorphic to $S^2$. Moreover, $\FFF$ is a KS-lattice.


\section{Three projections in finite von Neumann algebras}

We first recall some definitions in the theory of non selfadjoint
operator algebras. For basic facts on operator algebras, we refer to
\cite{KR, RR}.

Suppose $\HHH$ is a separable complex Hilbert space and $B(\HHH)$
the algebra of all bounded linear operators on $\HHH$. For a set
$\LLL$ of orthogonal projections in $B(\HHH)$, we denote by
$Alg\,\LLL$ the set of all bounded linear operators on $\HHH$
leaving (the range of) each projection in $\LLL$ invariant. Then
$Alg\,\LLL$ is a unital, weak-operator closed subalgebra of
$B(\HHH)$. Similarly, for a subset $\mathcal{S}$ of $B(\HHH)$, let
$Lat\,\mathcal{S}$ be the set of all projections (with their ranges)
invariant under each operator in $\mathcal{S}$. Then
$Lat\,\mathcal{S}$ is a strong-operator closed lattice of
projections. A subalgebra $\mathcal{A}$ of $B(\HHH)$ is said to be
reflexive if $\mathcal{A}=Alg\,Lat\,\mathcal{A}$. Similarly, a
lattice $\LLL$ of projections in $B(\HHH)$ is called reflexive if
$\LLL=Lat\,Alg\,\LLL$.

Throughout this section, $\MMM$ denotes a finite von Neumann algebra
acting on $\HHH$, and $tr$ is a faithful, normal trace on $\MMM$.
Let $\AAA = M_{2}(\C)\otimes\MMM$. Then $\AAA$ is a finite von
Neumann algebra acting on $\HHH\oplus \HHH$. Let $\{ E_{i,j} \}_{i,j
= 1}^{2}$ be the standard matrix unit system in $M_2(\C)$. Each
operator $T$ in $B(\HHH\oplus \HHH)$ can be expressed as an operator
matrix with respect to the units:
\begin{align*}
T=\left(\begin{array}{cc}T_{11} & T_{12} \\T_{21} &
T_{22}\end{array}\right),\,\,\,\,\, T_{ij}\in B(\HHH).
\end{align*}
Then the state $\tau$, defined by $\tau(T)=\frac{1}{2}(tr(T_{11}) +
tr(T_{22}))$ for $T\in \AAA$, is a faithful, normal trace on $\AAA$.
We use the same symbol $I$ to denote the identity operator in $\MMM$
and in $\AAA$ when there is no risk of ambiguity. Let $P_1
=\left(\begin{array}{cc}I & 0 \\0 & 0\end{array}\right)
(=E_{11}\otimes I)$ in $\AAA$. Now we state our first lemma, the
proof of which might be well-known.

\begin{lemma}
With the above notation, let $P$ be a projection in $\AAA$ such that
$P\vee P_1=I$ and $P\wedge P_1=0$. Then there are a unitary operator
$V$ in $\MMM$ and a  positive contractive operator $H$ in $\MMM$
such that $I-H$ is injective and
$$P=\left(\begin{array}{cc} H & \sqrt{H(I-H)}V\\ V^*\sqrt{H(I-H)} & V^*(I-H)V \end{array}\right).$$
Furthermore, $\tau(P)=\frac12$.
\end{lemma}

\noindent{\it Proof}\quad By Kaplansky formula (\cite[Theorem
6.1.7]{KR}), we have  $\tau(P_1\vee
P)=\tau(P_1)+\tau(P)-\tau(P_1\wedge P)$. Thus $\tau(P)=\frac12$. Let
us write formally
\begin{align*}
P= \left(\begin{array}{cc}H & H_{1}V \\V^* H_{1} &
H_{2}\end{array}\right),
\end{align*}
where $H$, $H_1$ and $H_2$ are positive operators in $\MMM$, and $V$
is a unitary operator in $\MMM$. By $P^2 = P$, we have
$$H = H^2+H_{1}^{2},\,\,\,\,\,
H_2= V^*H_1^2V+H_2^2\,\,\, \hbox{ and }\,\,\, H_1V =HH_{1}V +
H_1VH_2.$$ Hence we have
$$H_1=\sqrt{H(I-H)},\eqno{(1)}$$
$$V^*H(I-H)V=H_2(I-H_2),\eqno{(2)}$$
$$(I-H)\sqrt{H(I-H)}V=\sqrt{H(I-H)}VH_2.\eqno{(3)}$$

Now we show that $I-H$ and $H_2$ are injective. Indeed, for each
vector $\xi$ in $\HHH$ with $(I-H)\xi=0$, we have $\xi=H\xi$ and
$\sqrt{I-H}\xi=0$, which imply $\xi\oplus 0\in P_1\wedge P$. Hence
$\xi=0$, and thus, $I-H$ is injective. Similarly, for each vector
$\eta$ in $\HHH$ with $H_2\eta=0$, we have $H_{1}V\eta =
\sqrt{H(I-H)}V\eta=0$ by (1) and (3). Thus $0\oplus \eta \in
P_1^{\perp}\wedge P^{\perp}$, which implies $\eta=0$. Hence $H_2$ is
injective.


Next we show that $H_2=V^*(I-H)V$. Clearly, (3) and the injectivity
of $I-H$ imply that $H(I-H)V=HVH_2$. Hence $V^*H(I-H)V=V^*HVH_2$,
which, together with (2), gives us that $V^*HVH_2=H_2(I-H_2)$.
Therefore, by the injectivity of $H_2$, we have $H_2=V^*(I-H)V$.
This completes the proof of our lemma.


\vspace{2mm} Before we proceed to the proof of our main result, we
recall some basic facts on unbounded operators affiliated with a
finite von Neumann algebra.

For a closed, densely defined operator $T$, we denote its  domain,
range, null space, range projection and null projection by
$\DDD(T),\, Ran(T),\, Ker(T)$, $R(T)$ and $N(T)$, respectively. A
dense subspace $\DDD_{0}$ of $\DDD(T)$ is called a core for $T$ if
$\mathcal{G}(T|_{ \DDD_{0}})^{-} = \mathcal{G}(T)$, where $T|_{
\DDD_{0}}$ is the restriction of $T$ to $\DDD_{0}$, $\mathcal{G}(T)$
is the graph of $T$ and $\mathcal{G}(T|_{ \DDD_{0}})^{-}$ is the
closure of $\mathcal{G}(T|_{ \DDD_{0}})$. We say that the operator
$T$ is affiliated with a von Neumann algebra $\mathcal{R}$ when
$U^{*}TU = T$  for each unitary operator $U$ commuting with
$\mathcal{R}$. In \cite{MV}, Murray and von Neumann showed that the
family of all operators affiliated with a finite von Neumann algebra
$\MMM$ forms an associative algebra, denoted by $\widetilde{\MMM}$.
For arbitrary operators $X$ and $Y$ in $\widetilde{\MMM}$, $X +Y$
and $XY$ are densely defined and closable. Their closures, denoted
by $X \widehat{+} Y$ and $X\widehat{\cdot} Y$, respectively, are in
$\widetilde{\MMM}$. If $\xi \in \DDD(X \widehat{+} Y) \cap \DDD(X)$,
then $\xi \in \DDD((X \widehat{+} Y) - X)$. Since $\widetilde{\MMM}$
is an algebra and $(X \widehat{+} Y) \widehat{-} X = Y$, we have
$\xi \in \DDD(Y)$ and $(X \widehat{+} Y)\xi = X\xi + Y\xi$.
Moreover, $Ker(X \widehat{+} Y) = 0$ if and only if $Ker(X +Y) = 0$.
Indeed, if $Ker(X +Y) = 0$ and $Ker(X \widehat{+} Y) \neq 0 $, then
there exists a projection $Q$ in $\MMM$ such that $tr(Q)> 0$, and
for any $\xi$ in $Q(\HHH)$, $(X \widehat{+} Y)\xi = 0$. Hence we can
choose two projections $E$ and $F$ in $\MMM$ such that $E(\HHH)
\subset \DDD(X)$, $F(\HHH) \subset \DDD(Y)$ and $tr(E \wedge F
\wedge Q) > 0$. This means that we can find a nonzero vector $\xi$
in $(E \wedge F \wedge Q) \HHH$, which contradicts the fact that
$Ker(X +Y) = 0$. For more details on unbounded operators, we refer
to \cite[Exercises 6.9.54 and 8.7.60]{KR1} and \cite{MV}.


\begin{lemma}
Let $P_1$ be as before, $P_2$ and $P_3$ be two projections in $\AAA$
such that $P_1\wedge P_i=0$ and $P_1\vee P_i=I$ for $i=2,3$. We
assume
\begin{align*}
&P_2 = \left(\begin{array}{cc}H_1 & \sqrt{H_1 (I-H_1)}V_1 \\V_1^*
\sqrt{H_1 (I-H_1)} & V_1^{*}(I - H_1)V_1\end{array}\right)\,\,\,\,\,\hbox{ and }\\
&P_3 = \left(\begin{array}{cc}H_2 & \sqrt{H_2(I-H_2)}V_2 \\V_2^*
\sqrt{H_2(I-H_2)} & V_2^{*}(I - H_2)V_2\end{array}\right),
\end{align*}
where $V_i$ is a unitary operator and $H_i$ is a positive
contractive operator in $\MMM$ such that $Ker(I-H_i)=0$ for $i=1,2$.
Then
\begin{enumerate}
\item[(i)] $P_2 \wedge P_3 = 0$ if and only if
$Ker(\sqrt{H_1(I-H_1)^{-1}}V_1-\sqrt{H_2(I-H_2)^{-1}}V_2)=0;$
\item[(ii)] $P_2\vee P_3=I$ if and only if
$Ker(V_1^*\sqrt{H_1(I-H_1)^{-1}}-V_2^*\sqrt{H_2(I-H_2)^{-1}})=0.$
\end{enumerate}
\end{lemma}

\noindent{\it Proof}\quad (i) If $Ker(\sqrt{H_1(I-H_1)^{-1}}V_1 -
\sqrt{H_2(I-H_2)^{-1}}V_2)  \neq 0$, then there is a nonzero vector
$\xi$ in $\DDD(\sqrt{H_1(I-H_1)^{-1}}V_1)  \cap
\DDD(\sqrt{H_2(I-H_2)^{-1}}V_2)$ such that
\begin{align*}
\sqrt{H_1(I-H_1)^{-1}}V_1 \xi =  \sqrt{H_2(I-H_2)^{-1}}V_2 \xi \overset{\text{def}}{=} \eta.
\end{align*}
Then $\sqrt{H_i(I-H_i)}\eta=H_iV_i\xi$ for  $i=1,2$. So $\eta\oplus
\xi\in P_2\wedge P_3$, which yields $P_2 \wedge P_3\neq 0$.  We can
reverse the above argument to obtain the other direction.

Note that $P_i\vee P_j=I$ if and only if  $(I - P_i)\wedge (I-
P_j)=0$, for $i,j=1,2,3$ and $i \neq j$. We can obtain (ii) by
applying a similar argument to $I-P_1$, $I-P_2$ and $I-P_3$.
\vspace{2mm}

\begin{remark}
In the lemma above, we may assume that $V_1 = I$ by changing $P_2$
to $U^{*}P_2 U$, and $P_3$ to $U^* P_3 U$, where $U =
\left(\begin{array}{cc}I & 0 \\0 & V_{1}^{*}\end{array}\right)$.
\end{remark}

From now on, let $P_1$ be as before, $P_2$ and $P_3$ be two
projections in $\AAA$ such that $P_i \wedge P_j = 0$ and $P_i \vee
P_j = I$, for $i,j=1,2,3$ and $i \neq j$.  By Lemma 2.1, we have
$\tau(P_i)=\frac12$ for $i = 1,2,3$. Let $\LLL=\{0,P_1,P_2,P_3,I\}$.
Then $\LLL$ is a double triangle lattice of projections in $\AAA$.
In the following, we study the reflexive lattice determined by
$\LLL$, as well as the corresponding reflexive algebra. By Lemma 2.2
and its remark, we may assume
\begin{align*}
&P_2 = \left(\begin{array}{cc}H_1 & \sqrt{H_1 (I-H_1)} \\\sqrt{H_1
(I-H_1)} & I - H_1\end{array}\right)\,\,\,\,\,\,\,\,\hbox{and} \\
&P_3 = \left(\begin{array}{cc}H_2 & \sqrt{H_2(I-H_2)}V \\V^*
\sqrt{H_2(I-H_2)} & V^{*}(I - H_2)V\end{array}\right),
\end{align*}
where $H_i$ is a positive contractive operator in $\MMM$ such that
$Ker(I-H_i)=0$ for $i=1,2$, and $V$ is a unitary operator in $\MMM$.
Let
\begin{align*}
S = \sqrt{H_{1}(I-H_{1})^{-1}} \widehat{-}
\sqrt{H_{2}(I-H_{2})^{-1}}V.
\end{align*}
Then, by Lemma 2.2 and the discussion before it, we know that $S$ is
in general an unbounded, invertible operator affiliated with $\MMM$,
and its inverse is also in $\widetilde{\MMM}$.

\begin{lemma} With the above notation, suppose $T=\left(\begin{array}{cc} T_1 & T_2\\
T_4 & T_3\end{array}\right) \in M_2(\C)\otimes B(\HHH)$. Then $T$ is
in $Alg\,\L$ if and only if $T_4=0$ and the following equations
hold:
$$\sqrt{I-H_1}T_2\sqrt{I-H_1}=\sqrt{H_1}T_3\sqrt{I-H_1}-\sqrt{I-H_1}T_1\sqrt{H_1},\eqno{(4)}$$
$$\sqrt{I-H_2}T_2V^*\sqrt{I-H_2}=\sqrt{H_2}VT_3V^*\sqrt{I-H_2}-\sqrt{I-H_2}T_1\sqrt{H_2}.\eqno{(5)}$$
\end{lemma}

\noindent{\it Proof}\quad   Note that $P_2=W_2^*P_1W_2$ and
$P_3=W_3^*P_1W_3$, where $$W_2=\left(\begin{array}{cc} \sqrt{H_1} &
\sqrt{I-H_1}\\ \sqrt{I-H_1} & -\sqrt{H_1}
\end{array}\right)\,\, \hbox{ and }\,\,\,W_3=\left(\begin{array}{cc} \sqrt{H_2} &
\sqrt{I-H_2}V\\ \sqrt{I-H_2} & -\sqrt{H_2}V
\end{array}\right)$$
are unitary operators. Hence $T$ is in $ Alg\,\L$ if and only if
$(I- P_1)TP_1=(I-P_1)W_2TW_2^*P_1=(I-P_1)W_3TW_3^*P_1=0$. A
straightforward computation shows that (4),(5) and $T_4 = 0$ are
equivalent to these equalities. \vspace{2mm}

\begin{lemma} With the above notation, if $
T\in Alg\,\LLL$, then there is  an operator $T_1$ in $B(\HHH)$ such
that
\begin{align*}
T = \left(\begin{array}{cc}T_1 & \sqrt{H_{1}(I-H_{1})^{-1}}S^{-1}T_1 S -T_1 \sqrt{H_{1}(I-H_{1})^{-1}}  \\0 & S^{-1}T_1S \end{array}\right).
\end{align*}
Conversely, if $T_1 \in B(\HHH)$ such that
$\sqrt{H_{1}(I-H_{1})^{-1}}S^{-1}T_1S - T_1
\sqrt{H_{1}(I-H_{1})^{-1}} =T_2$ and $ S^{-1}T_1 S = T_3$ for some
bounded operators $T_2$ and $T_3$, then $T$ belongs to $ Alg\,\LLL$.
\end{lemma}

Before the proof of Lemma 2.4, we note that the equalities
$\sqrt{H_{1}(I-H_{1})^{-1}}S^{-1}T_1 S - T_1
\sqrt{H_{1}(I-H_{1})^{-1}} = T_2$ and $S^{-1}T_1 S= T_3$ in the
lemma hold in the sense that, for each vector $\xi$ in
$\DDD(\sqrt{H_1(I-H_1)^{-1}}) \cap \DDD(S)$ (=
$\DDD(\sqrt{H_1(I-H_1)^{-1}}) \cap \DDD(\sqrt{H_2(I-H_2)^{-1}}V) $),
$S^{-1}T_1 S\xi$ and $\sqrt{H_{1}(I-H_{1})^{-1}}S^{-1}T_1 S\xi - T_1
\sqrt{H_{1}(I-H_{1})^{-1}}\xi$ are defined and equal to $T_3\xi$ and
$T_2 \xi$, respectively. \vspace{2mm}


\noindent{\it Proof}\quad By Lemma 2.3, $T \in  Alg\,\LLL$ if and
only if $T$ has the
 form $T = \left(\begin{array}{cc}T_1 & T_2 \\0 &
T_3\end{array}\right)$, where $T_1$, $T_2$ and $T_3$ satisfy (4) and
(5). Since $\DDD(\sqrt{H_1(I-H_1)^{-1}}) \cap
\DDD(\sqrt{H_2(I-H_2)^{-1}}V)$ is a common core for both
$\sqrt{H_1(I-H_1)^{-1}}$ and $\sqrt{H_2(I-H_2)^{-1}}V$, we conclude
that (4) and (5) are true if and only if, for each vector $\xi$ in
$\DDD(\sqrt{(I-H_1)^{-1}}) \cap \DDD(\sqrt{(I-H_2)^{-1}}V)$,
\begin{align*}
T_2 \xi &= \sqrt{H_1(I - H_1)^{-1}} T_3 \xi - T_1 \sqrt{H_1(I-H_1)^{-1}} \xi \,\,\,\,\,\,\,\,\,\,\hbox{and} \\
T_2 \xi &= \sqrt{H_2(I - H_2)^{-1}}V T_3 \xi - T_1
\sqrt{H_2(I-H_2)^{-1}}V \xi .
\end{align*}
These are equivalent to
\begin{align*}
T_2 \xi = \sqrt{H_1(I - H_1)^{-1}} T_3 \xi - T_1
\sqrt{H_1(I-H_1)^{-1}} \xi \,\,\,\,\, \hbox{and}\,\,\,\, T_3 \xi =
S^{-1}T_1S\xi.
\end{align*}
This completes our proof.



\begin{remark}
Actually, there exists a family $\{E_{\epsilon}:\,\epsilon>0\}$ of
projections in $\MMM$ such that $S^{-1}E_{\epsilon}$,
$E_{\epsilon}S$, $E_{\epsilon}\sqrt{H_{1}(I-H_{1})^{-1}}$ and
$\sqrt{H_{1}(I-H_{1})^{-1}}S^{-1}E_{\epsilon}$ are bounded for each
$\epsilon$, and $\{E_{\epsilon}\}$ is convergent to $I$ under the
strong operator topology (as $\epsilon\rightarrow 0$). Hence the set
of all operators $T_1$ that satisfy the conditions in Lemma 2.4 is
dense in $B(\HHH)$ under the strong operator topology.
\end{remark}

\begin{corollary} For $Q$ in $Lat\,Alg\,\L \setminus \{0,I, P_1\}$,
 we have $Q\wedge P_1=0$, $Q\vee P_1=I$ and $\tau(Q)=\frac12$.
Moreover, for any two distinct nontrivial projections $Q_1$ and
$Q_2$ in $Lat\,Alg\,\L$,
 we have $Q_1\wedge Q_2=0$ and $Q_1\vee Q_2=I$.
\end{corollary}


 \noindent{\it Proof}\quad For any given $Q$ as in the corollary, let $Q\wedge P_1=\left(\begin{array}{cc} P & 0 \\ 0 &
0\end{array}\right)$, where $P$ is a projection in $\MMM$. Since
$Q\wedge P_1 \in Lat\,Alg\,\L$, we have that $P$ is invariant under
$P_1TP_1|_{\HHH}$ for all $T$ in $Alg\,\L$. By Remark 2.2, $P=0$ or
$I$. Thus $Q\wedge P_1=0$ or $P_1$. Similarly, we have $Q\vee
P_1=P_1$ or $I$. It follows that $Q\wedge P_1=0$, $Q\vee P_1=I$ and
$\tau(Q)=\frac12$.

For any distinct projections $Q_1$ and $Q_2$ in $Lat\,Alg\,\L
\setminus \{0, I, P_1 \}$, we have $Q_i\wedge P_1=0$, $Q_i\vee
P_1=I$ and $\tau(Q_i)=\frac12$ for $i=1,2$. Clearly, $Q_1\wedge
Q_2\neq P_1$. Suppose that $Q_1\wedge Q_2\neq 0$. Since $Q_1\wedge
Q_2 \in Lat\,Alg\,\L \setminus \{0,I\}$, we have $\tau(Q_1\wedge
Q_2)=\frac12$. This implies that $Q_1\wedge Q_2=Q_1=Q_2$, which
contradicts  the fact that $Q_1 \neq Q_2$. Hence $Q_1\wedge Q_2=0$
and $Q_1\vee Q_2=I$. \vspace{2mm}



\begin{prop}
For any projection $Q$ in $Lat\,Alg\,\LLL \setminus \{0, I,
P_{1}\}$, there are $K_a$ and $U_a$ in $\MMM$ such that
\begin{align*}
Q = \left(\begin{array}{cc}K_{a} & \sqrt{K_{a}(I-K_{a})}U_{a} \\U_{a}^{*}\sqrt{K_{a}(I-K_{a})} &
U_{a}^{*}(I-K_{a})U_{a} \end{array}\right),
\end{align*}
where $K_{a}$ and $U_{a}$ are determined by the polar decomposition
$\sqrt{K_{a}(I-K_{a})^{-1}} U_{a}$ of $aS \widehat{+}
\sqrt{H_{1}(I-H_{1})^{-1}}$ for some scalar $a$ in $\C$. Conversely,
for any given $a$ in $\C$, the polar decomposition determines
$U_{a}$ and $K_{a}$, which give rise to a projection $Q$ (in the
above form) in $Lat\,Alg\,\LLL$.
\end{prop}


\noindent{\it Proof}\quad Suppose  $Q$ is in $Lat\,Alg\,\LLL
\setminus \{0, I, P_{1}\}$. From Corollary 2.1, we know that
$Q\wedge P_1=0$ and $Q\vee P_1=I$. It follows from Lemma 2.1 that
$Q$ has the form
\begin{align*}
Q = \left(\begin{array}{cc}K & \sqrt{K(I-K)}U \\U^{*}\sqrt{K(I-K)} &
U^{*}(I-K)U \end{array}\right),
\end{align*}
where $K$ is a positive contractive operator in $\MMM$ such that
$Ker(I-K)=0$, and $U$ is a unitary operator in $\MMM$. Let
$\{E_\epsilon \}_{\epsilon>0}$ be an increasing net (as $\epsilon
\rightarrow 0$) of projections in $\MMM$ with the strong-operator
limit $I$ such that, for each $\epsilon$, $E_\epsilon \sqrt{H_1 (I -
H_1)^{-1}}$, $S^{-1}E_\epsilon$, $\sqrt{H_1(I-H_1)^{-1}}S^{-1}
E_\epsilon$, $E_{\epsilon}S$ and $(SU^{*}\sqrt{I-K})^{-1}E_\epsilon$
are all bounded operators. Then, for each operator $A$ in $B(\HHH)$
with $E_\epsilon A = A E_\epsilon = A$, we have $S^{-1}AS$,
$\sqrt{H_{1}(I-H_{1})^{-1}}S^{-1}AS$ and
$A\sqrt{H_{1}(I-H_{1})^{-1}}$ are bounded operators. By Lemma 2.4,
\begin{align*}
T = \left(\begin{array}{cc}A & \sqrt{H_{1}(I-H_{1})^{-1}}S^{-1}AS
-A\sqrt{H_{1}(I-H_{1})^{-1}}  \\0 & S^{-1}AS \end{array}\right) \in
 Alg\,\LLL.
\end{align*}
Exactly as in the proof of  Lemma 2.3, the equation $(I-Q)TQ = 0$
implies that
\begin{align*}
\sqrt{I-K}&A\sqrt{K} +
\sqrt{I-K}\left[\sqrt{H_{1}(I-H_{1})^{-1}}S^{-1}AS \right.\cr & -
\left.A\sqrt{H_{1}(I-H_{1})^{-1}}\right]U^{*}\sqrt{I-K} -
\sqrt{K}US^{-1}ASU^{*}\sqrt{I-K} = 0.
\end{align*}
This is equivalent to
\begin{align*}
\sqrt{I-K}A&\left[\sqrt{K} -
E_{\epsilon}\sqrt{H_{1}(I-H_{1})^{-1}}U^{*}\sqrt{I-K}\right]\\
& =\left[\sqrt{K}US^{-1}E_{\epsilon} -
\sqrt{I-K}\sqrt{H_{1}(I-H_{1})^{-1}}S^{-1}E_{\epsilon}\right]ASU^{*}\sqrt{I-K}.
\end{align*}
Multiplying both sides of the equation by $\sqrt{(I-K)^{-1}}$ on the
left and $(SU^{*}\sqrt{I-K})^{-1}E_\epsilon$ on the right, we have
\begin{align*}
A&\left[\sqrt{K} -
E_\epsilon\sqrt{H_{1}(I-H_{1})^{-1}}U^{*}\sqrt{I-K}\right](SU^{*}\sqrt{I-K})^{-1}E_\epsilon \\
& \qquad = \sqrt{(I-K)^{-1}}\left[\sqrt{K}US^{-1}E_{\epsilon} -
\sqrt{I-K}\sqrt{H_{1}(I-H_{1})^{-1}}S^{-1}E_{\epsilon}\right] A E_\epsilon .
\end{align*}
Since$\left[ E_\epsilon\sqrt{K} -
E_\epsilon\sqrt{H_{1}(I-H_{1})^{-1}}U^{*}\sqrt{I-K}\right](SU^{*}\sqrt{I-K})^{-1}E_\epsilon
$ is bounded, by letting $A = E_\epsilon$, we know that
$\sqrt{(I-K)^{-1}}\left[\sqrt{K}US^{-1}E_{\epsilon} -
\sqrt{I-K}\sqrt{H_{1}(I-H_{1})^{-1}}S^{-1}E_{\epsilon}\right]$ is
also bounded. Hence there is a unique $a$ in $\mathbb{C}$ such that
\begin{align*}
&[E_\epsilon\sqrt{K} -
E_\epsilon\sqrt{H_{1}(I-H_{1})^{-1}}U^{*}\sqrt{I-K}](SU^{*}\sqrt{I-K})^{-1}E_\epsilon
\\
& \qquad =\sqrt{(I-K)^{-1}}\left[\sqrt{K}US^{-1}E_{\epsilon} -
\sqrt{I-K}\sqrt{H_{1}(I-H_{1})^{-1}}S^{-1}E_{\epsilon}\right] = a
E_\epsilon.
\end{align*}
Letting $\epsilon \rightarrow 0$, we have
\begin{align*}
\sqrt{K(I-K)^{-1}}U = aS\widehat{+} \sqrt{H_{1}(I-H_{1})^{-1}}.
\end{align*}

Conversely, if  $K$ and $U$ are given by the above equation as a
polar decomposition of the right hand side, then, by Lemma 2.4, one
can check easily that $Q$ given in the proposition is in
$Lat\,Alg\,\LLL$. \vspace{2mm}

By this proposition, we have the following corollary.

\begin{corollary}
For any two distinct projections $Q_1$ and $Q_2$ in $Lat\,Alg\,\LLL
\setminus \{ 0, I, P_1 \}$, we have  $ Alg\,\LLL =  Alg\,\{P_1, Q_1,
Q_2\}$ and $Lat\,Alg\,\{P_1, Q_1, Q_2 \} = Lat\,Alg\,\LLL.$
\end{corollary}

\noindent{\it Proof}\quad Obviously, $ Alg\,\LLL\subseteq
Alg\,\{P_1, Q_1, Q_2 \}$. On the other hand, following Proposition
2.1, we write
$$Q_i= \left(\begin{array}{cc}K_{a_i}
& \sqrt{K_{a_i}(I-K_{a_i})}U_{a_i} \\U_{a_i}^{*}\sqrt{K_{a_i}(I-K_{a_i})} &
U_{a_i}^{*}(I-K_{a_i})U_{a_i}
\end{array}\right),
$$ where  $K_{a_i}$ and $U_{a_i}$ are determined by the polar
decomposition $a_i S\widehat{+ }\sqrt{H_{1}(I-H_{1})^{-1}}=
\sqrt{K_{a_i}(I-K_{a_i})^{-1}} U_{a_i}$, for $a_i\in \C$, $i=1,2$
and $a_1\neq a_2$. Then $$\sqrt{K_{a_1}(I-K_{a_1})^{-1}}
U_{a_1}\widehat{-}\sqrt{K_{a_2}(I-K_{a_2})^{-1}}
U_{a_2}=(a_1-a_2)S.$$

A similar argument to the proof of Lemma 2.4 gives us that $T \in
Alg\,\{P_1,Q_1,Q_2\}$ if and only if there exists an operator $T_1$
in $B(\HHH)$ such that
$$T=\left(\begin{array}{cc} T_1 & \sqrt{K_{a_1}(I-K_{a_1})^{-1}}U_{a_1}S^{-1}T_1 S-T_1\sqrt{K_{a_1}(I-K_{a_1})^{-1}}U_{a_1}\\
0 & S^{-1}T_1 S\end{array}\right).$$ However, for any $\xi$ in
$\DDD(\sqrt{K_{a_1}(I-K_{a_1})^{-1}}U_{a_1}) \cap
\DDD(\sqrt{K_{a_2}(I-K_{a_2})^{-1}}U_{a_2})$ which is equal to $
\DDD(\sqrt{H_1(I-H_{1})^{-1}}) \cap
\DDD(\sqrt{H_2(I-H_{2})^{-1}}V)$, we have
$$\begin{array}{l}
 \sqrt{K_{a_1}(I-K_{a_1})^{-1}}U_{a_1}S^{-1}T_1 S\xi -T_1\sqrt{K_{a_1}(I-K_{a_1})^{-1}}U_{a_1}\xi\\
=(a_1S\widehat{+} \sqrt{H_1(I-H_1)^{-1}})S^{-1}T_1S\xi- T_1 (a_1S \widehat{+} \sqrt{H_1(I-H_1)^{-1}})\xi\\
=\sqrt{H_1(I-H_1)^{-1}}S^{-1}T_1S\xi-T_1\sqrt{H_1(I-H_1)^{-1}}\xi.\end{array}$$
It follows from Lemma 2.4 that $T$ is in $ Alg\,\L$. Consequently,
$Alg\,\LLL =  Alg\,\{P_1, Q_1, Q_2\}$. \vspace{2mm}

\begin{prop}
With the above notation, the mapping, from $\C$ into
$Lat\,Alg\,\LLL$, given by
\begin{align*} a \rightarrow Q_{a} &=
\left(
          \begin{array}{cc}
            K_{a} & \sqrt{K_{a}(I-K_{a})}U_{a} \\
            U_{a}^{*}\sqrt{K_{a}(I-K_{a})} & U_{a}^{*}(I-K_{a})U_{a} \\
          \end{array}
        \right),
\end{align*}
is one to one and continuous with respect to the usual topology on
$\C$ and the trace norm on $Lat\,Alg\,\LLL$. Moreover, as $a
\rightarrow \infty$, $\|Q_{a} - P_1 \|_{2} \rightarrow 0$. Hence
$Lat\,Alg\,\LLL\setminus\{0,I\}$ is homeomorphic to $S^2$.
\end{prop}

\noindent{\it Proof}\quad For two scalars $a$ and $a_0$ in $\C$, let
$Q_a$ and $Q_{a_0}$ be defined as in the proposition. By Corollary
2.1, $\tau(Q_a)=\tau(Q_{a_0})=\frac12$. We only need to show that
$\|Q_{a} - Q_{a_0}\|_{2} \rightarrow 0$ as $a$ tends to $a_{0}$.
Since
\begin{align*}
\|Q_{a} - Q_{a_0}\|_{2}^{2} &= 1 - 2\tau(Q_{a}Q_{a_{0}}) \\
                            &= 1 - tr(K_{a}K_{a_0}) -
                            tr(\sqrt{K_{a}(I-K_{a})}U_{a}U^{*}_{a_0}\sqrt{K_{a_0}(I-K_{a_0})})\\
                            & \qquad -
                            tr(U^{*}_{a}\sqrt{K_{a}(I-K_{a})}\sqrt{K_{a_0}(I-K_{a_0})}U_{a_0})\\
                            & \qquad -
                            tr(U^{*}_{a}(I-K_{a})U_{a}U^{*}_{a_0}(I-K_{a_0})U_{a_0}),
\end{align*}
it is sufficient to show that the following statements hold as $a
\rightarrow a_{0}$:
$$|tr((K_{a}-K_{a_0})K_{a_0})| \rightarrow 0, \eqno{(6)}$$
$$|tr((\sqrt{K_{a}(I-K_{a})}U_{a}
-\sqrt{K_{a_0}(I-K_{a_0})}U_{a_0})U^{*}_{a_0}\sqrt{K_{a_0}(I-K_{a_0})})
| \rightarrow 0,\eqno{(7)}$$
$$|tr((U^{*}_{a}\sqrt{K_{a}(I-K_{a})}-U^{*}_{a_0}\sqrt{K_{a_0}(I-K_{a_0})})\sqrt{K_{a_0}(I-K_{a_0})}U_{a_0})|
\rightarrow 0,\eqno{(8)}$$
$$|tr(K_{a} - K_{a_0})| \rightarrow 0,
\quad |tr((U^{*}_{a}K_{a}U_{a}
-U^{*}_{a_0}K_{a_0}U_{a_0})U^{*}_{a_0}K_{a_0}U_{a_0})| \rightarrow
0.\eqno{(9)}$$ Recall that $\sqrt{K_{a}(I-K_{a})^{-1}}U_a = aS
\widehat{+} \sqrt{H_{1}(I-H_{1})^{-1}}$. Let
\begin{align*}
F(a) &= K_{a}(I-K_{a})^{-1}\\
&= |a|^{2}SS^{*} \widehat{+} aS\sqrt{H_{1}(I-H_{1})^{-1}}
\widehat{+} \overline{a}\sqrt{H_{1}(I-H_{1})^{-1}}S^{*} \widehat{+}
(\sqrt{H_{1}(I-H_{1})^{-1}})^2.
\end{align*}
Then $I - K_{a} = (I + F(a))^{-1}$.

For each positive number $\epsilon$, there exists a projection
$F_\epsilon$ in $\MMM$ such that $SS^{*}E_{\epsilon}$,
$S\sqrt{H_{1}(I-H_{1})^{-1}}E_{\epsilon}$,
$\sqrt{H_{1}(I-H_{1})^{-1}}S^{*}E_{\epsilon}$ and
$H_{1}(I-H_{1})^{-1}E_{\epsilon}$ are bounded operators, and
$tr(I-F_{\epsilon})<\epsilon^{2}$, where $E_\epsilon = \RRR((I +
F(a_0))^{-1}F_\epsilon)$. Note that
$tr(I-F_{\epsilon})=tr(I-E_{\epsilon}) < \epsilon^{2}$. So there
exists a constant $\delta > 0$ such that, if $|a-a_{0}| < \delta$,
then $\|F(a)E_{\epsilon} - F(a_0)E_{\epsilon}\| < \epsilon$ and
\begin{align*} |tr((K_{a}-K_{a_0})K_{a_0})| &\leq
|tr(K_{a_0}(K_{a}-K_{a_0})F_{\varepsilon})| +
|tr(K_{a_0}(K_{a}-K_{a_0})(I-F_{\varepsilon}))| \\
&\leq
|tr(K_{a_0}(I+F(a))^{-1}(F(a)-F(a_0))(I+F(a_0))^{-1}F_{\varepsilon})|
+ 2\varepsilon \\ & \leq 3\varepsilon.
\end{align*}
This implies (6). By a similar argument, we have $|tr(K_{a} -
K_{a_0})| \rightarrow 0$ as $a\rightarrow a_0$.

Next we prove (7). For each positive number $\epsilon$, we choose a
projection $P_\epsilon$ in $\MMM$ such that $SP_{\varepsilon}$ and
$\sqrt{H_{1}(I-H_{1})^{-1}}P_{\varepsilon}$ are bounded operators,
and  $tr(I - P_{\varepsilon}) < \varepsilon^{2}$. Thus there is a
positive constant $\delta_{1}$ such that, if $|a-a_{0}| <
\delta_{1}$, then
\begin{align*}
\| \sqrt{K_a(I-K_a)^{-1}}U_{a}P_{\epsilon} -
\sqrt{K_{a_0}(I-K_{a_{0}})^{-1}}U_{a_0}P_{\epsilon}  \| = \|(a - a_0)SP_{\epsilon}  \| \leq
\epsilon .
\end{align*}
Note that
\begin{align*}
&\sqrt{K_{a}(I-K_{a})}U_{a}P_{\varepsilon}
-\sqrt{K_{a_0}(I-K_{a_0})}U_{a_0}P_{\varepsilon}\\
& \qquad =(I-K_a)\sqrt{K_{a}(I-K_{a})^{-1}}U_{a}P_{\varepsilon}-(I-K_{a_0})\sqrt{K_{a_{0}}(I-K_{a_{0}})^{-1}}U_{a_0}P_{\varepsilon}\\
& \qquad =(I-K_a)(\sqrt{K_{a}(I-K_{a})^{-1}}U_{a}P_{\varepsilon}-\sqrt{K_{a_{0}}(I-K_{a_{0}})^{-1}}U_{a_0}P_{\varepsilon})\\
& \qquad    -(K_a-K_{a_0})\sqrt{K_{a_{0}}(I-K_{a_{0}})^{-1}}U_{a_0}P_{\varepsilon}.
\end{align*}
Hence we have
\begin{align*}
&|tr((\sqrt{K_{a}(I-K_{a})}U_{a}
-\sqrt{K_{a_0}(I-K_{a_0})}U_{a_0})U^{*}_{a_0}\sqrt{K_{a_0}(I-K_{a_0})})|
\\
&\qquad \leq
|tr(U^{*}_{a_0}\sqrt{K_{a_0}(I-K_{a_0})}(\sqrt{K_{a}(I-K_{a})}U_{a}
-\sqrt{K_{a_0}(I-K_{a_0})}U_{a_0})P_{\varepsilon})
| \\
&\qquad +
|tr(U^{*}_{a_0}\sqrt{K_{a_0}(I-K_{a_0})}(\sqrt{K_{a}(I-K_{a})}U_{a}
-\sqrt{K_{a_0}(I-K_{a_0})}U_{a_0})(I-P_{\varepsilon}))
| \\
&\qquad\leq
|tr(U^{*}_{a_0}\sqrt{K_{a_0}(I-K_{a_0})}(I-K_{a})[\sqrt{K_a(I-K_a)^{-1}}U_{a}P_{\varepsilon}
- \sqrt{K_{a_0}(I-K_{a_{0}})^{-1}}U_{a_0}P_{\varepsilon}]) | \\
& \qquad +
|tr(\sqrt{K_{a_0}(I-K_{a_{0}})^{-1}}U_{a_0}P_{\varepsilon}U^{*}_{a_0}\sqrt{K_{a_0}(I-K_{a_0})}(K_{a_0}
-K_a))| + 2\varepsilon \\
&\qquad \leq
|tr(\sqrt{K_{a_0}(I-K_{a_{0}})^{-1}}U_{a_0}P_{\varepsilon}U^{*}_{a_0}\sqrt{K_{a_0}(I-K_{a_0})}(K_{a_0}
-K_a))| + 3\varepsilon.
\end{align*}
To show
$|tr(\sqrt{K_{a_0}(I-K_{a_{0}})^{-1}}U_{a_0}P_{\varepsilon}U^{*}_{a_0}\sqrt{K_{a_0}(I-K_{a_0})}(K_{a_0}
-K_a))| \rightarrow 0$, we only need to apply the same argument as
in the proof of (6). Therefore, we have (7). The proof of (8) is
similar to that of (7).

For the second statement in (9), we choose, for each positive number
$\epsilon $, a projection $P_\epsilon$ in $\MMM$ such that
$SP_{\varepsilon}$, $\sqrt{H_{1}(I-H_{1})^{-1}}P_{\varepsilon}$,
$P_{\varepsilon}S^*$ and $P_{\varepsilon}\sqrt{H_{1}(I-H_{1})^{-1}}$
are all bounded operators, and  $tr(I - P_{\varepsilon}) <
\varepsilon^{2}$. Hence there is a positive constant $\delta_{1}$
such that, if $|a-a_{0}| < \delta_{1}$, then
\begin{align*}
\| \sqrt{K_a(I-K_a)^{-1}}U_{a}P_{\epsilon} -
\sqrt{K_{a_0}(I-K_{a_{0}})^{-1}}U_{a_0}P_{\epsilon}  \| = \|(a - a_0)SP_{\epsilon}  \| \leq
\epsilon\,\,\,\,\hbox{and} \\
 \|P_{\epsilon}U_{a}^*\sqrt{K_a(I-K_a)^{-1}} -
P_{\epsilon}U_{a_0}^*\sqrt{K_{a_0}(I-K_{a_{0}})^{-1}} \| = \|(a - a_0)P_{\epsilon}S^{*}  \| \leq
\epsilon.
\end{align*}
Note that
\begin{align*}
&P_{\epsilon}U_a^*K_{a}U_aP_{\varepsilon}
-P_{\epsilon}U_{a_0}^*K_{a_0}U_{a_0}P_{\varepsilon}\\
& \qquad =P_{\epsilon}U_a^*\sqrt{K_{a}(I-K_{a})^{-1}}(1-K_{a})\sqrt{K_{a}(I-K_{a})^{-1}}U_aP_{\varepsilon}\\
& \qquad - P_{\epsilon}U_{a_0}^*\sqrt{K_{a_0}(I-K_{a_0})^{-1}}(1-K_{a_0})\sqrt{K_{a_0}(I-K_{a_0})^{-1}}U_{a_0}P_{\varepsilon}.
\end{align*}
By the same argument as in the proof of (7), we can obtain (9).


Finally, we show that if $a \rightarrow \infty$, then $\|Q_{a} - P_1
\|_{2} \rightarrow 0$. Since $\|Q_{a} - P_1 \|_{2}^{2} = 1 -
tr(K_a)$, we only need to show that, when $a \rightarrow \infty$,
$tr(I-K_a) \rightarrow 0$.

Since $SS^*$ is invertible, for any positive number $\epsilon $, we
can choose a positive constant $\beta$ and a projection $E$ in
$\MMM$ such that $tr(E)
> 1 - \epsilon^2$, $ESS^{*}E \geq \beta E > 0$, and $ESS^{*}E$,
$ES\sqrt{H_{1}(I-H_{1})^{-1}}E$ and $EH_{1}(I-H_{1})^{-1}E$ are all
bounded. Thus there exists a positive constant $c$ such that if $|a|
> c$, we have
\begin{align*}
ESS^{*}E &+ \frac{1}{\overline{a}}ES\sqrt{H_{1}(I-H_{1})^{-1}}E \\
& + \frac{1}{a}E\sqrt{H_{1}(I-H_{1})^{-1}}S^{*}E +
\frac{1}{|a|^2}EH_{1}(I-H_{1})^{-1}E > \frac{\beta}{2}E.
\end{align*}
From the definition of $F(a)$, we have
$EF(a)E>|a|^{2}\frac{\beta}{2}E$. By \cite[Lemma 3.2]{BV}, if we let
$e$ denote the spectral projection of $F(a)$ on
$[|a|^{2}\frac{\beta}{2}, +\infty)$, then $tr(e)\geq tr(E)> 1-
\epsilon^2$ and $e F(a)e \geq \frac{|a|^{2}\beta}{2}e$. Choosing a
scalar $a$ with $|a|$ large enough, we will have
\begin{align*}
tr(I-K_a) = tr((I + F(a))^{-1}e) + tr((I + F(a))^{-1}(1-e))
          \leq \frac{2}{|a|^2\beta} + \epsilon \leq 2\epsilon.
\end{align*}
This implies that, when $a \rightarrow \infty$, $tr(I-K_a)
\rightarrow 0$. This finishes the proof of the proposition.
\vspace{2mm}

Let us now recall the definitions of  Kadison-Singer algebras and
Kadison-Singer lattices:

\begin{df}
A subalgebra $\AAA$ of $B(\HHH)$ is called a Kadison-Singer algebra
(or KS-algebra) if $\AAA$ is reflexive and maximal with respect to
the diagonal subalgebra $\AAA \cap \AAA^*$ of $\AAA$, in the sense
that if there is another reflexive subalgebra $\mathcal{B}$ of
$B(\HHH)$ such that $\AAA \subseteq\mathcal{B}$ and
$\mathcal{B}\cap\mathcal{B}^*=\AAA \cap\AAA^*$, then
$\AAA=\mathcal{B}$. A lattice $\L$ of projections in $B(\HHH)$ is
called a Kadison-Singer lattice (or KS-lattice) if $\L$ is a minimal
reflexive lattice that generates the von Neumann algebra $\L''$, or
equivalently, if $\L$ is reflexive and $Alg\,\L$ is a Kadison-Singer
algebra.
\end{df}

Next we will show that the lattice given in Proposition 2.1 is a KS-lattice in general.

\begin{lemma}
For any three distinct projections $Q_1, Q_2$ and $Q_3$ in
$Lat\,Alg\,\LLL \setminus \{0, I \}$, we have $P_{1} \in
Lat\,Alg\,\{ Q_{1}, Q_{2}, Q_{3}\}$. Hence $Lat\,Alg\,\{Q_{1},
Q_{2}, Q_{3}\}=Lat\,Alg\,\L.$
\end{lemma}

\noindent{\it Proof}\quad We may assume that $Q_i\in Lat\,Alg\,\LLL
\setminus \{0, I, P_1 \}$ for $i = 1,2,3$. By Proposition 2.1,
\begin{align*}
Q_{i}& =\left(
          \begin{array}{cc}
            K_{i} & \sqrt{K_{i}(I-K_{i})}U_{i} \\
            U_{i}^{*}\sqrt{K_{i}(I-K_{i})} & U_{i}^{*}(I-K_{i})U_{i} \\
          \end{array}
        \right),
\end{align*}
where $K_i$ and $U_i$ are determined by the polar decomposition
$a_{i}S \widehat{+}
\sqrt{H_{1}(I-H_{1})^{-1}}=\sqrt{K_{i}(I-K_{i})^{-1}} U_{i}$ for
distinct scalars $a_{i}$ in $\C$.

To prove the lemma, we only need to show that if
\begin{align*}
A =\left(
     \begin{array}{cc}
       A_{11} & A_{12} \\
       A_{21} & A_{22} \\
     \end{array}
   \right)\in
 Alg\,\{Q_{1}, Q_{2}, Q_{3}\},
\end{align*}
then $A_{21} = 0$. Since $(I-Q_{i})AQ_{i} = 0$ for $i=1,2,3$, by a
similar argument to the proof of  Lemma 2.3, we have
\begin{align*}
\sqrt{I-K_i}( A_{11}\sqrt{K_i} +A_{12}U_i^*\sqrt{I-K_i})=\sqrt{K_i}U_i(A_{21}\sqrt{K_i}+A_{22}U_i^*\sqrt{I-K_i}).
\end{align*}
Since the set $
(\cap_{i=1}^{3}\DDD(\sqrt{(I-K_i)^{-1}}U_i)\cap\DDD(S)\cap\DDD(\sqrt{H_{1}(I-H_{1})^{-1}})$,
denoted by $\mathfrak{D}$, is dense in $\HHH$, the invertibility of
$I-K_i$ (and $\sqrt{I-K_i}$) implies that, for each $\xi$ in
$\mathfrak{D}$,
\begin{align*}
A_{11}\sqrt{K_{i}(I-K_{i})^{-1}}U_{i}\xi + A_{12}\xi =
\sqrt{K_{i}(I-K_{i})^{-1}}U_{i}[A_{21}\sqrt{K_{i}(I-K_{i})^{-1}}U_{i}\xi
+ A_{22}\xi].
\end{align*}
Let $\{E_\epsilon \}_{\epsilon>0}$ be an increasing net (as
$\epsilon \rightarrow 0$) of projections in $\MMM$ with the
strong-operator topology limit $I$ such that $E_\epsilon S$ and
$E_\epsilon \sqrt{H_{1}(I-H_{1})^{-1}}$ are bounded operators for
each $\epsilon$. Then
\begin{align*}
E_\epsilon \sqrt{K_{i}(I-K_{i})^{-1}}U_i = a_{i}E_\epsilon S +
E_\epsilon \sqrt{H_{1}(I-H_{1})^{-1}}
\end{align*}
is bounded for $i=1,2,3$. Thus, for $i$, $j \in \{1, 2, 3\}$, we
have
\begin{align*}
(a_{i} -a_{j})[E_{\varepsilon}A_{11} &S\xi -
E_{\varepsilon}SA_{22}\xi]= \\
&E_\epsilon\sqrt{K_{i}(I  -K_{i})^{-1}}
U_{i}A_{21} \sqrt{K_{i}(I-K_{i})^{-1}}U_{i}\xi \\
&-E_{\epsilon}\sqrt{K_{j}(I-K_{j})^{-1}}
U_{j}A_{21}\sqrt{K_{j}(I-K_{j})^{-1}}U_{j}\xi.
\end{align*}
Replacing $E_\epsilon \sqrt{K_{i}(I-K_{i})^{-1}}U_i$ with
$a_{i}E_\epsilon S + E_\epsilon \sqrt{H_{1}(I-H_{1})^{-1}}$ in the
above equation, we show that
\begin{align*}
(a_{i} -a_{j})[E_{\varepsilon} &A_{11}S\xi -
E_{\varepsilon}SA_{22}\xi]\\
&= (a_i^2-a_j^2)E_{\epsilon}SA_{21}S\xi+(a_i-a_j)E_{\epsilon}SA_{21}\sqrt{H_{1}(I-H_{1})^{-1}}\xi\\
&+ (a_i-a_j)E_{\epsilon}\sqrt{H_{1}(I-H_{1})^{-1}}A_{21}S\xi.
\end{align*}
A simple calculation gives us that $(a_1+a_2)E_{\epsilon}SA_{21}S\xi
= (a_1+a_3)E_{\epsilon}SA_{21}S\xi$. Since $a_2\neq a_3$,
$E_{\epsilon}SA_{21}S\xi=0$. Note that $Ker(S) = 0$ and
$\{S\xi:\,\,\xi\in \mathfrak{D}\}$ is dense in $\HHH$. Letting
$\epsilon \rightarrow 0$, we have $A_{21} = 0$.

By Corollary 2.2, we have $Lat\,Alg\,\L=Lat\,Alg\,\{P_1, Q_1,
Q_2\}\subseteq Lat\,Alg\,\{Q_1, Q_2, Q_3\}\subseteq Lat\,Alg\,\L$.
Hence $Lat\,Alg\,\{Q_1, Q_2, Q_3\}= Lat\,Alg\,\L$. This proves our
lemma.


\begin{prop} With the above notation,
 $Lat\,Alg\,\LLL$ is  a Kadison-Singer
 lattice if $\LLL''$, the von Neumann algebra generated by $\LLL$, cannot be generated by two nontrivial
 projections.
\end{prop}

 \noindent{\it Proof}\quad Suppose that $\mathcal{F}$ is a reflexive lattice in
 $M_2(\C)\otimes B(\HHH)$ such that $\mathcal{F}\subseteq Lat\,Alg\,\L$ and $\mathcal{F}''=(Lat\,Alg\,\L)''$.
 From our assumption, there are at least three nontrivial projections
 $Q_1$, $Q_2$ and $Q_3$ in $\mathcal{F}$. Since $\mathcal{F}$ is reflexive, we have $Lat\,Alg\,\{Q_1,Q_2,Q_3\}\subseteq\mathcal{F}\subseteq
 Lat\,Alg\,\L$. By Lemma 2.5, we have $\mathcal{F}=Lat\,Alg\,\L$. Hence
 $Lat\,Alg\,\LLL$ is a Kadison-Singer lattice.


\begin{remark} When $\LLL$ is a double triangle lattice of projections and $\LLL''$ is generated by two nontrivial projections,
$Lat\,Alg\,\L$ may not be a Kadison-Singer lattice. For example, let
$P_1=\left(
     \begin{array}{cc}
     I & 0 \\
       0 & 0 \\
     \end{array}
   \right)$, $P_2=\left(
     \begin{array}{cc}
     0 & 0 \\
       0 & I \\
     \end{array}
   \right)$ and $P_3=\left(
     \begin{array}{cc}
     \frac I2 &\frac I2 \\
      \frac I2 & \frac I2 \\
     \end{array}
   \right)$. Then, for $i\neq j$, $P_i\wedge P_j=0$ and $P_i\vee P_j=I$. In this case, if let $\L=\{0,I,P_1,P_2,P_3\}$,
   then $\L$ is a double triangle lattice of projections, which generates the von Neumann algebra $M_2(\C)\otimes \C I$.
   Thus $\L''$ can be generated by two nontrivial
   projections. Let $\LLL_1=\left\{0, I,\left(
     \begin{array}{cc}
     I & 0 \\
       0 & 0 \\
     \end{array}
   \right), \left(
     \begin{array}{cc}
     \frac I2 &\frac I2 \\
      \frac I2 & \frac I2 \\
     \end{array}
   \right)\right\}$. Then $\L_1$ is a reflexive sublattice of $\LLL$ that generates the same von Neumann
   algebra as $\LLL$. Hence $Lat\,Alg\,\LLL$ is not a Kadison-Singer
 lattice.
\end{remark}

\begin{prop}
Let $Q_1$, $Q_2$ and $Q_3$ be any three projections acting on a
separable Hilbert space $\mathcal{K}$ such that $Q_i \wedge Q_j = 0$
and $Q_i \vee Q_j = I$ for $i \neq j$. Assume that the von Neumann
algebra $\AAA$ generated by these three projections is finite. Then
$Q_1$ is equivalent to $I-Q_1$ in $\AAA$ and $\AAA$ is
*-isomorphic to $Q_1 \AAA Q_1 \otimes M_2(\C)$.
\end{prop}

\noindent{\it Proof}\quad Let $\tau$ be a faithful, normal, tracial
state  on $\AAA$. By Kaplansky formula,  we have $\tau(Q_i) =
\frac{1}{2}$ for $i=1,2,3$. Let $\WWW =\AAA
* M_{2}(\C)$ be the reduced (von Neumann algebra) free product of
$\AAA$ with $M_2(\C)$. Since there exists a trace half projection
which is free with $Q_1$, it follows that $Q_1$ is equivalent to
$I-Q_1$ in $\WWW$ (\cite{VDN,GY2}). Therefore, we may choose a
system $\{ E_{ij} \}_{i,j = 1}^{2}$ of matrix units  in $\WWW$ such
that $E_{11} = Q_1$. In this case, $\WWW$ is $\ast$-isomorphic to
$Q_1\WWW Q_1 \otimes M_2(\C)$.


By Lemma 2.1, we write
\begin{align*}
&Q_2 = \left(\begin{array}{cc}H_1 & \sqrt{H_1 (I-H_1)}V_1 \\V_1^*
\sqrt{H_1 (I-H_1)} & V_1^{*}(I - H_1)V_1\end{array}\right)\,\,\,\,\,\hbox{ and }\\
&Q_3 = \left(\begin{array}{cc}H_2 & \sqrt{H_2(I-H_2)}V_2 \\V_2^*
\sqrt{H_2(I-H_2)} & V_2^{*}(I - H_2)V_2\end{array}\right),
\end{align*}
where $V_i$ is a unitary operator in $Q_1\WWW Q_1$ and  $H_i$ is a
positive contractive operator in $Q_1\WWW Q_1$ such that $Ker(I-H_i)
=0$ for $i = 1,2$.

Note that
\begin{align*} \left(
  \begin{array}{cc}
    H_{i-1} &  0\\
    0 & 0 \\
  \end{array}
\right)=Q_1Q_iQ_1 \mbox{ and }
\left(
  \begin{array}{cc}
    0 &  \sqrt{H_{i-1}(I-H_{i-1})}V_{i-1}\\
    0 & 0 \\
  \end{array}
\right)=Q_1Q_i(I-Q_1).
\end{align*}
Let
\begin{align*}
T = \left(
  \begin{array}{cc}
    0 &  \sqrt{H_1(I-H_1)^{-1}}V_1 \widehat{-}\sqrt{H_2(I-H_2)^{-1}}V_2)\\
    0 & 0 \\
  \end{array}
\right).
\end{align*}
Then $T$ is affiliated with $\AAA$. By Lemma 2.2,
$Ker(\sqrt{H_1(I-H_1)^{-1}}V_1-\sqrt{H_2(I-H_2)^{-1}}V_2)$ $=0$ and
$Ker(V_1^*\sqrt{H_1(I-H_1)^{-1}}-V_2^*\sqrt{H_2(I-H_2)^{-1}})=0$.
Let $HU$ be the polar decomposition of $T$. Then $U\in \AAA$, $U^*U
= I-Q_1$ and $UU^{*} = Q_1$. Thus $Q_1$ is equivalent to $I- Q_1$ in
$\AAA$. This proves our proposition. \vspace{2mm}

We summarize what we have proved into the following theorem.

\begin{theorem} Let $P_1$, $P_2$ and $P_3$ be three projections acting on a separable complex Hilbert space $\HHH$ such that
$P_i \wedge P_j = 0$ and $P_i \vee P_j = I$ for $i \neq j$. Suppose
the von Neumann algebra $\AAA$ generated by $P_1, P_2$ and $P_3$ is
finite. Then $Lat\,Alg\,\{P_1,P_2,P_3\}\setminus \{0, I \}$ is
homeomorphic to $S^{2}$. In addition, each nontrivial projection in
$Lat\,Alg\,\{P_1,P_2,P_3\}$ has trace $\frac 12$ and the reflexive
lattice can be generated by arbitrary three nontrivial projections
in it.

Furthermore, if the von Neumann algebra $\AAA$ cannot be generated
by two nontrivial projections, then $Lat\,Alg\,\{P_1,P_2,P_3\}$ is a
KS-lattice.
\end{theorem}

With the notation in the above theorem, let $\FFF =
Lat\,Alg\{P_1,P_2,P_3\}$. For any three projections in $\FFF
\setminus \{0,I\} $, they determine a coordinate chart of $\FFF
\setminus \{0,I\} $ (exclude one point corresponding to $\infty$) by
Proposition 2.1. It can be shown that the transition map between two
coordinate charts is a M\"{o}bius  transformation. We will address
this issue in a forthcoming paper. As for now, we will determine all
``connected" reflexive lattices acting on a finite-dimensional
Hilbert space.


\section{Connected reflexive lattices in $M_n(\C)$}

For $n\geq 3$, we know that $M_n(\C)$ cannot be generated by two
projections. When $n$ is an even number greater than $3$, $M_n(\C)$
can be generated by three projections of (normalized) trace
$\frac12$. By Theorem 2.1, we know that every nontrivial projection
in the reflexive lattice generated by a double triangle lattice of
projections in $M_n(\C)$ has trace $\frac12$. In this case, such a
reflexive lattice (with the $\|\cdot\|_2$-topology) is homeomorphic
to $S^2$ (plus $0$ and $I$). In this section, we will show that if
each nontrivial projection in a reflexive lattice $\mathcal{F}$ of
projections in $M_n(\C)$ has trace $\frac12$ and $\FFF$ has more
than three nontrivial projections, then
$\mathcal{F}\setminus\{0,I\}$ (with the $\|\,\|_2$-topology) is
homeomorphic to $S^2$. Moreover, if $n \geq 3$ and $\FFF$ generates
$M_n(\C)$, then $\mathcal{F}$ is a Kadison-Singer lattice.

\begin{prop}
 Let $\FFF$ be a reflexive lattice of projections
in $M_n(\C)$. If $\FFF \setminus \{0, 1 \}$ has only one connected
component under the $\|\,\|_2$-topology, then $\FFF=\{0,P,I\}$ for
some nontrivial projection $P$, or $\FFF$ is generated by a double
triangle lattice of projections in $M_n(\C)$. Furthermore, $n$ must
be even in the latter case.
\end{prop}

\noindent{\it Proof}\quad  Suppose $\FFF$ contains at least two
nontrivial projections. Let $\tau$ be the normalized trace on
$M_{n}(\C)$. Since  $\FFF \setminus \{0, 1 \}$ has only one
connected component, the range of the restriction of $\tau$ to $\FFF
\setminus \{0, 1 \}$ contains only one point. This implies that
$P\vee Q=I$, $P\wedge Q=0$ and $\tau(P)=\tau(Q)=\frac12$ for any two
distinct projections $P$ and $Q$ in $\FFF \setminus \{0, 1 \}$.
Thus, $n$ must be even. Let $n=2k$ and write $M_n(\C)=M_2(\C)\otimes
M_k(\C)$.

Next we show that $\FFF$ is generated by a double triangle lattice
of projections in $M_n(\C)$. Note that if $\FFF$ contains at least
two nontrivial projections, then $\FFF$ must contain infinitely many
projections. Otherwise, $\FFF \setminus \{0,I\}$ has more than one
connected components. Given $P_1,P_2$ and $P_3$ in
$\FFF\setminus\{0,I\}$, let $\L=\{0,P_1,P_2,P_3,I\}$. Then $\LLL$ is
a double triangle lattice of projections in $M_n(\C)$, and
$Lat\,Alg\,\LLL\subseteq\FFF$, since $\FFF$ is reflexive. Now we
show that $Lat\,Alg\,\LLL=\FFF$.

If there is a projection $P_4$ in $\FFF \setminus Lat\,Alg\,\LLL$,
then $P_i\wedge P_j=0$ and $P_i\vee P_j=I$
for $i,j=1,2,3,4$, and $i\neq j$. Up to unitary equivalence, we may assume that $P_1=\left(\begin{array}{cc}  I_k & 0\\
0 &0\end{array}\right)$, where $I_k$ is the identity matrix in
$M_k(\C)$. Then, by Lemma 2.1, we have
\begin{align*}
P_j=\left(\begin{array}{cc} H_{j-1} &
\sqrt{H_{j-1}(I-H_{j-1})^{-1}}V_{j-1}\\
V_{j-1}^{*}\sqrt{H_{j-1}(I-H_{j-1})^{-1}} &
V_{j-1}^{*}(I-H_{j-1})V_{j-1}\end{array}\right),\,\,\,
\hbox{for}\,\,\, j=2,3,4,
\end{align*}
where $H_j$ is a positive matrix and $V_j$ is a unitary matrix in
$M_k(\C)$ such that $I-H_j$ is positive definite (and hence,
invertible). Without loss of generality, we may assume
$V_1=I(=I_k)$.

Let $\L_0=\{0,P_1,P_2,P_4,I\}$. Then we have $
Alg\,\{P_1,P_2,P_3,P_4\}= Alg\,\L\cap Alg\,\L_0$. By Lemma 2.4, we
obtain that
\begin{align*}
 Alg\,\{&P_1,P_2,P_3,P_4\}\\
 &=\left\{\left(\begin{array}{cc} A & \sqrt{H_1(I-H_1)^{-1}}S_1^{-1}AS_1-A\sqrt{H_1(I-H_1)^{-1}}\\
0 & S_1^{-1}AS_1\end{array}\right):\, \begin{array}{l} A\in M_k(\C)\\
A\widetilde{S}=\widetilde{S}A\end{array}\right\},
\end{align*}
here $\widetilde{S}=S_2S_1^{-1}$,
$S_1=\sqrt{H_1(I-H_1)^{-1}}-\sqrt{H_2(I-H_2)^{-1}}V_2$ and
$S_2=\sqrt{H_1(I-H_1)^{-1}}-\sqrt{H_3(I-H_3)^{-1}}V_3$. When $k =
1$, $\widetilde{S}$, $S_1$ and $S_2$ are all scalars. It is easy to
see that $Alg\,\L=Alg\,\{P_1,P_2,P_3,P_4\}= Alg\,\L_0= \C I$. Hence
we have a contradiction with the assumption that $P_4\notin
Lat\,Alg\,\LLL$. The proposition follows.

For $k > 1$, we first remark that $\widetilde{S} \notin\C I$.
Otherwise, $S_2 = \lambda S_1$. Then $S_2^{-1}AS_2=S_1^{-1}AS_1$ for
any $A$ in $M_k(\C)$. Hence we have $ Alg\,\L= Alg\,\L_0$ by Lemma
2.4. Thus $P_4$ is in $Lat\,Alg\,\L$, which contradicts our
assumption.

Let $\lambda$ be an eigenvalue of $\widetilde{S}$ and $q$ be the
orthogonal projection onto the eigenspace of $\widetilde{S}$ at
$\lambda$. For each $A$ in $M_k(\C)$ with
$A\widetilde{S}=\widetilde{S}A$, we have $(I-q)Aq = 0$. Thus
$(I-Q)TQ = 0$ for any $T$ in $Alg\,\{P_1,P_2,P_3,P_4\}$, where
$Q=\left(\begin{array}{cc} q&0\\0&0\end{array}\right)$. This implies
that $Q$ is in $Lat\,Alg\{P_1,P_2,P_3,P_4\}$. Since $\widetilde{S}
\neq \lambda I$, we have $0 < q < I$ and $0<\tau(Q)<\frac 12$. Hence
$Q$ is a projection with trace less than $\frac12$ , which
contradicts the fact that $\FFF \setminus \{0, I\}$ has only one
connected component. Consequently, $\FFF$ is generated by $\LLL$.
This completes the proof.

\vspace{2mm}

By Proposition 3.1 and Theorem 2.1, we have the following corollary.

\begin{corollary}
 Let $\FFF$ be a reflexive lattice of projections in $M_n(\C)$. Suppose there exist at least two
nontrivial projections in $\FFF$. Then the following statements are
equivalent:
\begin{enumerate}
\item[(i)] $\FFF \setminus \{0, 1 \}$ has only one connected component under the $\|\cdot\|_2$-topology;
\item[(ii)] $\FFF$ is generated by three projections $P_1$,
$P_2$ and $P_3$ such that $P_i \wedge P_j = 0$ and $P_i \vee P_j =
I$ for $i \neq j$;
\item[(iii)] $\FFF \setminus \{0, 1 \}$ is  homeomorphic to
$S^2$.
\end{enumerate}

In addition, if any of the above conditions is satisfied, then $n$
must be even and each nontrivial projection in $\FFF$ has trace
$\frac12$.  Conversely, if $\FFF$ contains at least three nontrivial
projections, then the above conditions are equivalent to:

For each projection $P$ in $\FFF\setminus \{0,I\}$, $\tau(P)=\frac
12$, where $\tau$ is the normalized trace on $M_n(\C)$.

\end{corollary}

\noindent{\it Proof}\quad By Proposition 3.1, we have that (i)
implies (ii). By Theorem 2.1, we know that (ii) implies (iii). It is
obvious that (iii) implies (i). If any of these equivalent
conditions holds, by Theorem 2.1, we have $\tau(P)=\frac 12$ for
each nontrivial projection $P$ in $\FFF$. Hence $n$ is even.

Suppose $\FFF$ contains at least three nontrivial projections and
each nontrivial projection $P$ in $\FFF$ has trace $\frac 12$. Then
$n$ is even. For each pair of distinct nontrivial projections $P$
and $Q$ in $\FFF$, $P\wedge Q=0$ and $P\vee Q=I$. Let $P_1$, $P_2$
and $P_3$ be three distinct nontrivial projections in $\FFF$. By a
similar argument to the proof of Proposition 3.1, we can show that
$Lat\,Alg\,\{P_1, P_2, P_3\} = \FFF$. This proves that our last
statement in the corollary implies (ii). Hence our corollary
follows.

\begin{remark} When $\FFF$ contains only two nontrivial projections, the last statement in Corollary 3.1 cannot imply (i). For example, let
$\FFF=\left\{0,I, \left(\begin{array}{cc} I_k & 0\\
0&0\end{array}\right), \left(\begin{array}{cc} \frac12 I_k & \frac12 I_k\\
\frac12 I_k&\frac12 I_k\end{array}\right)\right\}$. Then $\FFF$ is a
reflexive lattice of projections in $M_2(\C)\otimes M_k(\C)$ such
that each nontrivial projection in $\FFF$ has  trace $\frac 12$.
Obviously, $\FFF \setminus \{0, I \}$ has two connected components.
\end{remark}

\begin{corollary} Let $\FFF$ be a reflexive lattice of projections in $M_n(\C)$ such that $\FFF$ generates $M_n(\C)$, for $n\geq 3$.
Suppose each nontrivial projection in $\FFF$ has trace $\frac 12$.
Then $\FFF\setminus\{0,I\}$ is homeomorphic to $S^2$. Moreover,
$\FFF$ is a Kadison-Singer lattice and the Kadison-Singer algebra
$Alg\,\FFF$ has dimension $\frac{n^2}{4}$.
\end{corollary}

\noindent{\it Proof}\quad When $n\geq 3$, $M_n(\C)$ cannot be
generated by two nontrivial projections. Thus $\FFF$ satisfies the
conditions in Corollary 3.1. Hence $\FFF\setminus\{0,I\}$ is
homeomorphic to $S^2$. Let $n=2k$. It follows from Lemma 2.4 that
$Alg\,\FFF$ is isomorphic to $M_k(\C)$, which has dimension
$\frac{n^2}{4}$.

\begin{remark} Let $n=2k>2$. Then there exists a reflexive lattice of projections in $M_n(\C)$ satisfying the conditions in Corollary 3.2. For
example, choose two positive matrices $H_1$ and $H_2$ in $M_k(\C)$
such that $I-H_i$ is positive definite for $i=1,2$,
$\sqrt{H_{1}(I-H_{1})^{-1}}-\sqrt{H_{2}(I-H_{2})^{-1}}$ is
invertible, and $\{H_1,H_2\}$ generates $M_k(\C)$. Let $\LLL$ be the
lattice generated  by $\left(\begin{array}{cc} I_k & 0\\
0&0\end{array}\right)$, $\left(\begin{array}{cc} H_1 & \sqrt{H_1
(I-H_1)} \\ \sqrt{H_1 (I-H_1)} & I - H_1\end{array}\right)$ and $
\left(\begin{array}{cc} H_2 & \sqrt{H_2(I-H_2)} \\
\sqrt{H_2(I-H_2)} & I - H_2\end{array}\right)$.  By Theorem 2.1,
$Lat\,Alg\,\LLL$ is a KS-lattice generating $M_n(\C)$ and satisfies
the conditions in Corollary 3.2.
\end{remark}

\begin{prop}$^{\cite{Hou,WY}}$ Let $E_{ij},\,\, i,j=1,2,\cdots,n$, be the standard matrix units of
$M_n(\C)$, where $n\geq 2$. Let $P_i=\sum\limits_{j=1}^{i}E_{ii}$,
for $i=1,2,\cdots, n$, and $Q= \frac 1n\sum\limits_{i,j=1}^nE_{ij}$.
Let $\FFF$ be the lattice generated by  $P_1,\cdots,P_{n}$ and $Q$.
Then $\FFF$ is a Kadison-Singer lattice of projections in $M_n(\C)$
and the corresponding Kadiosn-Singer algebra $Alg\,\FFF$ has
dimension $1+\frac{n(n-1)}{2}$.
\end{prop}

\begin{remark} When $n=2$ or $3$, it was
proved that if a Kadison-Singer lattice $\L$ generates $M_n(\C)$,
then $\L$ or $I-\L$ is similar to the one given in Proposition 3.2
(\cite{Tan}). Hence each Kadison-Singer algebra in $M_2(\C)$ (or
$M_3(\C)$) with diagonal $\C I$ has dimension $2$ (or $4$). For a
general $n$, we have the following conjecture.

\vspace{2mm}

\noindent{\bf Conjecture 3.1.}\quad Let $\AAA$  be a Kadison-Singer
algebra in $M_n(\C)$ with a trivial diagonal. Then
$\frac{n^2}{4}\leq dim(\AAA)\leq 1+\frac{n(n-1)}{2}$, where
$dim(\AAA)$ is the dimension of $\AAA$.
\end{remark}


\section*{\bf Acknowledgments}\quad  The authors would like to thank the editor/referee for his/her helpful comments and
suggestions. Research was supported in part by the National Natural
Science Foundation of China (Grant No. 10971117), Natural Science
Foundation of Shandong Province (Grant No. Y2006A03, ZR2009AQ005)
and Morningside Mathematic Center.




%\newpage
\begin{thebibliography}{00}


\bibitem{Ar} Arveson W.: Operator algebras and invariant subspaces. Ann. Math. 100, 433-532 (1974)

\bibitem{BV} Bercovici H., Voiculescu, D.: Free convolution of measures with unbounded
support. Indiana Univ. Math. J. 42, 733-773 (1993)

\bibitem{Da} Davidson, K. R.:
{\sl Nest algebras}. Longman Scientific \& Technical, {\bf $\pi$}
Pitman Research Notes in Mathematics Series, {No. 191}. New York
(1988)

\bibitem{GS} Ge, L., Shen, J.: On the generator problem of von Neumann algebras. AMS/IP Studies in adv. Math. 42, 257-275 (2008)

\bibitem{GY1} Ge, L., Yuan, W.: Kadison-Singer algebras,
I: hyperfinite case. Proc Natl Acad Sci (USA) 107(5), 1838-1843
(2010)

\bibitem{GY2} Ge, L., Yuan, W.: Kadison-Singer algebras,
II: general case. Proc Natl Acad Sci (USA) 107(11), 4840-4844 (2010)

\bibitem{Ha} Halmos, P.: Reflexive lattices of subspaces. J. of London Math. Soc. 4,
257-263 (1971)

\bibitem{Hou} Hou, C.: Cohomology of a class of Kadison-Singer algebras. Science China Mathematics
53(7), 1827-1839 (2010)

\bibitem{KR} Kadison, R., Ringrose, J.: {\sl Fundamentals of the theory
of operator algebras. Vol. I: Elementary theory.} No. 15 in Graduate
Studies in Mathematics; {\sl Vol. II: Advanced theory.} No. 16 in
Graduate Studies in Mathematics. American Mathematical Society,
Providence, RI (1997)

%Vols. I and II. Academic Press, Orlando (1983 and 1986)

\bibitem{KR1} Kadison, R., Ringrose, J.: {\sl Fundamentals of the theory
of operator algebras.Vol. IV: Special topics, advanced theory-- An
exercise approach.}  Birkh\"{a}user, Boston (1992)

\bibitem{KS} Kadison R., Singer, I.: Triangular operator algebras. Fundamentals and hyper-reducible theory. Amer. J. of Math. 82, 227-259 (1960)

\bibitem{Lan} Lance C.: Cohomology and perturbations of nest algebras. Proc. London Math. Soc. 43, 334-356 (1981)

\bibitem{La} Larson D.: Similarity of nest algebras. Ann. Math. 121, 409-427 (1983)

%\bibitem{Zhe} Liu, Z.: On some mathematical aspects of the Heisenberg Relation. Science  China series Mathematics: Kadison's
%proceedings. (To appear)

\bibitem{MV} Murray, F., von Neumann, J.: On rings of operators, II. Trans. Amer. Math.
Soc. 41, 208-248 (1937)


\bibitem{RR} Radjavi, H., Rosenthal P.: {\sl Invariant subspaces}. Springer-Verlag, Berlin (1973)

\bibitem{Ri} Ringrose, J.: On some algebras of operators, II. Proc. London
Math. Soc. 16(3), 385--402 (1966)


\bibitem{Tan} Tan, J.: Classification on Kadison-Singer algebras. Graduation Thesis, Academy of Mathematics and Systems Science, CAS (2010)

\bibitem{VDN} Voiculescu, D., Dykema, K., Nica, A.: {\sl Free
random variables}. CRM Monograph Series, vol. 1 (1992)

\bibitem{WY} Wang, L., Yuan, W.: A new class of Kadison-Singer algebras. Expositiones Mathematicae (2010). Doi: 10.1016
/j.exmath. 2010.08.001


\end{thebibliography}



\end{document}
