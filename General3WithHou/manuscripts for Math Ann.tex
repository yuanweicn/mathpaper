\documentclass[12pt]{article}

%Setup Begin-------------------------------------------------------------------------------------
\usepackage{amssymb}
\usepackage{latexsym}
\usepackage{amsmath}
\usepackage{amsfonts}

\textwidth=145truemm \textheight=240truemm
%\headsep=4truemm
\topmargin= 0pt
%\oddsidemargin=0pt \evensidemargin=0pt
\parindent=18pt

\newtheorem{theorem}{Theorem}[section]
\newtheorem{corollary}{Corollary}[section]
\newtheorem{main}{Main Theorem}[section]
\newtheorem{lemma}{Lemma}[section]
\newtheorem{prop}{Proposition}[section]
\newtheorem{df}{Definition}[section]
\newtheorem{remark}{Remark}[section]
\newtheorem{example}{Example}[section]
\newtheorem{question}{Question}[section]

\newcommand{\AAA}{\mathfrak A}
\newcommand{\BBB}{\mathcal B}
\newcommand{\CCC}{\mathcal C}
\newcommand{\DDD}{\mathcal D}
\newcommand{\GGG}{\mathcal G}
\newcommand{\HHH}{\mathcal H} %for Hilbert space
\newcommand{\LLL}{\mathcal L} % for lattice
\newcommand{\MMM}{\mathcal M}
\newcommand{\NNN}{\mathcal N} %for nest
\newcommand{\RRR}{\mathcal R}
\newcommand{\SSS}{\mathcal S}
\newcommand{\WWW}{\mathcal W}
\newcommand{\FFF}{\mathcal F}

\newcommand{\Lat}{\mathcal Lat}
\newcommand{\Alg}{\mathcal Alg}
\newcommand{\tensor}{\mathop{\bar \otimes}}
\newcommand{\tr}{\tau}

\newcommand{\C}{\mathbb C} %for complex number
\newcommand{\R}{\mathbb R}  %for real number
\newcommand{\Z}{\mathbb Z} %for integer
\newcommand{\N}{\mathbb N} % for nature number
\newcommand{\Q}{\mathbb Q} %for rational number


%The following is defined by hou

\def\L{{\mathcal{L}}}
\def\H{{\mathcal{H} }}
\def\D{{\mathscr{D} }}
\def\M{{\mathscr{ M}}}
\def\F{{\mathscr{F}}}
%\def\P{{\mathscr{P}}}
\def\P{{\mathcal{P}}}
\def\N{{\mathbb{N}}}
\def\I{{\mathbb{I}}}
\def\Z{{\mathbb{Z}}}\def\C{{\mathbb{C}}}
\def\Lat{{\mathcal Lat}}

\def\l{{\mathcal{L}}}

%%%%%%%%%%%%%%%%%%
%short cuts
%%%%%%%%%%%%%%%%%%

%Setup End----------------------------------------------------------------------------------------
\begin{document}

\begin{center}
{\Large \bf Minimal Generating Reflexive Lattices of Projections in
Finite von Neumann Algebras}
\end{center}

\begin{center}

{\bf Chengjun Hou}\\

Department of Mathematics, Qufu Normal University, Qufu 273165,
China\\
e-mail: cjhou@mail.qfnu.edu.cn\\
\vspace{2mm}

{\bf Wei Yuan}\\
L.K.Hua Key Laboratory of Mathematics, Chinese Academy of Sciences,
Beijing 100190, China\\
e-mail: wyuan@math.ac.cn

\end{center}

\noindent{\small {\small\bf Abstract} \ \ We show that the reflexive
lattice generated by each double triangle lattice of projections in
a finite von Neumann algebra is topologically homeomorphic to
$S^2$ (plus two distinct points corresponding to zero and
$I$). In particular, the reflexive lattice is in general minimally
generating for the von Neumann algebra it generates. As an
application, we show that if a reflexive lattice $\FFF$ generates a
matrix algebra $M_n(\C)$ for $n\geq 3$, then $\FFF\setminus\{0,I\}$
is topologically homeomorphic to $S^2$ if and only if each
nontrivial projection has (normalized) trace $\frac12$; moreover 
$\FFF$ is a Kadison-Singer lattice in the sense of Ge and
Yuan.



\vspace{2mm}\baselineskip 12pt

\noindent{\small\bf Keywords} \ \ Kadison-Singer algebra,
Kadison-Singer lattices, von Neumann algebra, unbounded operator.

\vspace{2mm}\baselineskip 12pt

\noindent{\small\bf MSC(2010)} 46L10, 47L75, 47L60



\date{}
\title{{\bf Minimal Generating Reflexive Lattices of Projections in
Finite von Neumann Algebras}}
\author{\bf Chengjun Hou$^{\ast}$, Wei Yuan}\thanks{
Corresponding author: Chengjun Hou\\
Department of Mathematics, Qufu Normal University, Qufu 273165,
China\\
e-mail: cjhou@mail.qfnu.edu.cn\\
Wei Yuan\\
L.K.Hua Key Laboratory of Mathematics, Chinese Academy of Sciences,
Beijing 100190, China\\
e-mail: wyuan@math.ac.cn} \maketitle


\noindent{\small {\small\bf Abstract} \ \ We show that the reflexive
lattice generated by each double triangle lattice of projections in
a finite von Neumann algebra is topologically homeomorphic to
$S^2$ (plus two distinct points corresponding to zero and
$I$). In particular, the reflexive lattice is in general minimally
generating for the von Neumann algebra it generates. As an
application, we show that if a reflexive lattice $\FFF$ generates a
matrix algebra $M_n(\C)$ for $n\geq 3$, then $\FFF\setminus\{0,I\}$
is topologically homeomorphic to $S^2$ if and only if each
nontrivial projection has (normalized) trace $\frac12$; moreover 
$\FFF$ is a Kadison-Singer lattice in the sense of Ge and
Yuan.}



\vspace{2mm}
\baselineskip 12pt

\noindent{\small\bf Keywords} \ \ Kadison-Singer algebra,
Kadison-Singer lattices, von Neumann algebra, unbounded operator.

\vspace{2mm}\baselineskip 12pt

 \noindent{\small\bf MSC(2010)}
46L10, 47L75, 47L60

\baselineskip 16pt

%\newpage
\section{Introduction}

In their 1960 paper (\cite{KS}), Kadison and Singer initiate the study of non selfadjoint
algebras of bounded operators on Hilbert spaces.  They introduce and study "triangular algebras,"
subalgebras of C$^{*}$-algebra whose diagonal is maximal abelian selfadjoint. 
Since then, the theory of non selfadjoint operator  algebras has undergone a vigorous 
development, considerable efforts has gone into the study of different types of  non selfadjoint operator algebras include
triangular algebras, nest algebras, reflexive algebras, etc., many definitive and interesting results are obtained.

Recently, motivated by the work in \cite{KS}, Ge and Yuan(\cite{GY1}) introduce a new
class of non selfadjoint algebras which they call Kadison-Singer algebras. 
These algebras combine triangularity
(\cite{KS}), reflexivity (\cite{Ha,Da}) and von Neumann algebra
properties in their structure. Accordingly, KS-lattices are the
latices of invariant projections of KS-algebras, which are reflexive and 
"minimally generating" in the sense that they generate the
commutant of the diagonals of KS-algebras as von Neumann algebras.
So KS-lattices are closely related to the generator problem for von Neumann
algebras(\cite{GS}). In \cite{GY1, GY2}, Ge and Yuan
constructed KS-algebras with hyperfinite factors as their diagonals, and proved that the KS-lattice
determined by three trace half free projections is homeomorphic
to the two-dimensional sphere $S^2$ (plus two distinct
points corresponding to zero and $I$). The lattice generated by three trace half free projections is 
a special case of the "double triangle lattice"(A lattice contains three nontrivial elements such that the intersection of any
two is zero and the join of any two is $I$). This type of subspace lattices is first studied by P. Halmos 
around 1970's. In \cite{Ha}, Halmos find that any double triangle subspace lattice of  finite dimensional space
is not reflexive, and ask whether any realization of the double triangle is not reflexive? Although some progress
has be made thereafter, this problem is still open in general. In \cite{GY2}, Ge and Yuan also claim if
a double triangle lattice of projections satisfies some specified conditions, then the reflexive lattice determined by it
(excludes 0 and I) is also homemorphic to $S^2$, thus Halmos's problem is settled for those special cases.
This result also indicates that many factors are minimally generated by a reflexive
lattice of projections that is topologically homeomorphic to
$S^2$. In \cite{Hou, WY}, the lattice
generated by a nest with a rank one projection determined by a
separating vector for the von Neumann algebra generated by the nest is studied, and more examples
of KS-lattices are given. 

The paper contains three sections. In section two, by using affiliated 
operators introduced by Murray and von Neumann,
we show that the reflexive lattice generated by any
double triangle lattice of projections in a finite von Neumann
algebra is topologically homeomorphic to $S^2$,
which generalizes the main result in \cite{GY2}. In particular, we
prove that every nontrivial projections in this reflexive lattice have
trace half, and if the von Neumann algebra generated by the double
triangle lattice can not be generated by two nontrivial projections,
then the reflexive lattice is a KS-lattice. In section three, we apply the results 
obtained in section two to describe the "connected" reflexive lattices
in matrix algebras, and show that
if a reflexive lattice $\FFF$ generates a matrix algebra
$M_n(\C)$ for $n\geq 4$, and each
nontrivial projection in it has same dimension, then $\FFF\setminus\{0,I\}$ is
topologically homeomorphic to $S^2$, moreover $\FFF$ is a KS-lattice.



\section{Three projections in finite von Neumann algebras}

Let us first recall some basic notation in the theory of non-self-adjoint operator
algebras. For general theory on operator algebras, we refer to \cite{KR, RR}.

Suppose $\HHH$ is a separable Hilbert space and $B(\HHH)$ the
algebra of all bounded linear operators on $\HHH$. For a set $\LLL$
of orthogonal projections in $B(\HHH)$, we denote by $Alg\LLL$ the
set of all bounded linear operators on $\HHH$ leaving each element
in $\LLL$ invariant. Then $Alg\LLL$ is an unital weak-operator closed
subalgebra of $B(\HHH)$. Similarly, for a subset $\mathcal{S}$ of
$B(\HHH)$, let $Lat\mathcal{S}$ be the set of  invariant
projections for every operators in $\mathcal{S}$. Then $Lat\mathcal{S}$ is a
strong-operator closed lattice of projections. A subalgebra
$\mathcal{A}$ of $B(\HHH)$ is said to be reflexive if
$\mathcal{A}=Alg\Lat\mathcal{A}$, similarly a lattice $\LLL$ of
projections is reflexive if $\LLL=LatAlg\LLL$.

Throughout this section $\MMM$ denotes a finite von Neumann algebra acting on 
a separable Hilbert space $\HHH$, $tr$ is a normal, faithful tracial state on
$\MMM$. Let $\AAA = M_{2}(\C)\otimes\MMM$. Then $\AAA$ is a finite
von Neumann algebra acting on $\HHH\oplus \HHH$. Let $\{ E_{i,j}
\}_{i,j = 1}^{2}$ be the canonical matrix units in $M_2(\C)$. With
respect to these matrix units, we can write any operator $T$ in
$B(\HHH\oplus \HHH)$ as a operator matrix:
\begin{align*}
T=\left(\begin{array}{cc}T_{11} & T_{12} \\T_{21} &
T_{22}\end{array}\right),
\end{align*}
where each entry $T_{ij}$ is in $B(\HHH)$. Then
$\tau(T)=\frac{1}{2}(tr(T_{11}) + tr(T_{22}))$, for $T\in \AAA$, is
a normal, faithful tracial state on $\AAA$. We use
the same symbol $I$ to denote the identity operator in $\MMM$ and in
$\AAA$  when there is no risk of ambiguity.
In the rest of this section, we fix a projection
$P_1 =\left(\begin{array}{cc}I & 0 \\0 & 0\end{array}\right)$ in
$\AAA$. The following lemma is probably well know, we give a proof for the convenience of the reader.


\begin{lemma}
With the above notations, let $P$ be a projection in $\AAA$ such that $P\vee P_1=I$ and
$P\wedge P_1=0$. Then there exist a  positive contractive operator
$H$ in $\MMM$ such that $I-H$ is injective, and an unitary $V$ in
$\MMM$ such that
$$P=\left(\begin{array}{cc} H & \sqrt{H(I-H)}V\\ V^*\sqrt{H(I-H)} & V^*(I-H)V \end{array}\right).$$
Furthermore, $\tau(P)=\frac12$.
\end{lemma}

\noindent{\it Proof}\quad By Kaplansky formula
(\cite[ Theorem 6.1.7]{KR}), we have  $\tau(P_1\vee
P)=\tau(P_1)+\tau(P)-\tau(P_1\wedge P)$, thus $\tau(P)=\frac12$.
If we write
\begin{align*}
P= \left(\begin{array}{cc}H & H_{1}V \\V^* H_{1} &
H_{2}\end{array}\right),
\end{align*}
where $V$ is a unitary in $\MMM$, $H$, $H_1$ and $H_2$ are positive
operators in $\MMM$. By $P^2 = P$, we have
$$H = H^2+H_{1}^{2},\,\,\,\,\,
H_2= V^*H_1^2V+H_2^2,\,\,\, H_1V =HH_{1}V + H_1VH_2.$$ Hence we have
$$H_1=\sqrt{H(I-H)},\eqno{(1)}$$
$$V^*H(I-H)V=H_2(I-H_2),\eqno{(2)}$$
$$(I-H)\sqrt{H(I-H)}V=\sqrt{H(I-H)}VH_2.\eqno{(3)}$$

We claim that $I-H$ and $H_2$ are injective. Indeed, for each vector
$\xi\in \HHH$ with $(I-H)\xi=0$, we have $\xi=H\xi$ and
$\sqrt{I-H}\xi=0$, this implies $\left(\begin{array}{c} \xi\\
0\end{array}\right)\in P_1\wedge P$( = 0). Hence, $\xi=0$ and $I-H$ is injective. 
Similarly, for each vector
$\eta\in\HHH$ with $H_2\eta=0$, we have
$H_{1}V\eta = \sqrt{H(I-H)}V\eta=0$ by (3). Thus $\left(\begin{array}{c}0\\
\eta\end{array}\right)\in P_1^{\perp}\wedge P^{\perp}$ (=0), which
implies $\eta=0$, and $H_2$ is injective.


Next we show that $H_2=V^*(I-H)V$. Clearly, (3) and the injectivity
of $I-H$ imply that $H(I-H)V=HVH_2$. Hence $V^*H(I-H)V=V^*HVH_2$,
which, together with (2), gives us $V^*HVH_2=H_2(I-H_2)$. Therefore, 
by the injectivity of $H_2$, we have $H_2=V^*(I-H)V$. This
completes the proof.


\vspace{2mm}
Before we proceed to the proof of our main result, we recall some terms and basic facts for
unbounded operators affiliated with finite von Neumann algebras for the readers convenience.

Let $T$ be a closed densely defined operator, i.e. the graph $\mathcal{G}(T) = \{(\xi, T\xi): \xi \in \DDD(T) \}$ is closed, 
we denote its  domain, range, null space,
range projection and null projection by $\DDD(T),\,
Ran(T),\, Ker(T)$, $R(T)$ and $N(T)$ respectively.
A dense linear manifold $\DDD_{0}$ in $\DDD(T)$ is a core for $T$ if 
$\mathcal{G}(T| \DDD_{0})^{-} = \mathcal{G}(T)$, where $\mathcal{G}(T| \DDD_{0})^{-}$ is
the closure of $\mathcal{G}(T| \DDD_{0})$. If $U^{*}TU = T$  for any unitary operator $U$ in a von Neumann algebra $\AAA$,
we say that $T$ is affiliated with $\AAA$.
In \cite{MV}, Murray and von Neumann show that the family of all operators affiliated with
a finite von Neumann algebra $\MMM$ forms an associative algebra, we denote this
algebra by $\widetilde{\MMM}$. 
For any $X$, $Y \in \widetilde{\MMM}$, $X
+Y$ and $XY$ are densely defined, closable, and their closures,
denoted by $X \widehat{+} Y$ and $X\widehat{\cdot} Y$, respectively,
are in $\widetilde{\MMM}$.  If $\xi \in \DDD(X \widehat{+} Y) \cap \DDD(X)$, then $\xi \in
\DDD((X \widehat{+} Y) - X)$, and thus $\xi \in \DDD(Y)$, $(X
\widehat{+} Y)\xi = X\xi + Y\xi$, since $\widetilde{\MMM}$ is an
algebra and $(X \widehat{+} Y) \widehat{-} X = Y$. Moreover, $Ker(X \widehat{+} Y) = 0$ if and only if $Ker(X +Y) =
0$. Indeed, if $Ker(X +Y) = 0$, but $Ker(X \widehat{+} Y) \neq 0 $,
then there exists a projection $Q \in \MMM$ such that $tr(Q)> 0$,
and for any $\xi \in Q \HHH$, $(X \widehat{+} Y)\xi = 0$. Hence we
can choose two projections $E$ and $F \in \MMM$ such that $E(\HHH)
\subset \DDD(X)$, $F(\HHH) \subset \DDD(Y)$, and $tr(E \wedge F
\wedge Q) > 0$. This means we can find a nonzero vector $\xi \in (E
\wedge F \wedge Q) \HHH$, which contradicts with the fact that
$Ker(X +Y) = 0$.  Readers interested in a more detailed treatment of this subject are
referred to \cite{Zhe}.



\begin{lemma}
Let $P_1$ be as before,  $P_2$ and $P_3$ be two projections in
$\AAA$ such that $P_1\wedge P_i=0$ and $P_1\vee P_i=I$ for $i=2,3$.
Let
\begin{align*}
&P_2 = \left(\begin{array}{cc}H_1 & \sqrt{H_1 (I-H_1)}V_1 \\V_1^*
\sqrt{H_1 (I-H_1)} & V_1^{*}(I - H_1)V_1\end{array}\right),\\
&P_3 = \left(\begin{array}{cc}H_2 & \sqrt{H_2(I-H_2)}V_2 \\V_2^*
\sqrt{H_2(I-H_2)} & V_2^{*}(I - H_2)V_2\end{array}\right),
\end{align*}
where $V_i$ is a unitary in $\MMM$, $H_i$ is a contractive positive
operator such that $Ker(I-H_i)=0$ for $i=1,2$. Then
\begin{enumerate}
\item[(i)] $P_2 \wedge P_3 = 0$ if and only if
$Ker(\sqrt{H_1(I-H_1)^{-1}}V_1-\sqrt{H_2(I-H_2)^{-1}}V_2)=0.$
\item[(ii)] $P_2\vee P_3=I$ if and only if
$Ker(V_1^*\sqrt{H_1(I-H_1)^{-1}}-V_2^*\sqrt{H_2(I-H_2)^{-1}})=0.$
\end{enumerate}
\end{lemma}

\noindent{\it Proof}\quad (i) If $Ker(\sqrt{H_1(I-H_1)^{-1}}V_1 -
\sqrt{H_2(I-H_2)^{-1}}V_2)  \neq 0$, there exists a nonzero
vector $\xi \in \DDD(\sqrt{H_1(I-H_1)^{-1}}V_1)  \cap
\DDD(\sqrt{H_2(I-H_2)^{-1}}V_2)$ such that
\begin{align*}
\sqrt{H_1(I-H_1)^{-1}}V_1 \xi =  \sqrt{H_2(I-H_2)^{-1}}V_2 \xi \overset{\text{def}}{=} \eta.
\end{align*}
Then
$\sqrt{H_i(I-H_i)}\eta=H_iV_i\xi$ for  $i=1,2$. So $\left(\begin{array}{c} \eta\\
\xi\end{array}\right)\in P_2\wedge P_3$, which yields $P_2 \wedge
P_3\neq 0$.  We can now reverse the above argument to obtain the other
direction.

Note that $P_i\vee P_j=I$ if and only if  $(I - P_i)\wedge
(I- P_j)=0$, $i,j=1,2,3$ and $i \neq j$, we can obtain (ii) by applying the same argument to $I-P_1$, 
$I-P_2$ and $I-P_3$.
\vspace{2mm}

\begin{remark}
In the lemma above, we could actually assume that $V_1 = I$, by
changing $P_2$ to $U^{*}P_2 U$, and $P_3$ to $U^* P_3 U$, where $U =
\left(\begin{array}{cc}I & 0 \\0 & V_{1}^{*}\end{array}\right)$.
\end{remark}

From now on, let $P_1$ be as before, and $P_2$, $P_3$ two
projections in $\AAA$ such that $P_i \wedge P_j = 0$ and $P_i \vee
P_j = I$ for $i,j=1,2,3$ and $i \neq j$.  By Lemma 2.1, we have $\tau(P_i)=\frac12$ for each $i = 1,2,3$. 
Let $\LLL=\{0,P_1,P_2,P_3,I\}$. Then $\LLL$ is a double triangle lattice
of projections. For the rest of this section, we study the structure of the
reflexive lattice, as well as the corresponding reflexive algebra,
determined by $\LLL$, up to the unitary equivalence. By Lemma 2.2 
and its remark, we may assume
\begin{align*}
&P_2 = \left(\begin{array}{cc}H_1 & \sqrt{H_1 (I-H_1)} \\\sqrt{H_1
(I-H_1)} & I - H_1\end{array}\right), \\
&P_3 = \left(\begin{array}{cc}H_2 & \sqrt{H_2(I-H_2)}V \\V^*
\sqrt{H_2(I-H_2)} & V^{*}(I - H_2)V\end{array}\right),
\end{align*}
where $H_i$ is a contractive positive operator in $\MMM$ such that
$Ker(I-H_i)=0$ for $i=1,2$, $V$ is a unitary operator in $\MMM$. Let
\begin{align*}
S = \sqrt{H_{1}(I-H_{1})^{-1}} \widehat{-}
\sqrt{H_{2}(I-H_{2})^{-1}}V.
\end{align*}
Then Lemma 2.2 and the discussion before it show that $S$ is an invertible, since an unbounded operator
affiliated with $\MMM$ has an inverse in $\widetilde{\MMM}$ if and only if 0 is not in the point spectrum of it.

\begin{lemma} With notation given above, $T=\left(\begin{array}{cc} T_1 & T_2\\
T_4 & T_3\end{array}\right) \in M_2(\C)\otimes B(\HHH)$ is in $Alg\L$ if and only if $T_4=0$ and the following equations
hold:
$$\sqrt{I-H_1}T_2\sqrt{I-H_1}=\sqrt{H_1}T_3\sqrt{I-H_1}-\sqrt{I-H_1}T_1\sqrt{H_1},\eqno{(4)}$$
$$\sqrt{I-H_2}T_2V^*\sqrt{I-H_2}=\sqrt{H_2}VT_3V^*\sqrt{I-H_2}-\sqrt{I-H_2}T_1\sqrt{H_2}.\eqno{(5)}$$
\end{lemma}

\noindent{\it Proof}\quad   Note that $P_2=W_2^*P_1W_2$ and
$P_3=W_3^*P_1W_3$, where $$W_2=\left(\begin{array}{cc} \sqrt{H_1} &
\sqrt{I-H_1}\\ \sqrt{I-H_1} & -\sqrt{H_1}
\end{array}\right),\,\,W_3=\left(\begin{array}{cc} \sqrt{H_2} &
\sqrt{I-H_2}V\\ \sqrt{I-H_2} & -\sqrt{H_2}V
\end{array}\right)$$
are  unitary. Hence $T$ is in $ Alg\L$ if and only if $(I- P_1)TP_1=(I-P_1)W_2TW_2^*P_1=(I-P_1)W_3TW_3^*P_1=0$.
A straightforwad computation then shows this equivalent to $T_4 = 0$ and (4),(5). \vspace{2mm}

\begin{lemma}
 If $
T\in Alg\LLL$, then there is an $T_1 \in \BBB(\HHH)$ such that
\begin{align*}
T = \left(\begin{array}{cc}T_1 & \sqrt{H_{1}(I-H_{1})^{-1}}S^{-1}T_1 S -T_1 \sqrt{H_{1}(I-H_{1})^{-1}}  \\0 & S^{-1}T_1S \end{array}\right).
\end{align*}
Conversely, if $T_1 \in\BBB(\HHH)$ such that 
$\sqrt{H_{1}(I-H_{1})^{-1}}S^{-1}T_1S - T_1 \sqrt{H_{1}(I-H_{1})^{-1}} =T_2$
and $ S^{-1}T_1 S = T_3$ for some bounded operators $T_2$ and $T_3$, then $T$ belongs to $
Alg\LLL$.
\end{lemma}


Note that the equalities , $\sqrt{H_{1}(I-H_{1})^{-1}}S^{-1}T_1 S -
T_1 \sqrt{H_{1}(I-H_{1})^{-1}} = T_2$ and $S^{-1}T_1 S= T_3$, in the preceding lemma is 
to be understood in the sense that for each vector $\xi$ in $\DDD(\sqrt{H_1(I-H_1)^{-1}}) \cap
\DDD(S)$(= $\DDD(\sqrt{H_1(I-H_1)^{-1}}) \cap \DDD(\sqrt{H_2(I-H_2)^{-1}}V) $) , $S^{-1}T_1 S\xi$ and  $\sqrt{H_{1}(I-H_{1})^{-1}}S^{-1}T_1 S\xi
- T_1 \sqrt{H_{1}(I-H_{1})^{-1}}\xi$ are defined, and equal to $T_3\xi$, $T_2 \xi$ respectively. \newline


\noindent{\it Proof}\quad By Lemma 2.3, $T \in  Alg\LLL$ if
and only if $T$ has the
 form $T = \left(\begin{array}{cc}T_1 & T_2 \\0 &
T_3\end{array}\right)$, where $T_1$, $T_2$ and $T_3$ satisfy (4) and
(5). Since $\DDD(\sqrt{H_1(I-H_1)^{-1}}) \cap
\DDD(\sqrt{H_2(I-H_2)^{-1}}V)$ is a common core for both $\sqrt{H_1(I-H_1)^{-1}}$ and $\sqrt{H_2(I-H_2)^{-1}}V$, 
we deduce (4) and (5)
are true if and only if, for each vector $\xi$ in
$\DDD(\sqrt{(I-H_1)^{-1}}) \cap \DDD(\sqrt{(I-H_2)^{-1}}V)$,
\begin{align*}
T_2 \xi &= \sqrt{H_1(I - H_1)^{-1}} T_3 \xi - T_1 \sqrt{H_1(I-H_1)^{-1}} \xi , \\
T_2 \xi &= \sqrt{H_2(I - H_2)^{-1}}V T_3 \xi - T_1
\sqrt{H_2(I-H_2)^{-1}}V \xi .
\end{align*}
These are equivalent to
\begin{align*}
T_2 \xi &= \sqrt{H_1(I - H_1)^{-1}} T_3 \xi - T_1 \sqrt{H_1(I-H_1)^{-1}} \xi, \\
T_3 \xi &= S^{-1}T_1S\xi.
\end{align*}

\begin{remark}
Actually there exists a family of projections, $\{E_{\epsilon}:\,\epsilon>0\}$, in $\MMM$ such that $S^{-1}E_{\epsilon}$,
$E_{\epsilon}S$, $E_{\epsilon}\sqrt{H_{1}(I-H_{1})^{-1}}$ and
$\sqrt{H_{1}(I-H_{1})^{-1}}S^{-1}E_{\epsilon}$ are bounded and
$E_{\epsilon}$ is strong-operator convergent to $I$ (as
$\epsilon\rightarrow 0$). Hence the set of all $A$ that satisfies the conditions of Lemma 2.4 is dense in
$B(\HHH)$ under the strong operator topology.
\end{remark}

\begin{corollary} For $Q\in LatAlg\L \setminus \{0,I, P_1\}$,
 we have $Q\wedge P_1=0$, $Q\vee P_1=I$, and $\tau(Q)=\frac12$.
Moreover, for any two distinct nontrivial projections $Q_1$ and $Q_2$ in $LatAlg\L \setminus \{0, I \}$,
 we have $Q_1\wedge Q_2=0$ and $Q_1\vee Q_2=I$.
\end{corollary}


 \noindent{\it Proof}\quad For $Q$ as in the corollary, let $Q\wedge P_1=\left(\begin{array}{cc} P & 0 \\ 0 &
0\end{array}\right)$, where $P$ is a projection in $\MMM$. Then $P$ is invariant under $P_1TP_1|_{\HHH}$ for all $T \in Alg\L$,
since $Q\wedge P_1 \in LatAlg\L$. By Remark 2.2, 
$P=0$ or $P=I$, thus $Q\wedge
P_1=0$ or $Q\wedge P_1=P_1$. Similarly, we have $Q\vee P_1=P_1$ or
$Q\vee P_1=I$. It follows that $Q\wedge P_1=0$, $Q\vee P_1=I$ and $\tau(Q)=\frac12$.

For $Q_1$ and $Q_2$ in $LatAlg\L \setminus \{0, I, P_1 \}$, we have 
$Q_i\wedge P_1=0$, $Q_i\vee P_1=I$ and $\tau(Q_i)=\frac12$.
Clearly, $Q_1\wedge Q_2\neq P_1$. Suppose that $Q_1\wedge Q_2\neq
0$. $\tau(Q_1\wedge Q_2)=\frac12$, since $Q_1\wedge Q_2 \in   LatAlg\L \setminus \{0,I\}$.  It follows that $Q_1\wedge
Q_2=Q_1=Q_2$, but this contradicts to the fact that $Q_1 \neq Q_2$. Hence $Q_1\wedge Q_2=0$ and
$Q_1\vee Q_2=I$. \vspace{2mm}



\begin{prop}
For any projection $Q$ in $LatAlg\LLL \setminus \{0, I, P_{1}\}$, there are $K_a$ and $U_a$ in $\MMM$ such that 
\begin{align*}
Q = \left(\begin{array}{cc}K_{a} & \sqrt{K_{a}(I-K_{a})}U_{a} \\U_{a}^{*}\sqrt{K_{a}(I-K_{a})} &
U_{a}^{*}(I-K_{a})U_{a} \end{array}\right),
\end{align*}
where $\sqrt{K_{a}(I-K_{a})}$ (or  $K_{a}$) and $U_{a}$ are determined
by the polar decomposition of $aS
\widehat{+} \sqrt{H_{1}(I-H_{1})^{-1}}$
$= \sqrt{K_{a}(I-K_{a})^{-1}} U_{a}$ for some $a \in \C$.
Conversely, for any given $a$ in $\C$, the polar decomposition
determines $U_{a}$ and $K_{a}$ uniquely which give rise to a projection $Q$ (in the
above form) in $LatAlg\LLL$.
\end{prop}


\noindent{\it Proof}\quad Suppose that $Q$ is in $LatAlg\LLL
\setminus \{0, I, P_{1}\}$. From Corollary 2.1, we know that
$Q\wedge P_1=0$ and $Q\vee P_1=I$. It follows from Lemma 2.1 that
$Q$ has the form
\begin{align*}
Q = \left(\begin{array}{cc}K & \sqrt{K(I-K)}U \\U^{*}\sqrt{K(I-K)} &
U^{*}(I-K)U \end{array}\right),
\end{align*}
where $K$ is a contractive positive operator in $\MMM$ such that
$Ker(I-K)=0$, $U$ is a unitary in $\MMM$. Let $\{E_\epsilon \}$ be a monotone increasing net of projections in $\MMM$ with 
strong-operator limit $I$ as $\epsilon \rightarrow 0$, such that for each $\epsilon > 0$, $E_\epsilon \sqrt{H_1 (I - H_1)^{-1}}$,
$S^{-1}E_\epsilon$, $\sqrt{H_1(I-H_1)^{-1}}S^{-1} E_\epsilon$,
$E_{\epsilon}S$, $(SU^{*}\sqrt{I-K})^{-1}E_\epsilon$ are all
bounded. Then for each operator $A \in \BBB(\HHH)$ with $E_\epsilon
A = A E_\epsilon = A$, we have $S^{-1}AS$,
$\sqrt{H_{1}(I-H_{1})^{-1}}S^{-1}AS$ and
$A\sqrt{H_{1}(I-H_{1})^{-1}}$ are bounded operators. By Lemma 2.4, 
\begin{align*}
T = \left(\begin{array}{cc}A & \sqrt{H_{1}(I-H_{1})^{-1}}S^{-1}AS
-A\sqrt{H_{1}(I-H_{1})^{-1}}  \\0 & S^{-1}AS \end{array}\right) \in
 Alg\LLL.
\end{align*}
Exactly as in the proof of  Lemma 2.3, $(I-Q)TQ = 0$ implies
\begin{align*}
\sqrt{I-K}&A\sqrt{K} +
\sqrt{I-K}\left[\sqrt{H_{1}(I-H_{1})^{-1}}S^{-1}AS \right.\cr & -
\left.A\sqrt{H_{1}(I-H_{1})^{-1}}\right]U^{*}\sqrt{I-K} -
\sqrt{K}US^{-1}ASU^{*}\sqrt{I-K} = 0,
\end{align*}
which equivalent to
\begin{align*}
\sqrt{I-K}A&\left[\sqrt{K} -
E_{\epsilon}\sqrt{H_{1}(I-H_{1})^{-1}}U^{*}\sqrt{I-K}\right]\\
& =\left[\sqrt{K}US^{-1}E_{\epsilon} -
\sqrt{I-K}\sqrt{H_{1}(I-H_{1})^{-1}}S^{-1}E_{\epsilon}\right]ASU^{*}\sqrt{I-K}.
\end{align*}
Multiply both sides of the equation by $\sqrt{(I-K)^{-1}}$ and $(SU^{*}\sqrt{I-K})^{-1}E_\epsilon$, we have 
\begin{align*}
A&[\sqrt{K} -
E_\epsilon\sqrt{H_{1}(I-H_{1})^{-1}}U^{*}\sqrt{I-K}](SU^{*}\sqrt{I-K})^{-1}E_\epsilon \\
& \qquad = \sqrt{(I-K)^{-1}}\left[\sqrt{K}US^{-1}E_{\epsilon} -
\sqrt{I-K}\sqrt{H_{1}(I-H_{1})^{-1}}S^{-1}E_{\epsilon}\right] A E_\epsilon .
\end{align*}
Since$[ E_\epsilon\sqrt{K} -
E_\epsilon\sqrt{H_{1}(I-H_{1})^{-1}}U^{*}\sqrt{I-K}](SU^{*}\sqrt{I-K})^{-1}E_\epsilon
$ is bounded, by letting $A = E_\epsilon$, we know that
$\sqrt{(I-K)^{-1}}\left[\sqrt{K}US^{-1}E_{\epsilon} -
\sqrt{I-K}\sqrt{H_{1}(I-H_{1})^{-1}}S^{-1}E_{\epsilon}\right]$ is also
bounded. So there is a unique $a \in \mathbb{C}$, such that
\begin{align*}
&[E_\epsilon\sqrt{K} -
E_\epsilon\sqrt{H_{1}(I-H_{1})^{-1}}U^{*}\sqrt{I-K}](SU^{*}\sqrt{I-K})^{-1}E_\epsilon
\\
& \qquad =\sqrt{(I-K)^{-1}}\left[\sqrt{K}US^{-1}E_{\epsilon} -
\sqrt{I-K}\sqrt{H_{1}(I-H_{1})^{-1}}S^{-1}E_{\epsilon}\right] = a
E_\epsilon.
\end{align*}
Let $\epsilon \rightarrow 0$, we have
\begin{align*}
\sqrt{K(I-K)^{-1}}U = aS\widehat{+} \sqrt{H_{1}(I-H_{1})^{-1}}.
\end{align*}

Conversely, if  $K$ and $U$ are given by the above equation, one checks easily that $Q$ given in the proposition is
in $LatAlg\LLL$ by Lemma 2.4. \vspace{2mm}

By this proposition, we have the following corollary.

\begin{corollary}
For any two distinct projections $Q_1$, $Q_2$ in $LatAlg\LLL
\setminus \{ 0, I, P_1 \}$, we have  $ Alg\LLL =  Alg\{P_1, Q_1,
Q_2\}$, thus $$LatAlg\{P_1, Q_1, Q_2 \} = LatAlg\LLL.$$
\end{corollary}

\noindent{\it Proof}\quad Obviously, $ Alg\LLL\subseteq Alg\{P_1,
Q_1, Q_2 \}$. On the other hand, by Proposition 2.1, let
$$Q_i= \left(\begin{array}{cc}K_{a_i}
& \sqrt{K_{a_i}(I-K_{a_i})}U_{a_i} \\U_{a_i}^{*}\sqrt{K_{a_i}(I-K_{a_i})} &
U_{a_i}^{*}(I-K_{a_i})U_{a_i}
\end{array}\right),
$$ where  $K_{a_i}$ and $U_{a_i}$ are determined by the polar
decomposition $a_i S\widehat{+ }\sqrt{H_{1}(I-H_{1})^{-1}}=
\sqrt{K_{a_i}(I-K_{a_i})^{-1}} U_{a_i}$ for $a_i\in \C$, $i=1,2$, 
$a_1\neq a_2$. Therefore we have $$\sqrt{K_{a_1}(I-K_{a_1})^{-1}}
U_{a_1}\widehat{-}\sqrt{K_{a_2}(I-K_{a_2})^{-1}} U_{a_2}=(a_1-a_2)S.$$

By the same reasoning as in the proof of Lemma 2.4, we have $T \in Alg\{P_1,Q_1,Q_2\}$ if and only if 
there exists $T_1$ in $B(\HHH)$ such that
$$T=\left(\begin{array}{cc} T_1 & \sqrt{K_{a_1}(I-K_{a_1})^{-1}}U_{a_1}S^{-1}T_1 S-T_1\sqrt{K_{a_1}(I-K_{a_1})^{-1}}U_{a_1}\\
0 & S^{-1}T_1 S\end{array}\right).$$ 
However, for any $\xi \in \DDD(\sqrt{K_{a_1}(I-K_{a_1})^{-1}}U_{a_1}) \cap \DDD(\sqrt{K_{a_2}(I-K_{a_2})^{-1}}U_{a_2})$\\  
$= \DDD(\sqrt{H_1(I-H_{1})^{-1}}) \cap  \DDD(\sqrt{H_2(I-H_{2})^{-1}}V)$, we have 
$$\begin{array}{l} 
 \sqrt{K_{a_1}(I-K_{a_1})^{-1}}U_{a_1}S^{-1}T_1 S\xi -T_1\sqrt{K_{a_1}(I-K_{a_1})^{-1}}U_{a_1}\xi\\
=(a_1S\widehat{+} \sqrt{H_1(I-H_1)^{-1}})S^{-1}T_1S\xi- T_1 (a_1S \widehat{+} \sqrt{H_1(I-H_1)^{-1}})\xi\\
=\sqrt{H_1(I-H_1)^{-1}}S^{-1}T_1S\xi-T_1\sqrt{H_1(I-H_1)^{-1}}\xi.\end{array}$$
It follows from Lemma 2.4 that $T$ is in $ Alg\L$.
Consequently, $Alg\LLL =  Alg\{0,P_1, Q_1, Q_2, I\}$.
\vspace{2mm}

\begin{prop}
With the notation in Proposition 2.1, the mapping, from $\C$ into $LatAlg\LLL$,
given by
\begin{align*} a \rightarrow Q_{a} &=
\left(
          \begin{array}{cc}
            K_{a} & \sqrt{K_{a}(I-K_{a})}U_{a} \\
            U_{a}^{*}\sqrt{K_{a}(I-K_{a})} & U_{a}^{*}(I-K_{a})U_{a} \\
          \end{array}
        \right),
\end{align*}
is one to one and continuous when $\C$ is endowed with the usual topology and
$LatAlg\LLL$ has the topology induced by the $\|.\|_{2}$-norm. Moreover, $\| Q_{a} - P_1 \|_{2}
$ tends to $0$ as $a$ tends to $\infty$. Hence $LatAlg\LLL\setminus\{0,I\}$ is homeomorphic
to $S^2$.
\end{prop}

\noindent{\it Proof}\quad For $a,\,a_0\in \C$, let $Q_a$ and
$Q_{a_0}$ be defined as in the proposition. By corollary 2.1,
$\tau(Q_a)=\tau(Q_{a_0})=\frac12$. We only need to show that $\|Q_{a} - Q_{a_0}\|_{2} \rightarrow 0$ as $a$ tends to $a_{0}$. Since
\begin{align*}
\|Q_{a} - Q_{a_0}\|_{2}^{2} &= 1 - 2\tau(Q_{a}Q_{a_{0}}) \\
                            &= 1 - tr(K_{a}K_{a_0}) -
                            tr(\sqrt{K_{a}(I-K_{a})}U_{a}U^{*}_{a_0}\sqrt{K_{a_0}(I-K_{a_0})})\\
                            & \qquad - 
                            tr(U^{*}_{a}\sqrt{K_{a}(I-K_{a})}\sqrt{K_{a_0}(I-K_{a_0})}U_{a_0})\\
                            & \qquad -
                            tr(U^{*}_{a}(I-K_{a})U_{a}U^{*}_{a_0}(I-K_{a_0})U_{a_0}),
\end{align*}
it is sufficient to show the following statements hold as $a \rightarrow a_{0}$,
$$|tr((K_{a}-K_{a_0})K_{a_0})| \rightarrow 0, \eqno{(6)}$$
$$|tr((\sqrt{K_{a}(I-K_{a})}U_{a}
-\sqrt{K_{a_0}(I-K_{a_0})}U_{a_0})U^{*}_{a_0}\sqrt{K_{a_0}(I-K_{a_0})})
| \rightarrow 0,\eqno{(7)}$$
$$|tr((U^{*}_{a}\sqrt{K_{a}(I-K_{a})}-U^{*}_{a_0}\sqrt{K_{a_0}(I-K_{a_0})})\sqrt{K_{a_0}(I-K_{a_0})}U_{a_0})|
\rightarrow 0,\eqno{(8)}$$
$$|tr(K_{a} - K_{a_0})| \rightarrow 0,
\quad |tr((U^{*}_{a}K_{a}U_{a}
-U^{*}_{a_0}K_{a_0}U_{a_0})U^{*}_{a_0}K_{a_0}U_{a_0})| \rightarrow
0.\eqno{(9)}$$
Remember $\sqrt{K_{a}(I-K_{a})^{-1}}U_a = aS \widehat{+}
\sqrt{H_{1}(I-H_{1})^{-1}}$, let 
\begin{align*}
F(a) &= K_{a}(I-K_{a})^{-1}\\
&= |a|^{2}SS^{*} \widehat{+} aS\sqrt{H_{1}(I-H_{1})^{-1}} \widehat{+}
\overline{a}\sqrt{H_{1}(I-H_{1})^{-1}}S^{*} \widehat{+}
(\sqrt{H_{1}(I-H_{1})^{-1}})^2,
\end{align*}
thus $I - K_{a} = (I + F(a))^{-1}$.

For each $\epsilon > 0$, there exists a projection $F_\epsilon \in
\MMM$ such that $tr(I-F_{\epsilon})<\epsilon^{2}$, and 
$SS^{*}E_{\epsilon}$, $S\sqrt{H_{1}(I-H_{1})^{-1}}E_{\epsilon}$,
$\sqrt{H_{1}(I-H_{1})^{-1}}S^{*}E_{\epsilon}$,
$H_{1}(I-H_{1})^{-1}E_{\epsilon}$ are bounded operators, where 
$E_\epsilon = \RRR((I + F(a_0))^{-1}F_\epsilon)$. Note
that $tr(I-F_{\epsilon})=tr(I-E_{\epsilon}) < \epsilon^{2}$. So
there exists a constant $\delta > 0$ such that if $|a-a_{0}| <
\delta$, we have $\|F(a)E_{\epsilon} - F(a_0)E_{\epsilon}\| <
\epsilon$, and thus
\begin{align*}
|tr((K_{a}-K_{a_0})K_{a_0})| &\leq
|tr(K_{a_0}(K_{a}-K_{a_0})F_{\varepsilon})| +
|tr(K_{a_0}(K_{a}-K_{a_0})(I-F_{\varepsilon}))| \\
&\leq
|tr(K_{a_0}(I+F(a))^{-1}(F(a)-F(a_0))(I+F(a_0))^{-1}F_{\varepsilon})|
+ 2\varepsilon \\ & \leq 3\varepsilon.
\end{align*}
This implies (6). By a similar argument, we have $|tr(K_{a} -
K_{a_0})| \rightarrow 0$, as $a\rightarrow a_0$.

In order to prove (7), for each $\epsilon > 0$, we choose a
projection $P_\epsilon \in \MMM$ such that $tr(I - P_{\varepsilon})
< \varepsilon^{2}$, $SP_{\varepsilon}$ and
$\sqrt{H_{1}(I-H_{1})^{-1}}P_{\varepsilon}$ are bounded operators.
Thus there exists $\delta_{1} > 0$ such that if $|a-a_{0}| <
\delta_{1}$, then
\begin{align*}
\| \sqrt{K_a(I-K_a)^{-1}}U_{a}P_{\epsilon} -
\sqrt{K_{a_0}(I-K_{a_{0}})^{-1}}U_{a_0}P_{\epsilon}  \| = \|(a - a_0)SP_{\epsilon}  \| \leq
\epsilon .
\end{align*}
Note that
\begin{align*}
&\sqrt{K_{a}(I-K_{a})}U_{a}P_{\varepsilon}
-\sqrt{K_{a_0}(I-K_{a_0})}U_{a_0}P_{\varepsilon}\\
& \qquad =(1-K_a)\sqrt{K_{a}(I-K_{a})^{-1}}U_{a}P_{\varepsilon}-(1-K_{a_0})\sqrt{K_{a_{0}}(I-K_{a_{0}})^{-1}}U_{a_0}P_{\varepsilon}\\
& \qquad =(1-K_a)(\sqrt{K_{a}(I-K_{a})^{-1}}U_{a}P_{\varepsilon}-\sqrt{K_{a_{0}}(I-K_{a_{0}})^{-1}}U_{a_0}P_{\varepsilon})\\
& \qquad    -(K_a-K_{a_0})\sqrt{K_{a_{0}}(I-K_{a_{0}})^{-1}}U_{a_0}P_{\varepsilon}.
\end{align*}
Hence we have
\begin{align*}
&|tr((\sqrt{K_{a}(I-K_{a})}U_{a}
-\sqrt{K_{a_0}(I-K_{a_0})}U_{a_0})U^{*}_{a_0}\sqrt{K_{a_0}(I-K_{a_0})})|
\\
&\qquad \leq
|tr(U^{*}_{a_0}\sqrt{K_{a_0}(I-K_{a_0})}(\sqrt{K_{a}(I-K_{a})}U_{a}
-\sqrt{K_{a_0}(I-K_{a_0})}U_{a_0})P_{\varepsilon})
| \\
&\qquad +
|tr(U^{*}_{a_0}\sqrt{K_{a_0}(I-K_{a_0})}(\sqrt{K_{a}(I-K_{a})}U_{a}
-\sqrt{K_{a_0}(I-K_{a_0})}U_{a_0})(I-P_{\varepsilon}))
| \\
&\qquad\leq
|tr(U^{*}_{a_0}\sqrt{K_{a_0}(I-K_{a_0})}(I-K_{a})[\sqrt{K_a(I-K_a)^{-1}}U_{a}P_{\varepsilon}
- \sqrt{K_{a_0}(I-K_{a_{0}})^{-1}}U_{a_0}P_{\varepsilon}]) | \\
& \qquad +
|tr(\sqrt{K_{a_0}(I-K_{a_{0}})^{-1}}U_{a_0}P_{\varepsilon}U^{*}_{a_0}\sqrt{K_{a_0}(I-K_{a_0})}(K_{a_0}
-K_a))| + 2\varepsilon \\
&\qquad \leq
|tr(\sqrt{K_{a_0}(I-K_{a_{0}})^{-1}}U_{a_0}P_{\varepsilon}U^{*}_{a_0}\sqrt{K_{a_0}(I-K_{a_0})}(K_{a_0}
-K_a))| + 3\varepsilon.
\end{align*}
To show $|tr(\sqrt{K_{a_0}(I-K_{a_{0}})^{-1}}U_{a_0}P_{\varepsilon}U^{*}_{a_0}\sqrt{K_{a_0}(I-K_{a_0})}(K_{a_0}
-K_a))| \rightarrow 0$, apply the same argument as in the proof of (6). Therefore we have (7).
(8) can be derived similarly.

For the second statement in (9), we choose, for any $\epsilon > 0$,
a projection $P_\epsilon \in \MMM$ such that $tr(I -
P_{\varepsilon}) < \varepsilon^{2}$, $SP_{\varepsilon}$,
$\sqrt{H_{1}(I-H_{1})^{-1}}P_{\varepsilon}$,  $P_{\varepsilon}S^*$
and $P_{\varepsilon}\sqrt{H_{1}(I-H_{1})^{-1}}$ are bounded
operators. Thus there exists $\delta_{1} > 0$ such that if
$|a-a_{0}| < \delta_{1}$, 
\begin{align*}
\| \sqrt{K_a(I-K_a)^{-1}}U_{a}P_{\epsilon} -
\sqrt{K_{a_0}(I-K_{a_{0}})^{-1}}U_{a_0}P_{\epsilon}  \| = \|(a - a_0)SP_{\epsilon}  \| \leq
\epsilon, \\
 \|P_{\epsilon}U_{a}^*\sqrt{K_a(I-K_a)^{-1}} -
P_{\epsilon}U_{a_0}^*\sqrt{K_{a_0}(I-K_{a_{0}})^{-1}} \| = \|(a - a_0)P_{\epsilon}S^{*}  \| \leq
\epsilon.
\end{align*}
Note that 
\begin{align*}
&P_{\epsilon}U_a^*K_{a}U_aP_{\varepsilon}
-P_{\epsilon}U_{a_0}^*K_{a_0}U_{a_0}P_{\varepsilon}\\
& \qquad =P_{\epsilon}U_a^*\sqrt{K_{a}(I-K_{a})^{-1}}(1-K_{a})\sqrt{K_{a}(I-K_{a})^{-1}}U_aP_{\varepsilon}\\
& \qquad - P_{\epsilon}U_{a_0}^*\sqrt{K_{a_0}(I-K_{a_0})^{-1}}(1-K_{a_0})\sqrt{K_{a_0}(I-K_{a_0})^{-1}}U_{a_0}P_{\varepsilon}.
\end{align*}
By the same argument as in the proof of (7), we can obtain (9).

\vspace{2mm}

Next we show that if $a \rightarrow \infty$, then
$\|Q_{a} - P_1 \|_{2} \rightarrow 0$. Since $\|Q_{a} - P_1
\|_{2}^{2} = 1 - tr(K_a)$, we only need to show that when $a
\rightarrow \infty$, $tr(I-K_a) \rightarrow 0$.

Let $F(a)$ be as before. Since $SS^*$ is invertible, for any
$\epsilon > 0$, we can choose a projection $E \in \MMM$ such that
$tr(E) > 1 - \epsilon^2$, $ESS^{*}E \geq \beta E > 0$, $ESS^{*}E$,
$ES\sqrt{H_{1}(I-H_{1})^{-1}}E$, $EH_{1}(I-H_{1})^{-1}E$ are all
bounded. Thus there exists a constant $c > 0$ such that if $|a|
> c$, we have
\begin{align*}
ESS^{*}E &+ \frac{1}{\overline{a}}ES\sqrt{H_{1}(I-H_{1})^{-1}}E \\
& + \frac{1}{a}E\sqrt{H_{1}(I-H_{1})^{-1}}S^{*}E +
\frac{1}{|a|^2}EH_{1}(I-H_{1})^{-1}E > \frac{\beta}{2}E.
\end{align*}
Hence $EF(a)E>|a|^{2}\frac{\beta}{2}E$. By \cite[Lemma 3.2]{BV}, if
we let $e$ denote the  the spectral projection of $F(a)$ on
$[|a|^{2}\frac{\beta}{2}, +\infty)$, then $tr(e)\geq tr(E)> 1-
\epsilon^2$ and $e F(a)e \geq \frac{|a|^{2}\beta}{2}e$. Choose $|a|$
large enough, we will have
\begin{align*}
tr(I-K_a) &= tr((I + F(a))^{-1}e) + tr((I + F(a))^{-1}(1-e)) \\
          &\leq \frac{2}{|a|^2\beta} + \epsilon \leq 2\epsilon.
\end{align*}
This implies when $a \rightarrow \infty$, $tr(I-K_a) \rightarrow 0$.
\vspace{2mm}

Let us now pause to recall the definition of  Kadison-Singer algebras and Kadison-Singer lattices:

\begin{df}
A subalgebra $\AAA$ of $B(\HHH)$ is called a Kadison-Singer algebra (or
KS-algebra), if $\AAA$ is reflexive and maximal with respect
to the diagonal subalgebra $\AAA \cap \AAA^*$ of
$\AAA$, in the sense that if there is another reflexive
subalgebra $\mathcal{B}$ of $B(\HHH)$ such that
$\AAA \subseteq\mathcal{B}$ and
$\mathcal{B}\cap\mathcal{B}^*=\AAA \cap\AAA^*$, then
$\AAA=\mathcal{B}$. A lattice $\L$ of projections in $B(\HHH)$ is
called a Kadison-Singer lattice (or KS-lattice) if $\L$ is a minimal
reflexive lattice that generates the von Neumann algebra $\L''$, or
equivalently, $\L$ is reflexive and $Alg\L$ is a Kadison-Singer
algebra.
\end{df}

Next we will show that the lattice given in Proposition 2.1 is a KS-lattice in general.

\begin{lemma}
For any three distinct projections $Q_1, Q_2$ and $Q_3$ in
$LatAlg\LLL \setminus \{0, I \}$, we have $P_{1} \in LatAlg\{
Q_{1}, Q_{2}, Q_{3}\}$. Hence $LatAlg\{Q_{1}, Q_{2},
Q_{3}\}=LatAlg\L.$
\end{lemma}

\noindent{\it Proof}\quad
We may assume that $Q_i\in LatAlg\LLL
\setminus \{0, I, P_1 \}$, $i = 1,2,3$. By Proposition 2.1,
\begin{align*}
Q_{i}& =\left(
          \begin{array}{cc}
            K_{i} & \sqrt{K_{i}(I-K_{i})}U_{i} \\
            U_{i}^{*}\sqrt{K_{i}(I-K_{i})} & U_{i}^{*}(I-K_{i})U_{i} \\
          \end{array}
        \right),
\end{align*}
where $K_i$ and $U_i$ are determined by the polar decomposition
$a_{i}S \widehat{+}
\sqrt{H_{1}(I-H_{1})^{-1}}=\sqrt{K_{i}(I-K_{i})^{-1}} U_{i}$ for
some $a_{i} \in \C$.

To prove the lemma, we only need to show that if
\begin{align*}
A =\left(
     \begin{array}{cc}
       A_{11} & A_{12} \\
       A_{21} & A_{22} \\
     \end{array}
   \right)\in
 Alg\{Q_{1}, Q_{2}, Q_{3}\} \mbox{£¨}
\end{align*}
then $A_{21} = 0$. Since $(I-Q_{i})AQ_{i} = 0$,
$i=1,2,3$, by the similar argument as in the proof of  Lemma 2.3, we have
\begin{align*}
\sqrt{I-K_i}( A_{11}\sqrt{K_i} +A_{12}U_i^*\sqrt{I-K_i})=\sqrt{K_i}U_i(A_{21}\sqrt{K_i}+A_{22}U_i^*\sqrt{I-K_i}).
\end{align*}
Since the set
$\mathfrak{D} \overset{\text{def}}{=} (\cap_{i=1}^{3}\DDD(\sqrt{(I-K_i)^{-1}}U_i)\cap\DDD(S)\cap\DDD(\sqrt{H_{1}(I-H_{1})^{-1}})$ is dense in $\HHH$, from the
invertibility of $I-K_i$, this implies, for each $\xi \in \mathfrak{D}$,
\begin{align*}
A_{11}\sqrt{K_{i}(I-K_{i})^{-1}}U_{i}\xi + A_{12}\xi = 
\sqrt{K_{i}(I-K_{i})^{-1}}U_{i}[A_{21}\sqrt{K_{i}(I-K_{i})^{-1}}U_{i}\xi
+ A_{22}\xi].
\end{align*}
Let $\{ E_\epsilon \}_{\epsilon}$ be a monotone increasing family of projections in $\MMM$ with strong-operator limit $I$ as $\epsilon \rightarrow 0$, also for each $\epsilon$, $E_\epsilon S$, $E_\epsilon
\sqrt{H_{1}(I-H_{1})^{-1}}$ are bounded operators. Then
\begin{align*}
E_\epsilon \sqrt{K_{i}(I-K_{i})^{-1}}U_i = a_{i}E_\epsilon S +
E_\epsilon \sqrt{H_{1}(I-H_{1})^{-1}}, \qquad i=1,2,3
\end{align*}
are bounded operators. Thus for $i$, $j \in \{1, 2, 3\}$, we have
\begin{align*}
(a_{i} -a_{j})[E_{\varepsilon}A_{11} &S\xi -
E_{\varepsilon}SA_{22}\xi]= \\
&E_\epsilon\sqrt{K_{i}(I  -K_{i})^{-1}}
U_{i}A_{21} \sqrt{K_{i}(I-K_{i})^{-1}}U_{i}\xi \\
&-E_{\epsilon}\sqrt{K_{j}(I-K_{j})^{-1}}
U_{j}A_{21}\sqrt{K_{j}(I-K_{j})^{-1}}U_{j}\xi.
\end{align*}
Substitute $a_{i}E_\epsilon S + E_\epsilon \sqrt{H_{1}(I-H_{1})^{-1}}$ into the above equation for $E_\epsilon \sqrt{K_{i}(I-K_{i})^{-1}}U_i$, a simple calculation will give the following relation,
\begin{align*} 
(a_{i} -a_{j})[E_{\varepsilon} &A_{11}S\xi -
E_{\varepsilon}SA_{22}\xi]\\
&= (a_i^2-a_j^2)E_{\epsilon}SA_{21}S\xi+(a_i-a_j)E_{\epsilon}SA_{21}\sqrt{H_{1}(I-H_{1})^{-1}}\xi\\
&+ (a_i-a_j)E_{\epsilon}\sqrt{H_{1}(I-H_{1})^{-1}}A_{21}S\xi].
\end{align*}
Divide both sides by $a_i - a_j$, we deduce that $(a_1+a_2)E_{\epsilon}SA_{21}S\xi = (a_1+a_3)E_{\epsilon}SA_{21}S\xi$. Since $a_2\neq a_3$, $E_{\epsilon}SA_{21}S\xi=0$. Let
$\epsilon$ tend to $0$, and note $Ker(S) = 0$,
$\{S\xi:\,\,\xi\in \mathfrak{D}\}$ is dense in $\HHH$, we have $A_{21} = 0$.

By Corollary 2.2, we have
$LatAlg\L=LatAlg\{P_1, Q_1, Q_2\}\subseteq LatAlg\{Q_1, Q_2,
Q_3\}\subseteq LatAlg\L$, which yields $LatAlg\{Q_1, Q_2,
Q_3\}= LatAlg\L$.

\vspace{2mm}

\begin{prop}
 $LatAlg\LLL$ is  a Kadison-Singer
 lattice if $\LLL''$ can not be generated by two nontrivial
 projections.
\end{prop}

 \noindent{\it Proof}\quad Suppose that $\mathcal{F}$ is a reflexive lattice in
 $M_2(\C)\otimes B(\HHH)$ such that $\mathcal{F}\subseteq LatAlg\L$ and $\mathcal{F}''=(LatAlg\L)''(=\LLL'')$.
 By assumption, there are at least three nontrivial projections
 $Q_1$, $Q_2$ and $Q_3$ in $\mathcal{F}$. Then $LatAlg\{Q_1,Q_2,Q_3\}\subseteq\mathcal{F}\subseteq
 LatAlg\L$. By Lemma 2.5, we have $\mathcal{F}=LatAlg\L$. Hence
 $LatAlg\LLL$ is  a Kadison-Singer lattice.
\vspace{2mm}

\begin{remark} The condition on $\LLL''$ can not be dropped. Let $P_1=\left(
     \begin{array}{cc}
     I & 0 \\
       0 & 0 \\
     \end{array}
   \right)$, $P_2=\left(
     \begin{array}{cc}
     0 & 0 \\
       0 & I \\
     \end{array}
   \right)$ and $P_3=\left(
     \begin{array}{cc}
     \frac I2 &\frac I2 \\
      \frac I2 & \frac I2 \\
     \end{array}
   \right)$. Then $P_i\wedge P_j=0$ and $P_i\vee P_j=I$,
   and $\{P_1,P_2,P_3\}''$ can be generated by two nontrivial
   projections. Particularly, $\LLL_1=\{0, I,\left(
     \begin{array}{cc}
     I & 0 \\
       0 & 0 \\
     \end{array}
   \right), \left(
     \begin{array}{cc}
     \frac I2 &\frac I2 \\
      \frac I2 & \frac I2 \\
     \end{array}
   \right)\}$ is a reflexive sublattice of $\LLL$ that generates the same von Neumann
   algebra as $\LLL$, thus $LatAlg\LLL$ is not  a Kadison-Singer
 lattice
\end{remark}

\begin{prop}
Let $Q_1$, $Q_2$ and $Q_3$ be any three projections acting on a
separable Hilbert space $\HHH$ such that $Q_i \wedge Q_j = 0$ and
$Q_i \vee Q_j = I$ for $i \neq j$, $i,j=1,2,3$. If the
von Neumann algebra $\AAA$ generated by these three projections is finite. Then $Q_1$ is equivalent to $I-Q_1$ in
$\AAA$, and $\AAA$ is *-isomorphic to $Q_1 \AAA Q_1 \otimes
M_2(\C)$. 
\end{prop}

\noindent{\it Proof}\quad Let $\tau$ be a faithful  normal tracial
state  on $\AAA$. By Kaplansky Formula,  we have $\tau(Q_i) =
\frac{1}{2}$. Let $\WWW =\AAA
* M_{2}(\C)$ be the reduced (von Neumann algebra) free product of
$\AAA$ with $M_2(\C)$. Since there exists a trace half projection which is free with $Q_1$, $Q_1$ is equivalent to $I-Q_1$ in $\WWW$ \cite{GY2}.
Therefore, we may choose a matrix units $\{ E_{ij} \}_{i,j = 1}^{2}$ in
$\WWW$ such that $\WWW$ is $\ast$-isomorphic to $Q_1\WWW Q_1 \otimes
M_2(\C)$ and $E_{11} = Q_1$..


By Lemma 2.1,
\begin{align*}
&Q_2 = \left(\begin{array}{cc}H_1 & \sqrt{H_1 (I-H_1)}V_1 \\V_1^*
\sqrt{H_1 (I-H_1)} & V_1^{*}(I - H_1)V_1\end{array}\right),\\
&Q_3 = \left(\begin{array}{cc}H_2 & \sqrt{H_2(I-H_2)}V_2 \\V_2^*
\sqrt{H_2(I-H_2)} & V_2^{*}(I - H_2)V_2\end{array}\right),
\end{align*}
where $V_1$, $V_2$ are unitary operators in $\MMM$, $H_1$ and $H_2$
are two contractive positive operators in $\MMM$ such that
$Ker(I-H_i) =0$ for $i = 1,2$.

Note both
\begin{align*} \left(
  \begin{array}{cc}
    H_{i-1} &  0\\
    0 & 0 \\
  \end{array}
\right)=Q_1Q_iQ_1 \mbox{ and }
\left(
  \begin{array}{cc}
    0 &  \sqrt{H_{i-1}(I-H_{i-1})}V_{i-1}\\
    0 & 0 \\
  \end{array} 
\right)=Q_1Q_i(I-Q_1)
\end{align*}
are in $\AAA$. Hence
\begin{align*}
T = \left(
  \begin{array}{cc}
    0 &  \sqrt{H_1(I-H_1)^{-1}}V_1 \widehat{-}\sqrt{H_2(I-H_2)^{-1}}V_2)\\
    0 & 0 \\
  \end{array}
\right)
\end{align*}
is affiliated with $\AAA$. By Lemma 2.2,
$Ker(\sqrt{H_1(I-H_1)^{-1}}V_1-\sqrt{H_2(I-H_2)^{-1}}V_2) = 0$,
let $HU$ be the polar decomposition of $T$, then $U\in \AAA$,
and $U^*U = I-Q_1$, $UU^{*} = Q_1$, thus $Q_1$ is equivalent to $I-
Q_1$ in $\AAA$.\vspace{2mm}

We assemble the foregoing results as a theorem:

\begin{theorem} Let $P_1$, $P_2$ and $P_3$ be three projections acting on a separable Hilbert space $\HHH$ such that
$P_i \wedge P_j = 0$ and $P_i \vee P_j = I$ for $i \neq j$. If the von Neumann algebra $\AAA$ generated by $P_1, P_2$ and
$P_3$ is finite. Then
$LatAlg\{P_1,P_2,P_3\}\setminus \{0, I \}$ is homeomorphic to
$S^{2}$; in addition, each nontrivial projection in
$LatAlg\{P_1,P_2,P_3\}$ has trace $\frac 12$, and the reflexive
lattice can be determined by arbitrary three nontrivial
projections in it.

Furthermore, if the von Neumann algebra $\AAA$ can not be generated
by two nontrivial projections, then $LatAlg\{0,I,P_1,P_2,P_3\}$ is a
KS-lattice.
\end{theorem}

With the notation in the above theorem, let $\LLL = LatAlg\{P_1,P_2,P_3\}$. By fixing any three projections in $\LLL \setminus  \{0,I\} $, we essentially give 
a coordinate chart of  $\LLL \setminus  \{0,I\} $ (exclude one point corresponding to $\infty$) by Proposition 2.1. It can be showed
that the transition map is M\"{o}bius  transformation. We will address this issue in our forthcoming paper. 
As for now, we will determine all the "connected" reflexive lattices acting on a finite-dimensional space.


\section{Connected reflexive lattices in $M_n(\C)$}

By Theorem 2.1, we know that every nontrivial projection in a
reflexive lattice determined by a double triangle lattice in
$M_n(\C)$ has trace $\frac12$, and this reflexive
lattice is homeomorphic to $S^2$ (plus two distinct points
corresponding to $0$ and $I$). In this section, we will show that if
every nontrivial projection of a reflexive lattice $\mathcal{F}$ in $M_n(\C)$, $n\geq 2$ 
has trace $\frac12$, and $\FFF$ has more than 3 projections, then
$\mathcal{F}\setminus\{0,I\}$ is  homeomorphic to $S^2$. 
Moreover if $n \geq 3$ and $\FFF'' = M_n(\C)$, $\mathcal{F}$ is also a Kadison-Singer lattice.

\begin{prop}
 Let $\FFF$ be a reflexive lattice of projections
in $M_n(\C)$, $n\geq 2$. If $\FFF \setminus \{0, 1 \}$ has only
one connected component under the $\|\,\|_2$-topology, then
$\FFF=\{0,P,I\}$ for some nontrivial projection $P$, or, $\FFF$ is
determined by three projections $P_1$, $P_2$, $P_3$ such that $P_i
\wedge P_j = 0$, and $P_i \vee P_j = I$, $i \neq j$. In particular,
if the latter case occurs, $n$ must be even.
\end{prop}

\noindent{\it Proof}\quad  Suppose $\FFF$ contains at least two
nontrivial projections. Let $\tau$ be the normalized trace on $M_{n}(\C)$. Since  $\FFF \setminus \{0, 1 \}$ has only
one connected component, the range of $\tau$: $\FFF \setminus \{0, 1 \} \rightarrow \Q$ can only contains one point.
This implies $P\vee Q=I$, $P\wedge Q=0$ and $\tau(P)=\tau(Q)=\frac12$, for any two distinct projections in $\FFF \setminus \{0, 1 \}$, thus $n$ must be even. Let $n=2k$, $M_n(\C) \cong M_2(\C)\otimes M_k(\C)$.

Next we show that $\FFF$ is determined by three nontrivial
projections. Note if $\FFF$ contains at least two nontrivial projections, then $\FFF$ must contains infinite many projections, otherwise $\FFF \setminus \{0,I\}$ has more than one connected component. 

Fix $P_1,P_2,P_3$ in $\FFF\setminus\{0,I\}$, and let
$\L=\{0,P_1,P_2,P_3,I\}$. Clearly, $LatAlg\LLL\subseteq\FFF$.
If there is a projection $P_4$
in $\FFF \setminus LatAlg\LLL$, we
know $P_i\wedge P_j=0$ and $P_i\vee P_j=I$,
for $i,j=1,2,3,4$ and $i\neq j$ by the argument above. Up to a unitary equivalence, we could assume $P_1=\left(\begin{array}{cc}  I_k & 0\\
0 &0\end{array}\right)$. Then by Lemma 2.1,
\begin{align*}
P_j=\left(\begin{array}{cc} H_{j-1} &
\sqrt{H_{j-1}(I-H_{j-1})^{-1}}V_{j-1}\\
V_{j-1}^{*}\sqrt{H_{j-1}(I-H_{j-1})^{-1}} &
V_{j-1}^{*}(I-H_{j-1})V_{j-1}\end{array}\right), 
\qquad j=2,3,4,
\end{align*}
where $H_j$ is a semi-positive definite matrix in
$M_k(\C)$ such that $I-H_j$ is positive definite (hence
invertible), $V_j$ is a unitary matrix in $M_k(\C)$. Without loss of generality, we may assume $V_1=I$.

Let $\L'=\{0,P_1,P_2,P_4,I\}$, we have $ Alg\{0,P_1,P_2,P_3,P_4,I\}=
Alg\L\cap Alg\L'$. By Lemma 2.4, 
\begin{align*}
 Alg\{0,&P_1,P_2,P_3,P_4,I\}\\ 
 &=\left\{\left(\begin{array}{cc} A & \sqrt{H_1(I-H_1)^{-1}}S_1^{-1}AS_1-A\sqrt{H_1(I-H_1)^{-1}}\\
0 & S_1^{-1}AS_1\end{array}\right):\, \begin{array}{l} A\in M_k(\C)\\
A\widetilde{S}=\widetilde{S}A\end{array}\right\},
\end{align*}
here
$\widetilde{S}=S_2S_1^{-1}$,
$S_1=\sqrt{H_1(I-H_1)^{-1}}-\sqrt{H_2(I-H_2)^{-1}}V_2$ and 
$S_2=\sqrt{H_1(I-H_1)^{-1}}-\sqrt{H_3(I-H_3)^{-1}}V_3$ are all
invertible matrices in $M_k(\C)$. If $k = 1$, $\widetilde{S}$, $S_1$ and $S_2$ are all scalars, so it is easy to see that $Alg\L=Alg\{P_1,P_2,P_3,P_4\}= Alg\L'= \C I$, thus we finish the proof in the case $k = 1$.

For $k > 1$, we first remark that $\widetilde{S} \neq \lambda I$, $\lambda \in \C$. Otherwise
$S_2 = \lambda S_1$ and $S_2^{-1}AS_2=S_1^{-1}AS_1$ for any $A\in M_k(\C)$,  we have $ Alg\L= Alg\L'$ by Lemma 2.4, thus $P_4$ is in $LatAlg\L$, which is impossible by assumption.

Let $\lambda$ be an eigenvalue of $\widetilde{S}$, $q$ be the
orthogonal projection onto the eigen space of $\lambda$. For each $A$ in
$M_k(\C)$ with $A\widetilde{S}=\widetilde{S}A$, $(I-q)Aq = 0$, therefore $(I-Q)TQ = 0$ for any $T \in Alg\{0,P_1,P_2,P_3,P_4,I\}$, where $Q=\left(\begin{array}{cc}
q&0\\0&0\end{array}\right)$. This means $Q$ is in $LatAlg\{0,P_1,P_2,P_3,P_4,I\}\subset \FFF$. Since $\widetilde{S} \neq \lambda I$, we have $0 < q < I$ and $0<\tau(Q)<\frac 12$. Hence $Q$ is a projection with trace less than $\frac12$ in $\FFF$, this contradicts the fact that $\FFF \setminus \{0, I\}$ only has one connect component. Consequently, $\FFF$ is determined by three nontrivial projections
$P_j$ such that $P_i\wedge P_j=0$ and $P_i\vee P_j=I$ for $i\neq j$.

\vspace{2mm}

By Proposition 3.1 and Theorem 2.1, we have the following corollary.

\begin{corollary}
 Let $\FFF$ be a reflexive lattice of projections in $M_n(\C)$, $n \geq 2$. Suppose there exist at least two
nontrivial projections in $\FFF$. Then the following statements are
equivalent:
\begin{enumerate}
\item[(i)] $\FFF \setminus \{0, 1 \}$ has only one connected component;
\item[(ii)] $\FFF$ is determined by three projections $P_1$,
$P_2$, $P_3$ such that $P_i \wedge P_j = 0$, and $P_i \vee P_j = I$,
$i \neq j$;
\item[(iii)] $\FFF \setminus \{0, 1 \}$ is  homeomorphic to
$S^2$;

\noindent In addition, if any of these conditions is satisfied, then $n$ is
even, and  for every $P\in\FFF\setminus\{0,I\}$, $\tau(P)=\frac12$.
Conversely, if $\FFF$ contains at least three nontrivial
projections, then above conditions are equivalent to:

\item[(iv)] For every projection $P\in\FFF\setminus \{0,I\}$, $\tau(P)=\frac 12$, where 
$\tau$ is the normalized trace on $M_n(\C)$.
\end{enumerate}
\end{corollary}

\noindent{\it Proof}\quad By Proposition 3.1, we have (i) implies
(ii); by Theorem 2.1, (ii) implies (iii). It is obvious that  (iii)
implies (i). If any of these equivalent conditions hold, by Theorem
2.1, we have $\tau(P)=\frac 12$ for each nontrivial projection $P$
in $\FFF$, hence $n$ is even.

Suppose $\FFF$ contains at least three nontrivial projections
and each nontrivial projection $P$ in $\FFF$ has trace $\frac 12$.
Then $n$ is even, and for each pair of distinct nontrivial $P$ and $Q$,
$P\wedge Q=0$ and $P\vee Q=I$. Let $P_1$, $P_2$ and $P_3$ be three distinct projections in $\FFF \setminus \{0, I\}$,  exactly as in the proof of Proposition 3.1, we can show $LatAlg\{P_1, P_2, P_3\} = \FFF$, hence (iv) implies (ii).

\begin{remark} Let $\FFF=\{0,\left(\begin{array}{cc} I_k & 0\\
0&0\end{array}\right), \left(\begin{array}{cc} \frac12 I_k & \frac12 I_k\\
\frac12 I_k&\frac12 I_k\end{array}\right), I\}$. Then $\FFF$ is a
reflexive lattice of $M_2(\C)\otimes M_k(\C)$ such that for each
nontrivial projection $P$ in $\FFF$, $\tau(P)=\frac 12$. Obviously,
$\LLL \setminus \{0, 1 \}$ has two connected component.
\end{remark}

\begin{corollary} Let $\FFF$ be a reflexive lattice in $M_n(\C)$, such that $\FFF'' = M_n(\C)$, $n\geq 3$. If for each nontrivial projection $P$ in $\FFF$, $\tau(P)=\frac 12$. Then
$\FFF\setminus\{0,I\}$ is homeomorphic to $S^2$. Moreover, $\FFF$ is a Kadison-Singer lattice, the Kadison-Singer algebra $Alg\FFF$ has dimension
$\frac{n^2}{4}$.
\end{corollary}

\noindent{\it Proof}\quad Obviously, for $n >2$, $M_n(\C)$ can not be generated by
two nontrivial projections, thus $\FFF$ satisfies the
conditions in Corollary 3.1. Hence $\FFF\setminus\{0,I\}$ is
homeomorphic to $S^2$. Let
$n=2k$, by Lemma 2.4, $Alg\FFF$ is isomorphic to $M_k(\C)$ which has dimension $\frac{n^2}{4}$. 

\begin{remark} For $n=2k>2$, the reflexive lattice of projections in $M_n(\C)$ satisfying the conditions in Corollary 3.2 exists. It is well-known that there exist two semi-positive definite matrices
$H_1$ and $H_2$ such that $I-H_i$ is positive definite, $i=1,2$,
$\sqrt{H_{1}(I-H_{1})^{-1}}-\sqrt{H_{2}(I-H_{2})^{-1}}$ is
invertible, and $\{H_1,H_2\}$ generates $M_k(\C)$. Let $\LLL$ be the lattice generated by $\left(\begin{array}{cc} I_k & 0\\
0&0\end{array}\right)$, $\left(\begin{array}{cc} H_1 & \sqrt{H_1
(I-H_1)} \\ \sqrt{H_1 (I-H_1)} & I - H_1\end{array}\right)$ and $
\left(\begin{array}{cc} H_2 & \sqrt{H_2(I-H_2)} \\
\sqrt{H_2(I-H_2)} & I - H_2\end{array}\right)$.  By Theorem
2.1, $\FFF=LatAlg\LLL$ is a KS-Lattice that generates $M_n(\C)$. 
\end{remark}

\begin{prop}${\mbox{\cite{Hou,WY}}}$ For $n\geq 2$, let
$E_{ij},\,\, i,j=1,2,\cdots,n$, be the standard matrix unit. Let
$P_i=\sum\limits_{j=1}^{i}E_{ii}$ for $i=1,2,\cdots, n-1$, and let
$Q= \frac 1n\sum\limits_{i,j=1}^nE_{ij}$. Then the lattice $\FFF$
algebraically generated by  $P_1,\cdots,P_{n-1}$ and $Q$ is a
Kadison-Singer lattice of $M_n(\C)$, and the corresponding Kadiosn-Singer
algebra has dimension $1+\frac{n(n-1)}{2}$.
\end{prop}

\begin{remark} For $n=2$, $3$, it is 
proved that for each Kadison-Singer lattice $\L$ of $M_n(\C)$, $\L$
or $I-\L$ is similar to the one given in Proposition 3.2 \cite{Tan}. Hence
all the Kadison-Singer algebras of $M_2(\C)$ or $M_3(\C)$  with
trivial diagonal has dimension $2$ or $4$. For general $n$, we have the following
conjecture.

\vspace{2mm}

\noindent{\bf Conjecture 3.1.}\quad Let $\AAA$  be a
Kadison-Singer subalgebra of $M_n(\C)$ with trivial diagonal, then
$\frac{n^2}{4}\leq dim(\mathcal{A})\leq 1+\frac{n(n-1)}{2}$.
\end{remark}


\section*{\bf Acknowledgments}\quad  Research was supported
in part by the National Natural Science Foundation of China (Grant
No. 10971117), Natural Science Foundation of Shandong Province
(Grant No. Y2006A03, ZR2009AQ005) and Morningside Mathematic Center.

 \vspace{3mm}


\begin{thebibliography}{00}
\bibitem{BV} Bercovici H., Voiculescu, D.: Free convolution of measures with unbounded
support. Indiana Univ. Math. J. 42 (1993), 733-773.


\bibitem{VDN} D. Voiculescu, K. Dykema and A. Nica: Free
Random Variables, CRM Monograph Series, vol. 1, 1992.

\bibitem{Da} Davidson, K. R.:
Nest Algebras. Longman Scientific \& Technical, {\bf $\pi$} Pitman
Research Notes in Mathematics Series, {No. 191}. New York (1988).

\bibitem{MV} F. J. Murray and J. von Neumann, On rings of operators, II, Trans. Amer. Math.
Soc. 41 (1937), 208-248.

\bibitem{GS} Ge, L., Shen, J. H.: On the generator problem of von Neumann algebras. Proc of ICCM 2004. Hong
Kong Intern. Press, Boston.

\bibitem{GY1} Ge, L., Yuan, Y.: Kadison-Singer algebras,
I: hyperfinite case. Proc Natl Acad Sci USA 107(5), 1838-1843(2010).

\bibitem{GY2} Ge, L., Yuan, Y.: Kadison-Singer algebras,
II: General case. Proc Natl Acad Sci USA 107(11), 4840-4844(2010).

\bibitem{Ha} Halmos, P.: Reflexive lattices of subspaces. Journal of London Math. Society 4,
257-263(1971).

\bibitem{Hou} Hou, C. J.: Cohomology of a class of Kadison-Singer algebras. Science  China series Mathematics
53(7), 1827-1839(2010).

\bibitem{KR} Kadison, R., Ringrose, J.: Fundamentals of the
Operator Algebras. vols. I and II. Academic Press, Orlando (1983 and
1986).

\bibitem{KS} Kadison R., Singer, I.: Triangular operator algebras. Fundamentals and hyper-reducible theory. Amer Journal of Math. (82), 227-259(1960).


\bibitem{Ri} Ringrose, J.: On some algebras of operators, II. Proc. London
Math. Soc. 16(3), 385--402(1966).

\bibitem{RR} Radjavi H, Rosenthal P.: Invariant Subspaces, Springer-Verlag, Berlin (1973)

\bibitem{Tan} Tan, J.: Classification on Kadison-Singer Algebras. Graduation Thesis, Academy of Mathematics and Systems Science, CAS (2010).

\bibitem{WY} Wang, L., Yuan, Y.: A new class of Kadison-Singer algebras. Expositiones Mathematicae (2010). Doi: 10.1016
/j.exmath. 2010.08.001.

\bibitem{Zhe} Liu, Z. :On Some Mathematical Aspects of The Heisenberg Relation. Science  China series Mathematics: Kadison's proceedings.
\end{thebibliography}



\end{document}
