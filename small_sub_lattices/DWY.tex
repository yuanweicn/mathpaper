\documentclass{jaums}

\usepackage{mathrsfs}
\usepackage{amssymb}
\usepackage{graphicx}
\usepackage{amsmath}

% Theorem environments

\theoremstyle{thmit} % Numbered and Italic
\newtheorem{theorem}{Theorem}[section]
\newtheorem{lemma}[theorem]{Lemma}
\newtheorem{corollary}[theorem]{Corollary}
\newtheorem{proposition}[theorem]{Proposition}
\newtheorem{definition}[theorem]{Definition}


\theoremstyle{thmrm} % Numbered and Roman
\newtheorem{example}{Example}
\newtheorem{remark}{Remark}
\newtheorem*{oldproof}{Proof}
\renewenvironment{proof}[1][{}]{\begin{oldproof}[#1]}{\qed\end{oldproof}}

% Some standard inputs

\title{On Small Subspace Lattices in Hilbert Space}
\author{Aiju Dong}
\address{College of Mathematics and Computer Engineering\\
Xi'an University of Arts and Science\\
Xi'an, 710065, China\\
E-mail: daj1965@163.com}

\author{Wenming Wu}
\address{College of Mathematics\\
Chongqing Normal University\\
Chongqing, 400047, China\\
E-mail: wuwm@amss.ac.cn}

\author{Wei Yuan}
\address{Academy of Mathematics and Systems Science\\
Chinese Academy of Science\\
Beijing, 100190, China\\
E-mail:  wyuan@math.ac.cn}

\keywords{Projections; Lattice; Double triangle; Reflexive; Transitive}

\thanks{
2010 MR Subject Classification: 47A62, 47L75.\\
This work was partially supported
by Xi'an Scientific Program CXY1134WL03, NSF of China
(No.11271390) and Natural Science Foundation Project of CQ CSTC
(No.2010BB9318).\\
Wenming Wu: The corresponding author.
}
\begin{document}

\maketitle

\begin{abstract}
Reflexivity and transitivity of a double triangle lattice of subspaces in a
Hilbert space are studied. We show that the double triangle lattice is neither
reflexive nor transitive when some invertibility condition is satisfied (by the
restriction of a projection under another). In this case, we show that the
reflexive lattice determined by the double triangle lattice contains infinitely
many projections which partially answers a problem of Halmos on small lattices
of subspaces in Hilbert spaces.
\end{abstract}

\section{Introduction}

In {\cite{[Hal]}}, Paul Halmos asked ten problems in operator theory, two of
which were concerned with small subspace lattices in Hilbert spaces. The tenth
problem asks whether every non-trivial strongly closed transitive atomic lattice
is either self-conjugate or medial, i.e. the intersection and union of any two
nontrivial projections is $0$ and $I$ respectively. Let $\mathcal H$ be a
Hilbert space and
$\mathcal B(\mathcal H)$ the algebra of all bounded linear operators on
$\mathcal H$. A subalgebra $\mathcal{A}\subset\mathcal{B}(\mathcal{H})$ is
called {\sl transitive} if the invariant subspace lattice 
$Lat(\mathcal{A})=\{P:(I-P)TP=0,\ \forall T\in\mathcal{A}\}$ of $\mathcal{A}$ 
is
$\{0,I\}$. A subspace lattice $\mathcal{L}$ is called {\sl transitive} if the
algebra
$Alg(\mathcal{L})=\{T\in\mathcal{B}(\mathcal{H}):(I-P)TP=0,\forall
P\in\mathcal{L}\}$ associated to $\mathcal{L}$ is equal to $\mathbb{C}I$ (the
algebra
consisting only scalar multiples of the identity operator $I$). In this paper,
we shall assume that all subspaces of a Hilbert space are closed and they are
represented by the orthogonal projections onto them; and that every subspace
lattice contains 0 and $I$.

It is easy to check that any subspace lattice with only two nontrivial elements
is not transitive. Halmos {\cite{[Hal]}} gave an example of a transitive lattice
of subspaces with only five nontrivial elements; and K. Harrison, H. Radjavi
and P. Rosenthal {\cite{[HRR]}} constructed an example with only four nontrivial
projections. The existence of a transitive subspace lattice with only three
nontrivial projections is still unknown up to date.  Partial results are
obtained by Hadwin, Longstaff and Rosenthal {\cite{[HLP]}}, K. Harrison
{\cite{[Har]}} and Ge and Yuan {\cite{[GY],[GY2]}}.

The ninth problem in {\cite{[Hal]}} asks whether every complete Boolean algebra
given by subspaces of a Hilbert space is reflexive. A subalgebra $\mathcal{A}$
of $\mathcal{B}(\mathcal{H})$
is called {\sl reflexive} if $Alg(Lat(\mathcal{A}))=\mathcal{A}$ and a subspace
lattice $\mathcal{L}$ is called {\sl reflexive} when
$Lat(Alg(\mathcal{L}))=\mathcal{L}$.
The reflexive algebras are in the central role of the non selfadjoint operator
algebras which are closely related to operator theory
and invariant subspaces of operators. Parallel to the theory of selfadjoint
operator algebras (C$^*$-algebras and von Neumann algebras), many important
results in non selfadjoint algebras were obtained in the past 50 years by many
mathematicians, e.g.,
Arveson {\cite{[Ar]}}, Larson {\cite{[La]}}, Davidson {\cite{[Da]}} and Lance
{\cite{[Lan]}}.

It is not easy to determine the reflexivity of a given subspace lattice or a
given subalgebra. Halmos {\cite{[Ha2]}} has shown that any atomic complete
Boolean algebra is reflexive.
Furthermore Harrison {\cite{[Har]}} has shown that a finite distributive
lattice is always reflexive.
Thus, if a lattice has two or fewer nontrivial elements, it is reflexive.
There are just two non-isomorphism types of
non-distributive subspace lattices with only 3 nontrivial elements: the pentagon
and the double triangle.
These two classes are the most interesting small invariant subspace lattices to
study.



Recently, L. Ge and W. Yuan  {\cite{[GY],[GY2]}} studied certain maximal
triangularity of reflexive algebras and discovered a large classes of
interesting reflexive algebras and lattices. Those lattices assume some nice
topological structures. For example, they show that the reflexive lattices
generated by many double triangle lattices are homeomorphic to the classical
two-dimensional sphere. This study initiated a new class of operator algebras
which they call "Kadison-Singer algebras" or "KS-algebras" for short
(correspondingly "Kadison-Singer lattices" or "KS-lattices"). Several people
followed their study and obtain many interesting reflexive algebras and lattices
(see, e.g. {\cite{[HY], [WY], [DH]} }). Although some of the techniques
developed in {\cite{[GY],[GY2]}} can be applied to study double triangle
lattices in infinite factors, the reflexive lattice determined by a double
triangle lattice in $\mathcal{B}(\mathcal{H})$ is in general unknown. The
existence of a transitive double triangle 
lattice would imply such a reflexive lattice is trivial.


In this paper, we shall study the double triangle lattice of projections in
$\mathcal{B}(\mathcal{H})$. When $\mathcal{B}(\mathcal{H})$ is replaced by a
finite von Neumann algebra, two or any finitely many unbounded operators
affiliated with the algebra have a common dense domain. This allows the
construction of many bounded operators which leave the subspaces in the lattice
invariant. For $\mathcal{B}(\mathcal{H})$, we believe some analogs result hold:
Suppose $A$ and $B$ are (unbounded) selfadjoint invertible operators (with
unbounded inverses). Then it is easy to show that the algebra
$\mathcal{A}_A=\{X\in \mathcal{B}(\mathcal{H}): A^{-1}XA \ {\rm is\ bounded}\}$
is a (weak-operator) dense subalgebra of $\mathcal{B}(\mathcal{H})$. Similarly,
we define $\mathcal{A}_B$.
We conjecture that $\mathcal{A}_A\cap \mathcal{A}_B$ is weak-operator dense in
$\mathcal{B}(\mathcal{H})$ when $A$ and $B$ are affiliated with some non-atomic
subalgebras of $\mathcal{B}(\mathcal{H})$ respectively.

The paper is organized as follows. In Section 2 (following
the introduction), some basic results concerning double triangle lattices in
$\mathcal{B}(\mathcal{H})$ are proved.
Section 2 also contains a main result which describes the algebra of operators
that leave all subspaces in a double triangle lattice invariant. In Section 3,
we show that the algebra is not trivial if some invertibility is satisfied by
the projections in a double triangle lattice in $\mathcal{B}(\mathcal{H})$.
Moreover we show that the reflexive lattice determined by the double triangle
lattice is infinite. Furthermore we will prove the conjecture under the
assumption that $A$ and $B$ are affiliated with a finite von Neumann algebra. In
last Section, by using some of our techniques (different from {\cite{[Lo]}}), we
can again reduce Halmos' transitivity problem for small lattices to the case of
double triangle lattices in $\mathcal{B}(\mathcal{H})$.



\section{Preliminary Results on Double Triangle Lattices}

We shall assume that $\mathcal{H}$ is a separable Hilbert space over the field
of the complex numbers. We will use $\xi \otimes \eta$
to denote the rank one
operator $\xi \otimes \eta(\zeta) = \langle \zeta, \xi \rangle \eta$ where
$\xi$, $\eta$ and $\zeta$ are vectors in $\mathcal{H}$.
Let $P,Q,R$ be nontrivial (orthogonal) projections
acting on $\mathcal{H}$ such that
$\mathcal{L}=\{0,I,P,Q,R\}$ forms a double triangle lattice, i.e.,
$P\wedge Q=P\wedge R=Q\wedge R=0$ and $P\vee Q=P\vee R=Q\vee R=I$.


\begin{proposition}
Suppose that $\mathcal{H}$ is infinite dimensional and
$\mathcal{L}=\{0,I,P,Q,R\}$ a double triangle lattice.
Then the ranges of any nontrivial projection in $\mathcal{L}$ and its orthogonal
complement must be infinite dimensional.
\end{proposition}
\begin{proof} On the contrary, suppose that $dim(P(\mathcal{H}))=n<\infty$. As
$P\wedge Q=0$ and $P\vee Q=I$,
then according to Kaplansky formula(see, e.g. \cite{[KR]}), we have $P\vee
Q-Q\thicksim P-P\wedge Q$
which implies
that $dim((I-Q)(\mathcal{H}))\leqslant n<\infty$, where $\thicksim$ denotes the
usual Murray-von Neumann equivalence of projections. Similarly the dimension of
$(I-R)(\mathcal{H})$ is also finite.
Thus $(I-Q)\vee (I-R)$ is a finite-rank projection and $Q\wedge R\neq0$.
\end{proof}

Throughout the rest of this paper, we will assume that $\mathcal{H}$ is
infinite-dimensional.
As $P(\mathcal{H})$ and $(I-P)(\mathcal{H})$ both are infinite-dimensional, then
we can assume that $\mathcal{H}=\mathcal{H}_0\oplus\mathcal{H}_0$ and
$P=\left(
                                        \begin{array}{cc}
                                          I & 0 \\
                                          0 & 0 \\
                                        \end{array}
                                      \right)$. If $Q$ satisfies that $P\vee
Q=I$, $P\wedge Q=0$, then according to
                                      the structure of two projections, we have
the following result.

\begin{lemma} With the above notation, if $P\wedge Q=0$ and $P\vee Q=I$, then
$Q$ must have the following operator matrix form
\begin{align}
Q=\left(
                                                           \begin{array}{cc}
                                                             H & \sqrt{H(I-H)}V
\\
                                                             V^*\sqrt{H(I-H)} &
I-V^*HV \\
                                                           \end{array}
                                                         \right)
\end{align}
where $H$ is a positive contraction and $V$ a partial isometry whose final
projection agrees with the range projection of the operator $\sqrt{H(I-H)}$.
\end{lemma}
\begin{proof} Suppose that $Q=\left(
                                                           \begin{array}{cc}
                                                             H_1 & H_2V \\
                                                             V^*H_2 & H_3 \\
                                                           \end{array}
                                                         \right)$ where $H_1$
and $H_3$ are positive contractions and $H_2V$ is
                                                         the polar decomposition
of the $(2,1)-$entry in the operator matrix.

As $P\wedge Q=0$, thus we have $Ker(I-H_1)=0$. Otherwise, we may assume that
$0\neq x\in Ker(I-H_1)$. Then 
$Q\left(
\begin{array}{c}
                                                                                
                        x \\
                                                                                
                        0 \\
                                                                                
                      \end{array}                                              
                    \right)
=\left(
   \begin{array}{c}
     x \\
     0 \\
   \end{array}
 \right)
$. Thus $P\wedge Q\neq0$ which contradicts our assumption. Similarly as $P\vee
Q=I$, we have $Ker(H_3)=0$ since that $0\neq x\in Ker(H_3)$ would imply that
$(P\vee Q)\left(
\begin{array}{c}
0 \\
x \\
\end{array}
\right)=0$
which is a contradiction.

Furthermore since $Q^2=Q$, we have
\begin{eqnarray*}
% \nonumber to remove numbering (before each equation)
  H_1 &=& H_1^2+H_2^2  \\
  H_2V &=& H_1H_2V+H_2VH_3  \\
  H_3 &=& V^*H_2^2V+H_3^2.
\end{eqnarray*}
Then we have $H_2=\sqrt{H_1(I-H_1)}$ and
$(I-H_1)\sqrt{H_1(I-H_1)}V=\sqrt{H_1(I-H_1)}VH_3$. As the kernel of the operator
$\sqrt{I-H}$ is trivial,
we find that $(I-H_1)H_1V=H_1VH_3$. According to the last equation of the
above system of equations and $Ker(H_3)=0$, we obtain that $H_3=I-V^*H_1V$.
\end{proof}

\begin{remark} If the projections $P,Q$ are in a finite factor, then the partial
isometry $V$ can be extended to a unitary. In this case,
we have $Q=\left(
                                                           \begin{array}{cc}
                                                             H & \sqrt{H(I-H)}V
\\
                                                             V^*\sqrt{H(I-H)} &
V^*(I-H)V \\
                                                           \end{array}
                                                         \right)$.
Note that in the proof of the above lemma, we have $P\wedge Q=0$ and $P\vee Q=I$
if and only if $Q$ has the operator matrix form in Lemma 2.2 and $Ker(I-H)=0$.
\end{remark}

In this paper, we shall use $Ran(A)$ to denote the range projection of the
operator $A$ and $Range(A)$ to denote the actual range of $A$ when the
underlying space is given. When an operator $S$ is unbounded, we denote the
domain of $S$ by $\mathcal{D}(S)$.

\begin{remark} Let us recall some properties of the operators $H$ and $V$.
Suppose that $E=VV^*$ and $F=V^*V$. Then we have
\begin{align*}
    Ran(\sqrt{H})&=Ran(\sqrt{H(I-H)})=Ran(\sqrt{H(I-H)^{-1}})=Ran(\sqrt{H}V)\\
    &=Ran(\sqrt{H(I-H)}V)=Ran(\sqrt{H(I-H)^{-1}}V)=Ran(\sqrt{I-H}V)=E.
\end{align*}
Furthermore the restriction of $H$ to $E(\mathcal{H}_0)$ is injective.
Let $$W=\left(               \begin{array}{cc}
                                        \sqrt{I-H}    & \sqrt{H}V  \\
                                        -V^*\sqrt{H}  & V^*\sqrt{I-H}V+(I-F) \\
                                       \end{array} \right).$$
Then $W^*W=WW^*=I$,  i.e. $W$ is a unitary. Moreover it is easy to check that
$Q=W\left(
\begin{array}{cc}
0 & 0 \\
0 & F \\
\end{array}
\right)W^*$.
\end{remark}


If $P \wedge Q = 0$, we have, for any $0 \neq \xi \in
Q\mathcal{H} \ominus Q(I-P)\mathcal{H}$,
\begin{align*}
0 = \langle\xi, Q(I-P)\beta\rangle = \langle\xi, (I-P)\beta\rangle, \qquad
\forall \beta \in \mathcal{H},
\end{align*}
which contradicts the fact that $Q \wedge P = 0$.
Therefore we have the following:

\begin{lemma}
If $P \wedge Q = 0$, then $Ran(Q) = Ran(Q(I-P))$. If $P \vee Q = I$,
we have $Ran(I-Q) = Ran((I-Q)P)$.
\end{lemma}




Now we can assume that in the double triangle lattice $\mathcal{L}$, the
projections $P,Q,R$ have the following matrix form
\begin{align*}
&P=\left(
      \begin{array}{cc}
        I & 0  \\
        0 & 0  \\
      \end{array}
    \right), Q=\left(
      \begin{array}{cc}
        H & \sqrt{H(I-H)}V  \\
        V^*\sqrt{H(I-H)} & I-V^*HV \\
      \end{array}
    \right)\\
&R=\left(
      \begin{array}{cc}
        K & \sqrt{K(I-K)}W  \\
        W^{*}\sqrt{K(I-K)} & I-W^{*}KW \\
      \end{array}
    \right)
\end{align*}
where $H, K$ are positive contractions and $V,W$ are partial isometries.


\begin{lemma} With the notation and assumptions given above, we have that
$Q\wedge R=0$ if and only if $Ker(\sqrt{H(I-H)^{-1}}V - \sqrt{K(I-K)^{-1}}W)
= 0$; and that $Q\vee R=I$ if and only if $Ker(V^{*}\sqrt{H(I-H)^{-1}} -
W^{*}\sqrt{K(I-K)^{-1}}) = 0 $.
\end{lemma}

\begin{proof} It is easy to see that
$\begin{pmatrix}
\xi_1 \\ \xi_2
\end{pmatrix} \in Q(\mathcal{H})$ if and only if
\begin{align*}
\begin{pmatrix}
I-H & -\sqrt{H(I-H)}V \\
-V^{*}\sqrt{H(I-H)} & V^{*}HV
\end{pmatrix}
\begin{pmatrix}
\xi_1 \\ \xi_2
\end{pmatrix} = 0.
\end{align*}
Since $(I-Q) \wedge (I-P) = 0$, the above is equivalent to
\begin{align*}
\sqrt{I-H}\xi_1 - \sqrt{H}V \xi_2 = 0,
\end{align*}
which implies that $\xi_1 = \sqrt{H(I-H)^{-1}}V\xi_2$.


Thus we have
\begin{align*}
Range(Q) &= \left\{
\begin{pmatrix}
\sqrt{H(I-H)^{-1}}V\xi \\
\xi
\end{pmatrix} | \xi \in \mathcal{D}(\sqrt{H(I-H)^{-1}}V) \right\} \\
Range(R) &= \left\{
\begin{pmatrix}
\sqrt{K(I-K)^{-1}}W\xi \\
\xi
\end{pmatrix} | \xi \in \mathcal{D}(\sqrt{K(I-K)^{-1}}W) \right\}.
\end{align*}
Therefore $Q \wedge R = 0$ if and only if $Ker(\sqrt{H(I-H)^{-1}}V -
\sqrt{K(I-K)^{-1}}W)
= 0 $.

Note that $I-Q = \begin{pmatrix}
I - H & -\sqrt{H(I-H)}V \\
-V^{*}\sqrt{H(I-H)} & V^{*}HV
\end{pmatrix}$. Also, we have that
$\begin{pmatrix}
\xi_1 \\ \xi_2
\end{pmatrix} \in (I-Q)\mathcal{H}$ if and only if
\begin{align*}
\begin{pmatrix}
H & \sqrt{H(I-H)}V \\
V^{*}\sqrt{H(I-H)} & I - V^{*}HV
\end{pmatrix}
\begin{pmatrix}
\xi_1 \\ \xi_2
\end{pmatrix} = 0.
\end{align*}

Since $Q \wedge P = 0$, the above is equivalent to
\begin{align*}
V^{*}\sqrt{H(I-H)}\xi_1 + (I-V^{*}HV)\xi_2 = 0.
\end{align*}

If $V^{*}V = F$, then
$I- V^{*}HV = (I-F) + V^{*}(I-H)V$. The above is equivalent to
\begin{align*}
V^{*}\sqrt{H(I-H)}\xi_1 + (I-F)\xi_2 + V^{*}(I-H)V\xi_2 = 0.
\end{align*}
This implies that $(I-F)\xi_2 = 0$ and
\begin{align*}
V^{*}\sqrt{H(I-H)}\xi_1 + V^{*}(I-H)V\xi_2 = 0.
\end{align*}
These two equations imply that $\xi_2 = -V^{*}\sqrt{H(I-H)^{-1}}\xi_1$.
Thus
\begin{align*}
Range(I-Q) &= \left\{
\begin{pmatrix}
\xi \\
-V^{*}\sqrt{H(I-H)^{-1}}\xi
\end{pmatrix} | \xi \in \mathcal{D}(\sqrt{H(I-H)^{-1}}) \right\} \\
Range(I-R) &= \left\{
\begin{pmatrix}
\xi \\
-W^{*}\sqrt{K(I-K)^{-1}}\xi
\end{pmatrix} | \xi \in \mathcal{D}(\sqrt{K(I-K)^{-1}}W) \right\}.
\end{align*}

Therefore we have $Q\vee R = I$ if and only if $(I-Q)\wedge (I-R)=0$ which in
turn is equivalent to
$Ker(V^{*}\sqrt{H(I-H)^{-1}} - W^{*}\sqrt{K(I-K)^{-1}}) = 0$.
\end{proof}

The following theorem gives a description of all elements in $Alg(\mathcal{L})$
in terms of operator equations and will be used in later sections. Similar
descriptions were given in \cite{[GY]} and \cite{[GY2]}.


\begin{theorem} Suppose that the projections $P,Q,R$ are the above-mentioned
operator matrix forms and $\mathcal{L}=\{0,I,P,Q,R\}$ is a double triangle
lattice.
If $T\in Alg(\mathcal{L})$, then there exist operators
$T_1,T_2,T_3\in\mathcal{B}(\mathcal{H}_0)$ which satisfy
\begin{align}
T_1\sqrt{H(I-H)^{-1}}V+T_2 &= \sqrt{H(I-H)^{-1}}VT_3\\
T_1\sqrt{K(I-K)^{-1}}W+T_2 &= \sqrt{K(I-K)^{-1}}WT_3,
\end{align} such that $T=\left(
\begin{array}{cc}
 T_1 & T_2 \\
 0   & T_3 \\
 \end{array}
 \right)$.

Conversely, if there are operators $T_1,T_2,T_3$ satisfying the above system of
equations, then the operator $T$ determined by the above operator matrix is in
the algebra $Alg(\mathcal{L})$.
\end{theorem}

\begin{proof} As $P\in Alg(\mathcal{L})$, thus $T$ must have the operator matrix
form
$\left(
\begin{array}{cc}
T_1 & T_2 \\
0   & T_3 \\
\end{array}
\right)$. Furthermore $(I-Q)TQ=0$ if and only if
\begin{eqnarray*}
% \nonumber to remove numbering (before each equation)
  \sqrt{I-H}(\sqrt{I-H}T_1\sqrt{H}+\sqrt{I-H}T_2V^*\sqrt{I-H}-
\sqrt{H}VT_3V^*\sqrt{I-H})\sqrt{H} &=& 0 \\
  \sqrt{I-H}(\sqrt{I-H}T_1\sqrt{H(I-H)}V+\sqrt{I-H}T_2(I-V^*HV)-
\sqrt{H}VT_3(I-V^*HV)) &=& 0 \\
  V^*\sqrt{H}(\sqrt{I-H}T_1\sqrt{H}+\sqrt{I-H}T_2V^*\sqrt{I-H}-
\sqrt{H}VT_3V^*\sqrt{I-H})\sqrt{H} &=& 0 \\
  V^*\sqrt{H}(\sqrt{I-H}T_1\sqrt{H(I-H)}V+\sqrt{I-H}T_2(I-V^*HV)-
\sqrt{H}VT_3(I-V^*HV)) &=& 0.
\end{eqnarray*}

Suppose that $VV^*=E$ and $V^*V=F$. As $Ker(I-H)=0$, thus the second equation in
the above system of equations implies that
\begin{align}
\sqrt{I-H}T_1\sqrt{H(I-H)}V+\sqrt{I-H}T_2(I-V^*HV)- \sqrt{H}VT_3(I-V^*HV)=0.
\end{align}
By right-multiplying $I-F$ to the both side of the equation (4), we have
\begin{align}
\sqrt{I-H}T_2(I-F)-\sqrt{H}VT_3(I-F) = 0.
\end{align}
Subtract (5) from (4) and we have
\begin{align}
\sqrt{I-H}T_1\sqrt{H(I-H)}V+\sqrt{I-H}T_2V^*(I-H)V-\sqrt{H}VT_3V^*(I-H)V = 0.
\end{align}

As $Ran(V^*(I-H)V)=F$ and $Ker(V^*(I-H)V|_{F})=0$, thus equation $(6)$
implies that
\begin{align}
(\sqrt{I-H}T_1\sqrt{H(I-H)^{-1}}V+\sqrt{I-H}T_2-\sqrt{H}VT_3)F=0.
\end{align}
Note that $V(I-F)=0$. By combining the equations $(5)$ and $(7)$, we see that
$(I-Q)TQ=0$ implies that
$$\sqrt{I-H}T_1\sqrt{H(I-H)^{-1}}V+\sqrt{I-H}T_2-\sqrt{H}VT_3=0.$$
By left-multiplying the unbounded operator $\sqrt{(I-H)^{-1}}$ to the both side
of the above equation,
we obtain that $(I-Q)TQ=0$ implies that
$$T_1\sqrt{H(I-H)^{-1}}V+T_2=\sqrt{H(I-H)^{-1}}VT_3.$$
Similarly, $(I-R)TR=0$ implies that
$$T_1\sqrt{K(I-K)^{-1}}W+T_2=\sqrt{K(I-K)^{-1}}WT_3.$$

Conversely, when $T_1\sqrt{H(I-H)^{-1}}V+T_2=\sqrt{H(I-H)^{-1}}VT_3$, by
right-multiplying $V^*\sqrt{I-H}$ and $I-F$ respectively, we have
$$T_1\sqrt{H}+T_2V^*\sqrt{I-H}=\sqrt{H(I-H)^{-1}}VT_3V^*\sqrt{I-H}$$
which implies that
\begin{align}
\sqrt{I-H}T_1\sqrt{H}+\sqrt{I-H}T_2V^*\sqrt{I-H}=\sqrt{H}VT_3V^*\sqrt{I-H}
\end{align}
and $T_2(I-F)=\sqrt{H(I-H)^{-1}}VT_3(I-F)$. Therefore we have
\begin{eqnarray*}
\sqrt{I-H}T_1\sqrt{H(I-H)}V+\sqrt{I-H}T_2(I-V^*HV)- \sqrt{H}VT_3(I-V^*HV)=0
\end{eqnarray*}
which implies $(I-Q)TQ=0$. Similarly, $T_1\sqrt{K(I-K)^{-1}}W+T_2 =
\sqrt{K(I-K)^{-1}}WT_3$ implies that $(I-R)TR=0$.
\end{proof}


In {\cite{[GY2]}}, the authors have shown that if
$$\mathcal{D}(\sqrt{H(I-H)^{-1}})\cap
\mathcal{D}(\sqrt{K(I-K)^{-1}}W)\cap\mathcal{D}(W^{*}\sqrt{K(I-K)^{-1}})\neq
0,$$
 then the algebra $Alg(\mathcal{L})$ is non-trivial. Now we can get a similar
result under a weaker assumption.


\begin{lemma} With the notation given above and with the assumptions of Theorem
2.5, if we
assume further that $\mathcal{D}(\sqrt{H(I-H)^{-1}}V)\cap
\mathcal{D}(\sqrt{K(I-K)^{-1}}W)\neq0$ and
$\mathcal{D}(V^*\sqrt{H(I-H)^{-1}})\cap
\mathcal{D}(W^{*}\sqrt{K(I-K)^{-1}})\neq0$,
then we have $Alg(\mathcal{L})\neq\mathbb{C}I$.
\end{lemma}

\begin{proof} Suppose that $$\xi\in \mathcal{D}(V^*\sqrt{H(I-H)^{-1}})\cap
\mathcal{D}(W^{*}\sqrt{K(I-K)^{-1}})$$
and $$\eta\in \mathcal{D}(\sqrt{H(I-H)^{-1}}V)\cap
\mathcal{D}(\sqrt{K(I-K)^{-1}}W).$$
We define two rank-one operators as following
 $$T_1=\xi\otimes(\sqrt{H(I-H)^{-1}}V-\sqrt{K(I-K)^{-1}}W)\eta$$
 and
 $$T_3=(V^*\sqrt{H(I-H)^{-1}}-W^{*}\sqrt{K(I-K)^{-1}})
\xi\otimes\eta,$$ respectively. Furthermore, we define the operator $T_2$ as the
following
$$T_2x=\sqrt{H(I-H)^{-1}}VT_3x-\langle
x,V^*\sqrt{H(I-H)^{-1}}\xi\rangle(\sqrt{H(I-H)^{-1}}V-\sqrt{K(I-K)^{-1}}W)\eta$$
for any $x\in\mathcal{H}$.
It is easy to check that $T_2$ is a bounded operator.


By a simple calculation, it follows that
\begin{align*}
\sqrt{I-H}T_1\sqrt{H}+\sqrt{I-H}T_2V^*\sqrt{I-H} &= \sqrt{H}VT_3V^*\sqrt{I-H}\\
\sqrt{I-K}T_1\sqrt{K}+\sqrt{I-K}T_2W^{*}\sqrt{I-K} &=
\sqrt{K}WT_3W^{*}\sqrt{I-K}.
\end{align*}
The above equations imply that
\begin{align}
\sqrt{I-H}T_1\sqrt{H(I-H)^{-1}}VF+\sqrt{I-H}T_2F &= \sqrt{H}VT_3F\\
\sqrt{I-K}T_1\sqrt{K(I-K)^{-1}}WF'+\sqrt{I-K}T_2F' &= \sqrt{K}WT_3F',
\end{align}
where $F=V^*V$ and $F'=W^*W$.

Furthermore, we have
\begin{align*}
\sqrt{I-H}T_2(I-F)  &= \sqrt{H}VT_3(I-F)\\
\sqrt{I-K}T_2(I-F') &= \sqrt{K}WT_3(I-F').
\end{align*}
Now we can define a bounded operator $T=\left(
      \begin{array}{cc}
        T_1 & T_2  \\
        0   & T_3 \\
      \end{array}
    \right)$. It follows that $T\in Alg(\mathcal{L})$.
\end{proof}






\section{Double Triangle Lattices in $\mathcal{B}(\mathcal{H})$}

In {\cite{[HLP]}}, the authors studied certain triangle lattices determined by
two bounded operators. Here is their construction: Suppose $\mathcal{K}$ is a
Hilbert space and $A, B$ are bounded linear operators acting on $\mathcal{K}$
whose ranges are dense in $\mathcal{K}$ and $Range(A)\cap Range(B)=0$.
Let $\mathcal{H}=\mathcal{K}\oplus\mathcal{K}$.
They constructed a double triangle lattice
$\{0,\mathcal{K}\oplus 0,
\mathcal{K}\oplus\mathcal{K},\mathcal{G}(A),\mathcal{G}(B)\}$ where
$\mathcal{G}(A)$ denotes the graph of $A$. The authors then proved that
if a bounded linear operator on $\mathcal{K}$ preserving the ranges of $A,B$
invariant must be a scalar multiple of the identity operator, then the double
triangle lattice
$\{0,\mathcal{K}\oplus 0,
\mathcal{K}\oplus\mathcal{K},\mathcal{G}(A),\mathcal{G}(B)\}$ is transitive.

Here we can represent the projections in the above lattice in an operator matrix
form. In fact, let
$H=(I+AA^*)^{-1}$ and $K=(I+BB^*)^{-1}$, and $V,W$ be the unitaries in the
polar decompositions of $A$
and $B$ respectively. Let $P(\mathcal{H})=\mathcal{K}\oplus 0$ and the
projections $Q$ and $R$ be the operator matrices represented by $H,V$ and $K,W$
respectively as in Section 2. Then the above
double triangle lattice is the same as the lattice $\{0,I,P,Q,R\}$. The positive
answer to our conjecture in the introduction implies that this double triangle
lattice may not be transitive.

In the above-mentioned case, the operators $H$ and $K$ are invertible and $V$
and $W$ are unitaries. As it is shown in the previous section,
in general, $H$ and $K$ are not invertible and $V$ and $W$ are only partial
isometries. In this section,
we will show that if one of the operators $I-H$ and $I-K$ is invertible, then
the double triangle lattice $\mathcal{L}=\{0,I,P,Q,R\}$ is neither transitive
nor reflexive.

Before proving our general result, we first study a special case.

Suppose that $\mathcal{H}_0$ is a separable infinite dimensional Hilbert space
and $\mathcal{H}=\mathcal{H}_0\oplus\mathcal{H}_0$.
Let $P, Q, R\in\mathcal{B}(\mathcal{H})$ be the projections having the following
matrix forms with respect to our space decomposition of $\mathcal H$:
\begin{align*}
&P=\left(
      \begin{array}{cc}
        I &     0  \\
        0 &     0  \\
      \end{array}
    \right), \quad Q=\frac{1}{2}\left(
      \begin{array}{cc}
        I &   I  \\
        I &   I  \\
      \end{array}
    \right)\\
    &R=\left(
      \begin{array}{cc}
        H                    & \sqrt{H(I-H)}V   \\
        V^{*}\sqrt{H(I-H)}   & I-V^{*}HV      \\
      \end{array}
    \right)
\end{align*}
where $0\leqslant H\leqslant I$ and $V$ is a partial isometry. Let
$\mathcal{L}=\{0,I,P,Q,R\}$ and suppose that $\mathcal{L}$ is a double triangle
lattice.


We define $S=\sqrt{H(I-H)^{-1}}V-I$. Then $S$ is a densely defined closed, in
general, unbounded operator.
From $Q\wedge R=0$, $Q\vee R=I$ and Lemma 2.4, we have
$Ker(S^*)=Ker(V^*\sqrt{H(I-H)^{-1}}-I)=0$
and $Ker(S)=Ker(\sqrt{H(I-H)^{-1}}V-I)=0$. Hence the ranges of $S$ and $S^*$ are
both dense in $\mathcal{H}_0$.

Suppose that $T\in Alg(\mathcal{L})$ is any given element. Since
$P\in\mathcal{L}$, $T$ must be an upper triangle operator matrix. From
$Q\in\mathcal{L}$, we have
$T=\left(
\begin{array}{cc}
 T_1 & T_2 \\
  0   & T_3 \\
   \end{array}
   \right)$
such that $T_2=T_3-T_1$. From Theorem 2.5,
we further have $T_1S-ST_3=0$. By using properties of unbounded operators, we
state the following result.


\begin{lemma} With the above notation, if $T\in Alg(\mathcal{L})$,
then there is an $A \in \mathcal{B}(\mathcal{H}_0)$ such that $S^{-1}AS$ is
bounded and
$$T =\left(
      \begin{array}{cc}
        A   &    S^{-1}AS-A  \\
        0   &    S^{-1}AS    \\
      \end{array}
    \right).
$$
Conversely, if $A\in \mathcal{B}(\mathcal{H}_0)$ such that $S^{-1}AS$ is
bounded, then the above operator $T$ belongs to $Alg(\mathcal{L})$.
\end{lemma}

\begin{remark} It is easy to see that there are many bounded operators $A$ such
that $S^{-1}AS$ is bounded. In fact, for any $\xi\in
\mathcal{D}(V^*\sqrt{H(I-H)^{-1}}-I)$
and $\eta\in\mathcal{D}(\sqrt{H(I-H)^{-1}}V-I)$, we define the rank-one
operators $T_1=\xi\otimes S\eta$
and $T_3=S^*\xi\otimes\eta$ respectively. Then the operator $T=\left(
      \begin{array}{cc}
        T_1   &    T_3-T_1  \\
        0     &    T_3      \\
      \end{array}
    \right)\in Alg(\mathcal{L})$.
Thus the algebra $Alg(\mathcal{L})$ is infinite dimensional.
Furthermore, if $E=VV^{*}$ and $F=V^{*}V$, then for any $\xi\in
(I-E)(\mathcal{H}_0)$ and $\eta\in
(I-F)(\mathcal{H}_0)$,
since $S\eta=-\eta$ and $S^*\xi=-\xi$, we have that the operator $\left(
      \begin{array}{cc}
        \xi\otimes\eta   &    0                   \\
        0                &    \xi\otimes\eta      \\
      \end{array}
    \right)$ is in the algebra $Alg(\mathcal{L})$.
Hence all the tensor products of the bounded operators from
$(I-E)(\mathcal{H}_0)$ into $(I-F)(\mathcal{H}_0)$ with the identity operator
$I_2$ in $\mathcal{M}_2(\mathbb{C})$ are in $Alg(\mathcal{L})$.

Furthermore since $S$ is a densely defined closed operator, similar to the
argument used in {\cite{[HY]}}, there
exists a net $\{F_{\lambda}:\lambda>0\}$ of projections such that
$S^{-1}F_{\lambda}, F_{\lambda}S$ are bounded and $F_{\lambda}\rightarrow I$
as $\lambda\rightarrow 0$ under strong-operator topology. Thus the $(1,1)$-entry
of the element in the algebra
$Alg(\mathcal{L})$ is dense in $\mathcal{B}(\mathcal{H}_0)$ under
strong-operator topology.
\end{remark}

\begin{lemma} With the notation given as above, for any $E'\in
Lat(Alg(\mathcal{L}))\setminus\mathcal{L}$, we have $P\wedge E'=0$ and $P\vee
E'=I$.
\end{lemma}


\begin{proof} Suppose that $P\wedge E'=\left(
      \begin{array}{cc}
        E'_0   &    0  \\
        0     &    0  \\
      \end{array}
    \right)\in Lat(Alg(\mathcal{L}))$, where $E'_0$ is a projection. Then for
any $\xi\in \mathcal{D}(S^*)$ and
    $\eta\in\mathcal{D}(S)$, by using of the operators $T_1$, $T_3$ and $T$
defined in the above remark,
    we have $(I-E'_0)T_1E'_0=0$. It means that for any $x\in\mathcal{H}_0$,
$\xi\in\mathcal{D}(S^*)$ and $\eta\in\mathcal{D}(S)$, we have
    $$\langle E'_0x,\xi\rangle S\eta=\langle E'_0x,\xi\rangle E'_0S\eta.$$
    As the domains of $S^*$ are dense in $\mathcal{H}_0$ and $Ker(S^*)=0$ which
implies that the image of $S$ is also dense in $\mathcal{H}_0$,
    thus we obtain that $E'_0=0$ or $E'_0=I$.

    If $E'_0=I$, then $E'=\left(
      \begin{array}{cc}
        I     &    0  \\
        0     &    E'_1  \\
      \end{array}
    \right)$. Thus for any $\xi\in \mathcal{D}(S^*)$ and $\eta\in\mathcal{D}(S)$
and
    $T_3=S^*\xi\otimes\eta$, we also have $(I-E'_1)T_3E'_1=0$. Similarly, we
also have $E'_1=0$ or $E'_1=I$. But $E'\notin \mathcal{L}$,
    it follows that $E'\wedge P=0$.

    Similarly we can show that $P\vee E'=I$.
    \end{proof}\\

The following result is similar to that in {\cite{[GY2],[HY]}}. But the
arguments used are different.  We can also determine the elements of the lattice
$Lat(Alg(\mathcal{L}))$. But its topological structure is undetermined.


\begin{theorem} For any projection $E'\in Lat(Alg(\mathcal{L})) \setminus
\{0,I,P\}$, there are a positive contraction $K$ and
a partial isometry $U$ with range projection $K$ in $\mathcal{B}(\mathcal{H}_0)$
such that
$$
E' = \left(
\begin{array}{cc} K & \sqrt{K(I-K)}U  \\
 U^{*}\sqrt{K(I-K)} & I-U^{*}KU     \\
 \end{array}
 \right),
$$
where $\sqrt{K(I-K)^{-1}}$ (or $K$) and $U$ are determined by the
polar decomposition of $aS + I$ for some $a \in \mathbb{C}$.

Conversely, for any given
complex number $a\in \mathbb{C}$, the polar decomposition of $I+aS$ uniquely
determines $U$ and $K$ which gives rise to a projection $E'$
(in the above form)
in $Lat(Alg(\mathcal{L}))\setminus\{0,I,P\}$.
\end{theorem}

\begin{proof} Suppose that $E'\in Lat(Alg(\mathcal{L}))\setminus\{0,I,P\}$.
According to the above Lemma, we have
$E'\wedge P=0$ and $E'\vee P=I$. Thus according to Lemma 2.2,
we have
$$E'=\left(
\begin{array}{cc} K & \sqrt{K(I-K)}U  \\
 U^{*}\sqrt{K(I-K)} & I-U^{*}KU     \\
 \end{array}
 \right)$$
where $K$ is a positive contraction with $Ker(I-K)=0$ and $U$ is a partial
isometry with the range projection of $K$ or $\sqrt{H(I-H)}$
 as final projection.

 For any $\xi\in\mathcal{D}(V^*\sqrt{H(I-H)^{-1}}-I)$ and
$\eta\in\mathcal{D}(\sqrt{H(I-H)^{-1}}V-I)$,
 let $T_1=\xi\otimes S\eta$
and $T_3=S^{*}\xi\otimes\eta$ respectively. Then the operator $T=\left(
      \begin{array}{cc}
        T_1   &    T_3-T_1  \\
        0     &    T_1      \\
      \end{array}
    \right)\in Alg(\mathcal{L})$. According to the proof of Theorem 2.5, the
equation $(I-E')TE'=0$ implies that
       
$$\sqrt{I-K}T_1(\sqrt{K(I-K)}U-(I-U^{*}KU))=(\sqrt{K}U-\sqrt{I-K})T_3(I-U^{*}
KU).$$
   Then for any $x\in\mathcal{H}_0$, we have
   \begin{align}
   \langle(\sqrt{K(I-K)}U-(I-U^{*}KU))x,\xi\rangle \sqrt{I-K}S\eta=\langle
(I-U^{*}KU)x,S^{*}\xi\rangle(\sqrt{K}U-\sqrt{I-K})\eta.
   \end{align}


If $(\sqrt{K(I-K)}U-(I-U^{*}KU))x=0$ for any $x\in\mathcal{H}_0$, then $U$ must
be injective and $Ran(\sqrt{K(I-K)})=\mathcal{H}_0$
as $Ker(I-U^*KU)=0$. Thus we obtain that $U$ is a unitary and
$\sqrt{K(I-K)}U=U^*(I-K)U$ which implies $E'=Q$.

If $E'\neq Q$, then we can choose a vector $x_0$
   such that $(\sqrt{K(I-K)}U-(I-U^{*}KU))x_0\neq0$. Moreover, as
$\mathcal{D}(V^*\sqrt{H(I-H)^{-1}}-I)$ is dense, thus we also can pick a vector
   $\xi_0\in\mathcal{D}(V^*\sqrt{H(I-H)^{-1}}-I)$ such that
   $$\langle(\sqrt{K(I-K)}U-(I-U^{*}KU))x_0,\xi_0\rangle\neq0.$$

   Thus the equation (11) implies that there is a non-zero constant
$a\in\mathbb{C}$ such that
   $$a\sqrt{I-K}S\eta=(\sqrt{K}U-\sqrt{I-K})\eta$$
   for any
   $\eta\in\mathcal{D}(\sqrt{H(I-H)^{-1}}V-I)$. Thus we have
   $$\sqrt{K(I-K)^{-1}}U=I+aS.$$

   Conversely, if $K$ and $U$ are defined by the above equation as the polar
decomposition of the right hand side, then it is easy to check that
   $E'\in Lat(Alg(\mathcal{L}))$.
\end{proof}


Note that $a=0$ corresponds to the projection $Q$ and $a=1$ the projection $R$.



In the rest part of this section, we will discuss more general double triangle
lattices in $\mathcal{B}(\mathcal{H})$. Let $A$ be an invertible bounded
operator
and $P$ a projection. We let $\overline{P}$ be the range projection of
$APA^{-1}$. Note that
$\overline{P}(\mathcal{H})=Range(APA^{-1})=Range(AP)=\{APx:x\in\mathcal{H}\}$.

For the reader's convenience, we provide the proof of the following two
well-known results on the similarity of lattices.

\begin{lemma} With the notation given above, suppose that $P$ and $Q$ are two
projections. Then we have
$$\overline{P}\wedge\overline{Q}=\overline{P\wedge Q}, \    \
\overline{P}\vee\overline{Q}=\overline{P\vee Q}.$$
\end{lemma}
\begin{proof} The first result follows from the following observations.
\begin{align*} \xi\in (\overline{P}\wedge
\overline{Q})(\mathcal{H})&\Leftrightarrow  APA^{-1}\xi=\xi=AQA^{-1}\xi
\\ &\Leftrightarrow A^{-1}\xi\in P\wedge Q\Leftrightarrow
\xi\in \overline{P\wedge Q}.
\end{align*}

Secondly, it is easy to show that
$\overline{P}\vee\overline{Q}\leq\overline{P\vee Q}$. Conversely we have to show
that if $\xi\in\overline{P\vee Q}$,
then $\xi\in\overline{P}\vee\overline{Q}$. Assume that is not true. Then there
is a
vector $\xi\in\overline{P\vee Q}(\mathcal{H})$ but $\xi\perp
(\overline{P}\vee\overline{Q})(\mathcal{H})$.
Thus there is a vector $\xi_0\in (P\vee Q)(\mathcal{H})$ such that for any $x\in
P(\mathcal{H})$ and any $y\in Q(\mathcal{H})$ we have
$$\langle A\xi_0, Ax\rangle=0, \   \ \langle A\xi_0,Ay\rangle=0.$$
This implies that for any $x\in P(\mathcal{H})$ and any $y\in Q(\mathcal{H})$ we
have $\langle A^{*}A\xi_0,x\rangle=0$ and $\langle A^{*}A\xi_0,y\rangle=0$.
Thus $A^{*}A\xi_0\perp (P+Q)(\mathcal{H})$. But $Ran(P+Q)=P\vee Q$.
Thus $A^{*}A\xi_0\perp(P\vee Q)(\mathcal{H})$ and $A^{*}A\xi_0=0$. It follows
that $\xi_0=0$.
\end{proof}

It follows that if $\{ 0, I, P, Q, R \}$ is a double triangle lattice, then so
is $\{0, I, \overline{P}, \overline{Q}, \overline{R} \}$.

\begin{remark} 
In general, we do not have $I-\overline{P}=\overline{I-P}$. For
example: let $P=\left(                                                         
          \begin{array}{cc} 
            I & 0 \\                                                     
            0 & 0 \\                                                            
          \end{array}
        \right)$ 
and $A=\left(\begin{array}{cc}
       I & -I \\
       0 & I \\
\end{array}
\right)$,
then $\overline{P}=Ran(APA^{-1})=P$. But $A(I-P)A^{-1}=\left(
\begin{array}{cc}
  0 & -I \\
  0 & I \\
\end{array}
\right)$ and 
$\overline{I-P}=Ran(A(I-P)A^{-1})=\frac{1}{2}\left(
                                  \begin{array}{cc}
                                   I & -I \\
                                  -I & I \\
                                  \end{array}
\right)$. However $I-\overline{P}=\left(
                        \begin{array}{cc}
                          0 & 0 \\
                          0 & I \\
                        \end{array}
                      \right)$.
\end{remark}

However we can prove the following result.

\begin{lemma} Suppose $\mathcal{L}$ is a lattice and $A$ is an invertible
operator. For any $P\in\mathcal{L}$, we define $\overline{P}=Ran(APA^{-1})$
and $\overline{\mathcal{L}}=\{\overline{P}:P\in\mathcal{L}\}$. Then we have
$Alg(\overline{\mathcal{L}})=A (Alg(\mathcal{L}))A^{-1}$ and
$Lat(Alg(\overline{\mathcal{L}}))=\{\overline{P}:P\in Lat(Alg(\mathcal{L}))\}$.
\end{lemma}

\begin{proof} First, from
\begin{align*} T\in Alg(\mathcal{L})&\Leftrightarrow \forall P\in\mathcal{L},
PTP=TP\\
&\Leftrightarrow \forall P\in\mathcal{L}, APA^{-1}\cdot ATA^{-1}\cdot
AP=ATA^{-1}\cdot AP\\
&\Leftrightarrow ATA^{-1}\in Alg(\overline{\mathcal{L}}),
\end{align*}
we have $A(Alg(\mathcal{L}))A^{-1}= Alg(\overline{\mathcal{L}})$.

Secondly, according to the previous result and the following
\begin{align*} Q\in Lat(Alg(\overline{\mathcal{L}}))&\Leftrightarrow \forall
T\in Alg(\mathcal{L}), QATA^{-1}Q=ATA^{-1}Q\\
&\Leftrightarrow\forall T\in Alg(\mathcal{L}), A^{-1}QATA^{-1}QA=TA^{-1}QA\\
&\Leftrightarrow Ran(A^{-1}QA)=Ran(A^{-1}Q)\in Lat(Alg(\mathcal{L})),
\end{align*}
we have $Lat(Alg(\overline{\mathcal{L}}))=\{\overline{P}:P\in
Lat(Alg(\mathcal{L}))\}$.
\end{proof}\\

Now suppose that the projections $P,Q,R\in\mathcal{B}(\mathcal{H})$ have the
following matrix forms:
\begin{align*}&P=\left(
                    \begin{array}{cc}
                      I & 0 \\
                      0 & 0 \\
                    \end{array}
                  \right),
                  Q=\left(
          \begin{array}{cc}
            H & \sqrt{H(I-H)}V \\
            V^*\sqrt{H(I-H)} & I-V^*HV \\
          \end{array}
        \right)\\
 &R=\left(
          \begin{array}{cc}
            K & \sqrt{K(I-K)}W \\
            W^*\sqrt{K(I-K)} & W^*(I-K)W \\
          \end{array}
        \right)
 \end{align*} where $H,K$ are positive contractions and $V,W$ are partial
isometries with the range projections of $H$ and $K$ as their final projections,
respectively.
 Then we have the following conclusion.

 \begin{theorem} With the notation given above, let $\mathcal{L}=\{0,I,P,Q,R\}$.
Suppose that
 $\mathcal{L}$ is a double triangle lattice and one of the operators $I-H$ and
$I-K$ is invertible. Then $\mathcal{L}$ is neither reflexive nor transitive.
 \end{theorem}

 \begin{proof} Without loss of generality, we assume that $I-H$ is invertible.
Then $\sqrt{I-H}$ is also invertible.

        Let $A=\left(
          \begin{array}{cc}
            I & I-\sqrt{H(I-H)^{-1}}V \\
            0 & I \\
          \end{array}
        \right)$. Then $A^{-1}=\left(
          \begin{array}{cc}
            I & \sqrt{H(I-H)^{-1}}V-I \\
            0 & I \\
          \end{array}
        \right)$.
It is easy to check that $\overline{P}=Ran(APA^{-1})=P$. Furthermore, as
\begin{align*} AQ=\left(
          \begin{array}{cc}
            V^*\sqrt{H(I-H)} & I-V^*HV \\
            V^*\sqrt{H(I-H)} & I-V^*HV \\
          \end{array}
        \right)
\end{align*} and
$\{V^*\sqrt{H(I-H)}x+(I-V^*HV)y:x,y\in\mathcal{H}_0\}=\mathcal{H}_0$, thus we
have
              \begin{align*}(AQA^{-1})(\mathcal{H})=\{\left(
                                                   \begin{array}{c}
                                                     x \\
                                                     x \\
                                                   \end{array}
                                                 \right):x\in\mathcal{H}_0\}
        \end{align*} which implies that $\overline{Q}=\frac{1}{2}\left(
          \begin{array}{cc}
            I & I \\
            I & I \\
          \end{array}
        \right)$.

        Let
$\overline{\mathcal{L}}=\{0,I,\overline{P},\overline{Q},\overline{R}\}$.
According to Lemma 3.4, $\overline{\mathcal{L}}$ is also a double triangle
lattice. According to Remark 3, we know that $Alg(\overline{\mathcal{L}})$ is
infinite dimensional. According to Theorem 3.3,
$\overline{\mathcal{L}}\varsubsetneq Lat(Alg(\overline{\mathcal{L}}))$.
Lastly according to Lemma 3.5, we know that $Alg(\mathcal{L})$ is infinite
dimensional and $\mathcal{L}\varsubsetneq
Lat(Alg(\mathcal{L}))$.
\end{proof}\\



When $P,Q,R$ are projections in a finite factor $\mathcal{M}$, suppose that
$\mathcal{L}=\{0,I,P,Q,R\}$ is a double triangle lattice.
Ge and Yuan {\cite{[GY2]}}, Hou and Yuan {\cite{[HY]}} have shown that for any
three distinct projections $E_i(i=1,2,3)$
in $Lat(Alg(\mathcal{L}))\setminus\{0,I\}$,
$Lat(Alg(\mathcal{L}))=Lat(Alg(\{E_1,E_2,E_3\}))$ and
$Alg(\mathcal{L})=Alg(\{E_1,E_2,E_3\})$. It implies that
$Alg(\mathcal{L})$ is a KS-algebra in the sense of Ge and
Yuan(\cite{[GY]}\cite{[GY2]}). Furthermore they have shown that
$Lat(Alg(\mathcal{L}))\setminus\{0,I\}$ is homeomorphic to the sphere
$\mathbb{S}^2$ under the strong operator topology.

In the proof of these results, there are two crucial points: the first is that
the domains of the concerned unbounded operators contain a
common dense subspace, the second is that there is
 an increasing net $\{E_{\lambda}:\lambda>0\}$ of projections with
strong-operator
limit $I$ as $\lambda\rightarrow 0$ such that the left or right multiplications 
of those unbounded operators with $E_{\lambda}$ are bounded. Thus
they can use the operations of bounded operators.

As for the first point, the concerned unbounded operators are affiliated with
the finite factor and
the domains of these operators contain a common dense subspace such that
these unbounded operators form an algebra. However it is not true in a type
$\mathrm{I}_{\infty}$ von Neumann algebra or $\mathcal {B}(\mathcal {H})$.

As for the second point, suppose that $\mathcal{M}$ is a factor of type
$\mathrm{II}_1$ with the unique trace $\tau$ and $T,S$ are two closed densely
defined
unbounded operators affiliated with $\mathcal{M}$. Let $T=HU$ and $S=KV$ be the
polar decompositions of $T$ and $S$ respectively. Note that $H$ and $K$ are
positive unbounded operators affiliated with $\mathcal{M}$. Thus the spectral
projections of $H$ and $K$ are contained in $\mathcal{M}$. For any
$\varepsilon>0$, we can choose the spectral projections $E_{\varepsilon}$ of $H$
and $F_{\varepsilon}$ of $K$ such that $E_{\varepsilon}H$
and $F_{\varepsilon}K$ are bounded and
$\tau(E_{\varepsilon})>1-\frac{\varepsilon}{2}$ and
$\tau(F_{\varepsilon})>1-\frac{\varepsilon}{2}$. We define
$P_{\varepsilon}=E_{\varepsilon}\wedge F_{\varepsilon}$. Then $P_{\varepsilon}T$
and $P_{\varepsilon}S$ are bounded and
$$\tau(P_{\varepsilon})=\tau(E_{\varepsilon})+\tau(F_{\varepsilon})-\tau(E_{
\varepsilon}\vee F_{\varepsilon})>1-\varepsilon.$$
Then the net $\{P_{\varepsilon}\}$ of projections gives what they need. Using
this fact, it is not hard to deduce the following lemma, and we leave the
proof as an exercise.

\begin{lemma} Suppose that $A$ and $B$ are closed densely defined invertible
(maybe with unbounded inverse) operators affiliated with the $\mathrm{II}_1$
factor $\mathcal{M}$. Then $\mathcal{A}_A\cap \mathcal{A}_B$ is weak-operator
dense in $\mathcal{B}(\mathcal{H})$.
\end{lemma}


\section{Transitivity of small lattices}

Longstaff {\cite{[Lo]}} has reduced the transitivity of lattices with three
nontrivial projections to the case of double triangle lattices. By using
completely different methods, we shall prove this result again in this section.

Suppose $\mathcal{H}$ is a separable Hilbert space and
$P,Q,R\in\mathcal{B}(\mathcal{H})$ are three projections.
Assume that $\mathcal{L}=\{0,I,P,Q,R\}$. Let
$\mathcal{L}^{\perp}=\{0,I,I-P,I-Q,I-R\}$. Now we have the following simple
facts.

\begin{lemma} $Alg(\mathcal{L})=\mathbb{C}I$ if and only if
$Alg(\mathcal{L}^{\perp})=\mathbb{C}I$.
\end{lemma}

This is an easy corollary of the following observation:

$$T\in Alg(\mathcal{L}) \Leftrightarrow  T^{*}\in Alg(\mathcal{L}^{\perp}).$$


 \begin{lemma} Suppose that $Alg(\mathcal{L})=\mathbb{C}I$ and $E=P\wedge
Q\neq0$. Let $R=\left(
      \begin{array}{cc}
        R_{11} & R_{12} \\
        R_{21} & R_{22} \\
      \end{array}
    \right)$ be the operator matrix form of $R$
with respect to the orthogonal decomposition $\mathcal{H}=E(\mathcal{H})\oplus
E^{\perp}(\mathcal{H})$.
Then it follows that $Ker(I-R_{11})=0$ and $Ker(R_{11})=0$.
\end{lemma}

\begin{proof} Assume that $\alpha\in Ker(I-R_{11})$ is a non-zero vector. Let
$\xi:=\left(
      \begin{array}{c}
        \alpha \\
        0 \\
      \end{array}
    \right)$ and $Tx:=\langle x,\xi\rangle\xi$ for any $x\in\mathcal{H}$. Then
$T$ is a rank-one operator and
     $T\in Alg(\mathcal{L})$ which is a contradiction.

    If $0\neq\alpha\in Ker(R_{11})$, then we similarly define the vector $\xi$
and the operator $T$
    as above. As $P\xi=\xi$ and $Q\xi=\xi$, thus we have
$$TP=PTP, \    \ TQ=QTQ.$$

Furthermore, as $$R\xi=\left(
      \begin{array}{c}
        0 \\
        R_{21}\alpha \\
      \end{array}
    \right)=R^{2}\xi=\left(
      \begin{array}{cc}
        R_{11} & R_{12} \\
        R_{21} & R_{22} \\
      \end{array}
    \right)\left(
      \begin{array}{c}
        0 \\
        R_{21}\alpha \\
      \end{array}
    \right)=\left(
      \begin{array}{c}
        R_{12}R_{21}\alpha \\
        R_{22}R_{21}\alpha \\
      \end{array}
    \right)$$ and $R_{12} = R_{21}^{*}$, thus we have $R_{21}\alpha=0$.
This implies that $TR=0=RTR$ and $T\in Alg(\mathcal{L})$.
\end{proof}\\

\begin{lemma} With the notation given above, suppose that $\mathcal{L}$ is
transitive and $P\wedge Q=E\neq0$. Let $R_{12}=HV$ be the polar decomposition
of $R_{12}$ where $H$ is positive and $V$ a partial isometry. Then we have
$Ker(H)=0$ and $Ran(V)=E$.
\end{lemma}
\begin{proof} We just need to show that $Ker(H)=0$. As $R_{12}^{*}=R_{21}$ and
$H^{2}=R_{12}R_{21}$, thus $Ker(H)=0$ if and only if $Ker(R_{21})=0$.
Suppose that $\alpha\in Ker(R_{21})$ is a non-zero vector. Let $\xi=\left(
      \begin{array}{c}
        \alpha \\
        0 \\
      \end{array}
    \right)$ and $T$ be the rank-one operator $x\mapsto\langle x,\xi\rangle\xi$.
Since $P\xi=Q\xi=\xi$, thus we have $PTP=TP$ and $QTQ=TQ$.

Furthermore, as $R\xi=\left(
      \begin{array}{c}
        R_{11}\alpha \\
        0 \\
      \end{array}
    \right)$ and $R^2\xi=\left(
      \begin{array}{c}
        R_{11}^2\alpha \\
        R_{21}R_{11}\alpha \\
      \end{array}
    \right)$, thus we have $R_{11}\alpha=R_{11}^{2}\alpha$. But
$Ker(I-R_{11})=0$, thus we have $R_{11}\alpha=0$ and $R\xi=0$.
It implies that $RTR=TR$ and $T\in Alg(\mathcal{L})$ which is a contradiction
with $Alg(\mathcal{L})=\mathbb{C}I$.
\end{proof}\\

By using the properties of two projections (see, e.g., {\cite{[KR]}} or
{\cite{[Ta]}}), we have the following:

\begin{corollary} With the notation given above,  we have have
$$R=\left(
      \begin{array}{cc}
        R_{11} & \sqrt{R_{11}(I-R_{11})}V  \\
        V^{*}\sqrt{R_{11}(I-R_{11})} & V^{*}(I-R_{11})V+F \\
      \end{array}
    \right)$$ where $F$ is a projection such that $F\perp V^{*}V$.
\end{corollary}


 Now we prove the main result of the section.
Let $P,Q,R\in\mathcal{B}(\mathcal{H})$ be projections and $\mathcal{L}=\{0, I,
P,Q,R\}$.
\begin{theorem} If $Alg(\mathcal{L})=\mathbb{C}I$, then the intersection and
union of any two nontrivial distinct projections in $\mathcal{L}$ are zero
and $I$ respectively.
\end{theorem}

\begin{proof} By Lemma 4.1, we only need to show that $P\wedge Q=0$.

On the contrary assume that $P\wedge Q\neq0$. According to the above lemmas,
the projections $P,Q,R$ must be the following operator matrix forms:
\begin{align*}
&P=\left(
      \begin{array}{cc}
        I & 0  \\
        0 & P_0 \\
      \end{array}
    \right), \   \ Q=\left(
      \begin{array}{cc}
        I & 0  \\
        0 & Q_0 \\
        \end{array}
    \right)\\
&R=\left(
      \begin{array}{cc}
        R_{11} & \sqrt{R_{11}(I-R_{11})}V  \\
        V^{*}\sqrt{R_{11}(I-R_{11})} & V^{*}(I-R_{11})V+F \\
      \end{array}
    \right)
\end{align*}
where $P_0$ and $Q_0$ are projections.

As $R_{11}$ is positive and $Ker(R_{11})=Ker(I-R_{11})=0$, then there exists a
spectral projection $E_{0}$ of $R_{11}$ such that both $E_{0}R_{11}$ and
$E_{0}(I-R_{11})$ are invertible on the subspace $E_{0}(\mathcal{H})$. Let
$T=E_{0}$ and $A=\left(
      \begin{array}{cc}
        T & -T\sqrt{R_{11}(T(I-R_{11}))^{-1}}V  \\
        0 & 0  \\
      \end{array}
    \right)$ where $(T(I-R_{11}))^{-1}$ is defined as the operator which is
$(T(I-R_{11}))^{-1}$ on the range of $E_{0}$ and zero on the range of $I-E_{0}$.

It is a simple calculation to check that $(I-P)AP=0$ and $(I-Q)AQ=0$.
Furthermore, we have
\begin{align*}
&(I-R)AR\\
&=\left(
      \begin{array}{cc}
        I-R_{11} & -\sqrt{R_{11}(I-R_{11})}V  \\
        -V^{*}\sqrt{R_{11}(I-R_{11})} & I-V^{*}(I-R_{11})V-F \\
      \end{array}
    \right)AR\\
    &=\left(
      \begin{array}{cc}
        (I-R_{11})T & -(I-R_{11})T\sqrt{R_{11}(T(I-R_{11}))^{-1}}V  \\
        -V^{*}\sqrt{R_{11}(I-R_{11})}T &
V^{*}\sqrt{R_{11}(I-R_{11})}T\sqrt{R_{11}(T(I-R_{11}))^{-1}}V \\
      \end{array}
    \right)R\\
    &=\left(
      \begin{array}{cc}
        (I-R_{11})T & -T\sqrt{R_{11}(I-R_{11})}V  \\
        -V^{*}\sqrt{R_{11}(I-R_{11})}T & V^{*}TR_{11}V \\
      \end{array}
    \right)\\
     & \  \  \times\left(
      \begin{array}{cc}
        R_{11} & \sqrt{R_{11}(I-R_{11})}V  \\
        V^{*}\sqrt{R_{11}(I-R_{11})} & V^{*}(I-R_{11})V+F \\
      \end{array}
    \right)=0.
\end{align*}
Hence $T\in Alg(\mathcal{L})\setminus\mathbb{C}I$ which is a contradiction.
\end{proof}\\

According to the above theorem, we can draw the following consequence.

\begin{corollary} Every pentagon lattice is not transitive.
\end{corollary}

\noindent\textbf{Acknowledgments} The authors would like to thank Professor
Liming Ge, Don Hadwin, Junhao Shen and the referee for many helpful comments
and suggestions.
Part of this work was carried out while first two authors visited the University
of New Hampshire while this work was carried out.

%\nocite{*}
\bibliographystyle{line}

\bibliographystyle{plain}
\begin{thebibliography}{21}

\bibitem{[Ar]} W. Arveson, {\it{Operator algebras and invariant subspaces}},
Ann. Math., 100, 433-532, 1974.

\bibitem{[Da]} K. R. Davidson, {\it{Nest Algebras}}, Longman Scientific $\&$
Technical, New York, 1988.

\bibitem{[DH]} A. Dong and C. Hou, On some automorphisms of a class of
Kadison-Singer algebras, Lin. Alg. Appl., 436, 2037-2053, 2012.

\bibitem{[GY]} L. Ge and W. Yuan, {\it{Kadison-Singer algebras, I: hyperfinite
case}}, PNAS. USA 107(5), 1838-1843, 2010.

\bibitem{[GY2]} L. Ge and W. Yuan, {\it{Kadison-Singer algebras, II: general
case}}, PNAS. USA 107(11), 4840-4844, 2010.


\bibitem{[HLP]} D. W. Hadwin, W. E. Longstaff, R. Peter, {\it{Small transitive
lattices}}, Proc. Amer. Math. Soc., 87(1),
121-124, 1983.

\bibitem{[Hal]} P. R. Halmos, {\it{Ten problems in Hilbert space}}, Bull. Amer.
Math. Soc. 76, 887-933, 1970.

\bibitem{[Ha2]} P. R. Halmos, {\it{Reflexive lattices of subspaces}}, J. London
Math. Soc., 4, 257--263, 1971.


\bibitem{[Har]} K. J. Harrison, {\it{Certain distributive lattices of subspaces
are reflexive}}, J. London Math. Soc., 8(2), 51-56, 1974.

\bibitem{[HRR]} K. J. Harrison, H. Radjavi and P. Rosenthal, {\it{A transitive
medial subspace lattice}}, Proc. Amer.
Math. Soc. 28, 119-121, 1971.

\bibitem{[HY]} C.  Hou and W. Yuan, {\it{Minimal generating reflexive lattices
of projections
in finite von Neumann algebras}}, Math. Annalen, 353, 499-517, 2012.



\bibitem{[KR]} R. V. Kadison and J. Ringrose, {\it Fundamentals of the
Operator Algebras,} vols. I and II, Academic Press, Orlando, 1983
and 1986.


\bibitem{[Lan]} C. Lance, {\it{Cohomology and perturbations of nest algebras}},
Proc. Lond. Math. Soc., 43, 334-356, 1981.

\bibitem{[La]} D. Larson, {\it{Similarity of nest algebras}}, Ann. Math., 121,
409-427, 1983.

\bibitem{[Lo]} W. E. Longstaff, {\it{Small transitive families of subspaces}},
Acta Math. Sinica, 19(3), 567-576, 2003.

\bibitem{[RR]} H. Radjavi and P. Rosenthal, {\it{Invariant Subspaces}},
Springer-Verlag, Berlin, 1973.

\bibitem{[Ta]} M. Takesaki, {\it{Theory of operator algebras $\mathrm{I}$}},
Springer-Verlag, Heidelberg, 1979.

\bibitem{[WY]} L. Wang and W. Yuan, A new class of Kadison-Singer algebras,
Expos. Math., 29(1), 126-132, 2011.

\end{thebibliography}

\end{document}
