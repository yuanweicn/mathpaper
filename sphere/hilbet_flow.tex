\documentclass[a4paper,10pt]{amsart}

\usepackage[protrusion=true,expansion=true]{microtype} 
\usepackage{fancyhdr}
\usepackage[utf8]{inputenc}
\usepackage{graphicx} 
\usepackage{wrapfig} 

\usepackage{mathpazo}
\usepackage[T1]{fontenc}
\usepackage{amsmath}
\usepackage{amssymb}
\usepackage{hyperref}
\usepackage{cleveref}
\usepackage{comment}
\usepackage{color}
\usepackage{enumerate}
\usepackage{braket}

\newtheorem{example}{Example}[section]
\newtheorem{theorem}{Theorem}[section]
\newtheorem{proposition}{Proposition}[section]
\newtheorem{corollary}{Corollary}[section]
\newtheorem{definition}{Definition}[section]
\newtheorem{lemma}{Lemma}[section]
\newtheorem{remark}{Remark}[section]
\newtheorem{question}{Question}[section]

\crefname{lemma}{Lemma}{lemmas}
\crefname{remark}{Remark}{remark}
\crefname{corollary}{Corollary}{corollary}
\crefname{theorem}{Theorem}{theorem}
\crefname{example}{Example}{example}
\crefname{definition}{Definition}{definition}

\newcommand{\AAA}{\mathfrak A}
\newcommand{\BBB}{\mathcal B}
\newcommand{\CCC}{\mathcal C}
\newcommand{\D}{\mathfrak D}
\newcommand{\F}{\mathfrak F}
\newcommand{\II}{\mathfrak I}
\newcommand{\HHH}{\mathcal H} %for Hilbert space
\newcommand{\LLL}{\mathcal L} % for lattice
\newcommand{\MMM}{\mathcal M}


\newcommand{\Lat}{\mathcal Lat}
\newcommand{\Alg}{\mathcal Alg}
\newcommand{\C}{\mathbb C} %for complex number
\newcommand{\R}{\mathbb R}  %for real number
\newcommand{\Z}{\mathbb Z} %for integer
\newcommand{\N}{\mathbb N} % for nature number
\newcommand{\I}{\mathcal I}
\newcommand{\RR}{\mathcal R}
% self defined vars
\newcommand{\titleinfo}{Holomorphic Fields of Hilbert Spaces 
in Finite von Neumann algebra}
\newcommand{\authorinfo}{Wei Yuan and Wenming Wu} 

\linespread{1.05}
\pagestyle{fancyplain}
\fancyhf{}
\fancyhf[HRE,HLO]{\authorinfo}
\fancyhf[FC]{\thepage}

\begin{document}

\title{\LARGE\textbf{\titleinfo}} 
\author{\large\textsc{\authorinfo}} 
\address{AMSS}  
\email{}

\date{}

%\renewcommand{\abstractname}{Summary} 
\begin{abstract}
Enter abstract here
\end{abstract}

% Keywords
\subjclass[2010]{Primary ; Secondary }
\keywords{}
\thanks{}
\maketitle


\section{Introduction}
Let $\HHH$ be a Hilbert space, $Proj(\HHH)$ be the set of self-adjoint projections
and $Uint(\HHH)$ be the set of unitary operators. We will routinely identify a 
closed subspace with its associated orthogonal projection in $B(\HHH)$.
For a set $\LLL$ of orthogonal projections in $Proj(\HHH)$, we denote by $Alg\LLL$ the set of all bounded linear operators on $\HHH$ leaving each element
in $\LLL$ invariant. Then $Alg\LLL$ is an unital weak-operator closed subalgebra of 
$B(\HHH)$. Similarly, for a subset $\mathcal{S}$ of
$B(\HHH)$, let $Lat\mathcal{S}$ be the set of  invariant projections for every 
operators in $\mathcal{S}$. Then $Lat\mathcal{S}$ is a
strong-operator closed lattice of projections. A subalgebra $\mathcal{A}$ of 
$B(\HHH)$ is said to be reflexive if
$\mathcal{A}=Alg\Lat\mathcal{A}$, similarly a lattice $\LLL$ of projections 
is reflexive if $\LLL=LatAlg\LLL$.

\section{Hilbert fields over sphere}
Let 
\begin{align*}
   \II_1 = \begin{pmatrix}
       0 & T_1 \\
       0 & I
   \end{pmatrix} \mbox{ and }
    \II_2 = \begin{pmatrix}
       0 & T_2 \\
       0 & I
   \end{pmatrix}
\end{align*}
be two (unbounded) idempotents acting on $\HHH \oplus \HHH$, 
where $T_1$ and $T_2$ are two closed operators affiliated with a 
finite von Neumann subalgebra of $\BBB(\HHH)$. 
The ranges of $\II_{1}$ and $\II_{2}$ are closed subspaces of 
$\HHH \oplus \HHH$.

Recall that the set of closed, densely defined operators 
affiliated with a finite von Neumann algebra $\AAA$
admits algebraic operations of addition and multiplication.
In another word, the set is an unital * algebra (cf. \cite{MV2}), 
and the elements in it can be manipulated as if they were bounded
operators. In the rest of the paper, we will repeatedly make use this fact 
without mentioning it explicitly. For a elegant treatment of this subject, 
we refer readers to \cite{Zhe}. 

Let $Q(\infty)=Ker(\II_1)$, $Q(0)=Ran(\II_{1})$ and $Q(-1)=Ran(\II_{2})$
be there trace half projections in $\AAA \otimes M_{2}(\C)$.
Note that
\begin{align*}
   \Alg(Q(\infty), Q(0), Q(-1)) = \{A | A\II_i \subset \II_{i}A, i = 1,2\}.
\end{align*}

By conjugating a unitary, we could assume that
\begin{align*}
   T_1 - T_2  
   = \begin{pmatrix}
      K & 0\\
      0 & 0
   \end{pmatrix} 
\end{align*}
where $K \geq 0$ and $K$ has a closed inverse. 

We also write $T_1$ as
$\begin{pmatrix}
       T_{11} & T_{12} \\
       T_{21} & T_{22}
\end{pmatrix}$. 
It is not hard to check that $\Alg(Q(\infty), Q(0), Q(-1))$ contains the
following elements 
\begin{align*}
    &\begin{pmatrix}
    A_1 & 0 & T_{11}K^{-1}A_{1}K-A_{1}T_{11}& -A_{1}T_{12}\\
       0 & 0 & T_{21}K^{-1}A_{1}K & 0\\
       0 & 0 & K^{-1}A_{1}K & 0 \\
       0 & 0 & 0 & 0
   \end{pmatrix} \qquad  
   \begin{pmatrix}
   0 & 0 & 0 & 0\\ 
   0 & A_{2} & -A_{2}T_{21} & -A_{2}T_{22} \\
   0 & 0 & 0 & 0\\ 
   0 & 0 & 0 & 0\\ 
   \end{pmatrix}\\
   &\begin{pmatrix}
   0 & A_{3} & -A_{3}T_{21} & -A_{3}T_{22}\\ 
   0 & 0 & 0 & 0 \\
   0 & 0 & 0 & 0\\ 
   0 & 0 & 0 & 0 
   \end{pmatrix} \qquad
   \begin{pmatrix}
   0 & 0 & T_{12}D_{1} & 0\\ 
   0 & 0 & T_{22}D_{1} & 0 \\
   0 & 0 & 0 & 0\\ 
   0 & 0 & D_{1} & 0 
   \end{pmatrix} \qquad
   \begin{pmatrix}
   0 & 0 & 0 & T_{12}D_{2}\\ 
   0 & 0 & 0 & T_{22}D_{2} \\
   0 & 0 & 0 & 0\\ 
   0 & 0 & 0 & D_{2}\\ 
   \end{pmatrix}
\end{align*}

If $Q \in \Lat(\Alg(Q(\infty), Q(0), Q(-1)))$ such that
$Q \wedge Q(\infty) = 0$ and $Q \vee Q(\infty) = I$, then there is
an idempotent
\begin{align*}
   \II = \begin{pmatrix}
       0 & T_{1} + S_{1} \\
       0 & I 
   \end{pmatrix} 
\end{align*}
such that $\II A = A\II$ for any $A \in \Alg(Q(\infty), Q(0), Q(-1))$. 
This implies that
\begin{align*}
   \begin{pmatrix}
      0 & 0\\
      0 & I
   \end{pmatrix}
   \begin{pmatrix}
       S_{11} & S_{12}\\
       S_{21} & S_{22}
   \end{pmatrix} = 0 \mbox{ and }
   \begin{pmatrix}
       S_{11} & S_{12}\\
       S_{21} & S_{22}
   \end{pmatrix}
    \begin{pmatrix}
      0 & 0\\
      0 & I
   \end{pmatrix} = 0.
\end{align*}
Therefore $S_{12}$, $S_{21}$ and $S_{22}$ are all equals $0$.
Since
\begin{align*}
   \begin{pmatrix}
       A_{1} & 0 \\
       0 & 0
   \end{pmatrix} 
   \begin{pmatrix}
       S_{11} & 0 \\
       0 & 0
   \end{pmatrix}
   = 
   \begin{pmatrix}
       S_{11} & 0 \\
       0 & 0
   \end{pmatrix}
   \begin{pmatrix}
       K^{-1}A_{1}K & 0 \\
       0 & 0
   \end{pmatrix} \qquad \mbox{ for all $A_{1} \in Ran(K)\AAA Ran(K)$}. 
\end{align*}
This implies that $S = z K$ for some $z \in \C$. We will denote 
this idempotent by $\II(z)$ and its range projection by $Q(z)$. 
Note that $\II_1 = \II(0)$ and $\II_{2} = \II(-1)$.

If $Q \wedge Q(\infty) = 0$ and $Q \vee Q(\infty) \neq I$, then
\begin{align*}
    E \equiv Q \vee Q(\infty) = 
    \begin{pmatrix}
        I & 0 & 0 & 0\\
        0 & I & 0 & 0\\
        0 & 0 & 0 & 0\\
        0 & 0 & 0 & I \\
    \end{pmatrix}.
\end{align*}
Therefore, there must exist a $\beta = (\xi_1, \xi_2, \xi_3, \xi_4)^{T}
\in Q\HHH$ such that $\xi_4 \neq 0$. This implies that
\begin{align*}
    \{(T_{12}\xi, T_{22}\xi, 0, \xi)^{t}| \xi \} \subset Q\HHH.
\end{align*}
Consider the trace of $Q$, we know that 
\begin{align*}
\{(T_{12}\xi, T_{22}\xi, 0, \xi)^{t}| \xi \}= Q\HHH \mbox{, and }
    Q = Ran(
    \begin{pmatrix}
        0 & T_{1}(I-Ran(T_1 - T_2))\\
        0 & (I-Ran(T_1 - T_2)) 
    \end{pmatrix}.
\end{align*}
It is not hard to check that
\begin{align*}
    Q(z_{1}) \wedge Q(z_{2}) =
    \{(T_{12}\xi, T_{22}\xi, 0, \xi)^{t}| \xi \}
\end{align*}
for any $z_1 \neq z_2 (\in \C))$.

If $Q \wedge Q(\infty) \neq 0$, then it is easy to see that
\begin{align*}
   F \equiv Q \wedge Q(\infty) = 
    \begin{pmatrix}
        I & 0 & 0 & 0\\
        0 & 0 & 0 & 0 \\
        0 & 0 & 0 & 0 \\
        0 & 0 & 0 & 0 \\
    \end{pmatrix}.
\end{align*}
If $Q \neq F$, we have 
$E \leq Q \vee Q(\infty)$ and $Q(z_1) \wedge Q(z_2) \leq Q$.
If $Q \vee Q(\infty) = E$, then $\tau(Q) = \frac{1}{2}$. Hence
\begin{align*}
    Q\HHH =
    \{(\xi_1, T_{22}\xi, 0,\xi)^{t}| \xi_1, \xi \} 
    = F \vee (Q(z_1) \wedge Q(z_2)). 
\end{align*}
The last possibility is $Q \vee Q(\infty) = I$. Then there must 
exists a vector $(\xi_1, \xi_2, \xi_3, \xi_4)^{T} \in Q$ such that
$\xi_3 \neq 0$. Note that $\tau(Q) = \frac{1}{2} + \tau(F)$ and
\begin{align*}
    \{(0,T_{21}\xi,\xi,0)^{t} | \xi \} \subset Q\HHH.
\end{align*}
Then it is not hard to see that
\begin{align*}
   Q=\{(\xi_1, T_{21}\xi_{2} + T_{22}\xi_3,\xi_{2},\xi_{3})^{t} 
   | \xi_1, \xi_2, \xi_3 \} = F \vee Q(z) = Q(z_1)\vee Q(z_2)   
\end{align*}
for any $z_1$ and $z_2 \in C$.

\begin{remark}
    $E = Q(\infty) \vee (Q(0) \wedge Q(-1))$ and 
    $F = (Q(0) \vee Q(-1)) \wedge Q(\infty)$.  
\end{remark}

\begin{example}[Tautological line bundle over $\C\mathbb{P}^1$]
Let
\begin{align*}
Q(\infty)&= \left(
        \begin{array}{cc}
          1 & 0 \\
          0 & 0 \\
        \end{array}
      \right),
Q(0)= \left(
        \begin{array}{cc}
          0 & 0 \\
          0 & 1 \\
        \end{array}
      \right),
Q(-1)= \left(
        \begin{array}{cc}
          \frac{1}{2} & -\frac{1}{2} \\
          -\frac{1}{2} & \frac{1}{2} \\
        \end{array}
      \right),
\end{align*}
then 
\begin{align*}
Q(z) = \frac{1}{1+|z|^2}
    \left(
        \begin{array}{cc}
          |z|^2 & z \\
        \overline{z} & 1\\
        \end{array}
      \right).  
\end{align*}
Note that the map $\phi: \C\mathbb{P}^1 \rightarrow S^{2}$ given by
$\phi([z_1, z_2]) = z_1/z_2$ is a homeomorphism. We have
\begin{align*}
\phi^{*}(Q)([z_1, z_2]) = \frac{1}{|z_1|^2+|z_2|^2}
    \left(
        \begin{array}{cc}
            |z_1|^2 & z_1 \overline{z_2} \\
        \overline{z_1}z_2 & |z_2|^2\\
        \end{array}
      \right) 
\end{align*}
is a line bundle over $\C\mathbb{P}^{1}$. 
It is the tautological line bundle (or Hopf bundle) over $\C\mathbb{P}^1$.
\end{example}

If $T_1$ and $T_2$ are not bouneded, then the connected 
component of $\Lat(\Alg(Q(\infty), Q(0), Q(-1))$ is not a Hilbert bundle.

\begin{definition}[Definition 2.2.1 and Definition 2.2.2 in \cite{LR}]
   Let $X$ be a smooth manifold. A field of Hilbert spaces in a map
   $p: \HHH \rightarrow X$, with each fiber $\HHH_{x} = p^{-1}(x)$
   endowed with the structure of a Hilbert space. A smooth structure on
   a field $\HHH \rightarrow X$ of Hilbert spaces in given by spcifiying
   a set $\Gamma^{\infty}$ of sections of $\HHH$, closed under
   addition and under multiplicatin by elements of $C^{\infty}(X)$,
   and linear operators $\nabla_\xi: \Gamma^{\infty} \rightarrow 
   \Gamma^{\infty}$ for each $\xi \in Vect(X)$, such that for 
   $\xi$, $\eta \in Vect(X)$, $f \in C^{\infty}(X)$ and
   $\varphi$, $\psi \in \Gamma^{\infty}$
   \begin{align}
       &\nabla_{\xi+\eta} = \nabla_{\xi} + \nabla{\eta},
       \nabla_{f\xi} = f\nabla_{\xi}, 
       \nabla_{\xi}(f\varphi)= (\xi f)\varphi + f\nabla_{\xi}\varphi;\\
       & \langle \varphi, \psi \rangle \in C^{\infty}(X) \mbox{ and }
       \xi \langle \varphi, \psi \rangle 
       = \langle \nabla_{\xi} \varphi, \pi \rangle +  
       \langle \varphi, \nabla_{\xi} \pi \rangle; \\
       & \{ \varphi(x): \varphi \in \Gamma^{\infty}\} \subset \HHH_{x} 
       \mbox{ is dense, for all } x \in X.
   \end{align}
\end{definition}

Let $Proj(1,\HHH)$ denote the set of projections in $\BBB(\HHH)$ such
that $dim P\HHH = 1$ for any $P \in Proj(1, \HHH)$.
Altougth $Proj(1, \HHH)$ is only a real analytic manifold, 
there exists a canonical complex structure on $Proj(1, \HHH)$.

In general, an almost complex structure on a Banach manifold $X$ is a 
smooth tensor field $J$ on $X$ which associates to each $x \in X$ a 
linear operator $J$ on the tangent space $T_{x}(X)$ to $X$ at $x$ 
which satisfies $J_{x}(J\xi) = -\xi$ for all $\xi \in T_{x}(X)$. 
An almost complex structure $J$ on $X$ is said to be a complex 
structure on $X$ if and only if, given any point $x \in X$, there exists
an open neighbourhood $\Omega$ of $x$ in $X$ and a smooth map
$\Phi: \Omega \rightarrow B$, where $B$ is some complex Banach
space which satisfies the following conditions:
\begin{enumerate}
    \item $\Phi$ maps $\Omega$ diffeomorphically onto some open set
        in $B$;
    \item for each $x \in \Omega$ the derivative $(d\Phi)_{x}: 
        T_{m}(\Omega) \rightarrow B$ of the map $\Omega$ at $x$
        satisfies $(d\Phi)_{m}(J_{m}\xi) = i(d\Phi)_{m}\xi$ for all
        $\xi \in T_{m}(X)$. 
\end{enumerate}

We will dsecibe the complex structure of $Proj(1, \HHH)$. 
Fixed a unit vector $\beta$ in $P\HHH$,
it easy to see that the tangent space $T_{P}Proj(1, \HHH)$ at $P$ is 
\begin{align*}
    T_{P}Proj(1, \HHH) &=
    \{H | H^{*}=H, PHP = 0, (I-P)H(I-P)=0, H \in \BBB(\HHH)\} \\
    & = \{\Ket{\xi}\Bra{\beta} + (\Ket{\xi}\Bra{\beta})^{*} 
    | \xi \in (I-P)\HHH \}
    \cong (I-P)\HHH.
\end{align*}
Let
\begin{align*}
    J_{P}: \{\Ket{\xi}\Bra{\beta} + (\Ket{\xi}\Bra{\beta})^{*} 
    &\rightarrow \Ket{i \xi}\Bra{\beta} + (\Ket{i \xi}\Bra{\beta})^{*}
\end{align*}
It is easy to see that $J_{P}^{2} = -I$. Actually the operator is just
the mutiplication of $i$ on the Hilbert space $(I-P)\HHH$.
Let $\Omega = \{Q | \langle Q\beta, \beta \rangle > 0\}$ and 
\begin{align*}
    \Phi: Q \rightarrow \frac{(I-P)Q\beta}{\langle Q\beta, \beta \rangle}. 
\end{align*}
Then
\begin{align*}
    d\Phi: H \rightarrow H\beta, \qquad \forall H \in T_{P}Proj(1, \HHH).
\end{align*}
Now it obvious that $(d\Phi)_{P}(J_{P}H) = iH\beta = i(d\Phi)_{P}H$
(Much general result is proved for the Flag manifolds in a C$^*$-algebra). 

\begin{remark}
    If we identify the projection with its range, then the complex
    structure descirbed above is the same as the complex structure of
    the 1-dimensional Grassmann manifold $Gr(1,\HHH)$,
    the set of all 1-dimensional subspaces of $\HHH$, as the
    complex projective space.
\end{remark}

\begin{definition}
   Suppose that $\Phi: X \rightarrow Proj(1, \HHH)$ is a 
   holomorphic embedding, where $X$ is a complex manifold. 
   We say the embedding is locally constant with respect to
   a von Neumann algebra $\AAA$ if for any $x \in X$
\end{definition}

\begin{definition} \label{def_f_2}
   Let $\AAA$ be a von Neumann algebra acting on a separable Hilbert space
   $\HHH$. Suppose that $\Phi: X \rightarrow Proj(1, \HHH)$ is a 
   holomorphic embeding, where $X$ is a complex manifold. Then
   the field of Hilbert spaces induce by the holomorphic embeding
   and $\AAA$ is 
   \begin{align*}
       \F(\Phi, \AAA) \equiv 
       \amalg_{x \in X} \bigvee_{U \in U(\AAA)}U^{*}\Phi(x)U.  
   \end{align*}
\end{definition}

The following fact is immediate from the \cref{def_f_2}.

\begin{proposition}
    Let $\AAA$ be a von Neumann algebra acting on a separable Hilbert
    space $\HHH$ and $\Phi: X \rightarrow Proj(1,\HHH)$ is a
    holomorphic embeding of a complex manifold. Then for any $x \in X$,
    $\F(\Phi, \AAA)(x) \in \AAA '$.
\end{proposition}



\section{Conclusion}



%-----------------------------------------------------------------------------------
%BIBLIOGRAPHY
%-----------------------------------------------------------------------------------
\begin{thebibliography}{9}
\bibitem{RK} R. Kadison and John R. Ringrose, {\em Fundamentals of the theory of Operator Algebras},  {\bf Volume II } (1997)

\bibitem{FK} B. Fuglede and R. Kadison, {\em Determinant Theory in Finite Factors}, The Annals of Mathematics, Second Series, Vol. 55, No. 3 (1952), 520-530

\bibitem{Ta} M. Takesaki, {\em Theory of Operator Algebras}, {\bf Volume I}, Encyclopedia of Mathematical Sciences, vol 124, Springer-Verlag, Berlin (2002)

\bibitem{LP} R. H. Levene, S. C. Power {\em Manifold of Hilbert space 
    projection}, Proc. London Math. Soc. 1-25 (2009)


\bibitem{Mo} Mohan Ravichandran, {University of New Hampshire Ph.D. Thesis} (2009) 
\bibitem{GYII} L. Ge and W. Yuan, {\em Kadison-Singer algebras,
II---General case,} to appear, 2009

\bibitem{HF} Uffe Haagerup and Flemming Larsen, {\em Brown's Spectral Distribution Measure for
R-diagonal Elements in Finite von Neumann Algebras,}, 1999

\bibitem{HH} Uffe Haagerup and Hanne Schultz, {\em Brown Measures of Unbounded Operators Affiliated with
 Finite von Neumann Algebras,}, 1999

\bibitem{HV} Hari Bercovici and Dan Voiculescu, {\em Free Convolution of Measures with Unbounded Support, }

\bibitem{KS} $K.S.K\ddot{O}LBIG$, {\em On The Value Of A Logarithmic-Trigonometric Integral}, BIT 11(1971), 21-28

\bibitem{MR} Mehmet Koca, Ramazan Koc and Muataz Al-Barwani, {\em Breaking SO(3) into its closed subgroups by Higgs mechanism},
J.Phys.A: Math.Gen. 30(1997) 2109-2125

\bibitem{BA} Beardon, Alan F, {\em The Geometry of Discrete Groups}, New York: Springer-Verlag GTM 91.

\bibitem{AR} Alexandru Nica and Roland Speicher, {\em Lectures on the Combinatorics of Free Probability},
London Mathematical Society Lecture Note Series:335

\bibitem{SS} Elias M. Stein and Rami Shakarchi, {\em complex analysis}, Priceton lectures in analysis II.

\bibitem{Hou} Chengjun Hou and Wei Yuan, {\em Minimal Generating Reflexive Lattices of Projections in Finite von Neumann Algebras}

\bibitem{WW} W.Wu and W. Yuan, {\em On generators of abelian 
        Kadison–Singer algebras in matrix algebras} 
        Linear Algebra and its Applications, Vol 440, 2014, Pages 197–205.

\bibitem{LU} R. C. Lyndon and J. L. Ullman {\em Groups of Elliptic Linear Fractional Transformations}, Proceedings of the American Mathematical Society,   Vol. 18, No. 6, (1967) 1119-1124

\bibitem{SA} S. Sakai, {\em On automorphism groups of II$_1$-factors}, T\^{o}koku Math. J. (2) 26 (1974), 423-430.

\bibitem{WA} William L. Green and Anthony To-Ming Lau {\em Strong finite von Neumann algebras }, Math. Scand. 40 (1977), 105- 112.

\bibitem{MRM} M. Koca, R. Koc and M. Al-Barwani {\em Breaking SO(3) into its closed subgroups by Higgs mechanism}, J.Phys. A: Math. Gen. 30 (1997) 2109-2125

\bibitem{VDN} D. Voiculescu, K. Dykema and A. Nica, {\em Free Random Variables}, CMR Monograph Series 1, American Mathematical Society (1992)

\bibitem{NS} A. Nica and R. Speicher, {\em Lectures on the Combinatorics of Free Probability}, Londom Mathematical Society Lecture Note Series: 335 (2006)

\bibitem{MV2} F. J. Murray and J. von Neumann, {\em On rings of operators, II}, Trans. Amer. Math. Soc. 41 (1937), 208 - 248

\bibitem{Zhe} Z. Liu, {\em On Some Mathematical Aspects of The Heisenberg Relation }, Science China series Mathematics: Kadison's proceedings

\bibitem{MK} Masoud Khalkhaili, {\em Basic Noncommutative Geometry}, EMS series of Lectures in Mathematics (2009)

\bibitem{S}
    N. Salinas
    {\em The Grassmann manifold of a C*-algebra, and Hermitian 
    holomorphic bundles}, 
    Operator Theory: Advances and Applications 28, 
    Birkhaiiser Verlag Basel, 1998, 267-289.

\bibitem{LR} L. Lempert and R. Szoke, {\em Direct Images, Fields of 
    Hilbert spaces, and Geometric Quantization}

\bibitem{WI} D. R. Wilkings, {\em The Grassmann manifold of a 
    C$^*$-algebra}, Proc. of the Royal Irish Acad. 90A, 1990, 99-116.

\bibitem{MN} M. Martin and N. Salinas {\em The Canonical Complex Structure
    of Flag Manifolds in a C$^{*}$-algebra},
    Operator Theory: Advances and Applications, Vol.104, 1998,
    173-187.
\end{thebibliography}
%-----------------------------------------------------------------------------------

\end{document}
