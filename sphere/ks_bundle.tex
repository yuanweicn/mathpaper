\documentclass[a4paper,10pt]{amsart}

\usepackage[protrusion=true,expansion=true]{microtype} 
\usepackage{fancyhdr}
\usepackage[utf8]{inputenc}
\usepackage{graphicx} 
\usepackage{wrapfig} 

\usepackage{mathpazo}
\usepackage[T1]{fontenc}
\usepackage{amsmath}
\usepackage{amssymb}
\usepackage{hyperref}
\usepackage{cleveref}
\usepackage{comment}
\usepackage{color}
\usepackage{enumerate}

\newtheorem{example}{Example}[section]
\newtheorem{theorem}{Theorem}[section]
\newtheorem{proposition}{Proposition}[section]
\newtheorem{corollary}{Corollary}[section]
\newtheorem{definition}{Definition}[section]
\newtheorem{lemma}{Lemma}[section]
\newtheorem{remark}{Remark}[section]
\newtheorem{question}{Question}[section]

\crefname{lemma}{Lemma}{lemmas}
\crefname{remark}{Remark}{remark}
\crefname{corollary}{Corollary}{corollary}
\crefname{theorem}{Theorem}{theorem}
\crefname{example}{Example}{example}
\crefname{definition}{Definition}{definition}

\newcommand{\AAA}{\mathfrak A}
\newcommand{\BBB}{\mathcal B}
\newcommand{\CCC}{\mathcal C}
\newcommand{\HHH}{\mathcal H} %for Hilbert space
\newcommand{\LLL}{\mathcal L} % for lattice
\newcommand{\MMM}{\mathcal M}


\newcommand{\Lat}{\mathcal Lat}
\newcommand{\Alg}{\mathcal Alg}
\newcommand{\tr}{\tau}
\newcommand{\C}{\mathbb C} %for complex number
\newcommand{\R}{\mathbb R}  %for real number
\newcommand{\Z}{\mathbb Z} %for integer
\newcommand{\N}{\mathbb N} % for nature number
\newcommand{\I}{\mathcal I}
\newcommand{\RR}{\mathcal R}
% self defined vars
\newcommand{\titleinfo}{Manifolds of Hilbert space homeomorphic to
    sphere in Finite von Neumann algebra}
\newcommand{\authorinfo}{Wei Yuan and Wenming Wu} 

\linespread{1.05}
\pagestyle{fancyplain}
\fancyhf{}
\fancyhf[HRE,HLO]{\authorinfo}
\fancyhf[FC]{\thepage}

\begin{document}

\title{\LARGE\textbf{\titleinfo}} 
\author{\large\textsc{\authorinfo}} 
\address{AMSS}  
\email{}

\date{}

%\renewcommand{\abstractname}{Summary} 
\begin{abstract}
Enter abstract here
\end{abstract}

% Keywords
\subjclass[2010]{Primary ; Secondary }
\keywords{}
\thanks{}
\maketitle


\section{Introduction}
Let $\HHH$ be a Hilbert space, $Proj(\HHH)$ be the set of self-adjoint projections
and $Uint(\HHH)$ be the set of unitary operators. We will routinely identify a 
closed subspace with its associated orthogonal projection in $B(\HHH)$.
For a set $\LLL$ of orthogonal projections in $Proj(\HHH)$, we denote by $Alg\LLL$ the set of all bounded linear operators on $\HHH$ leaving each element
in $\LLL$ invariant. Then $Alg\LLL$ is an unital weak-operator closed subalgebra of 
$B(\HHH)$. Similarly, for a subset $\mathcal{S}$ of
$B(\HHH)$, let $Lat\mathcal{S}$ be the set of  invariant projections for every 
operators in $\mathcal{S}$. Then $Lat\mathcal{S}$ is a
strong-operator closed lattice of projections. A subalgebra $\mathcal{A}$ of 
$B(\HHH)$ is said to be reflexive if
$\mathcal{A}=Alg\Lat\mathcal{A}$, similarly a lattice $\LLL$ of projections 
is reflexive if $\LLL=LatAlg\LLL$.

\section{Hilbert fields over sphere}
Suppose $Q(\infty)$, $Q(0)$ and $Q(-1)$ are three projections in a finite 
von Neumann algebra $\AAA$, and $Q(\infty)$, $Q(0)$ and $Q(-1)$ are 
in general position, i.e., the intersection of any two is zero and 
the join of 
any two is I, then $\AAA \cong Q(\infty)\AAA Q(\infty) \otimes M_{2}(\C)$
\cite[Proposition 2.4]{Hou}.
Moreover we can write $Q(\infty)$, $Q(0)$ and $Q(-1)$ in terms 
of $2 \times 2$ operator matrices (with respect to the canonical matrix 
units in $I \otimes M_{2}(\C)$) as follows:
\begin{equation}\label{eq0}
\begin{split}
    Q(\infty)&= \left(
        \begin{array}{cc}
          I & 0 \\
          0 & 0 \\
        \end{array}
      \right),
      Q(0) = \left(
          \begin{array}{cc}
            H_{1} & \sqrt{H_{1}(I-H_{1})} \\
            \sqrt{H_{1}(I-H_{1})} & I-H_{1} \\
          \end{array}
        \right) , \\
        Q(-1)&= \left(
          \begin{array}{cc}
            H_{2} & \sqrt{H_{2}(I-H_{2})}V \\
            V^{*}\sqrt{H_{2}(I-H_{2})} & V^{*}(I-H_{2})V \\
          \end{array}
        \right),
\end{split}
\end{equation}
where $H_i$ is a contractive positive operator in $Q(\infty)\AAA Q(\infty)$ 
such that $Ker(I - H_i) = 0$, $i = 1, 2$, and $V$ is a unitary operator 
in $Q(\infty)\AAA Q(\infty)$. 

In order to describe the invariant subspace lattice 
$\Lat(\Alg(\{Q(\infty), Q(0), Q(-1)\}))$, 
unbounded operator will be used. Let $\AAA$ be a von Neumann algebra, and
$\widetilde{\AAA}$ be the set of closed, densely defined operators 
affiliated with $\AAA$. When $\AAA$ is finite, the family of operators 
affiliated with $\AAA$ has remarkable properties, that is the operators in 
$\widetilde{\AAA}$ admits algebraic operations of addition and multiplication.
In another word,  $\widetilde{\AAA}$ is an unital * algebra (cf. \cite{MV2}), 
and the elements in $\widetilde{\AAA} $ can be manipulated as if they were bounded
operators. In the rest of the paper, we will repeatedly make use this fact 
without mentioning it explicitly. For a elegant treatment of this subject, 
we refer readers to \cite{Zhe}. 

\begin{theorem}[\cite{Hou}, Theorem 2.1] \label{1thm1}
With the above notation and assumptions, we have 
$\Lat(\Alg(\{Q(\infty), Q(0), Q(-1)\})) \setminus \{0, I\}$ 
endowed with the strong operator topology
is homeomorphic to $\C \cup \{\infty\} (\cong S^2)$ with
the homeomorphism given by
\begin{align*}
    \rho[Q(\infty), Q(0), Q(-1)](z) = \left(
     \begin{array}{cc}
      K_{z} & \sqrt{K_{z}(I-K_{z})}U_{z} \\
      U_{z}^{*}\sqrt{K_{z}(I-K_{z})} & U_{z}^{*}(I-K_{z})U_{z} \\
  \end{array}
\right) , \qquad \forall z \in \C,
\end{align*}
and $\rho[Q(\infty), Q(0), Q(-1)](\infty) 
= Q(\infty)$, 
where $K_z$ and $U_z$ are uniquely determined by the following polar decomposition:
\begin{equation}\label{eq1}
\begin{split}
&\sqrt{K_{z}(I-K_{z})^{-1}}U_{z} = (1+z)\sqrt{H_{1}(I-H_{1})^{-1}}-
                                   z\sqrt{H_{2}(I-H_{2})^{-1}}V \\
                                 &=zS + \sqrt{H_{1}(I-H_{1})^{-1}} \qquad 
                                   (S = \sqrt{H_{1}(I-H_{1})^{-1}}- 
                                   \sqrt{H_{2}(I-H_{2})^{-1}}V) .
\end{split}
\end{equation}
Moreover,  the reflexive lattice 
$\Lat(\Alg(\{Q(\infty), Q(0), Q(-1)\}))$ 
can be determined by arbitrary three nontrivial projections
$($not $0$ or I$)$ in it.
\end{theorem}

\begin{remark}\label{1re1}
Since $\sqrt{K_{0}(I-K_{0})^{-1}}U_{0} =
0\times S + \sqrt{H_{1}(I-H_{1})^{-1}}$ implies $K_{0} = H_{0}$ 
and $U_{0} = I$, 
we have $\rho[Q(\infty), Q(0), Q(-1)](0) = Q(0)$. Similarly, 
$\rho[Q(\infty), Q(0), Q(-1)](-1) = Q(-1)$. Therefore, we will also use 
$Q(z)$ to denote $\rho[Q(\infty), Q(0), Q(-1)](z)$ throughout the 
rest of this paper.
\end{remark}

\begin{definition}[Definition 2.1. (i) in \cite{LP}]\label{1exm1}
    A topological subspace manifold in $B(\HHH)$ of dimension $n$ is a set
    $\mathcal{M} \subset Proj(\HHH)$, considered with the relative strong 
    operator topology, which is locally homeomorphic to $\R^{n}$.
\end{definition}

Given any there projections $Q(\infty), Q(0), Q(-1)$ in a 
finite von Neumann
algebra, we have $\Lat(\Alg(\{Q(\infty), Q(0), Q(-1)\})) \setminus \{0, I\}$
is a topological subspace manifold of dimension $2$ 
if $Q(\infty), Q(0), Q(-1)$
are in general position by \cref{1thm1}.

\begin{example}[Tautological line bundle over $\C\mathbb{P}^1$]
Let
\begin{align*}
Q(\infty)&= \left(
        \begin{array}{cc}
          1 & 0 \\
          0 & 0 \\
        \end{array}
      \right),
Q(0)= \left(
        \begin{array}{cc}
          0 & 0 \\
          0 & 1 \\
        \end{array}
      \right),
Q(-1)= \left(
        \begin{array}{cc}
          \frac{1}{2} & -\frac{1}{2} \\
          -\frac{1}{2} & \frac{1}{2} \\
        \end{array}
      \right),
\end{align*}
then 
\begin{align*}
Q(z) = \frac{1}{1+|z|^2}
    \left(
        \begin{array}{cc}
          |z|^2 & z \\
        \overline{z} & 1\\
        \end{array}
      \right).  
\end{align*}
Note that the map $\phi: \C\mathbb{P}^1 \rightarrow S^{2}$ given by
$\phi([z_1, z_2]) = z_1/z_2$ is a homeomorphism. We have
\begin{align*}
\phi^{*}(Q)([z_1, z_2]) = \frac{1}{|z_1|^2+|z_2|^2}
    \left(
        \begin{array}{cc}
            |z_1|^2 & z_1 \overline{z_2} \\
        \overline{z_1}z_2 & |z_2|^2\\
        \end{array}
      \right) 
\end{align*}
is a line bundle over $\C\mathbb{P}^{1}$. 
It is actually the tautological line  bundle over $\C\mathbb{P}^1$.
\end{example}

In this paper will study the subgroup of automorphisms of $\AAA$ which leaves $Lat(\Alg(\{Q(\infty), Q(0), Q(-1)\}))$ invariant.
The rest of this paper is organized as follows.  
Next, we point out that the above result naturally 
induce a coordinate chart of 
the reflexive lattice generated by a double triangle lattice of projections (exclude $0$ and $I$) in a finite von Neumann algebra.
In section 3, we prove that the transition maps between the charts are  
M\"{o}bius transformations.
We then study the $\LLL$-invariant subgroup of automorphisms of  a von Neumann algebra $\AAA$, where $\LLL$ is a
reflexive subspace lattice contained in $\AAA$. In particular, 
we show if $\AAA$ is a finite factor or finite dimensional, and $\LLL$
is generated by a double triangle lattice of projections in 
$\AAA$, then the $\LLL$-invariant automorphism group is 
homeomorphic to a closed 
subgroup of $SO(3)$(Corollary \ref{2cor2}). In section 4, we compute the $\LLL$-invariant automorphism group when $\AAA $ 
is the interpolated free group factor $L_{F_{\frac{3}{2}}}$, 
and $\LLL$ is determined by the three free projection that 
generate $L_{F_{\frac{3}{2}}}$. 
In the appendix, we give a detailed proof of the following fact: 
If $\LLL$ is a reflexive lattice in a von Neumann algebra 
$\AAA$, and $\varphi$ is a *-isomorphism of 
$\AAA$, then $\varphi(\LLL)$ is also reflexive.\newline


Since any three projections in 
$\Lat(\Alg(\{Q(\infty), Q(0), Q(-1)\})) \setminus \{0, I\}$ 
are in the general position, form Theorem \cref{1thm1}, 
we have the following corollary.
\begin{corollary}\label{1cor1}
With the notations in the Theorem \ref{1thm1} and the Remark \ref{1re1}, 
suppose $Q(z_1)$, $Q(z_2)$ and $Q(z_3)$ are three nontrivial 
projections in $\Lat(\Alg(\{Q(\infty), Q(0), Q(-1)\}))$,
then there is a homeomorphism $\rho[Q(z_1), Q(z_2), Q(z_3)]$ form $S^2$ 
onto $\Lat(\Alg(\{Q(\infty), Q(0), Q(-1)\})) \setminus \{0, I\}$ 
such that $\rho[Q(z_1), Q(z_2), Q(z_3)](z)$ is determined by the 
following relation:
\begin{equation}\label{eq2}
\begin{split}
(I - Q(z_1)\rho[Q(z_1), Q(z_2), Q(z_3)]&(z)Q_{z_1})^{-1}
    [Q(z_1)\rho[Q(z_1), Q(z_2), Q(z_3)](z)(I-Q(z_1))]\\
=(1 &+z)(I - Q(z_1)Q(z_2)Q(z_1))^{-1}[Q(z_1)Q(z_2)(I-Q(z_1))] \\
 & - z(I - Q(z_1)Q(z_3)Q(z_1))^{-1}[Q(z_1)Q(z_3)(I-Q(z_1))].
 \end{split}
\end{equation}
By (\ref{eq2}), $\rho[Q(z_1), Q(z_2), Q(z_3)](\infty) = 
Q(z_1)$, $\rho[Q(z_1), Q(z_2), Q(z_3)](0) = Q(z_2)$ 
and $\rho[Q(z_1), Q(z_2), Q(z_3)](-1) = Q(z_3)$. 
\end{corollary}
  

The inverse of the homeomorphism $\rho[Q(z_1), Q(z_2), Q(z_3)]$ 
in the Corollary \ref{1cor1} actually gives a coordinate chart of 
$(\Lat(\Alg(\{Q(\infty), Q(0), Q(-1)\})) \setminus \{0, I,  Q(z_1)\}$, 
$\rho[Q(z_1), Q(z_2), Q(z_3)]^{-1})$ of 
$\Lat(\Alg(\{Q(\infty), Q(0), Q(-1)\})) \setminus \{0, I \}$.
So $\Lat(\Alg(\{Q(\infty), Q(0), Q(-1)\})) \setminus \{0, I\}$ 
is a 2-dimensional (topological) manifold with atlas
$\{\rho[Q(z_1), Q(z_2), Q(z_3)]^{-1} | 
z_1, z_2, z_3 \in \C \cup \{\infty \} \}$. 
In the next section, we will determine the transition maps between 
the charts in this atlas.

\section{Determined by two idempotent}

Let $Q(\infty)$, $Q(0)$ and $Q(-1)$ be trace half projections in 
a finite von Neumann algebra $\AAA$. If $Q(\infty) \wedge Q(i) = 0$,
$i = 0, -1$, then we have there exists a two closed idempotents 
$S_1$ and $S_2$ such that $S_1$ and $S_2$ are affiliated with $\AAA$ and
$\Alg(Q(\infty), Q(0), Q(-1)) = \{T | TS_i \subset S_{i}T, i = 1,2\}$
by Lemma 2.3 in \cite{WW}.

Now assume that $T_1$ and $T_2$ be two closed operator affiliated with
a finite von Neumann algebra $\AAA$. Consider two idempotents in 
$\AAA \otimes M_{2}(\C)$:
\begin{align*}
   S_1 = \begin{pmatrix}
       I & T_1\\
       0 & 0
   \end{pmatrix}, \qquad
   S_2 = \begin{pmatrix}
       I & T_2 \\
       0 & 0
   \end{pmatrix}.
\end{align*}
Let $Q(\infty) = Ran(S_1)$, $Q(0) = Ker(S_1)$ and $Q(-1)=Ker(S_2)$ be 
three trace half projections. Note that 
\begin{align*}
   \Alg(Q(\infty), Q(0), Q(-1)) = \{T | TS_i \subset S_{i}T, i = 1,2\}.
\end{align*}
By conjugating a unitary, we could assume that
\begin{align*}
   T_1 - T_2  
   = \begin{pmatrix}
      K & 0\\
      0 & 0
   \end{pmatrix} 
\end{align*}
where $K \geq 0$ and $K$ has a closed inverse. We also write $T_1$ as
$\begin{pmatrix}
       T_{11} & T_{12} \\
       T_{21} & T_{22}
\end{pmatrix}$. 

It is not hard to check that $\Alg(Q(\infty), Q(0), Q(-1))$ contains the
following elements 
\begin{align*}
    &\begin{pmatrix}
    A_1 & 0 & A_{1} T_{11}-T_{11}K^{-1}A_{1}K & A_{1}T_{12}\\
       0 & 0 & -T_{21}K^{-1}A_{1}K & 0\\
       0 & 0 & K^{-1}A_{1}K & 0 \\
       0 & 0 & 0 & 0
   \end{pmatrix} \qquad  
   \begin{pmatrix}
   0 & 0 & 0 & 0\\ 
   0 & A_{2} & A_{2}T_{21} & A_{2}T_{22} \\
   0 & 0 & 0 & 0\\ 
   0 & 0 & 0 & 0\\ 
   \end{pmatrix}\\
   &\begin{pmatrix}
   0 & A_{3} & A_{3}T_{21} & A_{3}T_{22}\\ 
   0 & 0 & 0 & 0 \\
   0 & 0 & 0 & 0\\ 
   0 & 0 & 0 & 0 
   \end{pmatrix} \qquad
   \begin{pmatrix}
   0 & 0 &  -T_{12}D_{1} & 0\\ 
   0 & 0 & -T_{22}D_{1} & 0 \\
   0 & 0 & 0 & 0\\ 
   0 & 0 & D_{1} & 0 
   \end{pmatrix} \qquad
   \begin{pmatrix}
   0 & 0 & 0 & -T_{12}D_{2}\\ 
   0 & 0 & 0 & -T_{22}D_{2} \\
   0 & 0 & 0 & 0\\ 
   0 & 0 & 0 & D_{2}\\ 
   \end{pmatrix}
\end{align*}
If $Q \in \Lat(\Alg(Q(\infty), Q(0), Q(-1)))$ such that
$Q \wedge Q(\infty) = 0$ and $Q \wedge Q(\infty) = I$, then there is
a idempotent
\begin{align*}
   S = \begin{pmatrix}
       I & T_{1} + S_{1} \\
       0 & 0
   \end{pmatrix} 
\end{align*}
such that $SA = AS$ for any $A \in \Alg(Q(\infty), Q(0), Q(-1))$. 
This implies that
\begin{align*}
   \begin{pmatrix}
      0 & 0\\
      0 & I
   \end{pmatrix}
   \begin{pmatrix}
       S_{11} & S_{12}\\
       S_{21} & S_{22}
   \end{pmatrix} = 0 \mbox{ and }
   \begin{pmatrix}
       S_{11} & S_{12}\\
       S_{21} & S_{22}
   \end{pmatrix}
    \begin{pmatrix}
      0 & 0\\
      0 & I
   \end{pmatrix} = 0.
\end{align*}
Therefore $S_{12}$, $S_{21}$ and $S_{22}$ are all equals $0$.
Since
\begin{align*}
   \begin{pmatrix}
       A_{1} & 0 \\
       0 & 0
   \end{pmatrix} 
   \begin{pmatrix}
       S_{11} & 0 \\
       0 & 0
   \end{pmatrix}
   = 
   \begin{pmatrix}
       S_{11} & 0 \\
       0 & 0
   \end{pmatrix}
   \begin{pmatrix}
       K^{-1}A_{1}K & 0 \\
       0 & 0
   \end{pmatrix} \qquad \mbox{ for all $A_{1} \in Ran(K)\AAA Ran(K)$}. 
\end{align*}
This implies that $S = z K$ for some $z \in \C$. We will denote 
this projection by $Q(z)$.

If $Q \wedge Q(\infty) = 0$ and $Q \vee Q(\infty) \neq I$, then
\begin{align*}
    Q \vee Q(\infty) = 
    \begin{pmatrix}
        I & 0 & 0 & 0\\
        0 & I & 0 & 0\\
        0 & 0 & 0 & 0\\
        0 & 0 & 0 & I \\
    \end{pmatrix} = E.
\end{align*}
Therefore, there must exist a $\beta = (\xi_1, \xi_2, \xi_3, \xi_4)^{T}
\in Q\HHH$ such that $\xi_4 \neq 0$. This implies that
\begin{align*}
    \{ \begin{pmatrix}
            -T_{12}\xi\\
            -T_{22}\xi\\
            0\\
            \xi
    \end{pmatrix} | \xi \} \subset Q\HHH.
\end{align*}
Consider the trace of $Q$, we know that 
\begin{align*}
    \{ \begin{pmatrix}
            -T_{12}\xi\\
            -T_{22}\xi\\
            0\\
            \xi
    \end{pmatrix} | \xi \} = Q\HHH.
\end{align*}
It is not hard to check that
\begin{align*}
    Q(z_{1}) \wedge Q(z_{2}) =
    \{ \begin{pmatrix}
            -T_{12}\xi\\
            -T_{22}\xi\\
            0\\
            \xi
    \end{pmatrix} | \xi \} 
\end{align*}
for any $z_1$ and $z_2 \in \C$.

If $Q \wedge Q(\infty) \neq 0$, then it is easy to see that
\begin{align*}
    Q \wedge Q(\infty) = 
    \begin{pmatrix}
        I & 0 & 0 & 0\\
        0 & 0 & 0 & 0 \\
        0 & 0 & 0 & 0 \\
        0 & 0 & 0 & 0 \\
    \end{pmatrix} = F.
\end{align*}
Since $E \leq Q \vee Q(\infty)$, $Q(z_1) \wedge Q(z_2) \leq Q$.
If $Q \vee Q(\infty) = E$, then $\tau(Q) = \frac{1}{2}$. Hence
\begin{align*}
    Q =
    \{ \begin{pmatrix}
            \xi_1\\
            -T_{22}\xi\\
            0\\
            \xi
    \end{pmatrix} | \xi_1, \xi \} = F \vee Q(z), \qquad z \in \C. 
\end{align*}
The last possibility is $Q \vee Q(\infty) = I$. Then there must 
exists a vector $(\xi_1, \xi_2, \xi_3, \xi_4)^{T} \in Q$ such that
$\xi_3 \neq 0$. Note that $\tau(Q) = \frac{1}{2} + \tau(F)$ and
\begin{align*}
    \{ \begin{pmatrix}
             0\\
            -T_{21}\xi\\
            \xi\\
            0 
    \end{pmatrix} | \xi \} \subset Q\HHH.
\end{align*}
Then it is not hard to see that
\begin{align*}
    \{ \begin{pmatrix}
             \xi_1 \\
             -T_{21}\xi_{2} - T_{22}\xi_3\\
             \xi_{2}\\
             \xi_{3} 
     \end{pmatrix} | \xi_1, \xi_2, \xi_3 \} = Q = Q(z_1)\vee Q(z_2)   
\end{align*}
for any $z_1$ and $z_2 \in C$.

\begin{remark}
    $E = Q(\infty) \vee (Q(0) \wedge Q(-1))$ and 
    $F = (Q(0) \vee Q(-1)) \wedge Q(\infty)$.  
\end{remark}


\section{Homotopy property of the bundle}

\begin{theorem}
    Let $Q(\infty)$, $Q(0)$, $Q(-1)$ and $P(\infty)$, $P(0)$, $P(-1)$
    be trace half projections in a finite factor $\AAA$. If 
    $Q(i) \wedge Q(j) = 0 = P(i) \wedge P(j)$ and 
    $Q(i) \vee Q(j) = I = P(i) \vee P(j)$, $i\neq j$ and $i$, $j \in
    \{0, -1\}$, then
    there exist continuous paths $P(i, t)$ of trace half projections such
    that $P(i, 0) = P(i)$ and $P(i, 1) = Q(i)$ where $i = \infty, 0, -1$.
    Furthermore, we can request that $P(\infty, t)$, $P(0, t)$ and
    $P(-1, t)$ are in general position and these maps can be 
    extended to be a continuous map form $S^{1} \times [0,t]$ into 
    the set of trace half projections such
    that $P(z, 0) = P(z)$ and $P(z, 1) = Q(z)$.
\end{theorem}

\begin{proof}
    We could assume that $Q(\infty) = \bigl(\begin{smallmatrix}
           I & 0\\
           0 & 0 
    \end{smallmatrix} \bigr)$, $Q(0) =  \bigl(\begin{smallmatrix}
           0 & 0\\
           0 & I 
    \end{smallmatrix} \bigr)$, $Q(-1) =  \bigl(\begin{smallmatrix}
            \frac{1}{2}I & \frac{1}{2}I\\
           \frac{1}{2}I & \frac{1}{2}I 
    \end{smallmatrix} \bigr)$.
    Let $W$ be a unitary such that $W^* P(\infty) W = Q(\infty)$. Since the
    set of unitaries $U(\AAA)$ is connected, there exists
    a path $W(t)$ in $U(\AAA)$ such that $W(0) = I$ and $W(1) = W$. By
    considering $W(t)^{*}P(i)W(t)$, we may assume that 
    $Q(\infty) = P(\infty)$.

    Now assume that
    \begin{align*}
        P(0) &= \begin{pmatrix}
        K_{1} & \sqrt{K_{1}(I-K_{1})}V_{1}\\
        V^{*}_{1}\sqrt{K_{1}(I-K_{1})} & V^{*}_{1}(I-K_1)V_{1}
        \end{pmatrix}, \quad
        P(-1) = \begin{pmatrix}
            K_{2} & \sqrt{K_{2}(I-K_{2})}V_{2}\\
            V_{2}^{*}\sqrt{K_{2}(I-K_{2})} & V_{2}^{*}(I-K_2)V_{2}
        \end{pmatrix}.
    \end{align*}
    Then $P(0)\HHH = \{(\sqrt{\frac{K_1}{I-K_1}}V_{1}\xi, \xi)\}$ and
    $P(-1)\HHH = \{(\sqrt{\frac{K_2}{I-K_2}}V_{2}\xi, \xi)\}$. Let
    \begin{align*}
        P(\infty, t) &= P(\infty) \mbox{, } 
        P(0, t)\HHH = \{(1-t)\sqrt{\frac{K_1}{I-K_1}}V_{1}\xi, \xi)\}
        \mbox{ and }\\ 
        P(-1, t)\HHH &= \{(\sqrt{\frac{K_2}{I-K_2}}V_{2} 
        -t\sqrt{\frac{K_1}{I-K_1}}V_{1})\xi, \xi)\}.
    \end{align*}
    Let 
    \begin{align*}
        P(z,t) = \begin{pmatrix}
            K(z,t) & \sqrt{K(z,t)(I-K(z,t))}V(z,t)\\
            V(z,t)^{*}\sqrt{K(z,t)(I-K(z,t))} & V(z,t)^{*}(I-K(z,t))V(z,t)
        \end{pmatrix}, 
    \end{align*}
where 
\begin{align*}
    \sqrt{\frac{K(z,t)}{I-K(z,t)}}V(z,t) = zS 
    + (1-t)\sqrt{\frac{K_{1}}{I-K_{1}}}V_{1}, \quad  
    S = \sqrt{\frac{K_{1}}{I-K_{1}}}V_{1} -
        \sqrt{\frac{K_{2}}{I-K_{2}}}V_{2}.
\end{align*}
We will show that $P(z,t)$ is continuous.
\begin{align*}
    &\|P(z,t_1) - P(z,t_2)\|_{2}^{2} = 1 - 2\tau(P(z,t_1)P(z,t_2))\\ 
                                   &= tr(K(z,t_1)(K(z,t_1) - K(z,t_2)))\\
                                   & + tr(V(z,t_1)^{*}(I-K(z,t_1))
    V(z,t_1)(V(z,t_2)^{*}K(z,t_2)V(z,t_2) - V(z,t_1)^{*}K(z,t_1)V(z,t_1)))\\
    &+ tr(\sqrt{K(z,t_1)(I-K(z,t_1))}V(z,t_1)(V(z,t_1)^{*}
    \sqrt{K(z,t_1)(I-K(z,t_1))}
    - V(z,t_2)^{*} \sqrt{K(z,t_2)(I-K(z,t_2))}) \\
    &+ tr(V(z,t_1)^{*}\sqrt{K(z,t_1)(I-K(z,t_1))}
    (\sqrt{K(z,t_1)(I-K(z,t_1)))}V(z,t_1)
    - \sqrt{K(z,t_2)(I-K(z,t_2))}V(z,t_2)).
\end{align*} 
Let 
\begin{align*}
    F(z,t) = \frac{K(z,t)}{I-K(z,t)} 
    = |z|^{2}SS^{*} + (1-t)(zSV_{1}^{*}\sqrt{\frac{K_1}{I-K_1}} 
    + \bar{z}\sqrt{\frac{K_1}{I-K_1}}V_{1}S^{*}) 
    + (1-t)^{2}\frac{K_1}{I-K_1}.
\end{align*}
Note that
\begin{align*}
    K(z,t_1) - K(z,t_2) &= 
    (I+F(z,t_2))^{-1}(F(z,t_1)-F(z,t_2))(I+F(z,t_1))^{-1}\\
    &=(t_2 - t_1)(I+F(z,t_2))^{-1}(zSV_{1}^{*}\sqrt{\frac{K_1}{I-K_1}} 
    + \bar{z}\sqrt{\frac{K_1}{I-K_1}}V_{1}S^{*} \\
    & \qquad \qquad \qquad \qquad 
    + (2-t_1 - t_2)\frac{K_1}{I-K_1})(I+F(z,t_1))^{-1}.
\end{align*}
For any $\varepsilon > 0$, let $E$ be a projection such that
$\tau(E) \geq 1-\varepsilon$ and 
\begin{align*}
 \|SV_{1}^{*}\sqrt{\frac{K_1}{I-K_1}}E\| \leq C,   
 \|\sqrt{\frac{K_1}{I-K_1}}V_{1}S^{*}E\| \leq C,
 \|\frac{K_1}{I-K_1}E\| \leq C,
 \|\sqrt{\frac{K_1}{I-K_1}}E\| \leq C,
\end{align*}
where $C$ is a constant determined by $E$.
Let $E_1(z,t)$ be the projection such that 
$(I-F(z,t))^{-1}E_1(z,t) = E(I-F(z,t))^{-1}E_1(z,t)$.
Then
\begin{align*}
    |tr(K(z,t_1)(K(z,t_1) - K(z,t_2)))| & \leq 
    |tr(K(z,t_1)(K(z,t_1) - K(z,t_2)E_1(z,t_1)))|\\
    &\qquad \qquad + |tr(K(z,t_1)(K(z,t_1) - K(z,t_2)(I-E_1(z,t_1))))|\\
    & \leq |t_2-t_1|(2|z|+2)C + 2\varepsilon.  
\end{align*}
Now consider
\begin{align*}
    &(\sqrt{K(z,t_1)(I-K(z,t_1)))}V(z,t_1)
    - \sqrt{K(z,t_2)(I-K(z,t_2))}V(z,t_2))
    V(z,t_1)^{*}\sqrt{K(z,t_1)(I-K(z,t_1))}\\
    &= (K(z,t_2)-K(z,t_1))K(z,t_1)    
    +(t_2-t_1)(1-K(z,t_2))\sqrt{\frac{K_1}{I-K_1}}V_{1}
    V(z,t_1)^{*}\sqrt{K(z,t_1)(I-K(z,t_1))}.
\end{align*}
Let $E_2(z,t)$ be a projection such that $tr(E_2(z,t)) \geq 
1 - \varepsilon$ and 
\begin{align*}
\|\sqrt{\frac{K_1}{I-K_1}}V_{1}
V(z,t)^{*}\sqrt{K(z,t)(I-K(z,t))}E_2(z,t)\| \leq C.
\end{align*}
Then
\begin{align*}
    |tr((\sqrt{K(z,t_1)(I-K(z,t_1)))}V(z,t_1)
    &- \sqrt{K(z,t_2)(I-K(z,t_2))}V(z,t_2))
    V(z,t_1)^{*}\sqrt{K(z,t_1)(I-K(z,t_1))})|\\ 
    &\leq  |t_2-t_1|(2|z|+3)C + 4\varepsilon.
\end{align*}
The similar argument will give us similar inequlities 
\begin{align*}
|tr(V(z,t_1)^{*}(I-K(z,t_1))
&V(z,t_1)(V(z,t_2)^{*}K(z,t_2)V(z,t_2) - V(z,t_1)^{*}K(z,t_1)V(z,t_1)))|\\
&\leq |t_2-t_1|(2|z|+2)C + 2\varepsilon\\
\end{align*}
and 
\begin{align*}
    |tr(\sqrt{K(z,t_1)(I-K(z,t_1))}V(z,t_1)&(V(z,t_1)^{*}
    \sqrt{K(z,t_1)(I-K(z,t_1))}
    - V(z,t_2)^{*} \sqrt{K(z,t_2)(I-K(z,t_2))})|\\
    &\leq |t_2-t_1|(2|z|+3)C + 4\varepsilon.
\end{align*}
This implies that $P(z,t)$ is continuous at all point except $(\infty, t)$.
Argue exactly as in the proof of Proposition 2.2 in \cite{Hou}, we can
show that $P(z,t)$ is also continuous at $(\infty, t)$.

Note that $P(0, 1) = Q(0,1)$. Thus we now assume that
$P(0) = Q(0)$ and
\begin{align*}
    P(-1) = \begin{pmatrix}
            K_{2} & \sqrt{K_{2}(I-K_{2})}V_{2}\\
            V_{2}^{*}\sqrt{K_{2}(I-K_{2})} & V_{2}^{*}(I-K_2)V_{2}
        \end{pmatrix}.
\end{align*}
By considering the path
\begin{align*}
    P(-1,t)\HHH = \{(\sqrt{\frac{K_2}{I-K_2}}V(t)\xi, \xi)\},
\end{align*}
where $V(t)$ is a continuous map form $[0,1]$ into the set of
unitaries of $P(\infty)\AAA P(\infty)$ such that $V(0) = V_2$ and
$V(1)=I$. We may assume that $V_2 = I$.
Now
\begin{align*}
P(-1,t)\HHH = \{(\frac{t}{2}I+(1-t)\sqrt{\frac{K_2}{I-K_2}})\xi, \xi)\},
\end{align*}
connect $P(-1)$ and $Q(-1)$.
\end{proof}

\begin{lemma}
    With the same notations as in \cref{1thm1}, 
    there is a dense subset of $\HHH$ for any fix $z$ such that
    $\frac{\partial Q(z)\xi}{\partial z}$ and 
    $\frac{\partial Q(z)\xi}{\partial \bar{z}}$
    exist for any $\xi$ in the
    subset. Actually, 
    \begin{align*}
        \frac{\partial Q(z)}{\partial z} =
            \begin{pmatrix}
                (I-K_z) & 0 \\
                0 & U^{*}_{z}\sqrt{K_{z}(I-K_{z})} \\
            \end{pmatrix}
            \times
            \begin{pmatrix}
                S & S \\
                -S & -S \\
            \end{pmatrix}
            \times     
            \begin{pmatrix}
                U^{*}_{z}\sqrt{K_{z}(I-K_{z})} & 0 \\
                0 & U^{*}_{z}(I- K_{z})U_z \\
            \end{pmatrix},
    \end{align*}
and $\frac{\partial Q(z)\xi}{\partial \bar{z}} 
= (\frac{\partial Q(z)\xi}{\partial z})^{*}$. 
\end{lemma}

If $S$ is bounded, then the map $(x,y) \rightarrow Q(x+iy)$  
is $C^{\infty}$. Next we will show that for any map $\R^2 \rightarrow
Proj(\HHH)$ defined by \cref{1thm1}, there exists a $C^{\infty}$ map
approximate it.

\begin{lemma} \label{h_lem1}
   Let $P$ and $Q$ be two trace half projections in a finite von Neumann 
   algebra $\AAA$. If $\|P - Q\|_2 \leq \varepsilon$, then there exists 
   a unitary $U \in \AAA$ such that $U^{*}PU = Q$ and 
   $\|U - I\|_2 \leq \sqrt{2}\varepsilon$.
\end{lemma}

\begin{proof}
   Let $E = P \wedge Q$, $F = P \vee Q$. If $E \neq 0$, then
   $(F-E)P \wedge (F-E)Q = 0$. 
   If there exists a $U_1 \in (F-E)\AAA(F-E)$ such that
   $U_{1}^{*}(F-E)PU_1 = (F-E)Q$ and 
   $\|U_1 -I \|_2 \leq \sqrt{2}\varepsilon$. Then
   $U=E + U_1 + (I-F)$ will satisfies the conditions in the lemma.

   Now, assume that $P \wedge Q = 0$, $P \vee Q = I$ and
   \begin{align*}
      P = \begin{pmatrix}
          I & 0\\
          0 & 0
      \end{pmatrix}, \qquad 
      Q = \begin{pmatrix}
          H & \sqrt{H(I-H)}\\
          \sqrt{H(I-H)} & I-H
      \end{pmatrix},
   \end{align*}
   where $H \geq 0$. Since $\|P - Q\|_2 \leq \varepsilon$,
   $tr(I-H) \leq \varepsilon^2$, 
   where $tr$ is the trace on $P\AAA P$. 
   Let
   \begin{align*}
      U = \begin{pmatrix}
          \sqrt{H} & \sqrt{I-H}\\
          -\sqrt{I-H} & \sqrt{H}
      \end{pmatrix}. 
   \end{align*}
   It is clear that $U^{*}PU = Q$. Since $1-\varepsilon^2 \leq 
   tr(H) \leq tr(\sqrt{H}) \leq 1$, 
   $\|U-I\|^2 = 2-2tr(\sqrt{H}) \leq 2\varepsilon^2$.
\end{proof}

\begin{corollary} \label{h_cor1}
    Let $P(\infty)$, $P(0)$, $P(-1)$ and $Q(\infty)$ be trace half projections in a 
    finite von Neumann algebra $\AAA$. If 
    $\|Q(\infty)-P(\infty)\|_2 \leq \varepsilon$, and
    $P(\infty)$, $P(0)$, $P(-1)$ are in general position, then
    there exist trace half projections $Q(0)$ and $Q(-1)$ such that
    $Q(\infty)$, $Q(0)$, $Q(-1)$ are in general position and
    \begin{align*}
        \|Q(z) - P(z)\| \leq 2\sqrt{2}\varepsilon, \qquad 
        \forall z \in \C \cup \{\infty\}, 
    \end{align*}
    where $Q(z)$ and $P(z)$ are the projections determined by  
    $P(\infty)$, $P(0)$, $P(-1)$ and $Q(\infty)$, $Q(0)$, $Q(-1)$ 
    as in \cref{1thm1}.
\end{corollary}

\begin{proof}
    By \cref{h_lem1}, we have a unitary $U$ in $\AAA$ such that $U^{*}P(\infty)U =
    Q(\infty)$ and $\|U - I\|_2 \leq \sqrt{2}\varepsilon$. It is clear that
    $Q(z) = U^{*}P(z)U$ satisfy the conditions.
\end{proof}

\begin{lemma} \label{h_lem2}
   Let $\AAA$ be a finite von Neumann algebra and $\tau$ be a
   faithful normal trace on $\AAA$. 
   Suppose that $P$ and $Q$ are two trace half projections in $\AAA$ such
   that $P \wedge Q = 0$ and $P \vee Q =I$. For any $\varepsilon > 0$, 
   there exists a $\delta > 0$ such that
   if $E$ and $F$ are two projections in $\AAA$ satisfying
   $\| E - P \|_{2} \leq \delta$ and $\| F - Q \|_{2} \leq \delta$, then
   $\tau(E \wedge F) \leq \varepsilon$.
\end{lemma}


\begin{proof}
   Since $P \wedge Q = 0$, there exists a $n \in \N$ such that
   $\tau((PQ)^n) \leq \frac{\varepsilon}{2}$. 
   We claim that $\delta = \frac{\varepsilon}{4n}$ satisfies the 
   condition. Indeed,
   if $\| E - P \|_{2} \leq \frac{\varepsilon}{4n}$ and 
   $\| F - Q \|_{2} \leq \frac{\varepsilon}{4n}$, then
   \begin{align*}
       |\tau(E \wedge F)| &\leq | \tau((EF)^{n}) |\\ 
       & \leq \tau((PQ)^n) +
       \sum_{i=0}^{n-1}|\tau((PQ)^{n-i-1}(EF)^{i}E(F-Q))|
       + |\tau(Q(PQ)^{n-i-1}(EF)^{i}(E-P))|\\
       &\leq \tau((PQ)^n) + 
       \sum_{i=0}^{n-1} \|F-Q\|_{2} + \|E-P\|_2 
       \leq \varepsilon.
   \end{align*}
\end{proof}

The following proposition is immediate from the lemma above.

\begin{proposition} \label{h_prop1}
    Let $Q(\infty) = \bigl(\begin{smallmatrix}
           I & 0\\
           0 & 0 
   \end{smallmatrix} \bigr)$ and 
   \begin{align*}
       \qquad Q(0) &= \begin{pmatrix}
           H_{1} & \sqrt{H_{1}(I-H_{1})}\\
           \sqrt{H_{1}(I-H_{1})} & I-H_1
        \end{pmatrix},
        Q(-1) = \begin{pmatrix}
            H_{2} & \sqrt{H_{2}(I-H_{2})}U_{2}\\
            U_{2}^{*}\sqrt{H_{2}(I-H_{2})} & U_{2}^{*}(I-H_2)U_{2}
        \end{pmatrix}\\ 
    \end{align*}  
    be trace half projections in a finite von Neumann algebras.
    Suppose that $Q(\infty)$, $Q(0)$ and $Q(-1)$ are in general 
    position.
    Then for any $\varepsilon$, there exists two trace half projections 
    \begin{align*}
        P(0)= \begin{pmatrix}
            K_{1} & \sqrt{K_{1}(I-K_{1})}V_{1}\\
            V^{*}_{1}\sqrt{K_{1}(I-K_{1})} & V^{*}_{1}(I-K_1)V_{1}
        \end{pmatrix},
        P(-1) = \begin{pmatrix}
            K_{2} & \sqrt{K_{2}(I-K_{2})}V_{2}\\
            V_{2}^{*}\sqrt{K_{2}(I-K_{2})} & V_{2}^{*}(I-K_2)V_{2}
        \end{pmatrix}
    \end{align*}
   in $\AAA$ such that 
    \begin{enumerate}
        \item $\|P(0) - Q(0)\| \leq \varepsilon$ and 
            $\|P(-1) - Q(-1)\| \leq \varepsilon$;
   \item $\tau(Ran(\sqrt{\frac{H_1}{I-H_1}} - \sqrt{\frac{K_1}{I-K_1}}V_1)) 
             \leq \varepsilon$ and
    $\tau(Ran(\sqrt{\frac{H_2}{I-H_2}}U_2 - \sqrt{\frac{K_2}{I-K_2}}V_2)) 
             \leq \varepsilon$;
        \item $\|I - V_1\| \leq \varepsilon$ and
            $\|U_2 - V_2\| \leq \varepsilon$;
        \item $P(0) \wedge Q(\infty) = 0$ and $P(-1) \wedge Q(\infty) = 0$;
        \item 0 < $\|K_1\| < 1$ and 
            0 < $\|K_2\| < 1$;
        \item $P(0)\wedge P(-1) = 0$.
   \end{enumerate}
\end{proposition}

\begin{proof}
    Let
    \begin{align*}
        h(x) = \begin{cases}
           \delta & x \in [0, \delta)\\
            x & x \in [\delta, 1-\delta]\\
            1-\delta & x \in (1-\delta, 1]
           \end{cases}. 
    \end{align*}
    For $\delta$ small enough, \cref{h_lem2} implies that 
    \begin{align*}
        P(0) = \begin{pmatrix}
            K_{1} & 
            \sqrt{K_{1}(I-K_{1})}\\
            \sqrt{K_{1}(I-K_{1})} 
            & I-K_1
        \end{pmatrix} \mbox{ and } 
        F = \begin{pmatrix}
            \widetilde{H_{2}} 
            & \sqrt{\widetilde{H_{2}}(I-\widetilde{H_{2}})}U_{2}\\
            U_{2}^{*}\sqrt{\widetilde{H_{2}}(I-\widetilde{H_{2}})} 
            & U_{2}^{*}(I-\widetilde{H_2})U_{2}
        \end{pmatrix}, 
    \end{align*}
    satisfy $\tau(P(0) \wedge F) \leq \frac{\varepsilon}{4}$,
    where $K_{1} = h(H_1)$ and $\widetilde{H_{2}} = h(H_2)$.
    It is clear that $\sqrt{\frac{K_1}{I-K_1}}$ and
    $\sqrt{\frac{\widetilde{H_2}}{I-\widetilde{H_2}}}
    \widetilde{U}_{2}$ are bounded. Let 
    $S = \sqrt{\frac{K_1}{I-K_1}} - 
    \sqrt{\frac{\widetilde{H_2}}{I-\widetilde{H_2}}}\widetilde{U}_2$
    and $V$ be the partial isometry from $KerS$ onto $(RanS)^{\perp}$
    ($V$ exists, because $\AAA$ is finite). We 
    have $\tau(Ran(V)) \leq \frac{\varepsilon}{4}$ and 
    $Ker(S + zV) = \{0\}$ for any $z \neq 0$.
    Let 
    \begin{align*}
        P(-1) = \begin{pmatrix}
            K_{2} & \sqrt{K_{2}(I-K_{2})}V_{2}\\
            K_{2}^{*}\sqrt{K_{2}(I-K_{2})} & K_{2}^{*}(I-K_2)V_{2}
        \end{pmatrix},
    \end{align*}
    where
    \begin{align*}
        \sqrt{\frac{K_2}{I-K_2}}V_2 = 
    \sqrt{\frac{\widetilde{H_2}}{I-\widetilde{H_2}}}\widetilde{U}_2 + 
     zV.
    \end{align*}
    If $z$ is small enough, we have $P(-1)$ satisfies all the
    conditions in the proposition.
\end{proof}

\begin{lemma}
Let $Q(\infty) = \bigl(\begin{smallmatrix}
           I & 0\\
           0 & 0 
   \end{smallmatrix} \bigr)$ and 
   \begin{align*}
       \qquad Q(0) &= \begin{pmatrix}
           H_{1} & \sqrt{H_{1}(I-H_{1})}\\
           \sqrt{H_{1}(I-H_{1})} & I-H_1
        \end{pmatrix}, \quad  
        Q(-1) = \begin{pmatrix}
            H_{2} & \sqrt{H_{2}(I-H_{2})}U_{2}\\
            U_{2}^{*}\sqrt{H_{2}(I-H_{2})} & U_{2}^{*}(I-H_2)U_{2}
        \end{pmatrix},\\ 
        P(0) &= \begin{pmatrix}
        K_{1} & \sqrt{K_{1}(I-K_{1})}V_{1}\\
        V^{*}_{1}\sqrt{K_{1}(I-K_{1})} & V^{*}_{1}(I-K_1)V_{1}
        \end{pmatrix}, \quad
        P(-1) = \begin{pmatrix}
            K_{2} & \sqrt{K_{2}(I-K_{2})}V_{2}\\
            V_{2}^{*}\sqrt{K_{2}(I-K_{2})} & V_{2}^{*}(I-K_2)V_{2}
        \end{pmatrix}
    \end{align*}
   be trace half projections in a finite von Neumann algebra $\AAA$.
   Suppose that $Q(\infty)$,
   $Q(0)$ and $Q(-1)$ are in general position and  
   $Q(\infty)$, $P(0)$ and $P(-1)$ are in general position. 
   If $\tau(Ran(\sqrt{\frac{H_1}{I-H_1}} - 
   \sqrt{\frac{K_1}{I-K_1}}V_1))\leq \varepsilon$
   and $\tau(Ran(\sqrt{\frac{H_2}{I-H_2}}U_2 - 
    \sqrt{\frac{K_2}{I-K_2}}V_2))\leq \varepsilon$,
   then
   \begin{align*}
       \|Q(z) - P(z)\|_2 \leq 12\varepsilon 
       \qquad \mbox{ for any $z \in \C$}. 
   \end{align*}
\end{lemma}

\begin{proof}
    Let $F_1 = ker(\sqrt{\frac{H_1}{I-H_1}} - 
    \sqrt{\frac{K_1}{I-K_1}}V_1)$, 
    $F_2 = ker(\sqrt{\frac{H_2}{I-H_2}}U_2 - 
    \sqrt{\frac{K_2}{I-K_2}}V_2)$ and $E = F_1 \wedge F_2$. Then
    $\tau(E) \geq 1 - 2\varepsilon$. Note that
    \begin{align*}
        \|Q(z) - P(z)\|_{2}^{2} &= 1 - 2\tau(Q(z)P(z))\\ 
                                &= tr(H_{z}(H_z - K_z))
    + tr(U^{*}_{z}(I-H_z)U_{z}(V^{*}_{z}K_{z}V_{z} - U^{*}_{z}H_{z}U_{z}))\\
       &+ tr(\sqrt{H_{z}(I-H_{z})}U_{z}(U^{*}_{z}\sqrt{H_{z}(I-H_{z})}
        - V^{*}_{z} \sqrt{K_{z}(I-K_{z})}) \\
        &+ tr(U^{*}_{z}\sqrt{H_{z}(I-H_{z})}(\sqrt{H_{z}(I-H_{z})}U_{z}
        - \sqrt{K_{z}(I-K_{z})}V_{z}),
    \end{align*}
    where $tr$ is the trace on $Q(\infty)\AAA Q(\infty)$.
    Let
    \begin{align*}
        F(z) &= \frac{H_z}{I-H_z} = \left((1+z)\sqrt{\frac{H_1}{I-H_z}}
        -z\sqrt{\frac{H_2}{I-H_2}}U_{2} \right)
        \left((1+\bar{z})\sqrt{\frac{H_1}{I-H_z}}
        -\bar{z}U^{*}_{2}\sqrt{\frac{H_2}{I-H_2}} \right)\\
        G(z) &= \frac{K_z}{I-K_z} = \left((1+z)\sqrt{\frac{K_1}{I-K_z}}V_1
        -z\sqrt{\frac{K_2}{I-K_2}}V_{2} \right)
        \left((1+\bar{z})V^{*}_{1}\sqrt{\frac{K_1}{I-K_z}}
        -\bar{z}V^{*}_{2}\sqrt{\frac{K_2}{I-K_2}} \right)
    \end{align*}
    Note that $E F(z)E = E G(z) E$ and $Ran(I-H_z)= 
    Ran(I-K_z) = I$. 
    Let $E_1$ and $E_2$ be the projections
    such that $(I+F(z))^{-1}E_1 = E(I+F(z))^{-1}E_1$ and 
    $E_2(I+G(z))^{-1} = E_2(I+G(z))^{-1}E$.
    Note that $tr(E_1) = tr(E_2) = tr(E) \geq 1 - \varepsilon$.
    Then
    \begin{align*}
        tr(H_z(H_z-K_z))
        &=tr(H_z(I-E_2)(H_z - K_z)) + tr(H_z E_2 (H_z-K_z)(I-E_1))\\ 
        &+tr(H_z E_2(H_z - K_z)E_1).
    \end{align*}
    Note that
    \begin{align*}
        E_2(H_z - K_z)E_1 &= E_2((I+G(z))^{-1} - (I+F(z))^{-1})E_1\\
                          &= E_2(I+G(z))^{-1}E(F(z) - G(z))E(I+F(z))^{-1}E_1
                            = 0
    \end{align*}
    Therefore, $|tr(H_z(H_z-K_z))| \leq 2\varepsilon$.
    Similarly, we can show that
    \begin{align*}
      |tr(U^{*}_{z}(I-H_z)U_{z}(V^{*}_{z}K_{z}V_{z} - U^{*}_{z}H_{z}U_{z}))|
        \leq 2\varepsilon.
    \end{align*}
    Let $E_3 = Ran( U^{*}_{z}\sqrt{H_{z}(I-H_{z})})$ and 
    $E_4 = E \wedge E_3$. Then
    $tr(E_4) \geq tr(E_3) - 2\varepsilon$.
    Then it is not hard to see that there exist a projection
    $E_5$ such that $tr(E_5) \geq 1 - 2\varepsilon$ and 
    $U^{*}_{z}\sqrt{H_{z}(I-H_{z})}E_5 
    = E_4U^{*}_{z}\sqrt{H_{z}(I-H_{z})}E_5$. Hence
    \begin{align*}
        &tr((\sqrt{H_{z}(I-H_{z})}U_{z}
        - \sqrt{K_{z}(I-K_{z})}V_{z})U^{*}_{z}\sqrt{H_{z}(I-H_{z})})\\
        &= tr((\sqrt{H_{z}(I-H_{z})}U_{z}
        - \sqrt{K_{z}(I-K_{z})}V_{z})U^{*}_{z}\sqrt{H_{z}(I-H_{z})}E_5)\\
        &+ tr((\sqrt{H_{z}(I-H_{z})}U_{z}
        - \sqrt{K_{z}(I-K_{z})}V_{z})U^{*}_{z}\sqrt{H_{z}(I-H_{z})}(1-E_5)).
    \end{align*}
    By the definition of $E_5$, we have
    \begin{align*}
        &tr((\sqrt{H_{z}(I-H_{z})}U_{z}
        - \sqrt{K_{z}(I-K_{z})}V_{z})U^{*}_{z}\sqrt{H_{z}(I-H_{z})}E_5)\\
        &= tr((K_z - H_z)\sqrt{\frac{H_{z}}{I-H_{z}}}U_{z}
        )U^{*}_{z}\sqrt{H_{z}(I-H_{z})}E_5) \\
        &= tr((K_z - H_z)H_z E_5).
    \end{align*}
    Therefore
    \begin{align*}
        |tr((\sqrt{H_{z}(I-H_{z})}U_{z}
        - \sqrt{K_{z}(I-K_{z})}V_{z})U^{*}_{z}\sqrt{H_{z}(I-H_{z})})|
        \leq 4 \varepsilon.
    \end{align*}
    Hence
    \begin{align*}
        \|Q(z) - P(z)\|_2 \leq 12\varepsilon.
    \end{align*}
\end{proof}


Let $dQ(z) = \frac{\partial Q(z)}{\partial z}dz + 
\frac{\partial Q(z)}{\partial \bar{z}}d\bar{z}$.

\begin{lemma}
    With the notations as in \cref{1thm1}, we have
    \begin{align*}
        Q(z)(dQ(z)) = (dQ(z))(I-Q(z)) \mbox{ and } 
        (dQ(z))Q(z) = (I-Q(z))(dQ(z)).
    \end{align*}
\end{lemma}

\begin{proposition}
    With the notations as in \cref{1thm1}, we have
    \begin{align*}
        \bigtriangledown_{Q}^{2}(z) \equiv Q(z)(dQ(z)\wedge dQ(z)) =
        (dQ(z))(I-Q(z))(dQ(z)) = (dQ(z)\wedge dQ(z))Q(z).   
    \end{align*}
\end{proposition}

\begin{remark}
   Let
   \begin{align*}
       W(z) = \begin{pmatrix}
           \sqrt{K_z} & \sqrt{I-K_z}\\
           V^{*}_{z}\sqrt{I-K_z}V_{z} & -V^{*}_{z}\sqrt{K_z}
       \end{pmatrix}.
   \end{align*}
   We have
   \begin{align*}
       W(z)^{*}(dQ(z))^{2}W(z) &= 
       \begin{pmatrix}
           \sqrt{I-K_z}U_{z}S^{*}(I-K_z)SU_{z}^{*}\sqrt{I-K_z} & 0\\
           0 & \sqrt{I-K_z}SU_{z}^{*}(I-K_z)U_{z}S\sqrt{I-K_z} 
       \end{pmatrix}dz \wedge d\bar{z}, \\
       W(z)^{*}\bigtriangledown_{Q}^{2}(z)W(z) &= 
       \begin{pmatrix}
           \sqrt{I-K_z}U_{z}S^{*}(I-K_z)SU_{z}^{*}\sqrt{I-K_z} & 0\\
           0 & 0
       \end{pmatrix}dz \wedge d\bar{z}.
   \end{align*}
\end{remark}





\section{Geodesic of the bundle}
Let $\AAA$ be a finite von Neumann algebra.
If $S \in \mathcal{I} = \{X^{2}=I : X \in \AAA \}$, then
\begin{align*}
   S = \begin{pmatrix}
       I & 2B \\
       0 & -I
   \end{pmatrix} = 
   \begin{pmatrix}
       I & -B \\
       0 & I
   \end{pmatrix}
   \begin{pmatrix}
       I & 0 \\
       0 & -I
   \end{pmatrix}
   \begin{pmatrix}
       I & B \\
       0 & I
   \end{pmatrix}, 
\end{align*}
and $T = \frac{S+I}{2}$ is an idempotent.
Then the tangent space to $S$ is given by
\begin{align*}
    T_{S} &= \{X \in \AAA : XS + SX = 0 \}\\
          &= 
    \{ \begin{pmatrix}
       I & -B \\
       0 & I
   \end{pmatrix}
   \begin{pmatrix}
        0 & X_{12} \\
        X_{21} & 0
    \end{pmatrix} 
    \begin{pmatrix}
       I & B \\
       0 & I
   \end{pmatrix}
   =
   \begin{pmatrix}
       -BX_{21} & -BX_{21}B+X_{12}\\
       X_{21} & X_{21}B
   \end{pmatrix}
   \}. 
\end{align*}

Let 
\begin{align*}
    E_{S}(X) = e^{\frac{XS}{2}}Se^{\frac{-XS}{2}}. 
\end{align*}
$E_{S}$ will be called the exponential map. Note that
$E_{S}(X) \in \mathcal{I}$ and 
$E_{S}(X) = e^{XS}S$, if $X \in T_{S}$.

It is obvious that $T_{S}$ has a complex structure. Therefore $S$ is a
complex submanifold of $\AAA$. 

Let $\mathcal{R} = \{H^{*} = H : H \in \mathcal{I} \}$. Then 
$\mathcal{R}$ is only a (real) analytic submanifold of $\AAA$,
since 
\begin{align*}
    T_{H} = \{
        \begin{pmatrix}
            0 & X_{12}\\
            X_{12}^{*} & 0
    \end{pmatrix} \}
\end{align*}
is not a complex linear space.

Let $S \in \mathcal{I}$ and $\pi_{S}$ be the projection of $\AAA$ onto
$T_S$ given by
\begin{align*}
    \pi_{S}(X) = \frac{1}{2}(X -SXS).
\end{align*}

Recall the following definition.

\begin{definition}
   Suppose that $E_1$ and $E_2$ are Banach spaces, and that $U$ is an
   open subset of $E_1$. A continuous map $f: U \rightarrow E_2$ is said
   to be differentiable at the point $x_0 \in U$ if there exists a 
   continuous linear map $T: E_1 \rightarrow E_2$ such that 
   \begin{align*}
       \lim_{\|h\| \rightarrow 0} \frac{\| f(x_0 +h) -f(x_0) - T(h)\|}
       {\|h\|} = 0.
   \end{align*}
   $T$ is called the derivative of $f$ at $x_0$ and written as
   $Df(x_0)$. Note that
   \begin{align*}
       Df(x_0)h = \lim_{t \rightarrow 0} \frac{f(x_0 + th) - f(x_0)}{t}. 
   \end{align*}
\end{definition}

\begin{definition}
    Let $T(\I)$ be the tangent bundles over $\I$ and $D(\I)$ be the set of 
    all vector fields over $\I$, i.e. a smooth map form $\I$ into $T(\I)$.
    Given $X$ and $Y$ in $D(\I)$, let  
    \begin{align*}
        D(X,Y)(S) = \pi_{S}(DY(S) (X(S))) 
        = \pi_{S}(\frac{d}{dt}Y(c(t))|_{t=0}), 
    \end{align*}
    where $c: [-1,1] \rightarrow \I$ be a smooth curve such that $c(0) = S$
    and $c'(0) = X(S)$.
\end{definition}

It is easy to see that $D(\cdot, \cdot)$ satisfies the standard axioms of 
a connection. Following the standard terminology, we say that a smooth
curve $g$ is a geodesic if
\begin{align*}
    D(g', g') = \pi_{g(t)}(\frac{d^{2}}{dt^{2}} (g(t))) = 0. 
\end{align*}

\begin{lemma}[Lemma 2.9 in \cite{S}]
    Given $S \in \I$ and $X \in T_{S}$, there exists a unique total
    geodesic $g = E_{S}(tX)$ such that $g(0) = S$ and $g'(0) = X$.
\end{lemma}

\begin{example}
   Let $S = 2P -I$ and $T = 2Q-I$, where $P$ and $Q$ are two projections
   in $\AAA$. If $\| S - T\| = 2\|P - Q\| < 2$, then $P \sim Q$ in $\AAA$.
   Suppose that 
   \begin{align*}
       S &= \begin{pmatrix}
          I & 0\\
          0 & -I 
      \end{pmatrix}, \\
      T &= \begin{pmatrix}
          2H-I & 2\sqrt{H(I-H)} \\
          2\sqrt{H(I-H)} & I-2H
      \end{pmatrix} \\
      &=
      \begin{pmatrix}
          \sqrt{H} & \sqrt{I-H} \\
          \sqrt{I-H} & -\sqrt{H} 
      \end{pmatrix}
      \begin{pmatrix}
          I & 0\\
          0 & -I
      \end{pmatrix}
        \begin{pmatrix}
          \sqrt{H} & \sqrt{I-H} \\
          \sqrt{I-H} & -\sqrt{H} 
      \end{pmatrix}.
   \end{align*}
   Let
   \begin{align*}
       X & = \frac{\pi i}{2}\begin{pmatrix}
           I - \sqrt{H} & - \sqrt{I-H}\\
           - \sqrt{I-H} & I + \sqrt{H}
      \end{pmatrix} \\
      W &= \begin{pmatrix}
          \sqrt{H} & \sqrt{I-H} \\
          \sqrt{I-H} & -\sqrt{H} 
      \end{pmatrix} = e^{X}.  
   \end{align*}
   Then the geodesic connect $S$ and $T$ is
   $g(t) = e^{tX}Se^{-tX}$. And
   \begin{align*}
       g'(0) = i\pi \begin{pmatrix}
           0 & \sqrt{I-H}\\
           -\sqrt{I-H} & 0
       \end{pmatrix}. 
   \end{align*}
\end{example}

\begin{definition}
    Given a smooth curve $c: [a, b] \rightarrow \RR$, we define the 
    length $L(c)$ of $c$ by
    \begin{align*}
        L(c) = \int^{b}_{a} \|c'(t)\|_2 dt. 
    \end{align*}
\end{definition}

\begin{lemma}
   The $\| \cdot \|_2$ give the geodesic distance on $\RR$. 
\end{lemma}

\section{Conclusion}


%-----------------------------------------------------------------------------------
%BIBLIOGRAPHY
%-----------------------------------------------------------------------------------
\begin{thebibliography}{9}
\bibitem{RK} R. Kadison and John R. Ringrose, {\em Fundamentals of the theory of Operator Algebras},  {\bf Volume II } (1997)

\bibitem{FK} B. Fuglede and R. Kadison, {\em Determinant Theory in Finite Factors}, The Annals of Mathematics, Second Series, Vol. 55, No. 3 (1952), 520-530

\bibitem{Ta} M. Takesaki, {\em Theory of Operator Algebras}, {\bf Volume I}, Encyclopedia of Mathematical Sciences, vol 124, Springer-Verlag, Berlin (2002)

\bibitem{LP} R. H. Levene, S. C. Power {\em Manifold of Hilbert space 
    projection}, Proc. London Math. Soc. 1-25 (2009)


\bibitem{Mo} Mohan Ravichandran, {University of New Hampshire Ph.D. Thesis} (2009) 
\bibitem{GYII} L. Ge and W. Yuan, {\em Kadison-Singer algebras,
II---General case,} to appear, 2009

\bibitem{HF} Uffe Haagerup and Flemming Larsen, {\em Brown's Spectral Distribution Measure for
R-diagonal Elements in Finite von Neumann Algebras,}, 1999

\bibitem{HH} Uffe Haagerup and Hanne Schultz, {\em Brown Measures of Unbounded Operators Affiliated with
 Finite von Neumann Algebras,}, 1999

\bibitem{HV} Hari Bercovici and Dan Voiculescu, {\em Free Convolution of Measures with Unbounded Support, }

\bibitem{KS} $K.S.K\ddot{O}LBIG$, {\em On The Value Of A Logarithmic-Trigonometric Integral}, BIT 11(1971), 21-28

\bibitem{MR} Mehmet Koca, Ramazan Koc and Muataz Al-Barwani, {\em Breaking SO(3) into its closed subgroups by Higgs mechanism},
J.Phys.A: Math.Gen. 30(1997) 2109-2125

\bibitem{BA} Beardon, Alan F, {\em The Geometry of Discrete Groups}, New York: Springer-Verlag GTM 91.

\bibitem{AR} Alexandru Nica and Roland Speicher, {\em Lectures on the Combinatorics of Free Probability},
London Mathematical Society Lecture Note Series:335

\bibitem{SS} Elias M. Stein and Rami Shakarchi, {\em complex analysis}, Priceton lectures in analysis II.

\bibitem{Hou} Chengjun Hou and Wei Yuan, {\em Minimal Generating Reflexive Lattices of Projections in Finite von Neumann Algebras}

\bibitem{WW} W.Wu and W. Yuan, {\em On generators of abelian 
        Kadison–Singer algebras in matrix algebras} 
        Linear Algebra and its Applications, Vol 440, 2014, Pages 197–205.

\bibitem{LU} R. C. Lyndon and J. L. Ullman {\em Groups of Elliptic Linear Fractional Transformations}, Proceedings of the American Mathematical Society,   Vol. 18, No. 6, (1967) 1119-1124

\bibitem{SA} S. Sakai, {\em On automorphism groups of II$_1$-factors}, T\^{o}koku Math. J. (2) 26 (1974), 423-430.

\bibitem{WA} William L. Green and Anthony To-Ming Lau {\em Strong finite von Neumann algebras }, Math. Scand. 40 (1977), 105- 112.

\bibitem{MRM} M. Koca, R. Koc and M. Al-Barwani {\em Breaking SO(3) into its closed subgroups by Higgs mechanism}, J.Phys. A: Math. Gen. 30 (1997) 2109-2125

\bibitem{VDN} D. Voiculescu, K. Dykema and A. Nica, {\em Free Random Variables}, CMR Monograph Series 1, American Mathematical Society (1992)

\bibitem{NS} A. Nica and R. Speicher, {\em Lectures on the Combinatorics of Free Probability}, Londom Mathematical Society Lecture Note Series: 335 (2006)

\bibitem{MV2} F. J. Murray and J. von Neumann, {\em On rings of operators, II}, Trans. Amer. Math. Soc. 41 (1937), 208 - 248

\bibitem{Zhe} Z. Liu, {\em On Some Mathematical Aspects of The Heisenberg Relation }, Science China series Mathematics: Kadison's proceedings

\bibitem{MK} Masoud Khalkhaili, {\em Basic Noncommutative Geometry}, EMS series of Lectures in Mathematics (2009)

\bibitem{S}
    N. Salinas
    {\em The Grassmann manifold of a C*-algebra, and Hermitian 
    holomorphic bundles}, 
    Operator Theory: Advances and Applications 28, 
    Birkhaiiser Verlag Basel, 1998, 267-289.
\end{thebibliography}
%-----------------------------------------------------------------------------------

\end{document}
