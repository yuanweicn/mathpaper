\documentclass{amsart}

\usepackage[all]{xy}   
\usepackage{graphicx}  
\usepackage{bm}        
\usepackage{amsthm}
\usepackage{amsfonts}
\usepackage{latexsym}
\usepackage{amsmath}
\usepackage{amssymb}
\usepackage{graphicx}
\usepackage{float}
\usepackage{enumerate}



%%%%%%%%%%%%%%%%%%%%%%%%%%%%%%%%%%%%%%%%%%%%%%%%%%%%%%%%%%%%%%%%
%  Self-defined notations
%%%%%%%%%%%%%%%%%%%%%%%%%%%%%%%%%%%%%%%%%%%%%%%%%%%%%%%%%%%%%%%%
\newcommand{\A}{\mathcal A}
\newcommand{\AAA}{\mathfrak A}
\newcommand{\B}{\mathcal B}
\newcommand{\BBB}{\mathfrak B}
\newcommand{\CCC}{\mathcal C}
\newcommand{\DDD}{\mathcal D}
\newcommand{\F}{\mathcal F}
\newcommand{\G}{\mathcal G}
\newcommand{\HHH}{\mathcal H} %for Hilbert space
\newcommand{\KKK}{\mathcal K} %for Hilbert space
\newcommand{\LLL}{\mathcal L} % for lattice
\newcommand{\PPP}{\mathcal P}
\newcommand{\M}{\mathcal M}
\newcommand{\MMM}{\mathfrak M}
\newcommand{\NNN}{\mathcal N} %for nest
\newcommand{\RRR}{\mathcal R}
\newcommand{\SSS}{\mathcal S}
\newcommand{\W}{\mathcal W}
\newcommand{\ZZZ}{\mathcal Z}
\newcommand{\supp}{\mathop{\mathrm supp}}
\newcommand{\TT}{\mathcal T}
\newcommand{\II}{\mathrm{II}}


\newcommand{\e}[2][]{e^{#1}_{#2}} %for matrix unit \e[upper index]{lower index}

\newcommand{\Lat}{\mathrm{Lat}}
\newcommand{\Alg}{\mathrm{Alg}}
\newcommand{\tensor}{\mathop{\bar \otimes}}
\newcommand{\tr}{\tau}

\newcommand{\C}{\mathbb C} %for complex numbers
\newcommand{\R}{\mathbb R}  %for real numbers
\newcommand{\Z}{\mathbb Z} %for integers
\newcommand{\N}{\mathbb N} % for natural numbers

\newtheorem{theorem}{Theorem}[section]
\newtheorem{corollary}{Corollary}[section]
\newtheorem{main}{Main Theorem}[section]
\newtheorem{lemma}{Lemma}[section]
\newtheorem{prop}{Proposition}[section]
\newtheorem{df}{Definition}[section]
\newtheorem{remark}{Remark}[section]
\newtheorem{example}{Example}[section]
\newtheorem{question}{Question}[section]


%%%%%%%%%%%%%%%%%%%%%%%%%%%%%%%%%%%%%%%%%%%%%%%%%%%%%%%%%%%%%%%%
%  title configurations
%%%%%%%%%%%%%%%%%%%%%%%%%%%%%%%%%%%%%%%%%%%%%%%%%%%%%%%%%%%%%%%%
\title{On the Sphere-invariant automorphisms of finite von Neumann algebras}
\author{Liming Ge}
\address{L. K. Hua Key Laboratory of Mathematics, Chinese Academy of Sciences, Beijing, China and Department of Mathematics, University of New Hampshire, Durham, NH 03824}
\email{liming@math.ac.cn}
%\thanks{Research supported by the NSF }

\author{Wei Yuan}
\address{L. K. Hua Key Laboratory of Mathematics, Chinese Academy of Sciences, Beijing, China}
\email{wyuan@math.ac.com}

\subjclass[2010]{Primary }

\keywords{von Neumann algebra, reflexive lattice, automorphism}

\date{}

\begin{document}
\begin{abstract}
In this paper we show that the subgroup of automorphisms of a finite von Neumann algebra $\AAA$ that leave $\LLL$ invariant is  isomorphic to a subgroup of M\"{o}bius transformations,
where $\LLL$ is a reflexive subspace lattice generated by a double triangle lattice of projections in $\AAA$. 
 As a particular case, we show the group is $S_3$, the  symmetric group of 3 elements, when $\AAA $ is the interpolated free group factor $L_{F_{\frac{3}{2}}}$, and 
$\LLL$ is the reflexive lattice determined by the three free projection that generate $L_{F_{\frac{3}{2}}}$. 
\end{abstract}

\maketitle


\section{Introduction and Preliminaries}
Let $\HHH$ be a Hilbert space and $B(\HHH)$ be the algebra of all bounded linear operators on $\HHH$. For a set $\LLL$
of orthogonal projections in $B(\HHH)$, we denote by $Alg\LLL$ the set of all bounded linear operators on $\HHH$ leaving each element
in $\LLL$ invariant. Then $Alg\LLL$ is an unital weak-operator closed subalgebra of $B(\HHH)$. Similarly, for a subset $\mathcal{S}$ of
$B(\HHH)$, let $Lat\mathcal{S}$ be the set of  invariant projections for every operators in $\mathcal{S}$. Then $Lat\mathcal{S}$ is a
strong-operator closed lattice of projections. A subalgebra $\mathcal{A}$ of $B(\HHH)$ is said to be reflexive if
$\mathcal{A}=Alg\Lat\mathcal{A}$, similarly a lattice $\LLL$ of projections is reflexive if $\LLL=LatAlg\LLL$.

 Suppose $Q_{\infty}$, $Q_{0}$ and $Q_{-1}$ are three projections in a finite von Neumann algebra $\AAA$, and $Q_{\infty}$, $Q_{0}$ and $Q_{-1}$ are 
in general position, i.e., the intersection of any two is zero and the join of any two is I, then $\AAA \cong Q_{\infty}\AAA Q_{\infty} \otimes M_{2}(\C)$\cite[Proposition 2.4]{Hou}.
Morevoer we can write $Q_{\infty}$, $Q_{0}$ and $Q_{-1}$ in terms of $2 \times 2$ operator matrices (with respect to the standard matrix units in $I \otimes M_{2}(\C)$)  as follows:
\begin{equation}\label{eq0}
\begin{split}
Q_{\infty}&= \left(
        \begin{array}{cc}
          I & 0 \\
          0 & 0 \\
        \end{array}
      \right),
Q_{0} = \left(
          \begin{array}{cc}
            H_{1} & \sqrt{H_{1}(I-H_{1})} \\
            \sqrt{H_{1}(I-H_{1})} & I-H_{1} \\
          \end{array}
        \right) , \\
Q_{-1}&= \left(
          \begin{array}{cc}
            H_{2} & \sqrt{H_{2}(I-H_{2})}V \\
            V^{*}\sqrt{H_{2}(I-H_{2})} & V^{*}(I-H_{2})V \\
          \end{array}
        \right),
\end{split}
\end{equation}
where $H_i$ is a contractive positive operator in $Q_{\infty}\AAA Q_{\infty}$ such that $Ker(I - H_i) = 0$, $i = 1, 2$,
and $V$ is a unitary operator in $Q_{\infty}\AAA Q_{\infty}$. 

In order to describe the invariant subspace lattice $\Lat(\Alg(\{Q_{\infty}, Q_{0}, Q_{-1}\}))$, 
unbounded operator will be used. Let $\AAA$ be a von Neumann algebra, and
$\widetilde{\AAA}$ be the set of closed, densely defined operators affiliated with $\AAA$.
When $\AAA$ is finite, the family of operators affiliated with $\AAA$ has remarkable properties, 
that is the operators in $\widetilde{\AAA}$ admits algebraic operations of addition and multiplication.
In another word,  $\widetilde{\AAA}$ is an unital * algebra (cf. \cite{MV2}), and the elements in $\widetilde{\AAA} $ can be manipulated as if they were bounded operators.
In the rest of the paper, we will repeatedly make use this fact without mentioning it explicitly. 
For a elegant treatment of this subject, we refer readers to \cite{Zhe}. 

\begin{theorem}[\cite{Hou}, Theorem 2.1] \label{1thm1}
With the above notation and assumptions, we have $\Lat(\Alg(\{Q_{\infty}, Q_{0}, Q_{-1}\})) \setminus \{0, I\}$ is homeomorphic to $\C \cup \{\infty\} (\cong S^2)$ with
the homeomorphism given by $\rho[Q_{\infty}, Q_{0}, Q_{-1}](\infty) = Q_{\infty}$ and
\begin{align*}
\rho[Q_\infty, Q_0, Q_{-1}](z) = \left(
     \begin{array}{cc}
      K_{z} & \sqrt{K_{z}(I-K_{z})}U_{z} \\
      U_{z}^{*}\sqrt{K_{z}(I-K_{z})} & U_{z}^{*}(I-K_{z})U_{z} \\
  \end{array}
\right) , \qquad \forall z \in \C,
\end{align*}
where $K_z$ and $U_z$ are uniquely determined by the following polar decomposition:
\begin{equation}\label{eq1}
\begin{split}
&\sqrt{K_{z}(I-K_{z})^{-1}}U_{z} = (1+z)\sqrt{H_{1}(I-H_{1})^{-1}}-
                                   z\sqrt{H_{2}(I-H_{2})^{-1}}V \\
                                 &=zS + \sqrt{H_{1}(I-H_{1})^{-1}} \qquad (S = \sqrt{H_{1}(I-H_{1})^{-1}}- \sqrt{H_{2}(I-H_{2})^{-1}}V) .
\end{split}
\end{equation}
Moreover,  the reflexive lattice $\Lat(\Alg(\{Q_{\infty}, Q_{0}, Q_{-1}\}))$ can be determined by arbitrary three nontrivial projections$($not $0$ or I$)$ in it.
\end{theorem}

\begin{remark}\label{1re1}
Since $\sqrt{K_{0}(I-K_{0})^{-1}}U_{0} =
0\times S + \sqrt{H_{1}(I-H_{1})^{-1}}$ implies $K_{0} = H_{0}$ and $U_{0} = I$, 
we have $\rho[Q_{\infty}, Q_{0}, Q_{-1}](0) = Q_{0}$. Similarly, $\rho[Q_{\infty}, Q_{0}, Q_{-1}](-1) = Q_{-1}$. Therefore, we will also use $Q_{z}$ to denote $\rho[Q_{\infty}, Q_{0}, Q_{-1}](z)$ throughout the rest of this paper.
\end{remark}

\begin{example}[Hopf line bundle on the two-sphere]
To get the tautological line bundle over $CP_{1}$, let
\begin{align*}
Q_{\infty}&= \left(
        \begin{array}{cc}
          1 & 0 \\
          0 & 0 \\
        \end{array}
      \right),
Q_{0}= \left(
        \begin{array}{cc}
          0 & 0 \\
          0 & 1 \\
        \end{array}
      \right),
Q_{-1}= \left(
        \begin{array}{cc}
          \frac{1}{2} & -\frac{1}{2} \\
          -\frac{1}{2} & \frac{1}{2} \\
        \end{array}
      \right),
\end{align*}
then $Q_{z} = \left(
        \begin{array}{cc}
          \frac{x^2 + y^2}{1+x^2 + y^2 } & \frac{x + iy}{1+ x^2 + y^2} \\
          \frac{x - iy}{1+ x^2 + y^2} & \frac{1}{1+ x^2 + y^2}\\
        \end{array}
      \right)$, where $z = x+ iy$. It is can be shown that it is the line bundle associated to the Hopf fibration $S^{1} \rightarrow S^{3} \rightarrow S^{2}$.
\end{example}

In this paper will study the subgroup of automorphisms of $\AAA$ which leaves $Lat(\Alg(\{Q_{\infty}, Q_{0}, Q_{-1}\}))$ invariant.
The rest of this paper is organized as follows.  
Next, we point out that the above result naturally induce a coordinate chart of 
the reflexive lattice generated by a double triangle lattice of projections (exclude $0$ and $I$) in a finite von Neumann algebra.
In section 3, we prove that the transition maps between the charts are  M\"{o}bius transformations.
We then study the $\LLL$-invariant subgroup of automorphisms of  a von Neumann algebra $\AAA$, where $\LLL$ is a
reflexive subspace lattice contained in $\AAA$. In particular, we show if $\AAA$ is a finite factor or finite dimensional, and $\LLL$
is generated by a double triangle lattice of projections in $\AAA$, then the $\LLL$-invariant automorphism group is homeomorphic to a closed 
subgroup of $SO(3)$(Corollary \ref{2cor2}). In section 4, we compute the $\LLL$-invariant automorphism group when $\AAA $ 
is the interpolated free group factor $L_{F_{\frac{3}{2}}}$, and $\LLL$ is determined by the three free projection that generate $L_{F_{\frac{3}{2}}}$. 
In the appendix, we give a detailed proof of the following fact: If $\LLL$ is a reflexive lattice in a von Neumann algebra $\AAA$, and $\varphi$ is a *-isomorphism of 
$\AAA$, then $\varphi(\LLL)$ is also reflexive.\newline


Since any three projections in $\Lat(\Alg(\{Q_{\infty}, Q_{0}, Q_{-1}\})) \setminus \{0, I\}$ are in the general position, form Theorem \ref{1thm1}, we have the following corollary.
\begin{corollary}\label{1cor1}
With the notations in the Theorem \ref{1thm1} and the Remark \ref{1re1}, suppose $Q_{z_1}$, $Q_{z_2}$ and $Q_{z_3}$ are three nontrivial projections in $\Lat(\Alg(\{Q_{\infty}, Q_{0}, Q_{-1}\}))$,
then there is a homeomorphism $\rho[Q_{z_1}, Q_{z_2}, Q_{z_3}]$ form $S^2$ onto $\Lat(\Alg(\{Q_{\infty}, Q_{0}, Q_{-1}\})) \setminus \{0, I\}$ such that $\rho[Q_{z_1}, Q_{z_2}, Q_{z_3}](z)$ is determined by the following relation:
\begin{equation}\label{eq2}
\begin{split}
(I - Q_{z_1}\rho[Q_{z_1}, Q_{z_2}, Q_{z_3}]&(z)Q_{z_1})^{-1}[Q_{z_1}\rho[Q_{z_1}, Q_{z_2}, Q_{z_3}](z)(I-Q_{z_1})]\\
=(1 &+z)(I - Q_{z_1}Q_{z_2}Q_{z_1})^{-1}[Q_{z_1}Q_{z_2}(I-Q_{z_1})] \\
 & - z(I - Q_{z_1}Q_{z_3}Q_{z_1})^{-1}[Q_{z_1}Q_{z_3}(I-Q_{z_1})].
 \end{split}
\end{equation}
By (\ref{eq2}), $\rho[Q_{z_1}, Q_{z_2}, Q_{z_3}](\infty) = Q_{z_1}$, $\rho[Q_{z_1}, Q_{z_2}, Q_{z_3}](0) = Q_{z_2}$ and $\rho[Q_{z_1}, Q_{z_2}, Q_{z_3}](-1) = Q_{z_3}$. 
\end{corollary}
  

The inverse of the homeomorphism $\rho[Q_{z_1}, Q_{z_2}, Q_{z_3}]$ in the Corollary \ref{1cor1} actually gives a coordinate chart of  $(\Lat(\Alg(\{Q_{\infty}, Q_{0}, Q_{-1}\})) \setminus \{0, I,  Q_{z_1}\}, 
\rho[Q_{z_1}, Q_{z_2}, Q_{z_3}]^{-1})$ of $\Lat(\Alg(\{Q_{\infty}, Q_{0}, Q_{-1}\})) \setminus \{0, I \}$.
So $\Lat(\Alg(\{Q_{\infty}, Q_{0}, Q_{-1}\})) \setminus \{0, I\}$ is a 2-dimensional (topological) manifold with atlas
$\{\rho[Q_{z_1}, Q_{z_2}, Q_{z_3}]^{-1} | z_1, z_2, z_3 \in \C \cup \{\infty \} \}$. In the next section, we will determine the transition maps between the charts in this atlas.



\section{The transition maps}
A M\"{o}bius transformation is a 1-1 map of the Riemann sphere $\widehat{C}$ onto itself such that 
\begin{align*}
f(z) = \frac{az + b}{cz + d},   \qquad \qquad z \in \widehat{C},
\end{align*}
where $ad - bc \neq 0$. 
The set of all M\"{o}bius transformations forms a group under the composition called the M\"{o}bius group.
M\"{o}bius group is the automorphism group of the Riemann sphere, and denoted by $Aut(\widehat{\C})$.
It is well known that $Aut(\widehat{\C}) \cong PSL(2, \C)$. If $z_1$, $z_2$, $z_3$ is a triple of distinct points in $\widehat{\C}$ and 
let $w_1$, $w_2$, $w_3$ be another such triple, then there is a unique $f$ in $Aut(\widehat{\C})$ such that $f(z_i) = w_i$, $i = 1, 2,3$.
 
By corollary \ref{1cor1}, any three nontrivial projections $Q_{z_1}$, $Q_{z_2}$ and $Q_{z_3}$ will determine a continuous map $\rho[Q_{z_1}, Q_{z_2}, Q_{z_3}]$ form $S^2$ onto $\Lat(\Alg(\{Q_{\infty}, Q_{0}, Q_{-1}\})) \setminus \{0, I\}$, and  $f = \rho[Q_{\infty}, Q_{0}, Q_{-1}]^{-1} \circ \rho[Q_{z_1}, Q_{z_2}, Q_{z_3}]$ is a homeomorphism of $S^2$ satisfying  
$f(\infty) = z_1$, $f(0) = z_2$, and $f(-1)  = z_3$. We will see in the next theorem that $f$ is the unique M\"{o}bius transformation
determined by the value at $\infty$, $0$ and $-1$.

\begin{lemma}\label{2lemma0}
With the notations in the last section, let 
$Q_{z_1}$, $Q_{z_2}$ and $Q_{z_3}$ be three nontrivial projections in $\Lat(\Alg(\{Q_{\infty}, Q_{0}, Q_{-1}\})) \setminus \{0, I\}$.  
Then  $f =\rho[Q_{\infty}, Q_{0}, Q_{-1}]^{-1} \circ  \rho[Q_{z_1}, Q_{z_2}, Q_{z_3}]$ is the
unique M\"{o}bius transformation satisfying $f(\infty) = z_1$, $f(0) = z_2$, $f(-1) = z_3$ .
\end{lemma}

\begin{proof}
First assume that none of $z_1$, $z_2$ or $z_3$ is $\infty$.
Since $\rho[Q_{z_1}, Q_{z_2}, Q_{z_3}](z) = Q_{f(z)}$, by (\ref{eq2}), we have
\begin{equation}\label{eq3}
\begin{split}
(I - Q_{z_1} Q_{f(z)}&Q_{z_1})^{-1}[Q_{z_1}Q_{f(z)}(I-Q_{z_1})]\\
=(1 &+z)(I - Q_{z_1}Q_{z_2}Q_{z_1})^{-1}[Q_{z_1}Q_{z_2}(I-Q_{z_1})] \\
 & - z(I - Q_{z_1}Q_{z_3}Q_{z_1})^{-1}[Q_{z_1}Q_{z_3}(I-Q_{z_1})].
\end{split}
\end{equation}
Let $W = \left(
           \begin{array}{cc}
             \sqrt{K_{z_1}} & \sqrt{I-K_{z_1}}U_{z_1} \\
             U^*_{z_1}\sqrt{I-K_{z_1}} & -U^*_{z_1}\sqrt{K_{z_1}}U_{z_1} \\
           \end{array}
         \right)$. $W$ is a unitary in the von Neumann algebra $\{Q_{\infty}, Q_{0}, Q_{-1} \}''$ such that $WQ_{z_1}W = Q_{\infty}$ and $WQ_{\infty}W = Q_{z_1}$. Thus, by (\ref{eq3}), we have 
\begin{equation}\label{eq4}
\begin{split}
(I - Q_{\infty} WQ_{f(z)}W&Q_{\infty})^{-1}[Q_{\infty}WQ_{f(z)}W(I-Q_{\infty})]\\
=(1 &+z)(I - Q_{\infty}WQ_{z_2}WQ_{\infty})^{-1}[Q_{\infty}WQ_{z_2}W(I-Q_{\infty})] \\
 & - z(I - Q_{\infty}WQ_{z_3}WQ_{\infty})^{-1}[Q_{\infty}WQ_{z_3}W(I-Q_{\infty})].
\end{split}
\end{equation}
For any $z \in \C$, direct computation gives that 
\begin{align*}
(WQ_{z}W)_{1,1}  &= (Q_{\infty} WQ_{z}W Q_{\infty})|_{Q_{\infty}\HHH}= \sqrt{K_{z_1}}K_{z}\sqrt{K_{z_1}} + \sqrt{I - K_{z_1}}U_{z_1}U^{*}_{z}\sqrt{K_{z} (I - K_{z})}\sqrt{K_{z_1}} \\
                &  +\sqrt{K_{z_1}}\sqrt{K_{z}(I-K_{z})}U_{z}U_{z_1}^{*} \sqrt{I-K_{z_1}} + \sqrt{I-K_{z_1}}U_{z_1}U^{*}_{z}(I - K_z)U_{z}U^{*}_{z_1}\sqrt{I - K_{z_1}}.
\end{align*}
By (\ref{eq1}), we have
\begin{equation}\label{eq5}
\begin{split}
I - (WQ_{z}W)_{1,1} &= \sqrt{K_{z_1}}(I - K_{z})\sqrt{K_{z_1}} \\
                    & \qquad - \sqrt{I - K_{z_1}}U_{z_1}U^{*}_{z}\sqrt{K_{z} (I - K_{z})}\sqrt{K_{z_1}} \\
                    & \qquad - \sqrt{K_{z_1}}\sqrt{K_{z}(I-K_{z})}U_{z}U^{*}_{z_1}\sqrt{I-K_{z_1}} \\
                    & \qquad + \sqrt{I-K_{z_1}}U_{z_1}U^{*}_{z}K_{z}U_{z}U^{*}_{z_1}\sqrt{I - K_{z_1}}\\
                    &= \sqrt{K_{z_1}}(I-K_z)[\sqrt{K_{z_1}(I-K_{z_1})^{-1}}U_{z_1} \\
                    & \qquad - \sqrt{K_z(I - K_z)^{-1}}U_z]U_{z_1}^{*}\sqrt{I-K_{z_1}} \\
                    & \qquad- \sqrt{I-K_{z_1}}U_{z_1}U_{z}^{*}\sqrt{K_{z}(I- K_z)}[\sqrt{K_{z_1}(I-K_{z_1})^{-1}}U_{z_1}\\
                    & \qquad  - \sqrt{K_z(I - K_z)^{-1}}U_z]U^{*}_{z_1}\sqrt{I-K_{z_1}}\\
                    & = (z_1 - z)[\sqrt{K_{z_1}}(I-K_z)\\
                    & \qquad -\sqrt{I - K_{z_1}}U_{z_1}U^{*}_{z}\sqrt{K_{z}(I-K_z)}]SU_{z_1}^{*}\sqrt{I-K_{z_1}}\\
                    &= (z_1 - z)\sqrt{I-K_{z_1}}U_{z_1}(U^{*}_{z_1}\sqrt{K_{z_1}(I-K_{z_1})^{-1}}\\
                    & \qquad  - U^{*}_{z}\sqrt{K_z(I-K_z)^{-1}})(I-K_z)SU^{*}_{z_1}\sqrt{I-K_{z_1}}\\
                    &= |z_1 - z|^{2}\sqrt{I-K_{z_1}}U_{z_1}S^{*}(I-K_z)SU_{z_1}^{*}\sqrt{I-K_{z_1}}.
\end{split}
\end{equation}
Similarly,
\begin{align*}\label{eq6}
(WQ_{z}W)_{1,2} &= Q_{\infty}WQ_{z}W(I - Q_{\infty}) \\   
                &= \sqrt{K_{z_1}}K_{z}\sqrt{I - K_{z_1}}U_{z_1} \\
                & \qquad + \sqrt{I - K_{z_1}}U_{z_1}U^{*}_{z}\sqrt{K_{z} (I - K_{z})}\sqrt{I - K_{z_1}}U_{z_1} \\
                & \qquad -\sqrt{K_{z_1}}\sqrt{K_{z}(I-K_{z})}U_{z}U^{*}_{z_1}\sqrt{K_{z_1}}U_{z_1}\\
                & \qquad  - \sqrt{I-K_{z_1}}U_{z_1}U^{*}_{z}(I - K_z)U_{z}U^{*}_{z_1}\sqrt{K_{z_1}}U_{z_1}\\
                &= -\sqrt{K_{z_1}}(I-K_{z})\sqrt{I - K_{z_1}}U_{z_1} \\
                & \qquad + \sqrt{I - K_{z_1}}U_{z_1}U^{*}_{z}\sqrt{K_{z} (I - K_{z})}\sqrt{I - K_{z_1}}U_{z_1} \\
                & \qquad -\sqrt{K_{z_1}}\sqrt{K_{z}(I-K_{z})}U_{z}U^{*}_{z_1}\sqrt{K_{z_1}}U_{z_1} \\
                & \qquad + \sqrt{I-K_{z_1}}U_{z_1}U^{*}_{z}K_{z}U_{z}U^{*}_{z_1}\sqrt{K_{z_1}}U_{z_1}\\
                & = \sqrt{I-K_{z_1}}U_{z_1}[U^{*}_{z}\sqrt{K_z(I-K_z)^{-1}} \\
                & \qquad - U^{*}_{z_1}\sqrt{K_{z_1}(I - K_{z_1}]^{-1}})(I-K_z)\sqrt{I - K_{z_1}}U_{z_1} \\
                & \qquad + \sqrt{I-K_{z_1}}U_{z_1}[U^{*}_{z}\sqrt{K_z(I-K_z)^{-1}}  \\
                & \qquad - U^{*}_{z_1}\sqrt{K_{z_1}(I - K_{z_1})^{-1}}]\sqrt{K_{z}(I-K_z)}U_{z}U^{*}_{z_1}\sqrt{K_{z_1}}U_{z_1}\\
                &= \overline{(z-z_1)}\sqrt{I-K_{z_1}}U_{z_1}S^{*}[(I-K_z)\sqrt{I - K_{z_1}}U_{z_1} + \sqrt{K_{z}(I-K_z)}U_{z}U_{z_1}^{*}\sqrt{K_{z_1}}U_{z_1}]\\
                &= \overline{(z-z_1)}\sqrt{I-K_{z_1}}U_{z_1}S^{*}(I-K_z)(\sqrt{I-K_{z_1}}U_{z_1} + \sqrt{K_z(I - K_z)^{-1}}U_{z}U_{z_1}^{*}\sqrt{K_{z_1}}U_{z_1}).
\end{align*}
Note $\sqrt{K_z(I - K_z)^{-1}}U_{z} = ( z - z_1)S + \sqrt{K_{z_1}(I - K_{z_1})^{-1}}U_{z_1}$, we have  
\begin{align*}
(WQ_{z}W)_{1,2} = \overline{(z-z_1)}\sqrt{I-K_{z_1}}U_{z_1}S^{*}(I-K_z)[(z -z_1)SU^{*}_{z_1}\sqrt{K_{z_1}}U_{z_1} + \sqrt{(I-K_{z_1})^{-1}}U_{z_1}].
\end{align*}
This gives us
\begin{align*}
&[I - (WQ_{z}W)_{1,1}]^{-1}(WQ_{z}W)_{1,2} = \\
& |z-z_1|^{-2}\sqrt{(I-K_{z_1})^{-1}}U_{z_1}S^{-1}(I-K_z)^{-1}{S^{*}}^{-1}U^{*}_{z_1}\sqrt{(I-K_{z_1})^{-1}}\\
                                          & \times  \overline{(z-z_1)}\sqrt{I-K_{z_1}}U_{z_1}S^{*}(I-K_z)[(z -z_1)SU^{*}_{z_1}\sqrt{K_{z_1}}U_{z_1} + \sqrt{(I-K_{z_1})^{-1}}U_{z_1}] \\
                                          &= \sqrt{K_{z_1}(I-K_{z_1})^{-1}}U_{z_1} + (z -z_1)^{-1} \sqrt{(I-K_{z_1})^{-1}}U_{z_1}S^{-1}\sqrt{(I-K_{z_1})^{-1}}U_{z_1}.
\end{align*}
By (\ref{eq4}), we have 
\begin{align*}
&\sqrt{K_{z_1}(I - K_{z_1})^{-1}}U_{z_1} + (f(z) - z_1)^{-1}\sqrt{(I - K_{z_1})^{-1}}U_{z_1}S^{-1}\sqrt{(I - K_{z_1})^{-1}}U_{z_1} \\
&=(1+z)[\sqrt{K_{z_1}(I - K_{z_1})^{-1}}U_{z_1} + (z_2 - z_1)^{-1}\sqrt{(I - K_{z_1})^{-1}}U_{z_1}S^{-1}\sqrt{(I - K_{z_1})^{-1}}U_{z_1}]\\
&\qquad -z[\sqrt{K_{z_1}(I - K_{z_1})^{-1}}U_{z_1} + (z_3 - z_1)^{-1}\sqrt{(I - K_{z_1})^{-1}}U_{z_1}S^{-1}\sqrt{(I - K_{z_1})^{-1}}U_{z_1}],
\end{align*}
which implies that 
\begin{align*}
\frac{1}{f(z) - z_1} = \frac{1+z}{z_2 - z_1} - \frac{z}{z_3 - z_1}.
\end{align*}
Solve $f(z)$, we have 
\begin{align*}
f(z) = \frac{zz_1 (z_3 - z_2) + z_2(z_3 - z_1)}{z(z_3 - z_2) + (z_3 - z_1)}.
\end{align*}

If any $z_i$ equals $\infty$, $i \in \{ 1, 2, 3\}$, we can choose three complex number $z_{1}'$, $z_{2}'$ and $z_{3}'$ in $\C \cup \{\infty \} \setminus \{z_1, z_2, z_3\}$.
By considering $Q_{z_{1}'}$, $Q_{z_{2}'}$ and $Q_{z_{3}'}$ as $Q_{\infty}'$, $Q_{0}'$ and $Q_{-1}'$,  we have $\rho[Q_{z_{1}'},Q_{ z_{2}'}, Q_{z_{3}'}]^{-1}\circ \rho[Q_{z_1}, Q_{z_2}, Q_{z_3}]$ 
is a  M\"{o}bius transformation by the above argument. Therefore $\rho[Q_{\infty}, Q_{0}, Q_{-1}]^{-1}\circ \rho[Q_{z_1}, Q_{z_2}, Q_{z_3}]    = (\rho[Q_{\infty}, Q_{0}, Q_{-1}]^{-1}\circ \rho[Q_{z_{1}'}, Q_{z_{2}'}, Q_{z_{3}'}]) \circ
\rho[Q_{z_{1}'}, Q_{z_{2}'}, Q_{z_{3}'}]^{-1}\circ \rho[Q_{z_1}, Q_{z_2}, Q_{z_3}]$ is a M\"{o}bius transformation, and $\rho[Q_{\infty}, Q_{0}, Q_{-1}]^{-1}\circ \rho[Q_{z_1}, Q_{z_2}, Q_{z_3}](\infty) = z_1$, 
$\rho[Q_{\infty}, Q_{0}, Q_{-1}]^{-1}\circ \rho[Q_{z_1}, Q_{z_2}, Q_{z_3}](0) = z_2$ and $\rho[Q_{\infty}, Q_{0}, Q_{-1}]^{-1}\circ \rho[Q_{z_1}, Q_{z_2}, Q_{z_3}](-1) = z_3$.
\end{proof}

It is easy to see that $P \wedge Q = 0 (P \vee Q = I)$ if and only if $(I-P) \vee (I-Q) = I ((1-P) \wedge (I- Q) = 0)$.
Thus, there is a homeomorphism $\rho[I - Q_{\infty}, I - Q_{0}, I - Q_{-1}]$ form $S^{2}$ onto 
$\Lat(\Alg(\{ I - Q_{\infty}, I - Q_{0}, I - Q_{-1}\}))$ by Corollary \ref{1cor1}. Note
\begin{align*}
&\Lat(\Alg(\{ I - Q_{\infty}, I - Q_{0}, I - Q_{-1}\})) = \\
& \qquad I - \Lat(\Alg( \{ Q_{\infty},  Q_{0}, Q_{-1} \})) = \{ I - Q | Q \in \Lat(\Alg( \{ Q_{\infty}, Q_{0}, Q_{-1} \})),
\end{align*}
we have  
\begin{align*}
z \rightarrow  \rho[Q_{\infty}, Q_{0}, Q_{-1}]^{-1} (I - \rho[I - Q_{\infty}, I - Q_{0}, I-Q_{-1}](z)), \qquad  \forall z \in \C \cup \{\infty \},
\end{align*}
is a continuous map from $S^2$ onto itself.  

\begin{lemma}\label{2lemma1}
With the above notation, $\rho[Q_{\infty}, Q_{0}, Q_{-1}]^{-1} (I - \rho[I - Q_{\infty}, I - Q_{0}, I-Q_{-1}](z)) = \overline{z} (\overline{\infty} = \infty)$.
\end{lemma}

\begin{proof}
We use $\rho(z)$ to denote $\rho[Q_{\infty}, Q_{0}, Q_{-1}]^{-1} (I - \rho[I - Q_{\infty}, I - Q_{0}, I-Q_{-1}](z))$, for the sake of simplicity.
Since $\rho[I - Q_{\infty}, I - Q_{0}, I-Q_{-1}](z) = I - Q_{\rho(z)} $, from (\ref{eq2}), we have  
\begin{align*}
[I - (I - Q_{\infty})&(I - Q_{\rho(z)})(I - Q_{\infty})]^{-1}[(I - Q_{\infty})(I - Q_{\rho(z)})Q_{\infty}] \\
&= (1+z)[I - (I - Q_{\infty})(I - Q_{0})(I - Q_{\infty})]^{-1}[(I - Q_{\infty})(I - Q_{-1})Q_{\infty}] \\
& \quad - z[I - (I-Q_{\infty})(I - Q_{-1})(I - Q_{0})]^{-1}[(I - Q_{\infty})(I - Q_{-1})Q_{\infty}]. 
\end{align*}
By (\ref{eq0}), this implies
\begin{align*}
(1+z)\sqrt{H_{1}(I - H_{1})^{-1}} -zV^{*}\sqrt{H_{2}(I - H_{2})^{-1}} = U^{*}_{\rho(z)}\sqrt{K_{\rho(z)}(I - K_{\rho(z)})^{-1}}.
\end{align*}
Thus by (\ref{eq1}), we have $\rho(z) = \overline{z}$.
\end{proof}


If a von Nemuann algebra $\AAA$ is generated by a set of projections $\LLL$, then any automorphism of $\AAA$ that fixes all projections in 
$\LLL$ must be the identity mapping. However, there may exists nontrivial automorphisms that map $\LLL$ onto itself. 

\begin{df}
Suppose $\LLL$ is a set of projections in a von Neumann algebra $\AAA$. Let 
$G[\LLL, \AAA] = \{ \varphi \in Aut(\AAA) | \varphi(\LLL) = \LLL \}$. When $\AAA = \LLL''$, we will 
omit $\AAA$ and write $G[\LLL]$ for $G[\LLL, \AAA]$.
\end{df}

Let $\AAA$ be a von Neumann algebra and $\AAA_{*}$ be the predual of $\AAA$, i.e., the set of normal linear functionals on $\AAA$.
Then the automorphism group $Aut(\AAA)$ of $\AAA$ is a topological group with respect to a natural topology:  the topology of 
pointwise norm convergence in the predual $\AAA_{*}$ of $\AAA$. That is $\{\varphi_{\alpha} \} (\subset Aut(\AAA))$ converges 
towards $\varphi$ if and only if
 \begin{align*}
\lim_{\alpha} \| \omega \circ \varphi_{\alpha} - \omega \circ \varphi \| = 0,   \qquad \forall \omega \in \AAA_{*}.
\end{align*}

 
 
\begin{lemma}\label{2lemma2}
If $\LLL$ is a strong-operator  $($or weak-operator $)$ closed set of projections in a von Neumann algebra $\AAA( \subset \B(\HHH))$,
then $G[\LLL, \AAA]$ is a closed subgroup of of $Aut(\AAA)$ when $Aut(\AAA)$ is endowed with the natural topology.
\end{lemma}

\begin{proof}
$G[\LLL, \AAA]$ obviously is a group, and we only need to show that it is closed.

Let $\{\varphi_{\alpha} \} \subset G[\LLL, \AAA]$. If $\{\varphi_{\alpha} \}$ converges to $\varphi \in Aut(\AAA)$, then for any vector state
$\omega_{x, x}$ and $P$ in $\LLL$,
\begin{align*}
\|(\varphi_{\alpha}(P) - \varphi(P))x\|^{2} &= \omega_{x,x}(\varphi_{\alpha}(P) - \varphi(P))^{*}(\varphi_{\alpha}(P) - \varphi(P)))\\
                                                               &= \omega_{x,x}(\varphi_{\alpha}(P) + \varphi(P) - \varphi_{\alpha}(P)\varphi(P) - \varphi(P)\varphi_{\alpha}(P)) \\
                                                               & \rightarrow \omega_{x,x}(\varphi(P) + \varphi(P) - \varphi(P)\varphi(P) - \varphi(P)\varphi(P)) = 0.
\end{align*}
This implies that $\varphi_{\alpha}(P)$ tends to $\varphi(P)$ in strong-operator topology and $\varphi(\LLL) \subset \LLL$. Because 
$\{\varphi_{\alpha} \} ^{-1}$ converges to $\varphi^{-1}$, by the same argument as before, we have $\varphi^{-1}(\LLL) \subset \LLL$.
This complete the proof.
\end{proof}

Because any reflexive lattice of projections is strong-operator closed, we have the following corollary.

\begin{corollary}
Let $\LLL$ be a reflexive lattice of projections in a  von Neumann algebra $\AAA$, 
then $G[\LLL, \AAA]$ is a closed subgroup of of $Aut(\AAA)$ when $Aut(\AAA)$ is endowed with the natural topology.
\end{corollary}

\begin{lemma}\label{2lemma4}
Let $\varphi \in G[\Lat(\Alg( \{ Q_{\infty},  Q_{0}, Q_{-1} \})]$, then $\varphi(Q_{z}) = Q_{\rho_{\varphi} (z)}$, where
$\rho_{\varphi} = \rho[Q_\infty, Q_0, Q_{-1}]^{-1} \circ  \rho[\varphi(Q_{\infty}), \varphi(Q_{0}), \varphi(Q_{-1})]$.
And $\rho_{\varphi_{2} \circ \varphi_{1}} = \rho_{\varphi_{2}} \circ \rho_{\varphi_{1}}$, where $\varphi_1$, $\varphi_2 \in  G[\Lat(\Alg( \{ Q_{\infty},  Q_{0}, Q_{-1} \})]$.
\end{lemma}

\begin{proof}
Apply $\varphi$ on both sides of (\ref{eq2}), for $z_1 = \infty$, $z_2 = 0$ and $z_3 = -1$, we obtain 
\begin{align*}
(I - \varphi(Q_{\infty})& \varphi(Q_{z}) \varphi(Q_{\infty}))^{-1}[\varphi(Q_{\infty}) \varphi(Q_{z})  (I-\varphi(Q_{\infty}))]\\
=(1 &+z)(I - \varphi(Q_{\infty}) \varphi(Q_{0}) \varphi(Q_{\infty}))^{-1}[ \varphi(Q_{\infty}) \varphi(Q_{0}) (I-\varphi(Q_{\infty}))] \\
 & - z(I - \varphi(Q_{\infty}) \varphi(Q_{-1}) \varphi(Q_{\infty}))^{-1}[ \varphi(Q_{\infty}) \varphi(Q_{-1})(I- \varphi(Q_{\infty}))].
\end{align*}
Comparing this equation with (\ref{eq2}), it is clear that  $\varphi(Q_{z}) = Q_{\rho_{\varphi} (z)}$.
%$Q_{\rho_{\varphi_{2} \circ \varphi_{1}}(z)} = \varphi_{2} (\varphi_{1}(Q_z)) = \varphi_{2}(Q_{\rho_{\varphi_1}(z)}) = Q_{\rho_{\varphi_{2}}(\rho_{\varphi_{1}}(z))}$.
\end{proof}


\begin{lemma}\label{2lemma3}
Suppose $\LLL = \Lat(\Alg \{ P_{i} | i \in \mathcal{I} \}$ and $\widetilde{\LLL} = \Lat(\Alg \{Q_{i} | i \in \mathcal{I} \}$ are two reflexive lattices, where $\mathcal{I}$ be an index set. 
Let $\AAA$ and $\widetilde{\AAA} $ be two von Neumann algebras that contain 
$\LLL$ and $\widetilde{\LLL}$ respectively.If 
$\varphi$ is a *-isomorphism form $\AAA$ onto $\widetilde{\AAA}$ such that 
$\varphi(\{ P_{i} \}_{i \in I})= \{ Q_{i} \}_{i \in I}$, then  $\varphi(\LLL) = \widetilde{\LLL}$.
\end{lemma}

\begin{proof}
$\varphi(\LLL)$ is reflexive by \ref{athm2}. This implies 
\begin{align*}
\varphi(\LLL) \supseteq \Lat(\Alg(\{ Q_{i} \}_{i \in I})) = \widetilde{\LLL},
\end{align*}
since $\varphi(\LLL) \supseteq \varphi(\{ P_{i} \}_{i \in I}) = \{ Q_{i} \}_{i \in I}$.
Similarly, by considering $\varphi^{-1}$, we have
\begin{align*}
\varphi^{-1}(\widetilde{\LLL}) \supseteq \LLL. \qquad \qquad (*)
\end{align*}
Apply $\varphi$ to both sides of $(*)$ will give the opposite direction of the inclusion
$\widetilde{\LLL} \supseteq \varphi(\LLL)$.
\end{proof}

Let $d(z_1, z_2)$ be the chordal metric on $\widehat{\C} = \C \cup \{\infty \}$:
\begin{eqnarray}
d(z_1, z_2)=
\begin{cases}
\frac{ 2| z_1 - z_2|}{(1 + |z_1|^2)^{1/2} (1 + |z_2|)^{1/2}}, & z_1 , z_2  \in \C ; \cr 
\frac{2}{(1 + |z_1|^{2})^{1/2}} , & z_2 = \infty. 
\end{cases}
\end{eqnarray}
Then $Aut(\widehat{\C})$ is a topological group with respect to the following metric
\begin{align*}
\sigma(f,g) = sup_{z \in \widehat{\C}}d(f(z), g(z)).
\end{align*}

By Lemma \ref{2lemma4}, $\varphi (\in  G[\Lat(\Alg( \{ Q_{\infty},  Q_{0}, Q_{-1} \})]) \rightarrow \rho_{\varphi} $ is an isomorphic 
form $G[\Lat(\Alg( \{ Q_{\infty},  Q_{0}, Q_{-1} \})]$ onto some subgroup of $Aut(\widehat{\C})$ . As will be shown by the next lemma, if
the von Neumann algebra generated by $ Q_{\infty}$, $Q_{0}$ and $Q_{-1}$ is a factor, then the range 
of this map is closed. 

\begin{lemma}\label{2lemma6}
With the notation in Lemma \ref{2lemma4}, let $\mathcal{G} =G[\Lat(\Alg( \{ Q_{\infty},  Q_{0}, Q_{-1} \})]$, and $\Phi(\varphi) = \rho_{\varphi}$ be the M\"{o}bius transformations such that 
$\varphi(Q_{z}) = Q_{\rho_{\varphi}(z)}$,  $\varphi \in \mathcal{G}$. If the finite von Neuamann algebra $\AAA$ generated by $ Q_{\infty}$, $Q_{0}$ and $Q_{-1}$ has a faithful norm trace $\tau$ that is 
$\mathcal{G}$-invariant, i.e., $\tau \circ \varphi = \tau$, $\forall \varphi \in \mathcal{G}$,
then $ \Phi(\mathcal{G})$ is a closed subgroup of $Aut(\widehat{\C})$. Furthermore, $\Phi^{-1}$ is continuous from $\Phi(\mathcal{G})$ onto $\mathcal{G}$.
\end{lemma}

\begin{proof}
Let $\rho_{\varphi_{i}} = \Phi(\varphi_{i})$ be a sequence in $Aut(\widehat{\C})$, where $\varphi_i \in \mathcal{G}$ and $i = 1, 2 \ldots$.
We need to show that if $\rho_{\varphi_{i}} $ converges to $\rho$, then there is a $\varphi \in  \mathcal{G}$ such that $\Phi(\varphi) = \rho$.

Without loss of generality, we could assume that $\AAA$ acts on $L^{2}(\AAA, \tau)$, 
and $\Omega$ is a generating trace vector satisfying $\left\langle A \Omega, \Omega \right \rangle = \tau(A)$. Since $\tau$ is $\mathcal{G}$-invariant, we have $\|\varphi_{i}(A -B) \|_2 = \| A - B \|_2$, where $\|. \|_2$ is the trace norm induced by $\tau$, and $A , B \in \AAA$.
Thus the following mapping 
\begin{align*}
U_{i}:  P(Q_{\infty}, Q_{0}, Q_{-1}) \Omega & \rightarrow P(\varphi_{i}(Q_{\infty}), \varphi_{i}(Q_{0}), \varphi_{i}(Q_{-1})) \Omega\\
& \qquad  = P(Q_{\rho_{\varphi_{i}}(\infty)}, Q_{\rho_{\varphi_{i}}(0)}, Q_{\rho_{\varphi_{i}}(-1)}) \Omega, \qquad i =  1, 2 \ldots ,
\end{align*}
extends to a unitary operator $U_i$ such that $\varphi_{i}(A) = U_{i}AU_{i}^{*}$ for
$A \in \AAA$, where $P(Q_{\infty}, Q_{0}, Q_{-1})$ is a polynomial of $Q_{\infty}$, $Q_{0}$ and $Q_{-1}$. Since $\rho_{\varphi_{i}}$ converges to $\rho$, $Q_{\rho_{\varphi_{i}}(\infty)}$, $Q_{\rho_{\varphi_{i}}(0)}$ and $Q_{\rho_{\varphi_{i}}(-1)}$
converge to $Q_{\rho_{\varphi}(\infty)}$, $Q_{\rho_{\varphi}(0)}$ and $Q_{\rho_{\varphi}(-1)}$ in strong-operator topology. This implies that $P(Q_{\rho_{\varphi_{i}}(\infty)}, Q_{\rho_{\varphi_{i}}(0)}, Q_{\rho_{\varphi_{i}}(-1)})$ converge strongly to $P(Q_{\rho_{\varphi}(\infty)}, Q_{\rho_{\varphi}(0)}, Q_{\rho_{\varphi}(-1)})$, since sums and multiplications of oparators are jointly continuous on the bounded set. 
Note $\|Q_{\rho_{\varphi}(z_1)} -  Q_{\rho_{\varphi}(z_2)}\|_2 = \|Q_{z_1} - Q_{z_2} \|_{2}$, $z_1$, $z_2 \in \{ \infty, 0, -1 \}$, and $Q_{\rho_{\varphi}(\infty)}, Q_{\rho_{\varphi}(0)}, Q_{\rho_{\varphi}(-1)}$ also generates $\AAA$. 
Therefore the mapping 
\begin{align*}
P(Q_{\infty}, Q_{0}, Q_{-1}) \Omega  \rightarrow  P(Q_{\rho_{\varphi}(\infty)}, Q_{\rho_{\varphi}(0)}, Q_{\rho_{\varphi}(-1)}) \Omega
\end{align*}
extends to a unitary transformation $U$ of $L^{2}(\AAA, \tau)$ such that 
$UQ_{\infty}U^{*}  = Q_{\rho_{\varphi}(\infty)}$, $UQ_{0}U^{*}  = Q_{\rho_{\varphi}(0)}$ and 
$UQ_{-1}U^{*}  = Q_{\rho_{\varphi}(-1)}$. Let $\varphi(A) = UAU^{*}$, $\forall A \in \AAA$.
We have $\varphi(\AAA) = \AAA$. By Lemma \ref{2lemma3}, $\varphi$ is in $\mathcal{G}$.

Because each functional $\omega \in \AAA_{*}$ has the form $\sum_{n=1}^{\infty}  \omega_{x_n, y_n}$, with
\begin{align*}
\sum_{n = 1} (\|x_n\|^2 + \|y_n\|^2) < \infty.
\end{align*}
Then $\omega \circ \varphi_{i} = \sum_{n=1}^{\infty}  \omega_{U_{i}^{*}x_n, U_{i}^{*}y_n}$.
Since $\rho_{\varphi_i}$ converges to $\rho_{\varphi}$ implies $U_i^{*}$ tends to $U^{*}$ in strong-operator topology, $\lim_{i} \| \omega \circ \varphi_{i} - \omega \circ \varphi\| = 0$, and $\Phi^{-1}$ is continuous.
\end{proof}

The elements in $Aut(\widehat{\C})$ are commonly classified into three types:
parabolic, loxodromic and elliptic. The next theorem describes the dynamic 
behavior when a transformation is iterated.


\begin{theorem}[\cite{BA}, Theorem 4.3.10]\label{2thm1}
Let $g (\neq I)$ be any $M\ddot{o}bius$ transformation. Then
\begin{enumerate}
\item If $g$ is parabolic with fixed point $\alpha$. Then for all $z$ in $\widehat{\C}$, we have
$g^{n}(z) \rightarrow \alpha$ as $n \rightarrow +\infty$. The convergence being uniform on compact subsets of $\C \setminus \{ \alpha \}$.
\item If $g$ is loxodromic, then the fixed points $\alpha$ and $\beta$ of $g$ can be labeled so that $g^{n}(z) \rightarrow \alpha$ as
$n \rightarrow +\infty ($if $z \neq \beta)$. The convergence being uniform on compact subsets of $\widehat{\C} \setminus \{ \beta \}$.
\item If $g$ is elliptic with fexed points $\alpha$ and $\beta$, then $g$ leaves invariant each circle for which
$\alpha$ and $\beta$ are inverse points.
\end{enumerate}
\end{theorem}

\begin{lemma}
With the notation and assumption in Lemma \ref{2lemma6}, for any $\varphi \in G[\Lat(\Alg( \{ Q_{\infty},  Q_{0}, Q_{-1} \})]$, $\rho_{\varphi}$ must be elliptic.
\end{lemma}

\begin{proof}
Let $\rho_{\varphi^{n}} = \Phi(\varphi^{n})$, where $\varphi \in G[\Lat(\Alg( \{ Q_{\infty},  Q_{0}, Q_{-1} \}$, $ n = 1, 2, \ldots$. If $\rho_{\varphi}$ is not elliptic, without lose of generality we may assume that $\lim_{n \rightarrow +\infty} d(\rho_{n}(\infty), \rho_{n}(0)) = 0$ by Theorem \ref{2thm1}. This implies 
\begin{align*}
\| Q_{\infty} - Q_{0} \|_{2} = \| \varphi^{n}(Q_{\infty} - Q_{0}) \|_{2} =\| Q_{\rho_n(\infty)} - Q_{\rho_n(0)} \|_{2} \rightarrow 0, \quad n \rightarrow +\infty,
\end{align*}
where $\| \cdot \|_2$ is the 2-norm induced by the $ G[\Lat(\Alg( \{ Q_{\infty},  Q_{0}, Q_{-1} \})]$-invariant faithful norm trace.
This contradicts with the fact that $Q_{\infty} \neq Q_{0}$.
\end{proof}

A subgroup of $Aut(\widehat{\C})$ contains only elliptic transformations (and I) is conjugate to a subgroup of rotations of the sphere $S^2$, thus a subgroups of $SU(2)$ (or $SO(3)$).
Furthermore, if a closed subgroup contains only elliptic elements, then it must be compact. 

\begin{lemma}
With the notation and assumption in Lemma \ref{2lemma6},  
$\Phi$ is a homeomorphism between $\mathcal{G} (= G[\Lat(\Alg( \{ Q_{\infty},  Q_{0}, Q_{-1} \}])$ and $\Phi(\mathcal{G})$ when $\mathcal{G}$ and $\Phi(\mathcal{G})$ are endowed with the natural topology and 
the topology induce by the metric $\sigma$ respectively. 
\end{lemma}
\begin{proof}
By the proceeding discussion and Lemma \ref{2lemma6}, $\Phi(\mathcal{G})$ is compact. Since $\Phi^{-1}$ is continuous,
and any every continuous bijective map form compact space to a Hausdorff space is a homeomorphism, $\Phi^{-1}$ is homeomorphism.
\end{proof}


Because any finite factor and finite dimensional von Neumann algebra has a faithful 
norm trace that is fixed by any automorphism, we have the following corollary.

\begin{corollary}\label{2cor2}
With the notation in Lemma \ref{2lemma6}, if the von Neumann algebra generated by $Q_{\infty}$, $Q_{0}$ and $Q_{-1}$ is a finite factor or finite dimensional, then 
$\mathcal{G} = G[\Lat(\Alg( \{ Q_{\infty},  Q_{0}, Q_{-1} \})]$ is homeomorphic to a closed subgroup $\Phi(\mathcal{G})$ of $SO(3)$.
\end{corollary} 

Every finite subgroup of $SO(3)$ is isomorphic to one of the symmetry groups of the regular solids: cyclic groups $C_n$, dihedral groups $D_n$, 
tetrahedral group $T ( \approx A_4) $, octahedral group $O (\approx S_4)$, and the icosahedral group $Y (\approx A_5)$. 
There are also two infinite closed subgroups $C_{\infty} \approx SO(2)$ generated by an arbitrary
rotation around an axis and $D_{\infty}$ which is generated by $C_{\infty}$ and a rotation $\pi$ around an axis orthogonal to
the axis of rotation of $C_{\infty}$.

\begin{prop}\label{2prop1}
Suppose $\{ 0, Q_{\infty}, Q_{0} ,Q_{-1}, I\}$ is a double triangle lattice, and 
$Q_{\infty}$, $Q_{0}$ and $Q_{-1}$ generate a finite von Neumann algebra $\AAA$.
For each $i$ in a index set $\mathcal{I}$, let $z^{i}_{\infty}$, $z_{0}^{i}$ and $z_{-1}^{i}$ be three elements
in $ \widehat{\C}$, and $P_{k} = \sum_{i} Q_{z^{i}_{k}} \otimes E_{i}$ $( k = \infty, 0, -1)$be a projection in 
$\AAA \otimes l^{\infty}(\mathcal{I})$, where $E_{i}$ is the projection onto the space
spanned by $e_{i} = \delta_{i} (\in l^{2}(\mathcal{I})$. Furthermore, we assume that $0 \in \mathcal{I}$ and $z^{0}_{\infty} = \infty$, $z_{0}^{0} =0$ and $z_{-1}^{0}= -1$. Then $\{0, P_{\infty}, P_{0}, P_{\infty}, I \}$ is
a double triangle lattice, and any $P_{z} \in \Lat(\Alg(P_{\infty}, P_{0}, P_{\infty}))$ has the form
$P_{z} = \sum_{i} Q_{\rho^{i}(z)}$, where $\rho^{i}$ is the M\"{o}bius transformation satisfying 
$\rho^{i}(\infty)= z^{i}_{\infty}$, $\rho^{i}(0) = z^{i}_{0}$ and $\rho^{i}(-1) = z^{i}_{-1}$ $(\rho^{0}(z) = z)$.
\end{prop}

\begin{proof}
Since $\AAA \otimes l^{\infty}(\mathcal{I})$ is a finite von Neumann algebra, we could apply 
Theorem \ref{1thm1}, and proceed as in the proof of Lemma \ref{2lemma1}, we have the result.
\end{proof}

\begin{corollary}\label{2cor3}
For any subgroup $\mathrm{G}$ of M\"{o}bius transformations, there is a double triangle 
lattice $\{0, P_{\infty}, P_{0}, P_{-1} , I \}$ such that $\mathrm{G} \subset G[\Lat(\Alg( \{ P_{\infty},  P_{0}, P_{-1} \})]$.
\end{corollary}

\begin{proof}
Let 
\begin{align*}
Q_{\infty} = \left(\begin{array}{cc}1 & 0 \\0 & 0\end{array}\right),
Q_{0} = \left(\begin{array}{cc}0 & 0 \\0 & 1\end{array}\right),
Q_{-1} = \left(\begin{array}{cc}\frac{1}{2} & \frac{1}{2}  \\ \frac{1}{2}  & \frac{1}{2} \end{array}\right).
\end{align*}
From Theorem \ref{1thm1}, we have 
\begin{align*}
Q_{z} = \left(\begin{array}{cc}\frac{|z|^{2}}{1 + |z|^2} & \frac{|z|e^{i(\pi + \theta)}}{1 + |z|^2}  \\ 
 \frac{|z|e^{-i(\pi + i\theta)}}{1 + |z|^2} & \frac{1}{1 + |z|^2}\end{array}\right), \qquad z = |z|e^{i\theta}.
\end{align*}
Let $P_{k} = \sum_{\rho \in \mathrm{G}}Q_{\rho(k)} \otimes E_{\rho} (\in M_{2}(\C) \otimes l^{\infty}(\mathrm{G})$, $k = \infty, 0, -1$, and $U_{\rho} = I \otimes R_{\rho}$, where $R_{\rho}$ is the unitary in right regular representation $R_{\mathrm{G}}$. Then for $\rho_{0} \in \mathrm{G}$, we have 
\begin{align*}
U^{*}_{\rho_{0}} P_{k} U_{\rho_{0}} = \sum_{\rho \in \mathrm{G}}Q_{\rho(\rho^{-1}_{0}(k))} \otimes E_{\rho}, \qquad k = \infty, 0, -1,
\end{align*}
and $U^{*}_{\rho_{0}} P_{k} U_{\rho_{0}} $ is in $\Lat(\Alg( \{ P_{\infty}, P_{0}, P_{-1} \}))$ by Proposition \ref{2prop1}.
Thus, $U^{*}_{\rho_{0}} \{P_{\infty}, P_{0}, P_{-1} \}^{''} U_{\rho_{0}} =   \{P_{\infty}, P_{0}, P_{-1} \}^{''}$ by Lemma
\ref{2lemma3}.
\end{proof}

\begin{remark}
Let $P_{\infty}$, $P_{0}$ and $P_{-1}$ be the projections in the proof of Corollary \ref{2cor3}. If the 
group $\mathrm{G}$ contains a element that is not elliptic, then the von Neumanm algebra $\AAA$
generated by $P_{\infty}$, $P_{0}$ and $P_{-1}$ does not have a faithful norm trace that is preserved by
any automorphisms of $\AAA$, in another word, the center of $\AAA$ is infinite dimensional.  
\end{remark}

In the next section we will determine $ G[\Lat(\Alg( \{ Q_{\infty},  Q_{0}, Q_{-1} \})]$ when $Q_{\infty}$, $Q_{0}$ and $Q_{-1} $ are three trace half free projections.


\section{Automorphism of $L_{F_{\frac{3}{2}}}$ that fixes $S^2$}

We first recall some basic facts and terminology relative to free probability theory.
A non-commutative $W^{*}$-probability space $(\AAA, \tau)$ is a von Neumann algebra $\AAA$ with a normal state $\tau$.
In particular, we only consider the case when $\AAA$ is finite and $\tau$ is a faithful norm trace. 
A family of unital *-subalgebras $\{\AAA_{l} \}_{l \in \mathcal{I}}$ of $\AAA$ is called free if $\tau(a_1 a_2 \cdots a_n) = 0$ 
whenever $a_i \in A_{l_{i}}$, $l_1 \neq l_2 \neq \cdots \neq l_n$ and $\tau(a_{i}) = 0$ for all $i = 1, \ldots n$.
In particular, if $\AAA_l$ is the unital *-algebra generated by $T_l \in \AAA$, then $\{T_l \}_{l \in \mathcal{I}}$ is called *-free.  
Specially, if $\{G_l\}_{l \in \mathcal{I}}$ is a family of groups, then  
the group von Neumann algebras $\{L_{G_l} \}_{l \in \mathcal{I}}$ is free in $L_{G}$, where $G = *_{l \in \mathcal{I}} G_{l}$ is the group free 
product of $\{G_l\}$.For a comprehensive treatment on free probability, we refer to \cite{VDN, NS}. 

Let $F_{\frac{3}{2}} = \Z_{2}*\Z_{2}*\Z_{2}$. $F_{\frac{3}{2}} $ is an i.c.c group so its associated group von Neumann algebra $L_{F_{\frac{3}{2}}}$ is 
a factor of type II$_1$ acting on $l^2(F_{\frac{3}{2}})$.
Throughout this section we assume $Q_{\infty} = \frac{I - U_1}{2}$, $Q_{0} = \frac{I - U_2}{2}$ and $Q_{-1} = \frac{I - U_3}{2}$
where $U_1$, $U_2$ and $U_3$ are canonical generators for $L_{F_{\frac{3}{2}}}$ corresponding to the generators of $F_{\frac{3}{2}}$ with $U_{i}^2 = I$.
From the above discussion, we have $Q_{\infty}$, $Q_{0}$ and $Q_{-1} $ are three free projections with trace $\frac{1}{2}$, and 
satisfy all the requirements in the statements of the Theorem \ref{1thm1} and the Corollary \ref{2cor2}. Thus $G[\Lat(\Alg( \{ Q_{\infty},  Q_{0}, Q_{-1} \}))]$ is isomorphic to 
a closed subgroup of $SO(3)$. 

It is obvious that any permutation of the three free projections $Q_{\infty}$, $Q_{0}$ and $Q_{-1} $ will induce a automorphism of $L_{F_{\frac{3}{2}}}$.
More precisely, we have the following lemma.

\begin{lemma}\label{3lemma1}
With the notation given above, there are two  automorphisms $\varphi_{1}$ and $\varphi_2$ of $L_{F_{\frac{3}{2}}}$ such that 
$\varphi_{i}(Q_{z}) = Q_{\rho_{\varphi_i}(z)}$ for any $Q_{z} \in \Lat(\Alg( \{ Q_{\infty},  Q_{0}, Q_{-1}\}))$, where $\rho_{\varphi_1}(z) = \frac{1}{z}$ and  $\rho_{\varphi_2}(z) = -1 - z$. 
\end{lemma}
\begin{proof}
Let $\varphi_{1}$ be the automorphism satisfying $\varphi_{1}(Q_{\infty}) = Q_{0}$, $\varphi_{1}(Q_{0}) = Q_{\infty}$ and  $\varphi_{1}(Q_{-1}) = Q_{-1}$.
By Lemma \ref{2lemma4}, $\rho_{\varphi_{1}}(z) = \frac{1}{z}$. Similarly let $\varphi_{2}$ be the automorphism that switches $Q_{0}$ and $Q_{-1}$ and 
leave $Q_{\infty}$ fixed, we have $\rho_{\varphi_{2}}(z) = -1 -z$.
\end{proof}

The *-isomorphisms $\varphi_1$ and $\varphi_2$ in the last lemma generate a subgroup of $Aut( L_{F_{\frac{3}{2}}})$, which is isomorphic to $S_3$, the symmetric group of 3 elements.
Later it will be shown that $G[\Lat(\Alg( \{ Q_{\infty},  Q_{0}, Q_{-1} \}))] \approx S_3$. In order to do this, we must dig out more structure information about $\Lat(\Alg( \{ Q_{\infty},  Q_{0}, Q_{-1}\})) $. 
In doing so, we shall adopt the notation and definitions developed in \cite{HH, VDN, NS}. 

Let $\AAA$ be a finite von Neumann algebra. Recall that $\widetilde{\AAA}$ is the *-algebra formed 
by the operators affiliated with $\AAA$.
Let $T \in \widetilde{\AAA}$, by polar decomposition, we have 
\begin{align*}
T = U|T| = U\int^{\infty}_{0} t dE_{|T|}(t),
\end{align*}
where $U \in \AAA$ is a unitary, and $E_{|T|}$ is the spectral measure for $|T|$ taking values in $\AAA$. 
We may define a probability measure $\mu_{|T|}$ by 
\begin{align*}
\mu_{|T|}(B) = \tau(E_{|T|}(B)),  (B \in \mathbb{B}),
\end{align*}
where $\mathbb{B}$ is the $\sigma$-algebra of Borel sets in $\R$. 

\begin{df}[Definition 3.2, \cite{HH}]
$T \in \widetilde{\AAA}$ is said to be $R-diagonal$ if there exist a von Neumann algebra $\mathcal{N}$, with a 
faithful norm trace state $\phi$, and *-free elements $U$ and $H$ in $\widetilde{N}$, such that $U$ is Haar unitary $($ An element $U \in N$ is said to be Haar unitary 
if it is a unitary and $\phi(U^k) = 0$, $\forall k \in \Z \setminus \{0\})$, $H \geq 0$, and 
$T$ has the same *-distribution as $UH$.
\end{df}

We will use the Proposition 3.11 in \cite{HH} which we state here for the convenience of the reader:

\begin{prop}[Proposition 3.11, \cite{HH}] \label{3prop1}
Let $S$, $T \in \widetilde{\AAA}$ be *-free R-diagonal elements.
Then
\begin{align*}
\widetilde{\mu}_{|S+T|} = \widetilde{\mu}_{|S|}\boxplus \widetilde{\mu}_{|T|}.
\end{align*}
Here $\widetilde{\mu}_{|S|}(\widetilde{\mu}_{|T|})$ is the symmetrization of $\mu_{|S|} (\mu_{|T|})$, which is defined by 
\begin{align*}
\widetilde{\mu}_{|S|}(B) = \frac{1}{2}(\mu_{|S|}(B) + \mu_{|S|}(-B)), \qquad (B \in \mathbb{B}).
\end{align*}
\end{prop}

From now on, let $\LLL = \Lat(\Alg( \{ Q_{\infty},  Q_{0}, Q_{-1} \}))$. We will compute the distance form $Q_{z}$ to $Q_{\infty}$, $\forall$ $Q_{z} \in \LLL$.
By the discussion in section 1, we may assume that 
\begin{align*}
Q_{\infty}&= \left(
        \begin{array}{cc}
          I & 0 \\
          0 & 0 \\
        \end{array}
      \right),
Q_{0} = \left(
          \begin{array}{cc}
            H_{1} & \sqrt{H_{1}(I-H_{1})} \\
            \sqrt{H_{1}(I-H_{1})} & I-H_{1} \\
          \end{array}
        \right) , \\
Q_{-1}&= \left(
          \begin{array}{cc}
            H_{2} & \sqrt{H_{2}(I-H_{2})}V \\
            V^{*}\sqrt{H_{2}(I-H_{2})} & V^{*}(I-H_{2})V \\
          \end{array}
        \right).
\end{align*}
The freeness among  $Q_{\infty}$, $Q_{0}$ and $Q_{-1} $ ensures that $H_1$, $H_2$ and $V$ are free, 
$H_1$ and $H_2$ have the same distribution as $\cos^{2}(\frac{\pi}{2} \theta)$ on $[ 0, 1 ]$ with 
respect to Lebesgue measure (\cite[Exmaple 3.6.7]{VDN}) and $V$ a Haar unitary element. And $Q_{\infty} L_{F_{\frac{3}{2}}} Q_{\infty}$ is 
generated by $H_1$, $H_2$ and $V$. Let $tr$ be the faithful norm trace on $ L_{F_{\frac{3}{2}}}$, then 
\begin{align*}
tr\left (\left(
     \begin{array}{cc}
       A_{11} & A_{12} \\
       A_{21} & A_{22} \\
     \end{array}
   \right) \right) = \frac{1}{2}(\tau(A_{11}) + \tau(A_{22})).
\end{align*}
where $\tau$ is the faithful norm trace on $Q_{\infty} L_{F_{\frac{3}{2}}} Q_{\infty}$.

With the notation in Theorem \ref{1thm1}, we have 
\begin{align*}
\parallel Q_{z} - Q_{\infty} \parallel_{2}^{2} = 1-2tr(Q_{z}Q_{\infty}) = 1 - \tau(Q_{\infty}Q_{z}Q_{\infty}|_{Q_{\infty}l^{2}_{F_{\frac{3}{2}}}}) = \tau(I - K_z).
\end{align*}
Therefore we only need to determine the distribution of $K_z$.

\begin{lemma}\label{3lemma2}
With the above notation,  any $Q_{z}$ in $\LLL \setminus \{0, I \}$ can be written as
\begin{align*}
Q_{z} = \left(
     \begin{array}{cc}
      K_{z} & \sqrt{K_{z}(I-K_{z})}U_{z} \\
      U_{z}^{*}\sqrt{K_{z}(I-K_{z})} & U_{z}^{*}(I-K_{z})U_{z} \\
  \end{array}
\right) , \qquad \forall z \in \C,
\end{align*}
where $K_z$ and $U_z$ are determined by the polar decomposition:
\begin{align*}
\sqrt{K_{z}(I-K_{z})^{-1}}U_{z} = (1+z)\sqrt{H_{1}(I-H_{1})^{-1}}- z\sqrt{H_{2}(I-H_{2})^{-1}}V. 
\end{align*} 
Furthermore, we have 
\begin{align*}
d \mu_{\sqrt{\frac{K_z}{I-K_z}}}(t) = \frac{2}{\pi}\frac{|z|+|z+1|}{t^2 + (|z|+|z+1|)^2} 1_{(0, \infty)}(t) dt .
\end{align*}
\end{lemma}
\begin{proof}
The first part of the lemma is just a restatement of Theorem \ref{1thm1}. From the discussion following Proposition \ref{3prop1},
$H_i$($i = 1$, $2$) has the same distribution as $\cos^2\frac{\pi}{2}\theta$ on $[0, 1]$, thus $\sqrt{H_i(I - H_i)^{-1}}$ has
the same distribution as $\cot\frac{\pi}{2}\theta$ on $[0, 1]$. So we have 
\begin{align*}
d\mu_{\sqrt{\frac{H_i}{I - H_i}}}(t) = \frac{2}{\pi}\frac{1}{1+t^2} 1_{(0, \infty)} dt, \qquad i = 1, 2, 
\end{align*}
where $dt$ is the Lebesgue measure on $\R$. Then for any $r \in \R^{+}$,
\begin{align*}
d\mu_r(t)  \overset{\text{def}}{=} d\mu_{(r\sqrt{\frac{H_i}{I - H_i}})}(t) = \frac{2}{\pi}\frac{r}{r^2+t^2} 1_{(0, \infty)} dt.
\end{align*}
Thus the symmetrization of $\mu_r(t)$ is 
\begin{align*}
d\widetilde{\mu_r}(t) = \frac{1}{\pi}\frac{r}{r^2+t^2} 1_{(-\infty, +\infty)} dt.
\end{align*}
The Cauchy transform of $\widetilde{\mu_r}$ is 
\begin{align*}
G_{\widetilde{\mu_r}}(\omega) &= \frac{1}{\pi}\int^{+\infty}_{-\infty} \frac{1}{\omega -t}\frac{r}{r^2+t^2}dt = \frac{1}{z+ri}.   \qquad (Im\omega > 0) 
\end{align*}
Let $F_{\widetilde{\mu_r}}(\omega) = \frac{1}{G_{\widetilde{\mu_r}}(\omega)} = \omega +ri$, and $\varphi_{\widetilde{\mu_r}}(\omega) =
F^{-1}_{\widetilde{\mu_r}}(\omega) -\omega = -ri$.

Since $H_1$, $H_2$ and $V$ are free and $V$ is a Haar unitary, we may assume that $V = V_2V_{1}^*$ with $V_1$ and $V_2$ Haar unitaries, 
and $\{H_1, H_2, V_1, V_2 \}$ is a free family. Thus $\sqrt{K_{z}(I-K_{z})^{-1}}U_{z}V_{1} = (1+z)\sqrt{H_{1}(I-H_{1})^{-1}}V_1- z\sqrt{H_{2}(I-H_{2})^{-1}}V_2$ is 
a sum of two *-free R-diagonal elements.
Form Corollary 5.8 in \cite{HV} and Proposition \ref{3prop1}, we have for any $z \in \C$,
\begin{align*}
\varphi_{\widetilde{\mu_{|z|}}\boxplus \widetilde{\mu_{|z+1|}}}(\omega) &= \varphi_{\widetilde{\mu_{|z|}}}(\omega) + \varphi_{\widetilde{\mu_{|z+1|}}}(\omega) = -i(|z| + |z+1|),\\
F_{\widetilde{\mu_{|z|}}\boxplus \widetilde{\mu_{|z+1|}}}(\omega) &= \omega+ i(|z| + |z+1|),\\
G_{\widetilde{\mu_{|z|}}\boxplus \widetilde{\mu_{|z+1|}}}(\omega) &= \frac{1}{\omega+ i(|z| + |z+1|)},
\end{align*}
therefore
\begin{align*}
d \widetilde{\mu_{\sqrt{\frac{K_z}{I-K_z}}}}(t) &= d\widetilde{\mu_{|z|}}\boxplus \widetilde{\mu_{|z+1|}}(t) =
- \frac{1}{\pi} \lim_{u \rightarrow 0+} Im G_{\widetilde{\mu_{|z|}}\boxplus \widetilde{\mu_{|z+1|}}}(t + iu) \\
&= \frac{1}{\pi}\frac{|z|+|z+1|}{t^2 + (|z|+|z+1|)^2}1_{(-\infty, +\infty)}dt,
\end{align*}
and
\begin{align*}
d \mu_{\sqrt{\frac{K_z}{I-K_z}}}(t) = \frac{2}{\pi}\frac{|z|+|z+1|}{t^2 + (|z|+|z+1|)^2}1_{(0, +\infty)}dt.
\end{align*}
\end{proof}


\begin{lemma}
For any projection $Q_{z} \in \LLL \setminus \{0, I \}$,  $\parallel Q_{z} - Q_{\infty} \parallel_{2} = \sqrt{\frac{1}{1+ |z|+|z+1|}}$, where
$\| \cdot \|_2$ is the 2-norm induced by the faithful norm trace $tr$ on $L_{F_{\frac{3}{2}}}$.
\end{lemma}

\begin{proof}
Let $\Delta_{z} = |z|+|z+1|$. An easy application of the residue theorem gives that 
\begin{align*}
\tau(I- K_z) =\frac{\Delta_z}{2\pi i}\int^{+\infty}_{-\infty} \frac{2i}{(t^2+1)(t^2 + \Delta_{z}^2)}dt =  \frac{1}{1+ \Delta_{z}}.
\end{align*}
Thus
\begin{align*}
\parallel Q_{z} - Q_{\infty} \parallel_{2}^{2} = \tau(I - K_z) =  \frac{1}{1+ |z|+|z+1|}.
\end{align*}
\end{proof}

\begin{corollary}
For any $z \in \widehat{\C}$, we have $\parallel Q_{z} - Q_{0} \parallel_{2} = \sqrt{\frac{|z|}{1+ |z|+|z+1|}}$, and
$\parallel Q_{z} - Q_{-1} \parallel_{2} = \sqrt{\frac{|z+1|}{1+ |z|+|z+1|}}$.
\end{corollary}

\begin{proof}
Let $\varphi_{2}$ be the automorphism in Lemma \ref{3lemma1}. Then we have
\begin{align*}
\parallel Q_{z} - Q_{0} \parallel_{2} =  \parallel Q_{\frac{1}{z}} - Q_{\infty} \parallel_{2} = \sqrt{\frac{|z|}{1+ |z|+|z+1|}}.
\end{align*}
Similarly, the second equation can be proved by considering the automorphism $\varphi_{2} \circ \varphi_{1}$.
\end{proof}


\begin{corollary}
With the notation and assumption in this section, $\| Q_{z_1} - Q_{z_2} \|_{2} = \| Q_{\overline{z}_1} - Q_{\overline{z}_2} \|_{2}$, $\forall$ $z \in \widehat{\C}$.
\end{corollary}

\begin{proof}
Since $\{I-Q_{\infty}, I-Q_{0}, I - Q_{-1} \}$ is also a free family of trace half projections,  the map $Q_{i} \rightarrow I - Q_{i}$ induces an automorphism $\varphi$ of $L_{F_{\frac{3}{2}}}$, 
$i \in \{\infty, 0, -1 \}$. Use Lemma \ref{2lemma1} and argue as in the proof of Lemma \ref{2lemma4}, we have $\varphi(Q_{z}) = I - Q_{\overline{z}}$. Then the equation in the corollary can be 
verified as follows: 
\begin{align*}
\| Q_{z_1} - Q_{z_2} \|_{2} = \| \varphi(Q_{z_1}) - \varphi(Q_{z_2}) \|_{2} =  \|(I - Q_{\overline{z}_1}) - (I - Q_{\overline{z}_2}) \|_{2}
=  \| Q_{\overline{z}_1} - Q_{\overline{z}_2} \|_{2}.
\end{align*}
\end{proof}

Before proceeding to the proof of the fact that $G[\Lat(\Alg( \{ Q_{\infty},  Q_{0}, Q_{-1} \}))] \approx S_3$, we need to learn more about 
the metric on $\widehat{\C}$ given below:
\begin{eqnarray}
dist(z_1, z_2)=
\begin{cases}
\sqrt{\frac{2|z_1 - z_2|}{(1+|z_1|+|z_1 +1|)(1+|z_2|+|z_2 +1|)}}, & z_1 , z_2  \in \C ; \cr 
\sqrt{\frac{1}{(1+|z_1|+|z_1 +1|)}} , & z_2 = \infty. 
\end{cases}
\end{eqnarray}

It is not hard to check that $dist$ is a distance function on $\widehat{\C}$.

\begin{lemma}
With the notation above, $dist$ is a metric on $\widehat{\C}$.
\end{lemma}

\begin{proof}
To check the subadditivity $dist(z_1 , z_3) \leq dist(z_1, z_2) + dist(z_2, z_3)$ when $z_1$, $z_2$, $z_3 \in \C$, we only need to show 
\begin{align*}
|z_1 - z_3|(1 + |z_2| + |z_2 + 1|) \leq |z_1 - z_2|(1 + |z_3| + |z_3 + 1| + |z_2 - z_3|(1 + |z_1| + |z_1 + 1|,
\end{align*}
which is an immediate consequence of the following inequalities: 
\begin{align*}
|z_1 - z_3| &\leq |z_1 - z_2| + |z_2 - z_3|, \\
|z_2||z_1 - z_3| &\leq |z_3||z_1 - z_2| + |z_1||z_2 - z_3|,\\
|z_2 + 1||z_1 - z_3| &\leq |z_3 + 1||z_1 - z_2| + |z_1 + 1||z_2 - z_3|.
\end{align*}
The other cases can be proved similarly.
\end{proof}

The following lemma is an easy observation of plane geometry.

\begin{lemma}\label{3lemma5}
Suppose $E_1$ and $E_2$ are two ellipses that share the common foci at $(\pm c, 0)$.
Let $\frac{x^2}{a_{i}^{2}} + \frac{y^2}{a_{i}^2 - c^2} = 1$ be the equation of $E_i$, $i = 1,2 $.
Then the distance between any two points $(x_1, y_1)$, $(x_2, y_2)$ on $E_1$ and $E_2$ respectively takes the maximum value if
and only if $(x_1, y_1) = (\pm a_1, 0)$ and $(x_2, y_2) = (\mp a_2, 0)$, and the maximum distance is $a_1 + a_2$.
\end{lemma}

\begin{proof}
Let $(x_1, y_1) = (a_1 \cos\theta_1, b_1 \sin \theta_1)$ and $(x_2, y_2) = (a_2 \cos\theta_2, b_2 \sin \theta_2)$, where
$b_i = \sqrt{a_{i}^{2} - c^2}$, $\theta_i \in [0, 2\pi)$ ($i = 1,2$). Then it is easy to see that the following inequality holds, and the conclusion is thus obvious.  
\begin{align*}
&(a_2 \cos\theta_2 - a_1 \cos\theta_1)^2 + ( b_2 \sin \theta_2 -  b_1 \sin \theta_1)^2 \\
&= a_{2}^{2} \cos^{2}\theta_{2} + b^{2}_2 \sin^{2} \theta_2 + a_{1}^{2} \cos^{2}\theta_{1} +  b^{2}_1 \sin^{2} \theta_1 \\
& \qquad   -2a_1 a_2 \cos\theta_{2} \cos\theta_{1} - 2b_1 b_2 \sin \theta_2 \sin \theta_1 \leq (a_1 + a_2)^2.
\end{align*}
\end{proof}

\begin{corollary}\label{3con3}
With the above notation, $dist(z_1 , z_2) \leq \frac{1}{\sqrt{2}}$ and $dist(z_1, z_2) =\frac{1}{\sqrt{2}}$ if and only if
\begin{enumerate}[(i)]
 \item  $z_1 = \infty$ and $z_2 \in [-1, 0]$ or
 \item  $z_1 = 0$ and $z_2 \in (-\infty , -1]$ or
 \item  $z_1 = -1$ and $z_2 \in (0, +\infty)$.
\end{enumerate}
\end{corollary}

\begin{proof}
We need to show
\begin{align*}
4|z_1 - z_2| \leq (1 + |z_1| + |z_1 + 1|)(1 + |z_2| + |z_2 + 1|).
\end{align*}
Assume $z_1$ and $z_2$ are on the ellipses $|z|+|z+1| = r_1$ and
$|z|+|z+1| = r_2$ respectively, where $r_2 \geq r_1 \geq 1$. By Lemma \ref{3lemma5},
we have
\begin{align*}
LHS = 4|z_1 - z_2| \leq 4|\frac{r_2-1}{2} + \frac{r_1 + 1}{2}| = 2(r_1 + r_2).
\end{align*}
Thus $RHS - LHS \geq (r_1 - 1)(r_2 -1) \geq 0$.
\end{proof}

If a closed operator $T$ affiliated with a finite von Neumann algebra $\AAA$ satisfying
\begin{equation}\label{con1}
\begin{split}
\int_{0}^{+\infty} log^{+} t d\mu_{|T|}(t) < \infty,
\end{split}
\end{equation}
then the Fuglede-Kadison determinant of $T$ 
\begin{align*}
\triangle(T) = exp \left(\int_{0}^{\infty} log t d \mu_{|T|}(t) \right) \in [ 0, +\infty)
\end{align*}
is well-defined (\cite{FK}), where $log^{+} (t) = max( 0, log(t))$.

Let $L(T) = log(\triangle(T))$.
It is proved in \cite{HH} that if $S$, $T$ both satisfy the condition (\ref{con1}), then $ST$ also satisfies (\ref{con1}), and $\triangle(ST) = \triangle(T)\triangle(S)$.
Thus $L(ST) = L(S) + L(T)$. This fact will be used in the proof of the next lemma.

\begin{theorem}\label{3thm1}
$L(I - Q_{z_1}Q_{z_2}Q_{z_1}) = 2log(dist(z_1, z_2))$, $z_1$, $z_2 \in \C$.
\end{theorem} 

\begin{proof}
To avoid confusion, we will use $L_{1}$ and $L_2$ to denote the function $L$ defined in the proceeding discussion on $L_{F_{\frac{3}{2}}}$ and 
$Q_{\infty}L_{F_{\frac{3}{2}}} Q_{\infty}$ respectively. 
As in the proof of Lemma \ref{2lemma0}, let $W$ be the unitary such that 
$WQ_{z_1}W = Q_{\infty}$ and $WQ_{\infty}W = Q_{z_1}$. Since $L_{1}(U^{*}TU) = L_{1}(T)$ for any unitary $U$ in $L_{F_{\frac{3}{2}}}$,
we have $L_{1} (I - Q_{z_1}Q_{z_2}Q_{z_1}) = L_{1} (I - Q_{\infty}WQ_{z_2}WQ_{\infty})$. From (\ref{eq5}),
\begin{align*}
I - Q_{\infty}WQ_{z_2}WQ_{\infty} = 
 \left(
        \begin{array}{cc}
          |z_1 - z_2|^{2}\sqrt{I-K_{z_1}}U_{z_1}S^{*}(I-K_{z_2})SU_{z_1}^{*}\sqrt{I-K_{z_1}} & 0 \\
          0 & I \\
        \end{array}
      \right).
\end{align*}
Note $L_{2}(V) = 0$ for any unitary $V$, we have 
\begin{align*}
L_1(I - Q_{\infty}WQ_{z_2}WQ_{\infty}) &= \frac{1}{2}(L_{2}(|z_1 - z_2|^{2}\sqrt{I-K_{z_1}}U_{z_1}S^{*}(I-K_{z_2})SU_{z_1}^{*}\sqrt{I-K_{z_1}}) + L_{2}(I)) \\
                                                              &= \frac{1}{2}(2L_{2}(|z_1 - z_2|) + L_{2}(S^*) + L_{2}(S) + L_{2}(I-K_{z_1}) + L_{2}(I-K_{z_2}) ).
\end{align*}
Recall $d \mu_{\sqrt{\frac{K_z}{I-K_z}}}(t) = \frac{2}{\pi}\frac{\Delta_{z}}{t^2 + \Delta_{z}^2}1_{(0, +\infty)}dt$, where $\Delta_{z} = |z|+|z+1|$. Therefore
\begin{align*}
L_{2}(I - K_{z_{i}}) = -\frac{2}{\pi}\int^{+\infty}_{0}log(t^{2} + 1)\frac{\Delta_{z_i}}{t^{2} + \Delta_{z_i}^{2}}dt  = -2log(\Delta_{z_i} + 1).
\end{align*}
Proceed as in the proof of Lemma \ref{3lemma2}, we have $d_{\mu_{|S|}}(t) = \frac{4}{\pi} \frac{1}{t^2 + 4} 1_{(0, +\infty)}(t)dt$, thus
\begin{align*}
L_{2}(S) = \frac{4}{\pi}\int_{0}^{\infty} \frac{log (t)}{t^2 + 4} dt = log(2).
\end{align*}
It is clear that $L_{2}(S^{*}) = L_{2}(S)$ and $L_{2}(|z_1 - z_2|) = log(|z_1 - z_2|)$, so we have 
\begin{align*}
L_{1}(I - Q_{\infty}WQ_{z_2}WQ_{\infty}) & =  log(|z_1 - z_2|) + log(2) -log(\Delta_{z_1} + 1) - log(\Delta_{z_2} + 1)\\
                     &= log(\frac{2|z_2 - z|}{(1 + |z| + |z+1|)(1 + |z_2| + |z_2 + 1|)}).
\end{align*}
\end{proof}

\begin{corollary}
$G[\Lat(\Alg( \{ Q_{\infty},  Q_{0}, Q_{-1} \}))] \approx S_3$.
\end{corollary}

\begin{proof}
Suppose $\varphi$ is a automorphism in $G[\Lat(\Alg( \{ Q_{\infty},  Q_{0}, Q_{-1} \}))]$, and $\rho$ is the M\"{o}binus transformation such that
$\varphi(Q_{z}) = Q_{\rho_{z}}$, $\forall z \in \widehat{\C}$. Let $\rho(\infty) = z_1$, $\rho(0) = z_2$ and $\rho(-1) = z_3$.
First assume that $z_1$, $z_2$ and $z_3$ are in $\C$. From Theorem \ref{3thm1} and the freeness of $\{Q_{z_1}, Q_{z_2}, Q_{z_3} \}$, we have 
\begin{align*}
2log(\frac{1}{\sqrt{2}}) =  L(I - Q_{z_i}Q_{z_j}Q_{z_i}) = 2log(dist(z_i, z_j)), \quad i \neq j, i, j \in \{1, 2, 3 \}.
\end{align*}
This is impossible by Corollary \ref{3con3}. Thus one of $z_1$, $z_2$ and $z_3$ is $\infty$. Without loss of generality, 
we could assume that $z_1 = \infty$. Since $\|Q_{\infty} - Q_{z_k} \| = dist(\infty , z_k) = \frac{1}{\sqrt{2}}$, $k = 2, 3$.
$z_2, z_3$ must be in $ [-1, 0]$. Apply Corollary \ref{3con3} again, we have $\{z_2, z_3\} = \{ 0, -1 \}$, since $dist(z_2,z_3) = \frac{1}{\sqrt{2}}$.  
Therefore $\varphi$ must be in the group generated by the two automorphisms in Lemma \ref{3lemma1}.
\end{proof}

\appendix
\section{Some technique results on reflexive lattices}
The purpose of this appendix is to prove that the reflexivity of a lattice of projections in a von Neumann algebra is 
independent of any particular faithful representation. Precisely, if $\LLL$ is a reflexive subspace lattice in a von Neumann algebra $\AAA$, 
$\varphi$ is a *-isomorphism of $\AAA$, and $\varphi(\AAA)$ is in $\B(\KKK)$, then $\varphi(\LLL)$is also reflexive.
This result was mentioned by Morhan in \cite{Mo}, but his proof was incomplete. In the following we will provide a detailed proof of this fact.

\begin{lemma}\label{alemma1}
Let $\AAA$ $(\subset \B(\HHH))$ be a von Neumann algebra, $\LLL$ is a
subspace lattice in $\AAA$. Suppose $P', Q'$ are two projections in $\AAA'$ such that $P' \precsim Q'$, and $C_{P'} = C_{Q'}$, where $C_{P'}$ and $C_{Q'}$ are the central carriers of $P'$ and $Q'$ respectively. Let $P'\LLL = \{P'E | E \in \LLL \}$. If $P'\LLL$ is reflexive as a subspace lattice in $\B(P' \HHH)$, then we have $Q'\LLL (=  \{Q'E | E \in \LLL \})$ is also reflexive as a subspace lattice in $\B(Q' \HHH)$.
\end{lemma}

\begin{proof}
Without loss of generality, we may assume that $C_{P'} = I$ and $Q' = I$. So we only need to show that if $P'\LLL$ is reflexive and $C_{P'} = I$, then $\LLL$ is also reflexive.

For any  $T_1$ in $\Alg(P'\LLL)$, let $T$ be the operator
in $\B(\HHH)$ such that $T(I-P')= (I-P')T = 0$, and $TP'|_{P'\HHH} = T_1$.
If $E$ is in $\LLL$, then $(I-E)TE = (P' - P'E)TP'E = 0$, since $T_1$ is in $\Alg(P'\LLL)$.
Because $\Lat(\Alg(\LLL))$ is a subset of $\AAA$, we have
$(P' - P'F)P'TP'F = P'(I-F)TFP' = 0$ for any $F$ belongs to $\Lat(\Alg(\LLL))$.
This implies $FP'|_{P'\HHH} \in \Lat(\Alg(P'\LLL)) = P'\LLL$. 
The map
\begin{align*}
\AAA \rightarrow P'\AAA: A \rightarrow AP'|_{P'\HHH}
\end{align*}
is $*-$isomorphism since $C_{P'} = I$, therefore $F$ must be in $\LLL$.
\end{proof}

\begin{example}
The condition $C_{P'} = I$ in the above lemma cannot be removed. Let 
\begin{align*}
E_1 = \left(\begin{array}{ccc}1 & 0 & 0 \\0 & 0 & 0 \\0 & 0 & 0\end{array}\right),
E_2 = \left(\begin{array}{ccc}0 & 0 & 0 \\0 & 1 & 0 \\0 & 0 & 0\end{array}\right),
E_3= \left(\begin{array}{ccc}\frac{1}{2} & \frac{1}{2} & 0 \\ \frac{1}{2} & \frac{1}{2} & 0 \\0 & 0 & 0\end{array}\right),
P' = \left(\begin{array}{ccc}0 & 0 & 0 \\0 & 0 & 0 \\0 & 0 & 1\end{array}\right),
\end{align*}
and $\LLL = \{ 0, E_1, E_2, E_3, I \}$. Then $C_{P' } = P' \neq I$, and $P' \LLL $ is reflexive.
However, it is not hard to check that $\LLL$ is not reflexive.
\end{example}

In following lemma is easy, and we omit the proof.

\begin{lemma}\label{alemma2}
Suppose $\{\HHH_{i} \}_{i \in \mathcal{I}}$ is a family of Hilbert spaces, where $\mathcal{I}$ is an index set. 
Let $\HHH = \oplus_{i} \HHH_i$, and $E_i$ is the orthonormal projection form $\HHH$ onto $\HHH_i$.
If $\LLL$ is a subset of projections in $ {\{E_i\}'}_{i \in \mathcal{I}}$. For each projection $P \in \LLL$, let
$P_{i} = E_{i}P |_{\HHH_i}$. Then
\begin{align*}
\Alg(\LLL) = \{ A  \in  \B(\HHH) | (E_i - P_{i})E_i A E_j P_{j} = 0, \mbox{ for all } i, j \in \mathcal{I} \mbox{ and } P \in \LLL \}.
\end{align*}
\end{lemma}

With the same notations in the above lemma, we give the following definition.

\begin{df}\label{adef1}
An extension index $ex$ of $\LLL$ is a map assigning a cardinal number $ex(i) $ to each $i$ in $\mathcal{I}$.
Let $\mathcal{I}^{ex} = \{ (i, j) | i \in \mathcal{I} \mbox{, and } j \in \mathcal{I}(ex(i)) \}$, and 
 $\HHH^{ex} = \oplus_{i \in \mathcal{I}}\HHH_{i}\otimes l^{2}(ex(i)) = \oplus_{(i,j) \in \mathcal{I}^{ex}} \HHH_{i}$,
where $\mathcal{I}(ex(i))$ is a set with cardinal $ex(i)$, and 
$ l^{2}(ex(i))$ is the Hilbert space with dimension $ex(i)$.
For each $P \in \LLL$, let $P^{ex}$ be a projection in $\B(\HHH^{ex})$ such that $E_{i} \otimes F_{j}^{i} P^{ex} = P^{ex}E_{i} \otimes F_{j}^{i} = P_i |_{\HHH_{i}}$, 
where $E_{i} \otimes F_{j}^{i}$ is the projection form $\HHH^{ex}$ onto the jth copy of $\HHH_{i}$ in $\HHH^{ex}$. 
And we will use $\LLL^{ex}$  to denote the set $\{ P^{ex} | \mbox{for all } P \in \LLL \}$.
\end{df}

\begin{lemma}\label{alemma3}
With the notations in Lemma \ref{alemma2} and Definition\ref{adef1}, let $ex$ be an extension index, then 
$\LLL$ is reflexive if and only if $\LLL^{ex}$ is reflexive.
\end{lemma}

\begin{proof}
First note $\bigoplus_{i} I \otimes \B(l^{2}(ex(i))$ is in $\{\LLL^{ex}\}'$, this implies that $\Lat(\Alg(\LLL^{ex})) \subset 
\bigoplus_{i} \B(\HHH_i) \otimes I_{ex(i)}$. Thus for any 
$Q^{ex} \in \Lat(\Alg(\LLL^{ex}))$, there exist $Q_i \in \B(\HHH_i)$, $i \in \mathcal{I}$, and $
Q^{ex} = \bigoplus_{i \in \mathcal{I}} Q_{i} \otimes I_{ex(i)} = \oplus_{(i,j) \in \mathcal{I}^{ex}} Q_{i}$.

We need to show 
\begin{align*}
Q^{ex} = \oplus_{(i,j) \in \mathcal{I}^{ex}} Q_{i} \in \Lat(\Alg(\LLL^{ex})) \Longleftrightarrow
Q = \oplus_{i \in \mathcal{I}} Q_{i}  \in \Lat(\Alg(\LLL)).
\end{align*}

By Lemma\ref{alemma2}, $Q =\oplus_{i} Q_{i}$ is in $\Lat(\Alg(\LLL))$ if and only if 
\begin{align*}
(I - Q_{i})A_{i j}Q_{j} = 0, \forall i, j \in \mathcal{I},
\end{align*}
where $A_{i j}$ is any operator in $\B(\HHH_{j}, \HHH_{i})$ such that 
$(I-P_{i})A_{i j}P_{j} = 0$ for any $P \in \LLL$.
Similarly,  $\oplus_{(i,j) \in \mathcal{I}^{ex}} Q_{i} \in \Lat(\Alg(\LLL^{ex})$ if and only if 
\begin{align*}
(I - Q_{i})A_{(i,k),(j,l)}Q_{j} = 0, \forall (i,k), (j,l) \in \mathcal{I}^{ex},
\end{align*}
where $A_{(i,k),(j,l)}$ is any operator in $\B(E_{j} \otimes F_{l}^{j}\HHH^{ex}, E_{i} \otimes F_{k}^{i}\HHH^{ex})$ such that 
$(I-P_{i})A_{(i,k),(j,l)}P_{j} = 0$ for any $P \in \LLL$. This end the proof, since $E_{j} \otimes F_{l}^{j}\HHH^{ex}$ is just $\HHH_{j}$.
\end{proof}

Let $\AAA$ be a von Neumann algebra, we say that a projection $E$ with central carrier $C_{E}$ in $\AAA$ is
monic if $E \neq 0$ and there exist a positive integer $k$ and projections $E_1, \ldots , E_k$ in $\AAA$
such that
\begin{align*}
E_1 \sim E_2 \sim \cdots \sim E_k \sim E , \mbox{   } E_1 + E_2 + \cdots + E_k = C_{E}.
\end{align*}

The following lemma is Proposition 8.2.1 in \cite{RK}.
\begin{lemma}\label{alemma4}
Each non-zero projection $E$ in a finite von Neumann algebra $\AAA$ is the sum of an orthogonal family
of monic projections in $\AAA$.
\end{lemma}

The following corollary is then immediate consequence of this lemma.

\begin{corollary}\label{acor1}
Suppose $\AAA$ is a finite von Neumann algebra, then for any non-zero projection $P$ in $\AAA$, there exists a non-zero projection $Q$ in $\AAA$
such that  $Q \leq P$ and $C_{Q} = Q + \sum_{i}Q_i$, where $\{Q_i\}$ is a mutually orthogonal family of projections in $\AAA$ such that $Q_i \sim Q$ for 
each $i$.   
\end{corollary}

In the following, we will prove that the statement in Corollary \ref{acor1} is true for any von Neumann algebra. 
Since any von Neumann algebra $\AAA$ is a direct 
sum of algebras of  types I$_{n} ( n = 1, 2, 3 \ldots)$, II$_1$, II$_{\infty}$ and III,  it is suffices to consider
only the three cases in which $\AAA$ is type I$_{\infty}$, II$_{\infty}$ or III. For basic comparison theory of
projections, we refer to \cite{RK}.

\begin{lemma}\label{alemma5}
Suppose $\AAA$ is a type III von Neumann algebra, then for any non-zero projection $P$ in $\AAA$,  there exists a non-zero projection $Q$ in $\AAA$
such that  $Q \leq P$, and $C_{Q} = Q + \sum_{i}Q_i$, where $\{Q_i\}$ is a mutually orthogonal family of projections in $\AAA$ such that $Q_i \sim Q$ for 
each $i$.   
\end{lemma}

\begin{proof}
By considering $C_P \AAA$, we could assume that $C_P = I$. Since any properly infinite projection can be "halved," 
we can find two projection $P_1$ and $P_2$ such that $P = P_1 + P_2$ and $P \sim P_1 \sim P_2$. 
Let $\{P_\alpha \}$ be a maximal family of equivalent mutually orthogonal projections that contains
$\{P_1, P_2\}$. If $\sum_{\alpha} P_\alpha = I$, let $Q = P_1$. Otherwise, by maximality of $\{P_{\alpha} \}$, 
there exists a non-zero central projection $E$ such that
\begin{align*}
0 \neq F = (I - \sum_{\alpha} P_\alpha)E \prec P_1 E.
\end{align*}
Thus we have
\begin{align*}
EP_1 \precsim EP_2 + F \precsim E(P_1 + P_2) \sim EP_1,
\end{align*}
which implies $EP_1 \sim EP_2 + F$. So we could let $Q = EP_1$ (Note that $C_{EP_1} = E$).
\end{proof}

\begin{lemma}\label{alemma6}
If a von Neumann algebra $\AAA$ is type $I_{\alpha}$$(\alpha$ is a infinite cardinal number$)$ or type II$_{\infty}$, and $P$ is a non-zero projection in $\AAA$, 
then there exists a non-zero subprojection $Q$ of $P$ such that $C_{Q} = Q + \sum_{i}Q_i$, where $\{Q_i\}$ is a mutually orthogonal family of projections in $\AAA$ such that $Q_i \sim Q$ for 
each $i$.   
\end{lemma}

\begin{proof}
By choosing a proper finite projection with central carrier $I$, we may assume $\AAA = \AAA_1 \overline{\otimes} \B(\HHH)$ where 
$\HHH$ is a infinite dimensional Hilbert space, and $\AAA_1$ is a finite von Neumann algebra. Let $\{ E_{\alpha, \beta} \}$ be a system of matrix units for $\B(\HHH)$.
If there is a central projection $F$ such that $0 \neq FE_{\alpha, \alpha} \sim Q \leq FP$ for some $E_{\alpha, \alpha}$,
then $Q$ satisfies all the requests in the lemma. Indeed, since $FE_{\alpha, \alpha} \sim Q$ and $FE_{\alpha, \alpha}$ is a finite projection,
there exists a unitary $U$ in $\AAA$ such that $Q = U^* FE_{\alpha, \alpha}U$. It clear that $Q \sim U^{*} FE_{\beta,\beta} U$ and $\sum_{\beta} FE_{\beta, \beta} = F$, 
thus we have 
\begin{align*}
C_Q = F = U^{*}FE_{\alpha,\alpha} U + \sum_{\beta \neq \alpha}U^{*} FE_{\beta,\beta} U = Q + \sum_{\beta \neq \alpha}U^{*} FE_{\beta,\beta} U.  
\end{align*}

We may, thus, assume that $P \prec E_\alpha$. Let $U$ be a unitary in $\AAA$ such that $P \leq U^* E_\alpha U$. 
Replacing $\{E_{\alpha, \beta} \}$ by  $\{U^* E_{\alpha, \beta}U \}$, we may now assume that  
$P \leq E_{\alpha, \alpha}$. Remember $\AAA_{1} \cong E_{\alpha, \alpha} \AAA E_{\alpha, \alpha}$ is a finite von Neumann algebra. From Corollary \ref{acor1}, there is 
a monic projection $Q$ in $E_{\alpha, \alpha} \AAA E_{\alpha, \alpha}$ such that $Q  \leq P$ and $\widetilde{C_{Q}} = Q + \sum_{i}Q_i$, where $\widetilde{C_{Q}}$ is the central carrier of 
$Q$ in $E_{\alpha, \alpha} \AAA E_{\alpha, \alpha}$. Since the central carrier of $Q$ in $\AAA$ is $C_{Q} = \sum_{\beta}E_{\beta, \alpha}\widetilde{C_{Q}}E_{\alpha, \beta}$, we have
\begin{align*}
C_{Q} = Q + \sum_{i}Q_i + \sum_{\beta \neq \alpha} E_{\beta, \alpha} (Q + \sum_{i}Q_i)E_{\alpha, \beta}.
\end{align*}
\end{proof}

By combining the above lemmas, we have the following theorem.

\begin{theorem}\label{athm1}
Suppose $P$ is a non-zero projection in a von Neumann algebra $ \AAA$, then 
there exists  a non-zero subprojection $Q$ of $P$ and a mutually orthogonal family of projections  $\{Q_i\}$ in $\AAA$ such that $C_{Q} = Q + \sum_{i}Q_i$ and $Q_i \sim Q$ for 
each $i$.   
\end{theorem}

The following corollary is immediate by a maximality argument, and we omit the proof.

\begin{corollary}\label{acor2}
Let $\AAA$ be a von Neumann algebra and $P$ be a projection in $\AAA$, then there is an orthogonal family $\{Q_{\alpha} \}$ of subprojections of $P$ in $\AAA$ such that 
$C_P = \sum_{\alpha}C_{Q_\alpha} (C_{\alpha_1} \perp C_{\alpha_2}, \alpha_1 \neq \alpha_2)$. Moreover, 
$C_{Q_\alpha} = Q_\alpha + \sum_{i}Q_{\alpha}^{i}$ where $\{Q_{\alpha}^{i} \}$ is a orthogonal family of projections in $\AAA$
such that $ Q_{\alpha}^{i} \sim Q$ for each $i$.
\end{corollary}

\begin{theorem}\label{athm2}
Suppose $\LLL$ is a reflexive subspace lattice in a von Neumann algebra $\AAA_1 (\subset \B(\HHH_1))$. If $\varphi$ is a *-isomorphism of $\AAA_1$ onto 
$\AAA_2 (\subset \B(\HHH_2))$, then $\varphi(\LLL)$ is also reflexive as a subspace lattice in $\B(\HHH_2)$.
\end{theorem}

\begin{proof}
By \cite[Theorem 5.5]{Ta}, there exists a Hilbert space $\KKK$, a projection $P' \in \AAA' \overline{\otimes} \B(\KKK)$ with central carrier I, and 
an unitary $U$ of $P'(\HHH_1 \otimes \KKK)$ onto to $\HHH_2$ such that 
\begin{align*}
\varphi(A) = U(A \otimes I_{\KKK})P'|_{P'(\HHH_1\otimes \KKK)}U^{*}, A \in \AAA.
\end{align*} 
In other words, $\varphi$ can be decomposed into the composition of an amplification, an induction, and a spatial 
isomorphism. Since amplification and spatial isomorphism preserve reflexivity, we could assume that $\varphi$ is an 
induction. And we only need to show $P'\LLL \subset P'\AAA|_{P'\HHH}$ is reflexive if $\LLL \subset \AAA$ is reflexive
, where $P'$ is a projection in $\AAA'$ such that $C_{P'} = I$.

By Theorem \ref{athm2},  there is a family $\{{Q'}_{\alpha} \}$ of subprojections of $P'$ in $\AAA'$ such that
$\sum_{\alpha}C_{{Q'}_{\alpha}} = I = C_{P}$, and $C_{{Q'}_\alpha} = {Q'}_\alpha + \sum_{i}{Q'}_{\alpha}^{i}$, 
where ${Q'}_{\alpha}^{i} \sim {Q'}_{\alpha}$. Let $Q' = \sum_\alpha {Q'}_{\alpha} \leq P'$. It is obvious $C_{Q'} = I$, so
if $Q'\LLL \subset Q' \AAA |_{Q' \HHH}$ is reflexive, 
then $P'\LLL$ is reflexive  by Lemma \ref{alemma1}. 

Let $W^{i}_{\alpha}$ be a partial isometry in $\AAA'$ with initial projection ${Q'}^{i}_{\alpha}$ and final projection ${Q'}_{\alpha}$, i.e. 
${W^{i}_{\alpha}}^{*} W^{i}_{\alpha} = {Q'}^{i}_{\alpha}$ and $W^{i}_{\alpha} {W^{i}_{\alpha}}^{*} = {Q'}_{\alpha}$. Then the 
equation
\begin{align*}
V  = \oplus_{\alpha} ({Q'}_{\alpha} \oplus (\oplus_{i}W^{i}_{\alpha}))
\end{align*} 
defines a unitary operator from $\HHH$ onto $ \bigoplus_{\alpha}({Q'}_{\alpha}\HHH \oplus (\oplus_{i}{Q'}_{\alpha}\HHH))$.

Replacing $\HHH$ by $\bigoplus_{\alpha}({Q'}_{\alpha}\HHH \oplus (\oplus_{i}{Q'}_{\alpha}\HHH))$,
$\AAA$ by $V\AAA V^{*}$$(\AAA'$ by $V\AAA' V^{*})$ and $Q'$ by $VQ'V^{*} = \oplus_{\alpha} (I_{\alpha} \oplus_{i} (0)) $, we may now assume that $\AAA' = \bigoplus_{\alpha} {Q'}_{\alpha}\AAA' {Q'}_{\alpha} \otimes \B(l^2(ex(\alpha))$,
where $ex(\alpha)$ equals the cardinal of  set $\{ {Q'}_{\alpha} \} \cup \{ {Q'}_{\alpha}^{i} \}$, and $l^2(ex(\alpha))$ is a Hilbert space with dimension $ex(\alpha)$. 
Under this assumption, $\LLL$ is just $ (Q' \LLL)^{ex}$. Thus 
by Lemma \ref{alemma3}, $\LLL$ is reflexive if and only if $Q'\LLL$ is reflexive, and the theorem follows.
\end{proof}

\begin{corollary}\label{acor3}
Suppose $\LLL$ is a KS-lattice for a von Neumann algebra $\AAA$ acting on some Hilbert space $\HHH)$, 
if $\varphi$ is a *-isomorphism of $\AAA$ and $\varphi(\AAA) \subset \B(\KKK)$,
then $\varphi(\LLL)$ is also KS-lattice for $\varphi(\AAA)$.
\end{corollary}

\begin{proof}
If $\varphi(\LLL)$ is not a KS-lattice, then there is a reflexive sublattice $\LLL_0$ of $\varphi(\LLL)$ such that $\LLL_0$ 
generates $\varphi(\AAA)$ as von Neumann algebra. So $\varphi^{-1}(\LLL_0)$ is a reflexive lattice that generates $\AAA$ 
by Theorem \ref{athm2}, which contradicts the minimality of  $\LLL$.
\end{proof}


\begin{thebibliography}{21}

\bibitem{RK} R. Kadison and John R. Ringrose, {\em Fundamentals of the theory of Operator Algebras},  {\bf Volume II } (1997)

\bibitem{FK} B. Fuglede and R. Kadison, {\em Determinant Theory in Finite Factors}, The Annals of Mathematics, Second Series, Vol. 55, No. 3 (1952), 520-530

\bibitem{Ta} M. Takesaki, {\em Theory of Operator Algebras}, {\bf Volume I}, Encyclopedia of Mathematical Sciences, vol 124, Springer-Verlag, Berlin (2002)

\bibitem{Mo} Mohan Ravichandran, {University of New Hampshire Ph.D. Thesis} (2009) 
\bibitem{GYII} L. Ge and W. Yuan, {\em Kadison-Singer algebras,
II---General case,} to appear, 2009

\bibitem{HF} Uffe Haagerup and Flemming Larsen, {\em Brown's Spectral Distribution Measure for
R-diagonal Elements in Finite von Neumann Algebras,}, 1999

\bibitem{HH} Uffe Haagerup and Hanne Schultz, {\em Brown Measures of Unbounded Operators Affiliated with
 Finite von Neumann Algebras,}, 1999

\bibitem{HV} Hari Bercovici and Dan Voiculescu, {\em Free Convolution of Measures with Unbounded Support, }

\bibitem{KS} $K.S.K\ddot{O}LBIG$, {\em On The Value Of A Logarithmic-Trigonometric Integral}, BIT 11(1971), 21-28

\bibitem{MR} Mehmet Koca, Ramazan Koc and Muataz Al-Barwani, {\em Breaking SO(3) into its closed subgroups by Higgs mechanism},
J.Phys.A: Math.Gen. 30(1997) 2109-2125

\bibitem{BA} Beardon, Alan F, {\em The Geometry of Discrete Groups}, New York: Springer-Verlag GTM 91.

\bibitem{AR} Alexandru Nica and Roland Speicher, {\em Lectures on the Combinatorics of Free Probability},
London Mathematical Society Lecture Note Series:335

\bibitem{SS} Elias M. Stein and Rami Shakarchi, {\em complex analysis}, Priceton lectures in analysis II.

\bibitem{Hou} Chengjun Hou and Wei Yuan, {\em Minimal Generating Reflexive Lattices of Projections in Finite von Neumann Algebras}

\bibitem{LU} R. C. Lyndon and J. L. Ullman {\em Groups of Elliptic Linear Fractional Transformations}, Proceedings of the American Mathematical Society,   Vol. 18, No. 6, (1967) 1119-1124

\bibitem{SA} S. Sakai, {\em On automorphism groups of II$_1$-factors}, T\^{o}koku Math. J. (2) 26 (1974), 423-430.

\bibitem{WA} William L. Green and Anthony To-Ming Lau {\em Strong finite von Neumann algebras }, Math. Scand. 40 (1977), 105- 112.

\bibitem{MRM} M. Koca, R. Koc and M. Al-Barwani {\em Breaking SO(3) into its closed subgroups by Higgs mechanism}, J.Phys. A: Math. Gen. 30 (1997) 2109-2125

\bibitem{VDN} D. Voiculescu, K. Dykema and A. Nica, {\em Free Random Variables}, CMR Monograph Series 1, American Mathematical Society (1992)

\bibitem{NS} A. Nica and R. Speicher, {\em Lectures on the Combinatorics of Free Probability}, Londom Mathematical Society Lecture Note Series: 335 (2006)

\bibitem{MV2} F. J. Murray and J. von Neumann, {\em On rings of operators, II}, Trans. Amer. Math. Soc. 41 (1937), 208 - 248

\bibitem{Zhe} Z. Liu, {\em On Some Mathematical Aspects of The Heisenberg Relation }, Science China series Mathematics: Kadison's proceedings

\bibitem{MK} Masoud Khalkhaili, {\em Basic Noncommutative Geometry}, EMS series of Lectures in Mathematics (2009)

\end{thebibliography}
\end{document}
