\documentclass[a4paper,10pt]{amsart}

\usepackage[protrusion=true,expansion=true]{microtype} 
\usepackage{fancyhdr}
\usepackage[utf8]{inputenc}
\usepackage{graphicx} 
\usepackage{wrapfig} 

\usepackage{mathpazo}
\usepackage[T1]{fontenc}
\usepackage{amsmath}
\usepackage{amssymb}
\usepackage{hyperref}
\usepackage{cleveref}
\usepackage{comment}
\usepackage{color}


\newtheorem{example}{Example}[section]
\newtheorem{theorem}{Theorem}[section]
\newtheorem{proposition}{Proposition}[section]
\newtheorem{corollary}{Corollary}[section]
\newtheorem{definition}{Definition}[section]
\newtheorem{lemma}{Lemma}[section]
\newtheorem{remark}{Remark}[section]
\newtheorem{question}{Question}[section]

\crefname{lemma}{Lemma}{lemmas}
\crefname{remark}{Remark}{remark}
\crefname{corollary}{Corollary}{corollary}
\crefname{theorem}{Theorem}{theorem}
\crefname{example}{Example}{example}
\crefname{definition}{Definition}{definition}

\newcommand{\AAA}{\mathfrak A}
\newcommand{\BBB}{\mathcal B}
\newcommand{\CCC}{\mathcal C}
\newcommand{\HHH}{\mathcal H} %for Hilbert space
\newcommand{\LLL}{\mathcal L} % for lattice
\newcommand{\MMM}{\mathcal M}


\newcommand{\Lat}{\mathcal Lat}
\newcommand{\Alg}{\mathcal Alg}
\newcommand{\tr}{\tau}
\newcommand{\C}{\mathbb C} %for complex number
\newcommand{\R}{\mathbb R}  %for real number
\newcommand{\Z}{\mathbb Z} %for integer
\newcommand{\N}{\mathbb N} % for nature number
\newcommand{\M}{\mathcal M} % for manifold
\newcommand{\E}{\mathcal E}
\newcommand{\Gr}{\mathcal Gr} % for grassmann
% self defined vars
\newcommand{\titleinfo}{Geometry of Sphere of Projections}
\newcommand{\authorinfo}{Wei Yuan} 

\linespread{1.05}
\pagestyle{fancyplain}
\fancyhf{}
\fancyhf[HLE,HRO]{\titleinfo}
\fancyhf[HRE,HLO]{\authorinfo}
\fancyhf[FC]{\thepage}

\begin{document}

\title{\LARGE\textbf{\titleinfo}} 
\author{\large\textsc{\authorinfo}} 
\address{AMSS}  
\email{}

\date{}

\begin{abstract}
Enter abstract here
\end{abstract}

% Keywords
\subjclass[2010]{Primary 47L75; Secondary 15A30}
\keywords{}
\thanks{}
\maketitle


\section*{Introduction}
Introduction here!


\section{Preliminary}

\begin{definition}
   Let $\M$ be a manifold with a complex structure. 
   A rank $n$ \textbf{Hermitian holomorphic vector bundle} 
   over $\M$ consists of 
   a manifold $\E$ with a complex structure together with a 
   holomorphic map $\pi$ from $\E$ onto $\M$ such that each
   fibre $\E_{x} = \pi^{-1}(x)$ is isomorphic to a Hilbert space
   and such that for each $x_0$ there exists a neighborhood $\Delta$
   of $x_0$ and holomorphic functions $\xi_{i}(x)$, $i = 1, \ldots, n$,
   form $\Delta$ to $\E$ whose values form a basis for $\E_{x}$.
\end{definition}

For a separable Hilbert space $\HHH$, let 
\begin{align*}
    \Gr (n, \HHH) = \{\mbox{n-dimensional subspaces of $\HHH$} \}.  
\end{align*}

\subsection{Curvature Matrix}
Let $E \rightarrow X$ be a $\C$-vector bundle.
$\E(X, E)$ denotes the sections of $E$.

\begin{definition}
   Let $E \rightarrow X$ be a $\C$-vector bundle. 
   Then a connection 
   $D$ on $E \rightarrow X$ is a $\C$-linear mapping
   \begin{align*}
       D: \E(X,E) \rightarrow \E(X, T^{*}(X) \otimes_{\C} E)  
   \end{align*}
   which satisfies 
   \begin{align*}
       D(f \xi) = df \cdot \xi + f D\xi,
   \end{align*}
   where $f \in C^{\infty}(X)$ and $\xi \in \E(X, E)$.
\end{definition}

Suppose that $E \rightarrow X$ is a $\C$-vector bundle.
Let $\{e_1, \ldots, e_n \}$ be a frame for $E$ over a open
subset $U$. Then locally, $\xi \in \E(U, E)$ can be write as
\begin{align*}
   \begin{pmatrix}
       e_1 & e_2 & \ldots & e_n 
   \end{pmatrix} \times
   \begin{pmatrix}
       \xi_1\\
       \xi_2\\
       \vdots\\
       \xi_n
   \end{pmatrix} = \sum^{k}\xi_k e_k.
\end{align*}

If
\begin{align*}
    D e_{j} = \sum^{n}_{i} \theta_{ij}e_{i}, 
\end{align*}
where $\theta_{ij} \in \E(U, T^{*}(X)\otimes E)$,
then
\begin{align*}
    D(
    \begin{pmatrix}
       e_1 & e_2 & \ldots & e_n 
    \end{pmatrix} \times 
    \begin{pmatrix}
       \xi_1\\
       \xi_2\\
       \vdots\\
       \xi_n
   \end{pmatrix}) &= \\ 
    \begin{pmatrix}
       e_1 & e_2 & \ldots & e_n 
    \end{pmatrix} &\times \left (
    \begin{pmatrix}
       d\xi_1\\
       d\xi_2\\
       \vdots\\
       d\xi_n
   \end{pmatrix} + 
   \begin{pmatrix}
       \theta_{11} & \ldots & \theta_{1n}\\
       \vdots      & \ddots & \vdots \\
       \theta_{n1} & \ldots & \theta_{nn}
   \end{pmatrix} \times
    \begin{pmatrix}
       \xi_1\\
       \xi_2\\
       \vdots\\
       \xi_n
   \end{pmatrix} \right ).
\end{align*}

$\theta(e) =  \begin{pmatrix}
       \theta_{11} & \ldots & \theta_{1n}\\
       \vdots      & \ddots & \vdots \\
       \theta_{n1} & \ldots & \theta_{nn}
   \end{pmatrix}$ will be called the associated connection matrix.

If 
\begin{align*}
    \begin{pmatrix}
        e_{1}^{'} & e_{2}^{'} & \ldots & e_{n}^{'} 
    \end{pmatrix} =  
    \begin{pmatrix}
       e_1 & e_2 & \ldots & e_n 
    \end{pmatrix} \times
    \begin{pmatrix}
       g_{11} & \ldots & g_{1n}\\
       \vdots      & \ddots & \vdots \\
       g_{n1} & \ldots & g_{nn}
   \end{pmatrix}
\end{align*}
is another frames over $U$, then
\begin{align*}
   \begin{pmatrix}
        e_{1}^{'} & e_{2}^{'} & \ldots & e_{n}^{'} 
   \end{pmatrix} \times
   \begin{pmatrix}
       \xi_{1}^{'}\\
       \xi_{2}^{'}\\
       \vdots\\
       \xi_{n}^{'}
   \end{pmatrix} &= \xi = 
   \begin{pmatrix}
       e_1 & e_2 & \ldots & e_n 
   \end{pmatrix} \times
   \begin{pmatrix}
       \xi_1\\
       \xi_2\\
       \vdots\\
       \xi_n
   \end{pmatrix}
\end{align*}
implies that
\begin{align*}
   \begin{pmatrix}
        e_{1}^{'} & e_{2}^{'} & \ldots & e_{n}^{'} 
   \end{pmatrix} \times
    \begin{pmatrix}
       \xi_{1}^{'}\\
       \xi_{2}^{'}\\
       \vdots\\
       \xi_{n}^{'}
   \end{pmatrix} 
   = 
   \begin{pmatrix}
        e_{1}^{'} & e_{2}^{'} & \ldots & e_{n}^{'} 
   \end{pmatrix} \times
   \begin{pmatrix}
       g_{11} & \ldots & g_{1n}\\
       \vdots      & \ddots & \vdots \\
       g_{n1} & \ldots & g_{nn}
   \end{pmatrix}^{-1} \times
    \begin{pmatrix}
       \xi_1\\
       \xi_2\\
       \vdots\\
       \xi_n
   \end{pmatrix}
\end{align*}
or
\begin{align*}
    \begin{pmatrix}
       e_1 & e_2 & \ldots & e_n 
    \end{pmatrix} \times
    \begin{pmatrix}
       \xi_1\\
       \xi_2\\
       \vdots\\
       \xi_n
   \end{pmatrix}
  = 
    \begin{pmatrix}
       e_1 & e_2 & \ldots & e_n 
    \end{pmatrix} \times
  \begin{pmatrix}
       g_{11} & \ldots & g_{1n}\\
       \vdots      & \ddots & \vdots \\
       g_{n1} & \ldots & g_{nn}
   \end{pmatrix}\times
    \begin{pmatrix}
       \xi_{1}^{'}\\
       \xi_{2}^{'}\\
       \vdots\\
       \xi_{n}^{'}
   \end{pmatrix}.
\end{align*}
Then
\begin{align*}
    &D(
    \begin{pmatrix}
       e_1 & e_2 & \ldots & e_n 
    \end{pmatrix}  \times 
    \begin{pmatrix}
       \xi_1\\
       \xi_2\\
       \vdots\\
       \xi_n
   \end{pmatrix}) = 
       D( \begin{pmatrix}
       e_1 & e_2 & \ldots & e_n 
    \end{pmatrix} \times
  \begin{pmatrix}
       g_{11} & \ldots & g_{1n}\\
       \vdots      & \ddots & \vdots \\
       g_{n1} & \ldots & g_{nn}
   \end{pmatrix}\times
    \begin{pmatrix}
       \xi_{1}^{'}\\
       \xi_{2}^{'}\\
       \vdots\\
       \xi_{n}^{'}
   \end{pmatrix}) \\
   & = \begin{pmatrix}
       e_1 & e_2 & \ldots & e_n 
    \end{pmatrix} 
   \begin{pmatrix}
       dg_{11} & \ldots & dg_{1n}\\
       \vdots      & \ddots & \vdots \\
       dg_{n1} & \ldots & dg_{nn}
   \end{pmatrix}\times
    \begin{pmatrix}
       \xi_{1}^{'}\\
       \xi_{2}^{'}\\
       \vdots\\
       \xi_{n}^{'}
   \end{pmatrix}  
    +
    \begin{pmatrix}
       g_{11} & \ldots & g_{1n}\\
       \vdots      & \ddots & \vdots \\
       g_{n1} & \ldots & g_{nn}
   \end{pmatrix}\times
    \begin{pmatrix}
       d\xi_{1}^{'}\\
       d\xi_{2}^{'}\\
       \vdots\\
       d\xi_{n}^{'}
   \end{pmatrix} \\
   & +
    \begin{pmatrix}
       \theta_{11} & \ldots & \theta_{1n}\\
       \vdots      & \ddots & \vdots \\
       \theat_{n1} & \ldots & \theta_{nn}
   \end{pmatrix} \times
    \begin{pmatrix}
       g_{11} & \ldots & g_{1n}\\
       \vdots      & \ddots & \vdots \\
       g_{n1} & \ldots & g_{nn}
   \end{pmatrix} \times
    \begin{pmatrix}
       \xi_{1}^{'}\\
       \xi_{2}^{'}\\
       \vdots\\
       \xi_{n}^{'}
   \end{pmatrix}
    ) \\
    &=
    \begin{pmatrix}
        e_{1}^{'} & e_{2}^{'} & \ldots & e_{n}^{'} 
    \end{pmatrix} (
    \begin{pmatrix}
       d\xi_{1}^{'}\\
       d\xi_{2}^{'}\\
       \vdots\\
       d\xi_{n}^{'}
   \end{pmatrix} + 
    \begin{pmatrix}
       g_{11} & \ldots & g_{1n}\\
       \vdots      & \ddots & \vdots \\
       g_{n1} & \ldots & g_{nn}
    \end{pmatrix}^{-1} \times 
    \begin{pmatrix}
       g_{11} & \ldots & g_{1n}\\
       \vdots      & \ddots & \vdots \\
       g_{n1} & \ldots & g_{nn}
   \end{pmatrix} \times
    \begin{pmatrix}
       d\xi_{1}^{'}\\
       d\xi_{2}^{'}\\
       \vdots\\
       d\xi_{n}^{'}
   \end{pmatrix} \\
   &+ 
    \begin{pmatrix}
       g_{11} & \ldots & g_{1n}\\
       \vdots      & \ddots & \vdots \\
       g_{n1} & \ldots & g_{nn}
    \end{pmatrix}^{-1} \times 
    \begin{pmatrix}
       \theta_{11} & \ldots & \theta_{1n}\\
       \vdots      & \ddots & \vdots \\
       \theta_{n1} & \ldots & \theta_{nn}
   \end{pmatrix} \times
    \begin{pmatrix}
       g_{11} & \ldots & g_{1n}\\
       \vdots      & \ddots & \vdots \\
       g_{n1} & \ldots & g_{nn}
   \end{pmatrix} \times
    \begin{pmatrix}
       \xi_{1}^{'}\\
       \xi_{2}^{'}\\
       \vdots\\
       \xi_{n}^{'}
   \end{pmatrix}
    ).
\end{align*}
Therefore
\begin{align*}
    \begin{pmatrix}
        \theta_{11}^{'} & \ldots & \theta_{1n}^{'}\\
       \vdots      & \ddots & \vdots \\
        \theta_{n1}^{'} & \ldots & \theta_{nn}^{'}
    \end{pmatrix} &= 
    \begin{pmatrix}
       g_{11} & \ldots & g_{1n}\\
       \vdots      & \ddots & \vdots \\
       g_{n1} & \ldots & g_{nn}
    \end{pmatrix}^{-1} \times 
    \begin{pmatrix}
       g_{11} & \ldots & g_{1n}\\
       \vdots      & \ddots & \vdots \\
       g_{n1} & \ldots & g_{nn}
   \end{pmatrix} \\
    &+ 
    \begin{pmatrix}
       g_{11} & \ldots & g_{1n}\\
       \vdots      & \ddots & \vdots \\
       g_{n1} & \ldots & g_{nn}
    \end{pmatrix}^{-1} \times 
    \begin{pmatrix}
       \theta_{11} & \ldots & \theta_{1n}\\
       \vdots      & \ddots & \vdots \\
       \theta_{n1} & \ldots & \theta_{nn}
   \end{pmatrix} \times
    \begin{pmatrix}
       g_{11} & \ldots & g_{1n}\\
       \vdots      & \ddots & \vdots \\
       g_{n1} & \ldots & g_{nn}
   \end{pmatrix} 
\end{align*}

\begin{definition}
   Let $E \rightarrow X$ be a $\C$-vector bundle with a connection 
   $D$ and $\theta(e)$ be the associated connection matrix for a
   frame $e = \begin{pmatrix}
        e_{1} & e_{2} & \ldots & e_{n} 
    \end{pmatrix}$. We define the curvature matrix associated with
    the connection matrix $\theta(e)$ to be 
   \begin{align*}
       \Theta(D,e) = d\theta(e) + \theta(e) \wedge \theta(e),
   \end{align*}
   an $n \times n$ matrix of 2-forms.
\end{definition}

\begin{lemma}
    Let $g$ be a change of frame, i.e.
    \begin{align*}
    \begin{pmatrix}
        e_{1}^{'} & e_{2}^{'} & \ldots & e_{n}^{'} 
    \end{pmatrix}= 
    \begin{pmatrix}
        e_{1} & e_{2} & \ldots & e_{n} 
    \end{pmatrix} \cdot g.
    \end{align*} 
     Then
   \begin{align*}
       \Theta(D, e \cdot g) = g^{-1} \Theta(D, e) g. 
   \end{align*}
\end{lemma}

\begin{lemma}
    $[d + \theta(e)][d+ \theta(e)]\xi = \Theta(D, e)\xi$.    
\end{lemma}

\begin{remark}
Let 
\begin{align*}
    \E^{p}(X, E) = \E(X, \wedge^{p}T^{*}(X) \otimes_{\C} E) 
\end{align*}
be the differential forms of degree $p$ on $X$ with coefficients in
$E$.
   We can extend $D$ to a linear map 
   \begin{align*}
       &D : \E^{p}(X,E) \rightarrow \E^{p+1}(X, E)\\
       &D \xi = d\xi + \theta(e) \wedge \xi \quad \mbox{for any $\xi
   \in \E^{p}(X,E)$}
   \end{align*} 
   Then, we have $D^{2}\xi = \Theta(D,e)\xi$. 
\end{remark}


\subsection{The Canonical Connection and Curvature of a Hermitian 
Holomorphic Vector Bundle}

\begin{definition}
   Let $E \rightarrow X$ be a $\C$-vector bundle. A Hermitian metric
   $h$ on $E$ is an assignment of a Hermitian inner product
   $\langle, \rangle_{x}$ to each fibre $E_{x}$ of $E$ such that
   for any open set $U \subset X$ and $\xi$, $\beta \in \E(U, E)$ the
   function
   \begin{align*}
      \langle \xi, \beta \rangle : U \rightarrow \C 
   \end{align*}
   given by 
   \begin{align*}
       \langle \xi, \beta \rangle (x) = \langle \xi(x), \beta(x) \rangle_{x}
   \end{align*}
   is $C^{\infty}$.
\end{definition}

\begin{remark}
  If $E$ is a Hermitian vector bundle over $X$. Then we can extend 
  the metric on $E$ in a natrual manner to act on $E$-valued covectors 
  in the following
  \begin{align*}
      \langle \omega \otimes \xi, \omega^{'}\otimes \xi^{'} \rangle_{x}
      = \omega \wedge \omega^{'} \langle \xi, \xi^{'} \rangle_{x}
  \end{align*}
  for $\omega \in \wedge^{p}T^{*}_{x}(X)$, $\omega^{'} \in 
  \wedge^{q}T^{*}_{x}(X)$, and $\xi$, $\xi^{'} \in E$, for
  $x \in X$.
\end{remark}

\begin{definition}
Let $E \rightarrow X$ be a holomorphic vector
bundle over a complex manifold $X$. If $E$, as a differentiable
bundle, is equipped with a differentiable Hermitian metric, then
$E$ is called a Hermitian holomorphic vector bundle.
\end{definition}

If $X$ is a complex manifold,
\begin{align*}
    \sum_{r}\E^{r}(X, E) = \sum_{p,q}\E^{p,q}(X,E), 
\end{align*}
where $\E^{p,q}(X,E) = \E(X \wedge^{p,q}T^{*}(X) \otimes_{\R} \C)
\otimes_{\E(X, \C)} \E(X, E)$.
If $D$ is a connection, then 
$D = D^{'} + D^{''}$ and
\begin{align*}
    & D^{'} = \partial + \theta: \E(X, E) \rightarrow \E^{1,0}(X, E) \\
    & D^{''} = \bar{\partial} : \E(X, E) \rightarrow \E^{0,1}(X, E).
\end{align*}

Note that $H = (\langle e_{i}, e_{j} \rangle)_{i,j}$ is a Hermitian matrix,
where $(e_1, \ldots, e_n)$ is a holomorphic frame. Then
\begin{align*}
    \theta = H^{-1}\partial H
\end{align*}
give a connection $D$ such that
\begin{itemize}
    \item $d \langle \xi, \beta \rangle = \langle D\xi, \beta \rangle
        + \langle \xi, D \beta \rangle$ for any $\xi$, $beta \in 
        \E(X, E)$;
    \item if $\xi$ is a holomorphic section of $E$, then $D^{''}\xi = 0$.
\end{itemize}

\begin{remark}
    If $(e_1, \ldots, e_n)$ is a holomorphic frame, then
    \begin{align*}
        & D^{'} = \partial + \theta\\
        & D^{''} = \bar{\partial}.   
    \end{align*}
    We also have
    \begin{itemize}
        \item $\theta$ is of type $(1,0)$, and 
            $\partial \theta = - \theta \wedge \theta$.
        \item $\Theta(D, e) = \bar{\partial}\theta$ and 
            $\Theta(D,e)$ is of type $(1,1)$.
        \item $\bar{\partial}\Theta(D,e) = 0$, and 
            $\partial \Theta(D,e) = [\Theta(D,e), \theta]$.
    \end{itemize}
\end{remark}

\subsection{Chern Classes}

Let $\widetilde{I_{k}}(M_{n}(\C))$ be the $\C$-vector space of all invariant
$\C$-linear forms on $M_{n}(\C)$, i.e. $\widetilde{\phi}
: M_{n}(\C) \times \cdots \times M_{n}(\C) \rightarrow \C$,  
$\in \widetilde{I_{k}}(M_{n}(\C)$ if
\begin{align*}
    \widetilde{\phi}(TA_{1}T^{-1}, \ldots, TA_{k}T^{-1}) = 
    \\widetilde{phi}(A_{1}, \ldots, A_{k})
\end{align*}
for and invertible element $T \in M_{n}(\C)$.

Suppose $\widetilde{\phi} \in \widetilde{I_{k}}(M_{n}(\C)$. Let
$\phi: M_{n}(\C) \rightarrow \C$ be
\begin{align*}
    \phi(A) = \widetilde{\phi}(A, \ldots, A). 
\end{align*}
Then $\phi$ is a homogeneous polynomial of degree $k$ in the
entries of $A$. Let 
\begin{align*}
    I_{k}(\M_{n}) = \{\phi : \phi(gAg^{-1} = \phi(A), \mbox{
    $\phi$ is homogeneous polynomials of degree $k$} \}.
\end{align*}
If $\phi \in I_{k}$, then
\begin{align*}
    \widetilde{\phi}(A_1, \ldots, A_{k}) = 
    \frac{(-1)^{k}}{k!} \sum^{k}_{j=1} \sum_{i_1 < \cdots < i_{j}}
    (-1)^{j}\phi(A_{i_1}+ \cdots + A_{i_j})
\end{align*}
is a element in $\widetilde{I_{k}}(M_{n}(\C)$.

Note that we can extend $\phi \in \widetilde{I_{k}}(M_{n}(\C)$ to 
$\E^{*}(X, Hom(E, E))$ in the following way,
\begin{align*}
    \phi(A_{1}\cdot \omega_{1}, \ldots, A_{k}\cdot \omega_{k})
    = \omega_{1} \wedge \cdots \wedge \omega_{k} \phi(A_{1}, \ldots,
    A_{k}).
\end{align*}

\begin{theorem}
   Let $E \rightarrow X$ be a differentiable $\C$-vector bundle, let
   $D$ be a connection on $E$, and suppose that $\phi \in I_{k}(M_{n}(\C))$.
   Then
   \begin{itemize}
       \item $\phi(\Theta(D, E))$ is closed.
        \item $\phi(\Theta(D, E))$ in the de Rham group $H^{2k}(X, \C)$ 
            is independent of the connection $D$.
   \end{itemize}
\end{theorem}

\begin{definition}
   Let $E \rightarrow X$ be a differentiable $\C$-vector bundle
   equipped with a connection $D$. Then the $k$th Chern form of $E$
   relative to the connection $D$ is defined to be 
   \begin{align*}
       c_{k}(E, D) = \Phi_{k}(\frac{i}{2\pi}\Theta(D, E)) \in 
       \E^{2k}(X, E),
   \end{align*}
   where $\Phi_{k} \in I_{k}(M_{n}(\C))$ are invariant polynomials 
   defined by 
   \begin{align*}
       det(I+A) = \sum_{k=0}^{n}\Phi_{k}(A), \quad A \in M_{n}(\C). 
   \end{align*}
  The total Chern form of $E$ relative to $D$ is defined to be 
  \begin{align*}
      c(E, D) = \sum^{n}_{k=0}c_{k}(E,D), \quad n = rank E. 
  \end{align*}
  The $k$th Chern class of the vector bundle $E$, denoted by $c_{k}(E)$,
  is the cohomology class of $c_{k}(E, D)$ in the de Rham group
  $H^{2k}(X, \C)$, and the total Chern class of $E$, denoted by $c(E)$,
  is the cohomology class of $c(E,D)$ in $H^{*}(X, \C)$;
  i.e., $c(E) = \sum^{n}_{k=0}c_{k}(E)$.
\end{definition}

\subsection{Complex Line Bundles}

\begin{proposition}
    Let $X$ be a (paracompact) differentiable manifold with topological
    dimension $n$. Suppose $E \rightarrow X$ is a differentiable vector
    bundle. Then there is $n+1$ open covers $\{U_{i}\}_{i=0}^{n}$ of 
    $X$ such that $E|_{U_{i}}$ is trivial.
\end{proposition}
%-----------------------------------------------------------------------------------
%BIBLIOGRAPHY
%-----------------------------------------------------------------------------------
\begin{thebibliography}{9}
\bibitem{CD}
    M. J. Cowen and R. G. Douglas, 
    \emph{Complex geometry and operator theory}, 
    Acta Math. 141 (1978) 187–261.
\end{thebibliography}
%-----------------------------------------------------------------------------------

\end{document}
