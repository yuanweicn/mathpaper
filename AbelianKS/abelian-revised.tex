\documentclass[a4paper,10pt]{amsart}

\usepackage[utf8]{inputenc}
\usepackage{hyperref}
\usepackage{cleveref}
\usepackage{amssymb}

\usepackage{comment}
\usepackage{color}
\usepackage{amsthm}



\newtheorem{example}{Example}[section]
\newtheorem{theorem}{Theorem}[section]
\newtheorem{proposition}{Proposition}[section]
\newtheorem{corollary}{Corollary}[section]
\newtheorem{definition}{Definition}[section]
\newtheorem{lemma}{Lemma}[section]
\newtheorem{remark}{Remark}[section]
\newtheorem{question}{Question}[section]

\newtheoremstyle{refs}{}{}{\itshape}{}{\bfseries}{.}{.5em}{#1 \thmnote{#3}}
\theoremstyle{refs}
\newtheorem*{lemma*}{Lemma}


\crefname{lemma}{Lemma}{lemmas}
\crefname{remark}{Remark}{remark}
\crefname{corollary}{Corollary}{corollary}
\crefname{theorem}{Theorem}{theorem}
\crefname{example}{Example}{example}
\crefname{definition}{Definition}{definition}

\newcommand{\AAA}{\mathfrak A}
\newcommand{\BBB}{\mathcal B}
\newcommand{\CCC}{\mathcal C}
\newcommand{\HHH}{\mathcal H} %for Hilbert space
\newcommand{\LLL}{\mathcal L} % for lattice
\newcommand{\MMM}{\mathcal M}

\newcommand{\PPP}[1]{ P_{#1}} %for projections
\newcommand{\QQQ}[1]{ Q_{#1}}


\newcommand{\Lat}{\mathcal Lat}
\newcommand{\Alg}{\mathcal Alg}
\newcommand{\tensor}{\mathop{\bar \otimes}}
\newcommand{\tr}{\tau}
\newcommand{\TT}{\cal T}
\newcommand{\EE}{\cal E}
\newcommand{\C}{\mathbb C} %for complex number
\newcommand{\R}{\mathbb R}  %for real number
\newcommand{\Z}{\mathbb Z} %for integer
\newcommand{\N}{\mathbb N} % for nature number


\begin{document}

\title{\textbf{On Generators of abelian Kadison-Singer algebras in matrix
algebras}}

\author{Wenming Wu$^{\dagger}$}
\address{College of Mathematical Sciences, Chongqing Normal University,
Chongqing, 400047, China}
\email{wuwm@amss.ac.cn}

\author{Wei Yuan$^{\ddagger}$}
\address{Academy of Mathematics and Systems Science, Chinese Academy of Science,
Beijing, 100190, China}
\email{wyuan@math.ac.cn}
\date{}

\begin{abstract}
Assume that $\HHH$ is a Hilbert space of dimension greater than two.
We prove that an abelian Kadison-Singer algebra acting on $\HHH$ can not
contain any non-trivial idempotent. Based on this, we show that
an abelian KS-algebra in matrix algebra $M_n(\mathbb{C})(n\geq3)$
can not be generated by a single
element. As a corollary, it is also proved that the lattice of an
abelian KS-algebra can not be completely distributive.
\end{abstract}

\subjclass[2010]{Primary 47L75; Secondary 15A30}
\keywords{Abelian KS-algebras; KS-lattices; Nilpotent; Generators; Completely
distributive lattices}


\thanks{$\dagger$ Research by the first author is supported by
    NSF of China (Grant No.11271390), NSFP of CQ CSTC
    (Grant No.2010BB9318) and
    the Chongqing Municipal Education Commission (Grant No.KJ120609).}
\thanks{
    $\ddagger$ Research by the second author is supported by NSF
    of China (Grant No.11301511).
}

\maketitle

\section{Introduction}
In 1960, Kadison and Singer introduced and studied a class of non
selfadjoint operator algebras which they called
``triangular algebras"(subalgebras of a
C$^*$-algebra whose diagonal is maximal abelian selfadjoint)\cite{KS}.
The most well understood non selfadjoint algebras are the Nest
algebras introduced by Ringrose \cite{R1}. It is a class of maximal
triangular algebras.
Reflexive algebras are generalizations of nest algebras.
These algebras are completely determined by their lattices of
invariant subspace projections.
In \cite{GY1} and \cite{GY2}, Ge and Yuan combine triangularity,
reflexivity and von Neumann algebra properties in a single class of algebras
and introduce Kadison-Singer algebras(KS-algebras for simplicity).

Before giving the definition of KS-algebras, we will fix some notation
and review some definitions. Throughout the paper, $\HHH$ will
be a Hilbert space and $\BBB(\HHH)$
the algebra of bounded linear operators acting on $\HHH$. If $\LLL$ is
a collection of self-adjoint projections of $\BBB(\HHH)$,
$\Alg(\LLL)$ is used to denote the set of bounded operators
that leave the range of every member of $\LLL$ invariant,
i.e. $\Alg(\LLL) = \{ T \in \BBB(\HHH) : (I-P)TP = 0 \mbox{, for any P in } \LLL
\}$. Dually, if $\AAA$ is a set of operators in $\BBB(\HHH)$, $\Lat(\AAA)$
is used to denote the collection of projections whose ranges are left invariant
by every element of $\AAA$, i.e. $\Lat(\AAA) = \{ P \in \BBB(\HHH) :
P^* = P \mbox{, } P^2 = P \mbox{ and } (I-P)TP = 0 \mbox{ for any $T$ in $\AAA$} \}$.
A subalgebra $\AAA$ of $\BBB(\HHH)$ is called reflexive
if $\AAA = \Alg(\Lat(\AAA))$.

\begin{definition}[Definition 1\cite{GY1}] \label{def1}
Let $\HHH$ be a Hilbert space. A subalgebra $\AAA$ of $\BBB(\HHH)$ is called a Kadison-Singer
algebra (or
KS-algebra) if $\AAA$ is reflexive and maximal with respect to the diagonal
subalgebra $\AAA \cap \AAA^{*}$ of $\AAA$, in the sense that if there is another
reflexive subalgebra $\mathfrak{B}$ of $\BBB(\HHH)$ such that $\AAA \subset
\mathfrak{B}$ and $\mathfrak{B} \cap
\mathfrak{B}^{*}=\mathfrak{A}\cap\mathfrak{A}^*$, then $\AAA =
\mathfrak{B}$.
\end{definition}

It is easy to deduce from the definition that the invariant subspace
lattice of a KS-algebra $\AAA$ is reflexive and minimal generating the
commutant of the diagonal subalgebra of $\AAA$ as a von Neumann algebra
(for a comprehensive treatment of the theory of von Neumann algebra
one may refer to
the book \cite{Ka}). So the
theory of KS-algebras is closely related with the theory of von Neumann
algebras. Actually, one of the main purposes to introduce the
KS-algebra, as stated in the introduction of
~\cite{GY1}, is to recapture the synergy that should exist between the powerful
techniques that have
developed in selfadjoint operator algebra theory and those of
the
non-selfadjoint theory. In \cite{GY1} and \cite{GY2}
many interesting examples of KS-algebras with
II$_1$ factor as their diagonal subalgebras have been constructed.
Since these
algebras contain II$_1$ factors, none of them can be commutative.
So it is quiet natural to ask whether there exist non-trivial
abelian KS-algebras. The following example provides an affirmative answer to
this question.

\begin{example}[Example 3.2\cite{LW}] \label{example1}
Let $S = \bigl(\begin{smallmatrix}
           1 & a \\
           0 & 0
           \end{smallmatrix} \bigr)$
be an idempotent in $M_2(\C)$. Then
\begin{align*}
    \AAA_{a} = \Alg(\{ \PPP{\xi_1}, \PPP{\xi_2} \}) = \{S\}^{'}
= \{\left( \begin{matrix}
      x  & a(x-y) \\
      0 & y \\
   \end{matrix}
   \right): x, y \in \C \},
\end{align*}
where $\xi_{1} = (1, 0)^{t}$, $\xi_{2} = (-a, 1)^{t}$, and $\PPP{\xi_{1}}$,
$\PPP{\xi_2}$
are the orthogonal projections onto the subspaces spanned by $\xi_{1}$ and
$\xi_2$ respectively. It is easy to check that $\AAA_{a}$ is reflexive and
similar to the diagonal algebra
$\AAA_{0}=   \{ \bigl(\begin{smallmatrix}
           x & 0 \\
           0 & y
           \end{smallmatrix} \bigr)
 : x, y \in \C \} $. We assert that $\AAA_{a}$ is an
 KS-algebra, provided that $a \neq 0$. If the assertion was false,
 then there
 would exist a reflexive algebra $\mathfrak{B}$
 contains $\AAA_a$ and $\mathfrak{B} \cap
\mathfrak{B}^{*}=\mathfrak{A_a}\cap\mathfrak{A_a}^* = \C I$.
It is clear that $\mathfrak{B}$ could only be
$M_2(\C)$, $\Alg(\{\PPP{\xi_1}\})$ or $\Alg(\{\PPP{\xi_2}\})$ if
$\mathfrak{B} \neq \AAA_{a}$. However, these algebras either contain
$\PPP{\xi_1}$ or $\PPP{\xi_2}$. This contradicts the fact that
$\mathfrak{B} \cap \mathfrak{B}^{*}=\C I$.
\end{example}

So far these are the simplest non-trivial examples of abelian KS-algebras we
have. Note that the algebras in ~\cref{example1} are similar to the algebra of
diagonal matrices in $M_2(\C)$. So it is interesting to ask whether
we can
find new abelian KS-algebras by "twisting" the algebra of
diagonal matrices in $M_n(\C)$ ($n \geq 3$). More generally, we can
ask
how to find new examples of abelian KS-algebras, besides the algebras
given in ~\cref{example1}, if there are any.
It is known that there are abelian reflexive subalgebras of
matrix algebras generated by a single matrix \cite{Ded-Fil}.
Hence, it is
also natural to ask whether there exist abelian KS-subalgebras
of matrix algebras
generated by one element. We will provide a negative answer
to the last question if the size of the matrices is greater than 2 and shed some light on other questions.


There are four sections in this paper. In section two, we give some properties
of abelian KS-algebras and prove an abelian KS-algebra acting on
Hilbert space of dimension greater than 2 can not contain any
non-trivial idempotent. Therefore, any abelian von
Neumann algebra of dimension greater than 2 can not similar to an
abelian
KS-algebra. In section 3, by using the results we get in section 2, we show
that an abelian KS-subalgebra of $M_n(\C)$ ($n \geq 3$) can not be a reflexive
algebra determined by a single element. In the last section, we
prove that the
lattice of an abelian KS-subalgebra of $\BBB(\HHH)$ ($dim\HHH \geq 3$) can not
be completely distributive.

\section{Properties of Abelian KS-algebras}

Note that the algebras in ~\cref{example1} contain non-trivial
idempotents in their centers. However, if an algebra is a
KS-algebra(~\cref{def1}), its center can not contain any non-trivial orthogonal
projections.
More precisely, we will show that the center of a KS-algebra contains
no non-trivial selfadjoint elements.

The following fact, which will be needed in the proof of ~\cref{lma1} and 
in Section 4, is proved in \cite{R1} and \cite{L}. 

Let $\xi$ and $\beta$ be two vectors in $\HHH$ and $\xi \otimes \beta$ the
rank one operator defined by $\eta \rightarrow \langle\eta, \xi\rangle\beta$
for all $\eta$ in $\HHH$. If $\LLL$ is a subspace lattice on $\HHH$,
then the operator $\xi \otimes \beta$ belongs
to $\Alg(\LLL)$ if
and only if there is a projection $P$ in $\LLL$ such that $\beta \in P\HHH$
and $\xi \in (I-P\_)\HHH$ where $P\_ = \vee \{ Q \in \LLL : P \nleq Q \}$.

\begin{lemma}\label{lma1}
The center $\CCC(\AAA)$ of a KS-algebra $\AAA ( \subset \BBB(\HHH))$
contains no non-trivial selfadjoint elements.
\end{lemma}

\begin{proof}
Suppose that $\CCC(\AAA)$ contains a non-trivial selfadjoint element
$P$. Since $\AAA$
is weak operator topology closed, we have
the spectrum projections of $P$ are in $\CCC(\AAA)$. Thus we may
assume that $P$ is a projection. Choose a proper orthonormal
basis so that $P$ can be written as
$\bigl(\begin{smallmatrix}
           I & 0 \\
           0 & 0
           \end{smallmatrix} \bigr)$. Because $P$
           commutes with every element in
$\AAA$ and $\Lat(\AAA)$, it is easy to see that $\AAA \cong
\AAA_1 \oplus \AAA_2$, where
$\AAA_1 = P\AAA P$ and $\AAA_2 = (I-P)\AAA(I-P)$.
Thus
\begin{align*}
\Lat(\AAA) = \{\left(\begin{array}{cc}Q_1 & 0 \\
                                      0 & Q_2\end{array}\right) : Q_1 \in
\Lat(\AAA_1) \mbox{ and } Q_2 \in \Lat(\AAA_2)\}.
\end{align*}
Let
\begin{align*}
    \widetilde{\LLL} &= \{Q \in \Lat( \AAA ) : Q \leq P \mbox{ or } P
\leq Q \} \\
& =  \{\left(\begin{array}{cc}Q_1 & 0 \\
                                      0 & 0 \end{array}\right) : Q_1 \in
\Lat(\AAA_1)\} \cup
 \{\left(\begin{array}{cc} I & 0 \\
                          0 & Q_{2} \end{array}\right) : Q_2 \in \Lat(\AAA_2)\} .
\end{align*}
 Then $\widetilde{\LLL}$ is a sub-lattice of
$\Lat(\AAA )$. Since $P$ is in $\widetilde{\LLL}$,
the von Neumann algebras generated by $\widetilde{\LLL}$ and
$\Lat(\AAA)$ are the same. Therefore, $\widetilde{\LLL}'' = \Lat(\AAA)''$
and $\widetilde{\LLL}' = \Lat(\AAA)'$.
This implies
that $\Alg(\widetilde{\LLL})^{*} \cap \Alg(\widetilde{\LLL})
= \widetilde{\LLL}' = \Lat( \AAA )' = \AAA^{*} \cap
\AAA$. Let $\xi$ be a vector in $(I - P) \HHH$ and
$\beta$ a vector in $P \HHH$. Note that
$P\_ \subseteq P$,
where $P\_ = \vee \{Q \in \widetilde{\LLL} | P \nsubseteq Q\}$.
It follows that $\xi \in (I-P\_)\HHH$ and
the rank one operator $\xi \otimes \beta$ is in $\Alg(\widetilde{\LLL})$.
However, $P (\xi \otimes \beta) (I-P) = \xi \otimes \beta \neq 0$.
Since $I-P$ is in $\Lat(\AAA)$, $\xi \otimes \beta$ is not in $\AAA$.
This contradicts the maximality of $\AAA$ and the
initial assumption must be false.
\end{proof}

In particular, if the algebra $\AAA$ is abelian, then the center of
$\AAA$ equals
to itself and we have the following corollary.

\begin{corollary} \label{cor1}
If $\AAA (\subset \BBB(\HHH))$ is an abelian KS-algebra, then
$\AAA^{*} \cap \AAA = \C I$ and $\Lat(\AAA)'' = \BBB(\HHH)$.
\end{corollary}

The following is a technical result that will be used later.
This fact might be well known, we sketch a proof here for the reader's
convenience.

\begin{lemma}\label{lma2}
Suppose that $\AAA$ is an abelian
subalgebra of $M_n(\C)$ and $\LLL = \Lat(\AAA)$.
If $\PPP{1}$, $\PPP{2}$ are two projections
in $\LLL$ such that $\PPP{1} < \PPP{2}$ and $dim(\PPP{2}-\PPP{1}) \geq 2$, then
there exists a $\PPP{3} \in \LLL$ satisfying $\PPP{1} < \PPP{3} < \PPP{2}$.
\end{lemma}

\begin{proof}
Since $dim(\PPP{2}-\PPP{1}) \geq 2$, the abelian algebra $(P_2 - P_1)\AAA(P_2 -
P_1)$ has a non-trivial invariant subspace $Q$. Let $P_{3} = P_{1} + Q$. Then
$\PPP{1} < \PPP{3} < \PPP{2}$ and it is not hard to check that $P_{3} \in
\LLL$.
\end{proof}

Before proceeding further, let us recall some definitions.
A nest $\mathcal{N}$ is a complete totally ordered family of
selfadjoint projections on a Hilbert space $\HHH$ containing $0$ and $I$.
A nest is maximal if it is not contained in any larger nest. The algebra
$\Alg(\mathcal{N})$ is called the nest algebra with respect to the nest
$\mathcal{N}$. The relevant background of the theory of nest algebras
can be found in \cite{Dv} and \cite{RR}.

Combine ~\cref{lma3} with the fact that nest algebras are non abelian,
we proved the following corollary.

\begin{corollary} \label{cor2}
If $\AAA$ is an abelian algebra in $M_n(\C)$ and $n \geq 2$, then $\Lat(\AAA)$ contains at
least two maximum nests.
\end{corollary}

Let $S$ be a closed idempotent(may be unbounded), i.e., $S$ is a densely defined
closed operator such that $S^2 = S$. The equality is to be understood in the
strict
sense that $S^2$ and $S$ have the same domain $\mathcal{D}(S)$ and $S^{2}\xi =
S\xi$ for
any $\xi \in \mathcal{D}(S)$. It is not hard to see that $I-S$ is also a closed
idempotent, and the range of $S$ and $I-S$ are both closed.
Let $P$, $Q$ be the range and kernel of $S$ respectively.
If $\xi \in \mathcal{D}(S)$, then $\xi = S\xi + (I-S)\xi$. Thus $\mathcal{D}(S)
= P
+ Q = \{\xi + \beta : \xi \in P \mbox{, } \beta \in Q \}$. Note that
$P$ equals to the kernel of $I-S$, $Q$ equals to the range of $I-S$.
It is easy to check that $P \wedge Q = 0$ and $P \vee Q = I$.

Conversely, if $P$ and $Q$ are two projections such that $P \wedge Q = 0$ and $P
\vee Q = I$. Since $\{(\xi + \beta, \xi) : \xi \in P \mbox{, } \beta \in Q \}$
($\subset \HHH \oplus \HHH$) is closed, $S: \xi + \beta \rightarrow \xi$ is a
closed operator with dense domain $\mathcal{D}(S) = \{ \xi + \beta : \xi \in P
\mbox{, } \beta \in Q \}$. It is clear that $S^2 = S$. Therefore, each pair of
projections $P$, $Q$ satisfying $P \wedge Q = 0$, $P\vee Q = I$ determines a
closed idempotent and vice versa.

From the above discussion, the following lemma is immediate, and we omit
the proof.

\begin{lemma} \label{lma3}
Let $\HHH$ be a Hilbert space.
If $\AAA$ is a subalgebra of $\BBB(\HHH)$, then there are two
projections
$P$, $Q$ in $\Lat(\AAA)$ satisfying $P \wedge Q = 0$, $P\vee Q = I$ if and
only if
there is a closed idempotent $S$ such that $TS \subset ST$ for any $T \in
\AAA$. In particular, if $\HHH$ is finite-dimensional,
then $S$ is bounded. Therefore,
$P$ and $Q$ are in $\Lat(\AAA)$ iff every elements in $\AAA$ commute
with $S$.
\end{lemma}

\begin{lemma} \label{lma4}
Let $\AAA$ be a subalgebra of $\BBB(\HHH)$. If the center of $\AAA$ contains an
idempotent $S$, then the range projections $P$, $Q$ of $S$ and $I-S$ are in
$\Lat(\AAA)$, and $E = (E \wedge P)\vee(E \wedge Q)$ for each $E \in
\Lat(\AAA)$.
\end{lemma}

\begin{proof}
The first part of the lemma is just a restatement of ~\cref{lma3}. For any $\xi
\in E\HHH$, we have $S\xi \in E \wedge P$ and $(I-S)\xi \in Q \wedge E$.
This clearly implies that $E = (E \wedge P)\vee(E \wedge Q)$.
\end{proof}

The following observation is certainly known, but we could not find
a reference. For the sake of completeness, we provide a short proof.

\begin{lemma} \label{lma5}
Let $\AAA$ be a reflexive
subalgebra of $B(\HHH)$. If $S$ is an idempotent in the center
of $\AAA$, then $P\AAA P |_{P\HHH}$ and $(I-P)\AAA (I-P)|_{(I-P)\HHH}$ are
reflexive algebras, where $P$ is the range projections of $S$.
\end{lemma}

\begin{proof}
To prove $P \AAA P |_{P\HHH}$ is reflexive, we only need to show that
$P \AAA P|_{P\HHH} = \Alg(\LLL)$ where
$\LLL = \{ E|_{P\HHH} : E \leq P \mbox{, }  E \in \Lat(\AAA) \}$.
$P \AAA P|_{P\HHH} \subseteq \Alg(\LLL)$
is clear. Suppose now that $T$ is an element in $\Alg(\LLL)$, let
$\widetilde{T} \in \BBB(\HHH)$ be the operator such that $\widetilde{T}(I-P) =
(I-P)\widetilde{T} = 0$ and $P\widetilde{T}P|_{P\HHH} = T$. By ~\cref{lma4}, it
is easy to check that $\widetilde{T}S$ is in $\AAA$ and
$P\widetilde{T}SP|_{P\HHH} = T$. Thus $P \AAA P |_{P\HHH} = \Alg(\LLL)$.

Note that $S^{*}$ is in the center of the reflexive algebra $\AAA^{*}$ and $I-P$
is the range projection of $I-S^{*}$. Therefore, by the argument above, we have
$(I-P)\AAA^{*} (I-P) |_{(I-P)\HHH}$ is reflexive which implies $(I-P)\AAA
(I-P) |_{(I-P)\HHH}$ is reflexive.
\end{proof}

\begin{remark}\label{rem1}
Note that the projection $Q$ onto the kernel of $S$ is also the range
projection of $I-S$, thus by ~\cref{lma5}, we have $Q\AAA Q |_{Q\HHH}$ and
$(I-Q)\AAA (I-Q)|_{(I-Q)\HHH}$ are reflexive algebras.
\end{remark}

\begin{remark} \label{rem2}
Suppose $\AAA_1 (\subset \BBB(\HHH_1))$ and $\AAA_2 (\subset \BBB(\HHH_2))$ are
two reflexive algebras. If $S =
\bigl(\begin{smallmatrix}
           I & T \\
           0 & 0
           \end{smallmatrix} \bigr)$
is an idempotent in $\BBB(\HHH_1 \oplus \HHH_2)$, then the following algebra
\begin{align*}
\AAA = \left \{ \begin{pmatrix}
 A & AT - TB\\
 0 & B
\end{pmatrix} : A \in \AAA_1, B \in \AAA_2 \right \}(\cong \AAA_1 \oplus
\AAA_2)
\end{align*}
is also reflexive.
\end{remark}



\begin{lemma} \label{lma6}
Let $\AAA$ be a subalgebra of $B(\HHH)$.
If $S$ is an idempotent in the center of $\AAA$,
then $P\AAA P |_{P\HHH} \cong (I-Q)\AAA
(I-Q) |_{(I-Q)\HHH}$ and $Q\AAA Q |_{Q\HHH} \cong (I-P)\AAA
(I-P) |_{(I-P)\HHH}$, where $P$ and $Q$ are the range projections
of $S$ and $I-S$ respectively.
\end{lemma}

\begin{proof}
We could assume that $S =
\bigl(\begin{smallmatrix}
           I & T \\
           0 & 0
           \end{smallmatrix} \bigr)$.
It is not difficult to see that $\AAA =
\left \{
\bigl(\begin{smallmatrix}
           A & AT-TB\\
           0 & B
           \end{smallmatrix} \bigr)
: A \in P\AAA P|_{P\HHH}, B \in (I-P) \AAA (I-P) |_{(I-P)\HHH}
\right \}$ and both $P\AAA P|_{P\HHH}$ and $(I-P) \AAA (I-P) |_{(I-P)\HHH}$
are algebras.
Let $S = \sqrt{SS^{*}}V$ be the polar decomposition of $S$, then $V =
\bigl(\begin{smallmatrix}
           (I+TT^{*})^{-\frac{1}{2}} & (I+TT^{*})^{-\frac{1}{2}}T\\
           0 & 0
           \end{smallmatrix} \bigr)$ is a
partial isometry with initial space $(I-Q)\HHH$ and final space $P\HHH$.
Since $(I-Q) \AAA (I-Q) |_{(I-Q)\HHH} \cong V\AAA V^{*}|_{P\HHH} =\{
\bigl(\begin{smallmatrix}
 (I+TT^{*})^{-\frac{1}{2}}A(I+TT^{*})^{\frac{1}{2}} & 0\\
 0 & 0
\end{smallmatrix} \bigr) : A \in P \AAA P |_{P\HHH}\}$,
we have $P\AAA P |_{P\HHH} \cong (I-Q)\AAA (I-Q) |_{(I-Q)\HHH}$. Replacing $S$
by $I-S$, $P$ by $Q$ and $Q$ by $P$, we
have $Q\AAA Q |_{Q\HHH} \cong (I-P)\AAA
(I-P) |_{(I-P)\HHH}$.
\end{proof}

\begin{lemma} \label{lma7}
Let $\AAA$ be a KS-subalgebra of $B(\HHH)$.
If the center of $\AAA$ contains an idempotent $S =
\bigl(\begin{smallmatrix}
           I & T\\
           0 & 0
           \end{smallmatrix} \bigr)$, then $P \AAA P$,
$(I-P)\AAA (I-P)$, $Q \AAA Q$ and $(I-Q)\AAA (I-Q)$ are
all KS-algebras, where $P$ and $Q$ are the range projections of $S$ and $I-S$.
\end{lemma}

\begin{proof}
If there exist reflexive subalgebras $\AAA_1$ of $\BBB(P\HHH)$ and
$\AAA_2$ of $\BBB((I-P)\HHH)$ such that $P \AAA P \subset \AAA_1 $,
$(I-P)\AAA(I-P) \subset \AAA_2$ and
$\AAA_1 \cap \AAA_1^{*}= P \AAA P \cap P\AAA^{*} P$, $\AAA_2 \cap
\AAA_2^{*}=(I-P) \AAA (I-P)\cap(I-P)\AAA^{*} (I-P)$. Let
$\widetilde{\AAA}= \{
\bigl(\begin{smallmatrix}
           A & AT-TB\\
           0 & B
           \end{smallmatrix} \bigr) : A
\in \AAA_1, B \in \AAA_2 \}$. By ~\cref{rem2}, the algebra
$\widetilde{\AAA}$ is reflexive. Note that
\begin{align*}
\AAA = \left \{\begin{pmatrix} A &
AT-TB \\ 0 & B \end{pmatrix}: A \in
P\AAA P|_{P\HHH}, B \in (I-P)\AAA(I-P)|_{(I-P)\HHH} \right \}.
\end{align*}
Thus $\widetilde{\AAA}\cap\widetilde{\AAA}^{*} = \AAA\cap\AAA^{*}$.
By the maximality of $\AAA$, we have $\widetilde{\AAA} = \AAA$. Therefore
$\AAA_1 = P \AAA P$ and $\AAA_2 = (I-P) \AAA (I-P)$.

Replacing $S$ by $I-S$, $P$ by $Q$ and $Q$ by $P$, we have the same results for
$Q \AAA Q$ and $(I-Q)\AAA (I-Q)$.
\end{proof}

With the help of the preceding lemmas we can now prove the
main theorem of the section.

\begin{theorem} \label{thm1}
Let $\HHH$ be a Hilbert space such that $dim\HHH \geq 3$.
Suppose that $\AAA$ is a reflexive subalgebra of $\BBB(\HHH)$ such that
$\AAA \cap \AAA^{*} = \C I$.
If the center of $\AAA$ contains an idempotent $S$ such that
$\LLL_{1}' = \{E \wedge P |_{P\HHH} : E \in \Lat(\AAA)\}' = \C I$ and
$\LLL_{2}' =\{ (E \vee P - P)|_{(I-P)\HHH} : E \in\Lat(\AAA)\}' = \C I$ where
$P$ is the range
projection of $S$, then $\AAA$ is not a KS-algebra.
\end{theorem}

\begin{proof}
Note that  $I-S^{*}$ is in the center of $\AAA^{*}$ and $I-P$ is the range
projection of $I-S^{*}$. Since $\Lat(\AAA^{*}) = \{I-E : E
\in \Lat(\AAA) \}$, we also have
\begin{align*}
\widetilde{\LLL_1} &= \{[(I-P) \wedge E] |_{(I-P)\HHH} : E \in
\Lat(\AAA^{*})\} \\
&= \{[(I-P) \wedge (I-E)] |_{(I-P)\HHH} : E \in \Lat(\AAA) \}
\\
& = \{(I - E \vee P) |_{(I-P)\HHH} : E \in \Lat(\AAA) \} = I
- \LLL_{2},\\
\widetilde{\LLL_2} &= \{[(I-P) \vee E -(I-P)]|_{P\HHH} : E \in
\Lat(\AAA^{*}) \}\\
&= \{[(I-P) \vee (I-E) -(I-P)]|_{P\HHH} : E \in \Lat(\AAA)
\}\\
&= \{(P - P \wedge E)|_{P\HHH} : E \in \Lat(\AAA)\} = I
-\LLL_1.
\end{align*}
Therefore $\widetilde{\LLL_1}' = \C I$ and $\widetilde{\LLL_2}' = \C I$.
If $dim(I-P)\HHH < 2$, we could replace $S$ by $I-S^{*}$, $P$ by $I-P$, $\LLL_1$
by $\widetilde{\LLL_1}$ and $\LLL_2$ by $\widetilde{\LLL_2}$. Since $\AAA$ is
a KS-algebra if and only if $\AAA^{*}$ is a KS-algebra, we could assume that
$dim(I-P)\HHH \geq 2$.

If $\AAA$ is a KS-algebra, then $S$ can not be a projection by ~\cref{lma1}.
Therefore we assume that $S =
\bigl(\begin{smallmatrix}
           I & T\\
           0 & 0
           \end{smallmatrix} \bigr)$ where $T \neq 0$. Let $Q$ be the range
projection of $I-S$.
We claim that there is a projection $E \in \Lat(\AAA)$ such that
$E < Q$ and $PE(I-P) \neq 0$.

By ~\cref{lma4}, we have
$\LLL_2 = \{(E \vee P - P)|_{(I-P)\HHH} : E \in \Lat(\AAA) \mbox{ and }
E \leq Q \}$. If the claim is false,
then $PE=EP=0$ for any $E < Q$. This implies that $E \leq I-P$ and
$\LLL_2 = \{E |_{(I-P)\HHH} : E \in \Lat(\AAA) \mbox{ and }
E < Q \mbox{ or } E = I\}$.
Since $\LLL_{2}' = \C I$ and $dimQ\HHH = dim(I-P)\HHH \geq 2$, we must have $I-P
= \vee \{ E : E \in \Lat(\AAA) \mbox{, } E < Q \} \leq Q$. Since $Q \wedge P =
0$, we have $Q = I-P$.
This is in contradiction to $T \neq 0$. Thus the claim holds.

Let $\LLL$ be the sublattice of $\Lat(\AAA)$ generated by $\LLL_1$ and
$P \vee \LLL_{2} = \{P \vee E : E \in \Lat(\AAA) \}$. By ~\cref{lma4},
we have $\Lat(\AAA)$ is generated by $\LLL \cup\{ Q \}$. Let
$E$ be a projection in $\Lat(\AAA)$ such
that $E < Q$ and $PE(I-P)\neq0$. Since $\LLL_1' = \C I$ and $\LLL_2' = \C
I$, we have $(\LLL \cup \{E \})'=\C I$. Therefore $\Alg(\LLL\cup\{E\})
\cap \Alg^{*}(\LLL\cup\{E\}) = \C I$. Next we will show that
$\AAA \subsetneq \Alg(\LLL\cup\{E\})$.

Let $V =
\bigl(\begin{smallmatrix}
           I & T\\
           0 & I
           \end{smallmatrix} \bigr)$.
It is easy to check that the range projection of $VQV^{-1}$ is $I-P$. Thus we
may assume that the range projection
of $VEV^{-1}$ is $
\bigl(\begin{smallmatrix}
           0 & 0\\
           0 & E_{0}
           \end{smallmatrix} \bigr)$ where $E_{0}$ is a projection in
$\BBB((I-P)\HHH)$. Note that
\begin{align*}
V \AAA V^{-1} &=\{\left(\begin{array}{cc}A_{1} & 0\\0 & A_{2}\end{array}\right):
A_{1}\in\Alg(\LLL_{1}),A_{2}\in\Alg(\LLL_{2})\} \mbox{ and} \\
V \Alg(\LLL\cup\{E\}) V^{-1}
&=\{\left(\begin{array}{cc}A_{1}&A_{12}\\0&A_{2}\end{array}\right):
A_{1}\in\Alg(\LLL_{1}),A_{2}\in\Alg(\LLL_{2}),A_{12}E_{0}=0\}.
\end{align*}
Since $0<E_{0}<I-P_{1}$, there is an operator $A\in\BBB((I-P)\HHH,P\HHH)$ such
that $A \neq 0$ and
$AE_{0}=0$. Thus we have
$\bigl(\begin{smallmatrix}
           0 & A\\
           0 & 0
           \end{smallmatrix} \bigr) \in
V\Alg(\LLL\cup\{E\})V^{-1} \setminus V\AAA V^{-1}$, and $\AAA$ is not
a KS-algebra.
\end{proof}

\begin{corollary} \label{cor3}
Let $\HHH$ be a Hilbert space such that $dim\HHH \geq 3$. If $\AAA$ is an
abelian KS-algebra, then $\AAA$ contains no non-trivial idempotent.
\end{corollary}

\begin{proof}
If $S$ is a non-trivial idempotent in $\AAA$. Let $P$ be the range projection
of $S$. By ~\cref{lma7}, we know that $P\AAA P$ and $(I-P)\AAA
(I-P)$ both are
abelian KS-algebra. Thus $\Lat(P\AAA P)' = \C I$ and $\Lat((I-P)\AAA (I-P))' =
\C I$ by ~\cref{cor1}. Therefore ~\cref{thm1} implies that $\AAA$ is not
a KS-algebra.
\end{proof}


\begin{corollary} \label{cor4}
Let $\AAA\subset M_n(\C)(n\geq 3)$ be an abelian
KS-algebra, then $\AAA=\C I+\AAA_{0}$ where $\AAA_{0}$ is an abelian
algebra contains only nilpotent matrices.
\end{corollary}

\begin{proof}
If $A$ in $\AAA$ has more than one eigenvalue, then $\AAA$ must contains
a non-trivial idempotent which is impossible by ~\cref{cor3}.
\end{proof}

\begin{corollary} \label{cor5}
Let $\AAA$ be an abelian von Neumann algebra such that $dim\AAA > 2$, then
$\AAA$ can not similar to an abelian KS-algebra.
\end{corollary}

\begin{proof}
Only need to note that every von Neumann algebra contains non-trivial
idempotent, and ~\cref{cor3} implies the result.
\end{proof}


\section{Generator of abelian KS-algebras}

If an subalgebra $\AAA$ of $M_n(\C)(n \geq 3)$ is an
abelian KS-algebra, then ~\cref{cor3} implies that the spectrum of every element
in $\AAA$ contains only one point. Therefore for any $T \in \AAA$, there is
a $\lambda \in \C$ such that $T - \lambda I$ is nilpotent. In this section, we
will prove that if $T \in M_n(\C)(n \geq 2)$ is a nilpotent, then
the $\Alg(\Lat(T))$ can not be an abelian KS-algebra.


\begin{lemma} \label{lma8}
Suppose that $\AAA$ is an abelian subalgebra of the matrix algebra $M_n(\C)$.
Let $\mathcal{S}=\{T : dim(Ker(T)) = m \}$ where $m = sup_{T \in \AAA
\setminus \{0\}} dimKer(T)$.
If $T \in \mathcal{S}$ and $A$ is a nilpotent in $\AAA$, then we have $TA=AT=0$.
\end{lemma}

\begin{proof}
Let $\mathcal{K} = Ran(T)$. Since $AT = TA$, we have $A\mathcal{K} \subset
\mathcal{K}$. Note that
$A|_{\mathcal{K}}$ is also a nilpotent, therefore
there exists a vector $\xi (\neq 0)$ in $\mathcal{K}$ such that $A\xi = 0$. If
$AT
\neq 0$, we must have
$dim(Ker(AT)) - dim(Ker(T)) \geq 1$ which is in contradiction to the fact that
$T$ is
in $\mathcal{S}$.
\end{proof}

\begin{remark} \label{rem3}
With the notations in ~\cref{lma8}. If $T$ in $\mathcal{S}$
is a nilpotent, then we have $T^{2}=0$. This implies that the range of $T$
is contained in the kernel of
$T$. Let $m = dimKer(T)$. Choosing a right orthonormal basis, $T$ can be
written as
$\bigl(\begin{smallmatrix}
           0 & T_1 \\
           0 & 0
           \end{smallmatrix} \bigr)$ where
$T_{1}\in\mathcal{M}_{m,n-m}(\mathbb{C})$. Note that the
columns of the matrix $T_{1}$ are linearly independent.
Therefore, $m \geq n-m$ and $m \geq {\frac{n}{2}}$.
\end{remark}


\begin{lemma} \label{lma9}
Suppose that $A\in M_{n}(\mathbb{C})$ is a nilpotent matrix with the order $k$.
Then there exists an invertible matrix
$W$ such that
$W^{-1}AW=\bigl(\begin{smallmatrix}
           J & A_{0}\\
           0 & 0\end{smallmatrix} \bigr)$,
where $J$ is in Jordan normal form such that $J^{k-1}=0$
and $J^{k-2}\neq0$, and the rank of $A_0$ equals to the number of
its columns.
\end{lemma}


Before proving this lemma, we feel that it is worthwhile to
examine a special case.

\begin{example} Let
    \begin{align*}
        A =  \left (
        \begin{smallmatrix}
          J_{3} &  \\
          & J_{3}  \\
        \end{smallmatrix}
        \right )
    =
       \left ( \begin{smallmatrix}
               \begin{smallmatrix}
               0 & 1 & 0\\
               0 & 0 & 1\\
               0 & 0 & 0
              \end{smallmatrix}  &  \\
                &  \begin{smallmatrix}
               0 & 1 & 0\\
               0 & 0 & 1\\
               0 & 0 & 0
              \end{smallmatrix} \\
              \end{smallmatrix} \right ).
    \end{align*}
It is not hard to see  that we may put the matrix into the
form which is described in the lemma by switching the rows and columns.
Indeed, let
\begin{align*}
U =  \left (\begin{smallmatrix}
        I_2 &  & \\
         & \bigl(\begin{smallmatrix}
              0 & 1 & 0\\
              0 & 0 & 1\\
              1 & 0 & 0
              \end{smallmatrix}\bigr) & \\
         &  & 1
       \end{smallmatrix}\right).
\end{align*}
Then
$UAU^{*}= \bigl(\begin{smallmatrix}
                  J' & A_{0}\\
                  0 & 0
                  \end{smallmatrix}\bigr)$
where
$A_0^{t}= \bigl(\begin{smallmatrix}
                 0 & 1 & 0 & 0\\
                 0 & 0 & 0 & 1
                \end{smallmatrix}\bigr)$
, $J'=daig(J_{2},J_{2})$ and
$J_{2} =
\bigl(\begin{smallmatrix}
       0 & 1\\
       0 & 0
      \end{smallmatrix}\bigr)$ is
the $2\times2 $ Jordan block of a nilpotent matrix.
\end{example}

The same strategy gives us the proof of the general case as follows.

\begin{proof}[Proof of ~\cref{lma9}]
We first assume that $M_n(\C) \cong M_k(\C) \otimes M_l(\C)$ and
$A = J_k \otimes I_{l}$ where $J_k$ is
the $k \times k$ Jordan block.
Let $\{e_{jk + i}\}_{i=1,\ldots,k; j = 0, \ldots,l-1}$ be an
orthonormal
basis of $\C^{n}$ such that $A e_{jk+1} =
0$ and $Ae_{jk+i} =  e_{jk+i-1}$, $i = 2, \ldots, k$, $j= 0, \ldots l-1$.

Let $W_{0}$ be a unitary matrix defined by
\begin{align*}
 W_{0} e_{jk + i} &= e_{j(k-1) + i} \qquad 0 \leq j \leq l, \quad 1 \leq i \leq
k-1,\\
 W_{0} e_{jk + k} &= e_{l(k - 1) + j + 1} \qquad 0 \leq j \leq l-1.
\end{align*}
Then an easy computation shows that $W_{0}AW_{0}^{*} = \bigl(\begin{smallmatrix}
       J & A_{0}\\
       0 & 0
      \end{smallmatrix}\bigr)$ such that $A_{0}$
is a $l(k-1) \times l$-matrix with $rank(A_{0})=l$ and $J = J_{k-1} \otimes
I_{l}$ where $J_{k-1}$ is the $k-1 \times k-1$ Jordan black of a single
nilpotent matrix.

For the general case, we could assume that
$A = \bigl(\begin{smallmatrix}
       J_0 & 0\\
       0 & J_1
      \end{smallmatrix}\bigr)$
where $J_{0}$ is a nilpotent matrix in Jordan normal form such that
$J_{0}^{k-1} = 0$, $J_{1} = J_{k} \otimes I = diag(J_{k}, \ldots,
J_{k})$. Then we can construct $W_0$ for $J_{k}\otimes I$ as in the first
part of the proof, and it is easy to see that $W = diag(I, W_{0})$
satisfies the conditions of the lemma.
\end{proof}


\begin{theorem}\label{thm2}
Let $A$ be a nilpotent in $M_n(\C)$ such that $dim(Ker(A)) \geq 2$. Then there
exists a reflexive sublattice $\LLL$ of $\Lat(A)$ such that $\LLL' = \C I$ and
$\Alg(\LLL) \setminus \Alg(\Lat(A))$ contains a non-trivial idempotent. In
particular $\Lat(A)' = \C I$ and $\Alg(\Lat(A))$ is not a KS-algebra.
\end{theorem}

\begin{proof}
We will prove this result by induction on the size $n$ of
$M_n(\C)$.
If $n=2$, the only nilpotent satisfies the condition in the proposition is $0$,
and the conclusion is trivial(see ~\cref{example1}).

Let $n \geq 3$ be given and suppose the conclusion of the proposition is true
for $l \leq n-1$. Let $A$ be a nonzero nilpotent in $M_{n}(\C)$ such that
$dim(Ker(A)) \geq 2$ and $A^{k} = 0$ and $A^{k-1} \neq 0$ where $k \geq 2$.

Let $P =
\bigl(\begin{smallmatrix}
       I_{m} & 0\\
       0 & 0
      \end{smallmatrix}\bigr)$ be the
projection onto the kernel of $A^{k-1}$, where $m = dimKer(A^{k-1})$.
By the argument in ~\cref{rem3}, we may write $A^{k-1}$ as
$\bigl(\begin{smallmatrix}
       0 & \tilde{A}\\
       0 & 0
      \end{smallmatrix}\bigr)$ such that
$rank(\tilde{A})=n-m$. Since $(I-P)AP = 0$, we have $A =
\bigl(\begin{smallmatrix}
       A_1 & A_2\\
       0 & A_3
      \end{smallmatrix}\bigr)$ where $A_1 \in M_{m}(\C)$, $A_2 \in
M_{m,n-m}(\C)$ and $A_3 \in M_{n-m}(\C)$.
Also note that $A^{k-1} \times A = 0$ implies $A_3 = 0$. Therefore we assume
that $A = \bigl(\begin{smallmatrix}
       A_{1} & A_{2}\\
       0 & 0
      \end{smallmatrix}\bigr)$.


Since $rank(\tilde{A}) = rank(A_{1}^{k-2}A_{2})=n-m$, we have
$A_{1}^{k-2}\xi \neq 0$, where $\xi = A_{2}e_1$ is the 1-st column of $A_{2}$
and $e_1 = (1, 0, \ldots, 0)^{t} \in (I-P)\C^{n+1}$.
It is not hard to see that
\begin{align*}
span\{\xi,A_{1}\xi,\cdots,A^{k-2}_{1}\xi \} \cap Ker(A_{1}) =
span\{ A_{1}^{k-2}\xi \}.
\end{align*}
Since the kernel of $A$ is contained in $P\C^{n}$, we have
$dim(Ker(A_{1}))\geq 2$ and there exists a nonzero vector $\beta\in Ker(A_{1})$
such that $\beta\perp
A_{1}^{k-2}\xi$. Note that
$\beta$ and $\{\xi,A_{1}\xi,\cdots,A_{1}^{k-2}\xi\}$ are linearly
independent.

Let $\alpha= \beta \oplus e_1 = (\beta,1,0,\cdots,0)^{t}$ and $Q$ be the
orthogonal projection onto to the subspace spanned by
$\bigl(\begin{smallmatrix}
 \xi\\
 0
\end{smallmatrix} \bigr)
,
\bigl(\begin{smallmatrix}
 A_{1}\xi\\
 0
\end{smallmatrix} \bigr),\cdots,
\bigl(\begin{smallmatrix}
 A_{1}^{k-2}\xi\\
 0
\end{smallmatrix} \bigr),
\alpha$. Since
$A\alpha=
\bigl(\begin{smallmatrix}
 A_1 & A_2\\
 0 & 0
\end{smallmatrix} \bigr)\alpha
= \bigl(\begin{smallmatrix}
 \xi\\
 0
\end{smallmatrix} \bigr)$,
we have $Q\in\Lat(A)$.
Let
\begin{align*}
\LLL_1 &= \{
\bigl(\begin{smallmatrix}
 E & 0\\
 0 & 0
\end{smallmatrix} \bigr) : E\in
\Lat(A_{1})\}\\
\LLL_2 &=
\{
\bigl(\begin{smallmatrix}
 I & 0\\
 0 & E
\end{smallmatrix} \bigr)
: E \mbox{ is a projection in } M_{n-m}(\mathbb{C})\}
\end{align*}
and
$\LLL = \Lat(\Alg(\LLL_1 \cup \LLL_2 \cup \{Q \}))$.
It is clear that $\LLL \subset \Lat(A)$.
By the induction assumption, we know that $\LLL_1 |_{P\C^{n}} ' = \C I_{m}$.
This implies that $\LLL' \subset \{\bigl(\begin{smallmatrix}
 aI_{m} & 0\\
 0 & bI_{n-m}
\end{smallmatrix} \bigr) : a \mbox{, } b \in \C \}$.
If $\LLL' \neq \C I$, then we have $P \in \LLL'$ and $P\alpha =
\bigl(\begin{smallmatrix}
 \beta\\
 0
\end{smallmatrix} \bigr) \in Q\C^{n}$. However, this can not be true since
$\beta$ and $\{\xi,A_{1}\xi,\cdots,A_{1}^{k-2}\xi\}$ are linearly
independent. Therefore $\LLL'$ must equals to $\C I$.

Let $B \in M_{m, n-m}(\C)$ be a matrix such that $Be_1 = \beta$. Then
$\bigl(\begin{smallmatrix}
 0 & B\\
 0 & I
\end{smallmatrix} \bigr)$ is a non-trivial idempotent in $Alg(\LLL)$.
Let $\widetilde{Q}$ be the projection onto the subspace spanned by
$\bigl(\begin{smallmatrix}
 \xi\\
 0
\end{smallmatrix} \bigr)
,
\bigl(\begin{smallmatrix}
 A_{1}\xi\\
 0
\end{smallmatrix} \bigr),\cdots,
\bigl(\begin{smallmatrix}
 A_{1}^{k-2}\xi\\
 0
\end{smallmatrix} \bigr),
0 \oplus e_1$.
Since
$A (0 \oplus e_1)= \bigl(\begin{smallmatrix}
 \xi\\
 0
\end{smallmatrix} \bigr)$,
we have $\widetilde{Q} \in \Lat(A)$. However,
$\bigl(\begin{smallmatrix}
 0 & B\\
 0 & I
\end{smallmatrix} \bigr)(0 \oplus e_1) = \alpha \notin \widetilde{Q} \C^{n}
$
and $\bigl(\begin{smallmatrix}
 0 & B\\
 0 & I
\end{smallmatrix}\bigr) \notin \Alg(\Lat(A))$.
\end{proof}

It is well-known that if $T \in M_n(\C)$ is a nilpotent and $dim(Ker(T)) = 1$,
then $\Lat(T)$ is a maximal nest and $\Alg(\Lat(T))$ is non-commutative(it is a
KS-algebra by \cite{GY1}). Therefore we have the following corollary.

\begin{corollary} \label{cor6}
Let $A$ be a nilpotent element in $M_n(\C)$, $n \geq 2$, then $\Alg(\Lat(A))$
is not an abelian KS-algebra.
\end{corollary}

\begin{corollary}\label{cor7} Suppose that $A\in M_n(\mathbb{C})(n\geq
3)$. Then the algebra generated by
$A$ and $I$ is not a KS-algebra.
\end{corollary}



\section{Abelian KS-algebras and Completely Distributive Lattices}


In this part, we will show that if $\AAA$ is an abelian KS-subalgebra of
$\BBB(\HHH)$, $dim\HHH \geq 3$, then $\Lat(\AAA)$ can not be completely
distributive.

It is well known that a subspace lattice $\mathcal{L}$ is completely
distributive if and only if $E_{\sharp}=E$ for any $E \in \mathcal{L}$
\cite{L}, where
$E_{\sharp}=\vee\{F\in\mathcal{L}:E \nleq F_{-}\}$ and
$E_{-}=\vee\{F\in\mathcal{L}:E \nleq F\}$.

\begin{lemma} \label{lma10}
If $\AAA \subset \BBB(\HHH)(dim\HHH \geq 3)$ is an abelian KS-algebra,
then $E \leq E_{-}$ for any $E\in\Lat(\AAA)$.
\end{lemma}

\begin{proof}
If $E=0$ or $I$, it is trivial. Assume that $E \in \Lat(\AAA)$ is a
projection such that $E \nleq E_{-}$, then there is a vector
$\xi\in E\HHH$ such that $\eta=(E_{-})^{\perp} \xi \neq 0$.
Let $ r =
\langle\xi,\eta\rangle=\langle\xi,(E_{-})^{\perp}\xi\rangle=||(E_{-})^{\perp}
\xi||^2>0$ and
$\eta ' = \frac{\eta}{r}$. Then the rank one operator $\eta '
\otimes \xi$ is in $\AAA$. Since $(\eta ' \otimes
\xi)^{2}=\langle\xi,\eta ' \rangle(\eta ' \otimes\xi) = \eta ' \otimes\xi$,
$\eta ' \otimes \xi$ is a non-trivial idempotent.
This is contradict with ~\cref{cor3}, thus $E \leq E_{-}$.
\end{proof}

\begin{remark} \label{rem5}
Suppose $\AAA$ is an abelian KS-algebra such that $dim\AAA > 1$. If $I_{-}<I$,
then for any $E\in
\Lat(\AAA)$, we
have $EI_{-}=I_{-}E$. Thus $I_{-}$ is a non-trivial projection in $\AAA$(note
that $\Lat(\AAA) \neq \{0, I\}$).
This contradicts with ~\cref{cor1}. Hence $I_{-}=I$.
\end{remark}

\begin{lemma} \label{lma11}
Suppose that $\AAA\subset\BBB(\HHH)$ is an abelian
subalgebra. Let $\mathfrak{I}=\{E \in \Lat(\AAA): E \neq 0,
E_{-} \neq I\}$. Then for any $E_{1} \neq E_{2}\in\mathfrak{I}$, we have
$E_{1} \leq (E_{2})_{-}$ and $E_{2} \leq (E_{1})_{-}$.
\end{lemma}

\begin{proof} If $E_{1} \nleq E_{2}$ and $E_{2} \nleq
E_{1}$, then $E_{2}\leq (E_{1})_{-}$ and
$E_{1}\leq(E_{2})_{-}$.

Without lose generosity, we could assume  $E_{1}< E_{2}$, then
$E_{1}\leq(E_{2})_{-}$.
Let $\xi_{i}\in E_{i}\HHH$ and $\eta_{i}\in
((E_{i})_{-})^{\perp}\HHH(i=1,2)$ be nonzero unit vectors, then we have
$\eta_{i}\otimes\xi_{i}(i=1,2) \in \AAA$. Since $\AAA$ is abelian,
we have
\begin{align*}
(\eta_{1}\otimes\xi_{1})(\eta_{2}\otimes\xi_{2})=\langle\xi_{2},\eta_{1}
\rangle(\eta_{2}
\otimes\xi_{1})=\langle\xi_{1},\eta_{2}\rangle(\eta_{1}\otimes\xi_{2})=(\eta_{2}
\otimes
   \xi_{2})(\eta_{1}\otimes\xi_{1}).
   \end{align*}
Note that $E_{1} \leq (E_{2})_{-}$, $\xi_{1}\in E_{1}\HHH$ and
$\eta_{2}\in ((E_{2})_{-})^{\perp}\HHH$. We have $\langle \xi_1, \eta_2 \rangle
= 0$. Therefore $\langle\xi_{2},\eta_{1}\rangle$ must also equals $0$. This
implies $E_{2} \leq (E_{1})_{-}$.
\end{proof}

\begin{theorem} \label{thm4}
If $\AAA \subset \BBB(\HHH)(dim\HHH \geq 3)$ is an abelian
KS-algebra, then $\Lat(\AAA)$ is not completely distributive.
\end{theorem}

\begin{proof}
If $E_{-}=I$ for every nonzero $E\in\Lat(\AAA)$, then $I_{\sharp}=0\neq I$.
If $E_{-} < I$, we have $I_{\sharp} \leq E_{-} \neq I$ by ~\cref{lma10} and
~\cref{lma11}. Thus $\Lat(\AAA)$ is not completely distributive.
\end{proof}

At the end of this paper, we state the following question whose answer we
believe is
negative.

\begin{question}
Besides the algebras given in ~\cref{example1}, dose there exist any other
non-trivial abelian KS-algebras?
\end{question}

\subsection*{Acknowledgment}
The authors would like to thank Prof. Jiankui Li of East China
University of Science and Technology and Prof. ChengJun Hou
of Qufu Normal University for many helpful discussions. We are
also grateful to the anonymous referee for useful comments and
suggestions.

\begin{thebibliography}{1}

\bibitem{Ka} R. Kadison and J. R. Ringrose, 
    {\em{Fundamentals of the Theory of
Operator Algebras I, II}}, Academic Press, Orlando, 1983 and 1986.

\bibitem{KS} R. Kadison and I. Singer, {\em{Triangular operator algebras.
Fundamentals and hyperreducible
theory}}, Amer.J.Math. 82(1960), 227-259.

\bibitem{R1} J. R. Ringrose,
    {\em{Super-diagonal forms for compact linear operators}},
    Proc. London Math. Soc. (3) 12(1962), 367-384.
\bibitem{J} J. R. Ringrose,
    {\em{On some algebras of operators}},
    Proc. London Math. Soc. 15(1965), 61-83.

\bibitem{Halmos} P. Halmos, {\em{Reflexive lattices of subspaces}}, J.London
Math.Soc. 4(1971), 257-263.

\bibitem{RR} H. Radjavi and P. Rosenthal, {\em{Invariant Subspaces}},
Springer-Verlan, Berlin, 1973.

\bibitem{Arv} W. Arveson, {\em{Operator algebras and invariant subspaces}},
    Ann. of Math. 100(1974), 433-532.

\bibitem{Bri-Fil} L. Brickman and P. A. Fillmore,
    {\em{The invariant subspace
lattice of a linear transformation}}, Canad. J. Math. 19(1967), 810-822.

\bibitem{Dv} K. Davidson,
    {\em{Nest algebras : triangular forms for operator algebras on
    Hilbert space}}, Pitman research notes in mathematics series 191,
    Longman Scientific \& Technical, New York, Wiley, 1988.

\bibitem{Ded-Fil} J. A. Deddens and P. A. Fillmore, {\em{Reflexive linear
transformation}}, Linear Algebra Appl. 10(1975), 89-93.

\bibitem{L} W.E. Longstaff,
    {\em{Strongly reflexive lattices}},
    J.London Math.Soc. 11(1975), no. 2, 491-498.

\bibitem{Goh-Lan-Rod} I. Gohberg, P. Lancaster and L. Rodman,
{\em{Invariant Subspaces of Matrices with Applications}}, Society for
Industrial and Applied Mathematics, Philadelphia, 2006.

\bibitem{HC} C. Hou, {\em{Cohomology of a class of Kadison-Singer
algebras}}, Science China Mathematics 53(2010), 1827-1839.

\bibitem{GY1} L. Ge and W. Yuan,
    {\em{Kadison-Singer algebras: Hyperfinite case}},
    Proc.Natl.Acad.Sci. USA 107(2010), no.5, 1838-1843.

\bibitem{GY2} L. Ge and W. Yuan, {\em{Kadison-Singer algebras,II:
    General case}}, Proc.Natl.Acad.Sci. USA 107(2010), no.11, 4840-4844.

\bibitem{LW} L. Wang and W. Yuan,
    {\em{A new class of Kadison–Singer algebras}},
    Expo.Math. 29(2011), no.1, 126-132.



\end{thebibliography}

\end{document}
