%% PNAStmpl.tex
%% Template file to use for PNAS articles prepared in LaTeX
%% Version: Apr 14, 2008


%%%%%%%%%%%%%%%%%%%%%%%%%%%%%%
%% BASIC CLASS FILE
%% PNAStwo for two column articles is called by default.
%% Uncomment PNASone for single column articles. One column class
%% and style files are available upon request from pnas@nas.edu.
%% (uncomment means get rid of the '%' in front of the command)
%\documentclass{pnasone}
\documentclass{pnastwo}

%%%%%%%%%%%%%%%%%%%%%%%%%%%%%%
%% Changing position of text on physical page:
%% Since not all printers position
%% the printed page in the same place on the physical page,
%% you can change the position yourself here, if you need to:

% \advance\voffset -.5in % Minus dimension will raise the printed page on the
                         %  physical page; positive dimension will lower it.

%% You may set the dimension to the size that you need.

%%%%%%%%%%%%%%%%%%%%%%%%%%%%%%
%% OPTIONAL GRAPHICS STYLE FILE

%% Requires graphics style file (graphicx.sty), used for inserting
%% .eps files into LaTeX articles.
%% Note that inclusion of .eps files is for your reference only;
%% when submitting to PNAS please submit figures separately.

%% Type into the square brackets the name of the driver program
%% that you are using. If you don't know, try dvips, which is the
%% most common PC driver, or textures for the Mac. These are the options:

% [dvips], [xdvi], [dvipdf], [dvipdfm], [dvipdfmx], [pdftex], [dvipsone],
% [dviwindo], [emtex], [dviwin], [pctexps], [pctexwin], [pctexhp], [pctex32],
% [truetex], [tcidvi], [vtex], [oztex], [textures], [xetex]

\usepackage[pdftex]{graphicx}

%%%%%%%%%%%%%%%%%%%%%%%%%%%%%%
%% OPTIONAL POSTSCRIPT FONT FILES

%% PostScript font files: You may need to edit the PNASoneF.sty
%% or PNAStwoF.sty file to make the font names match those on your system.
%% Alternatively, you can leave the font style file commands commented out
%% and typeset your article using the default Computer Modern
%% fonts (recommended). If accepted, your article will be typeset
%% at PNAS using PostScript fonts.


% Choose PNASoneF for one column; PNAStwoF for two column:
%\usepackage{PNASoneF}
%\usepackage{PNAStwoF}

%%%%%%%%%%%%%%%%%%%%%%%%%%%%%%
%% ADDITIONAL OPTIONAL STYLE FILES

%% The AMS math files are commonly used to gain access to useful features
%% like extended math fonts and math commands.

\usepackage{amssymb,amsfonts,amsmath}

%%%%%%%%%%%%%%%%%%%%%%%%%%%%%%
%% OPTIONAL MACRO FILES
%% Insert self-defined macros here.
%% \newcommand definitions are recommended; \def definitions are supported

%\newcommand{\mfrac}[2]{\frac{\displaystyle #1}{\displaystyle #2}}
%\def\s{\sigma}

\newenvironment{proof}[1][Proof]{\begin{trivlist}
\item[\hskip \labelsep {\bfseries #1}]}{\end{trivlist}}

\newtheorem{example}{Example}


\newcommand{\A}{\mathcal A}
\newcommand{\AAA}{\mathfrak A}
\newcommand{\B}{\mathcal B}
\newcommand{\BBB}{\mathfrak B}
\newcommand{\CCC}{\mathcal C}
\newcommand{\DDD}{\mathcal D}
\newcommand{\F}{\mathcal F}
\newcommand{\G}{\mathcal G}
\newcommand{\HHH}{\mathcal H} %for Hilbert space
\newcommand{\LLL}{\mathcal L} % for lattice
\newcommand{\PPP}{\mathcal P}
\newcommand{\M}{\mathcal M}
\newcommand{\NNN}{\mathcal N} %for nest
\newcommand{\RRR}{\mathcal R}
\newcommand{\SSS}{\mathcal S}
\newcommand{\W}{\mathcal W}
\newcommand{\ZZZ}{\mathcal Z}
\newcommand{\supp}{\mathop{\mathrm supp}}
\newcommand{\TT}{\mathcal T}
\newcommand{\II}{\mathrm{II}}


\newcommand{\e}[2][]{e^{#1}_{#2}} %for matrix unit \e[upper index]{lower index}

\newcommand{\Lat}{\mathrm{Lat}}
\newcommand{\Alg}{\mathrm{Alg}}
\newcommand{\tensor}{\mathop{\bar \otimes}}
\newcommand{\tr}{\tau}

\newcommand{\C}{\mathbb C} %for complex numbers
\newcommand{\R}{\mathbb R}  %for real numbers
\newcommand{\Z}{\mathbb Z} %for integers
\newcommand{\N}{\mathbb N} % for natural numbers


%%%%%%%%%%%%%%%%%%%%%%%%%%%%%%
%% Don't type in anything in the following section:
%%%%%%%%%%%%
%% For PNAS Only:
\contributor{Submitted to Proceedings
of the National Academy of Sciences of the United States of America}
\url{www.pnas.org/cgi/doi/10.1073/pnas.0709640104}
\copyrightyear{2008}
\issuedate{Issue Date}
\volume{Volume}
\issuenumber{Issue Number}
%%%%%%%%%%%%

\begin{document}

%%%%%%%%%%%%%%%%%%%%%%%%%%%%%%


%% For titles, only capitalize the first letter
%% \title{Almost sharp fronts for the surface quasi-geostrophic equation}

\title{Kadison-Singer Algebras, II \\
        -----General Case}


%% Enter authors via the \author command.
%% Use \affil to define affiliations.
%% (Leave no spaces between author name and \affil command)

%% Note that the \thanks{} command has been disabled in favor of
%% a generic, reserved space for PNAS publication footnotes.

%% \author{<author name>
%% \affil{<number>}{<Institution>}} One number for each institution.
%% The same number should be used for authors that
%% are affiliated with the same institution, after the first time
%% only the number is needed, ie, \affil{number}{text}, \affil{number}{}
%% Then, before last author ...
%% \and
%% \author{<author name>
%% \affil{<number>}{}}

%% For example, assuming Garcia and Sonnery are both affiliated with
%% Universidad de Murcia:
%% \author{Roberta Graff\affil{1}{University of Cambridge, Cambridge,
%% United Kingdom},
%% Javier de Ruiz Garcia\affil{2}{Universidad de Murcia, Bioquimica y Biologia
%% Molecular, Murcia, Spain}, \and Franklin Sonnery\affil{2}{}}

\author{Liming Ge\affil{1}{L. K. Hua Key Laboratory of Mathematics,
Chinese Academy of Sciences, Beijing, China} \affil{2}{ University
of New Hampshire, Durham, USA} \and Wei Yuan\affil{1}{}}

\contributor{Submitted to Proceedings of the National Academy of Sciences
of the United States of America}

%% The \maketitle command is necessary to build the title page.
\maketitle

%%%%%%%%%%%%%%%%%%%%%%%%%%%%%%%%%%%%%%%%%%%%%%%%%%%%%%%%%%%%%%%%
\begin{article}

\begin{abstract}
A new class of operator algebras, Kadison-Singer algebras, is
introduced. These highly noncommutative, non selfadjoint algebras
generalize triangular matrix algebras. They are determined by
certain minimally generating lattices of projections in the von
Neumann algebras corresponding to the commutant of the diagonals of
the Kadison-Singer algebras. A new invariant for the lattices is
introduced to classify these algebras.
\end{abstract}

%% When adding keywords, separate each term with a straight line: |
\keywords{von Neumann algebra | Kadison-Singer algebra |
Kadison-Singer lattice |triangular algebra | reflexive algebra}

%% Optional for entering abbreviations, separate the abbreviation from
%% its definition with a comma, separate each pair with a semicolon:
%% for example:
%% \abbreviations{SAM, self-assembled monolayer; OTS,
%% octadecyltrichlorosilane}

% \abbreviations{}

%% The first letter of the article should be drop cap: \dropcap{}
%\dropcap{I}n this article we study the evolution of ''almost-sharp'' fronts

%% Enter the text of your article beginning here and ending before
%% \begin{acknowledgements}
%% Section head commands for your reference:
%% \section{}
%% \subsection{}
%% \subsubsection{}

\section{Introduction}

Kadison-Singer algebras were introduced in \cite{GY}.
Examples of Kadison-Singer algebras with hyperfinite diagonals
were given and studied. In this article, we shall continue our
study of Kadison-Singer algebras and mostly deal with the case
when the diagonal is a finite von Neumann algebra. We shall use
the notation and definitions introduced in \cite{GY}.

Suppose $\HHH$ is a Hilbert space and $\B(\HHH)$ is the algebra of all
bounded operators on $\HHH$. Recall that a Kadison-Singer algebra is
a maximal reflexive subalgebra of $\B(\HHH)$ with respect to a given
von Neumann algebra as its diagonal algebra. In \cite{GY}, we
constructed Kadison-Singer algebras with hyperfinite factors as
their diagonal algebras and provided many new reflexive lattices.
Our main result in this paper is to prove that the reflexive
lattice generated by a double triangle (a special lattice with
only three nontrivial projections) is, in general, isomorphic to
the two-dimensional sphere $S^2$ (plus two distinct points
corresponding to $0$ and $I$), and the corresponding reflexive
algebra is a Kadison-Singer algebra. In particular, we show that
the algebra leave three free projections invariant is a
Kadison-Singer algebra. This shows that many factors are
(minimally) generated by a reflexive lattice of projections which
is topologically homeomorphic to  $S^2$. Noncommutative algebraic
structures on $S^2$ determined by the projections give rise to non
isomorphic factors and Kadison-Singer factors.

The paper contains five sections. In Section 2, maximal
triangularity is discussed in different aspects. In Section 3, we
describe the reflexive lattice generated by three free projections
and show that it is homeomorphic to $S^2$. In Section 4, we show
that this lattice is a Kadison-Singer lattice and thus the
corresponding algebra is a Kadison-Singer algebra. Certain
generalization of the result is also discussed. In Section 5, we
introduce a notion of connectedness of projections in a lattice of
projections in a finite von Neumann algebra and show that all
connected components form another lattice, called a {\it reduced
lattice}. Reduced lattices of most of our examples were computed.

The authors wish to thank J. Shen and M. Ravichandran for many
helpful discussions.

\section{Maximality conditions}

In the definition of Kadison-Singer algebras, we require
that the algebra be maximal in the class of reflexive algebras
with the same diagonal. Our examples of KS-algebras given in \cite{GY}
are ``maximal triangular'' in the class of all algebras with the
same diagonal, i.e., an algebraic maximality without reflexiveness
or closedness assumptions. In general, the algebraic maximality
assumption is a much stronger requirement. We call a subalgebra
$\AAA$ of $\B(\HHH)$ {\it maximal triangular with respect to its
diagonal C*- (or von Neumann) algebra $\AAA\cap\AAA^*$} if, for any
subalgebra $\BBB$ of $\B(\HHH)$, $\BBB$ contains $\AAA$ and has
the same diagonal as $\AAA$, then $\BBB$ is equal to $\AAA$. This
may lead to new interesting classes of non selfadjoint algebras.
Many similar questions as those in \cite{KS}, e.g., the closedness of
$\AAA$, can be asked accordingly.

In the following, we give a canonical method to construct a maximal
triangular algebra in the class of weak-operator closed algebras
with respect to a given von Neumann algebra as its diagonal.

Suppose $\M$ is a von Neumann algebra acting on a Hilbert space
$\HHH_0$ and $H_1,\ldots,H_n$ are positive elements in $\M$ such
that $H_1^2,H_2^2,\ldots, H_n^2$ generate $\M$ as a von Neumann
algebra. From \cite{GS}, we know that many von Neumann algebras can be
generated by such positive elements, especially all type III and
properly infinite von Neumann algebras. Let $\HHH$ be the direct sum
of $n+1$ copies of $\HHH_0$. Then $\B(\HHH)\cong M_{n+1}(\B(\HHH_0))$.
With this identification, we shall view both $M_{n+1}(\C)$ and
$\B(\HHH_0)$ as subalgebras of $\B(\HHH)$. Let $E_{ij}$,
$i,j=1,\ldots,n+1$ be a matrix unit system for $M_{n+1}(\C)$. We
shall write elements in $\B(\HHH)$ in a matrix form with respect to
this unit system (with entries from $\B(\HHH_0)$).

\begin{theorem}
Define
\begin{align*}
\AAA= \left\{ \left( \begin{array}{cccc}
                 A & * &  \ldots & * \\
                 0 & H_1^{-1} A H_1 & \ldots & * \\
                 \vdots & \vdots & \ddots & \vdots\\
                 0 & 0 & \ldots & H_n^{-1} A H_n \\
                 \end{array}
     \right) :\quad  A \in \B(\HHH_0) \right\},
\end{align*}
where $*$ denotes all possible elements in $\B(\HHH_0)$. Then $\A$
is maximal upper triangular with respect to the diagonal
$M_{n+1}(\M)'$ $(\cong \M'\cap\B(\HHH_0))$.
\end{theorem}

\begin{proof}
It is easy to check that $\AAA\cap
\AAA^*= M_{n+1}(\M)'$. Similar techniques we used in the proof of
Theorem 3 in \cite{GY} will show that $\AAA$ is maximal upper
triangular with the given diagonal. We omit the details
here.
\end{proof}

With $\AAA$ given in Theorem 1, suppose $P$ is a projection in
$\Lat(\AAA)$ with $P=(P_{ij})_{i,j=1}^{n+1}$, $P_{ij}\in\B(\HHH_0)$.
It is easy to see that $P$ must be diagonal and
$(I-P_{ii})TP_{jj}=0$ for all $i<j$ and any $T$ in $\B(\HHH_0)$.
Thus $P=\sum_{j=1}^k E_{jj}$ for some $k$. We know that such a $P$
lies in $\Lat(\AAA)$. This shows that $\Lat(\AAA)=\{0, E_{11},\ldots,
\sum_{j=1}^{n}E_{jj}, I\}$. Clearly this implies that $\AAA$ is not
reflexive. It is interesting to know if $\AAA$ contains a subalgebra
which is a Kadison-Singer algebra with the same diagonal.

The semidirect product of a von Neumann algebra $\M$ with a
semigroup $S$ (embedded in the automorphism group of $\M$) will
give us another construction of triangular algebras. In general,
such construction will not give us a maximal triangular algebra.
Whether there is a KS-algebra containing such a triangular algebra
for certain ergodic actions is another interesting question.


On the other hand, if we start with a ``minimal'' lattice $\LLL$ of
projections in a von Neumann algebra so that the lattice generate
the von Neumann algebra, then is $\Alg(\LLL)$ a Kadison-Singer
algebra? In general, $\LLL\neq\Lat(\Alg(\LLL))$. Thus $\LLL$ may not be
a Kadison-Singer lattice. But we shall see that $\Alg(\LLL)$ is
often a Kadison-Singer algebra in next section.

\section{Reflexive lattices generated by three projections}

Lattices generated by two projections are always
reflexive \cite{H}. But lattices generated by three projections are
complicated. Most of factors acting on a separable Hilbert space
are known to be generated by three projections or a projection and
a positive operator (see \cite{GS}).

\begin{example}
Suppose $\M$ is a factor
acting on $\HHH$ and write $\M=M_2(\NNN)$, where $\NNN$ is a subfactor
of $\M$. We assume that $\NNN$ is generated by a projection $P$ and
a positive element $H$ with $0\le H\le I$ and
$\supp(H)=\supp(I-H)=I$. Here we view $M_2(\C)$ as a subalgebra of
$\M$ and $\NNN$ as the relative commutant of $M_2(\C)$ in $\M$. Let
$E_{11}, E_{12}, E_{21}$ and $E_{22}$ be the standard matrix unit
system for $M_2(\C)$. Define $P_1=E_{11}$, $P_2=E_{11}H+E_{12}
\sqrt{H(I-H)}+E_{21} \sqrt{H(I-H)} +E_{22}(I-H)$,
$P_3=E_{11}+E_{22}P$ and $P_4=P_2\wedge P_3$. Assume that $P_1\wedge
P_2=0$, $P_1\vee P_2=I$. Then $\LLL=\{0,I, P_1, P_2, P_3, P_4\}$
generated by $\{P_1,P_2,P_3\}$ is a distributive lattice and thus
reflexive \cite{Ha}. From our construction, we know that $\M$ is
generated by $\LLL$ as a von Neumann algebra. One also easily checks
that any proper sublattice of $\LLL$ does not generate $\M$. Thus $\LLL$
is a Kadison-Singer lattice and $\Alg(\LLL)$ is a Kadison-Singer
factor. From this construction, we can realize most of the factors
as diagonals of Kadison-Singer algebras. For example, one may choose
$\NNN$ as a factor of type $\II_1$ generated by a projection $P$ of
trace $\frac12$ and a positive operator $H$ such that $P$ and $H$
are free, and $H$ has the same distribution (with respect to the
trace on $\NNN$) as the function $\cos^2\frac\pi2\theta$ on $[0,1]$
(with respect to Lebesgue measure). Let $\tau$ be the trace on $\M$.
In this case, $\tau(P_1)=\tau(P_2)=\frac12$, $\tau(P_3)=\frac34$ and
$\tau(P_4)=\frac14$. Then $\Alg(\LLL)$ is a Kadison-Singer factor of
type $\II_1$.
\end{example}

It is hard to determine when a lattice is reflexive even for a
finite lattice. Finite distributive lattices are reflexive \cite{Ha}.
But most of the lattice are not distributive. The simplest non
distributive lattice is a {\it double triangle} where it contains
$0, I$ and three projections $P_1,P_2$ and $P_3$ so that $P_i\vee
P_j=I$ and $P_i\wedge P_j=0$ for any $i\neq j$ and $i,j=1,2,3$.
Any lattice that contains a double triangle sublattice is not
distributive. Three free projections with trace $\frac12$ in a
factor of type $\II_1$ (together with $0,I$) form a double
triangle lattice. In the following, we first describe factors
generated by free projections. For basic theory on freeness and
distributions, we refer to \cite{VDN}.

Let $G_n$ be the free product of $\ZZZ_2$ with itself $n$ times, for
$n\ge2$, or $=\infty$. When $n\ge3$, $G_n$ is an i.c.c. group so
its associated group von Neumann algebra $\LLL_{G_n}$ is a factor of
type $\II_1$ acting on $l^2(G_n)$ (see \cite{KR}). If $U_1,\ldots,U_n$
are canonical generators for $\LLL_{G_n}$ corresponding to the
generators of $G_n$ with $U_j^2=I$. Then $\frac{I-U_j}2$,
$j=1,\ldots,n$, are projections of trace $\frac12$.  Let $\F_n$ be
the lattice consisting of these $n$ free projections and $0, I$.

Clearly $\F_n$ is a minimal lattice which generates $\LLL_{G_n}$ as
a von Neumann algebra. Is $\Alg(\F_n)$, $n\ge3$, a Kadison-Singer
algebra? What is $\Lat(\Alg(\F_n))$? When $n=2$, Halmos \cite{H} showed
that $\F_2$ is reflexive and thus $\Alg(\F_2)$ is maximal and
hence a Kadison-Singer algebra. We shall answer the above
questions for the case when $n=3$ and show that $\Alg(\F_3)$ is a
KS-algebra and $\Lat(\Alg(\F_3))\setminus \{0,I\}$ is homeomorphic
to $S^2$, the two-dimensional sphere.

We shall realize $\LLL_{G_3}$ as the von Neumann algebra generated
by $M_2(\C)$ and its relative commutant $\M$ in $\LLL_{G_3}$ and
write projection generators of $\LLL_{G_3}$ in terms of $2\times 2$
matrices (with respect to the standard matrix units in $M_2(\C)$)
given by the following equations:

\begin{align*}
P_1&= \left(
        \begin{array}{cc}
          I & 0 \\
          0 & 0 \\
        \end{array}
      \right),
P_{2} = \left(
          \begin{array}{cc}
            H_{1} & \sqrt{H_{1}(I-H_{1})} \\
            \sqrt{H_{1}(I-H_{1})} & I-H_{1} \\
          \end{array}
        \right) , \\
P_{3}&= \left(
          \begin{array}{cc}
            H_{2} & \sqrt{H_{2}(I-H_{2})}V \\
            V^{*}\sqrt{H_{2}(I-H_{2})} & V^{*}(I-H_{2})V \\
          \end{array}
        \right) \mbox{.}
\end{align*}
The freeness among $P_1,P_2,P_3$ require that $H_1, H_2$ and $V$
be free, $H_1$ and $H_2$ have the same distribution as
$\cos^2\frac{\pi}2\theta$ on $[0,1]$ with respect to Lebesgue
measure and $V$ a Haar unitary element. Then the subalgebra $\M$
of $\LLL_{G_3}$ is the von Neumann algebra generated by $H_1,H_2$
and $V$. Now $\F_3=\{0,I, P_1, P_2, P_3\}$, $\HHH=l^2(G_3)$.

When $M_2(\C)$ is a subalgebra of $\LLL_{G_3}$, we may also view
$\B(\HHH)=M_2(\C)\otimes \B$, where $\B$ is the commutant of
$M_2(\C)$ in $\B(\HHH)$. Thus all operators can be written as
$2\times 2$ matrices with entries from $\B$. In fact, when
$\HHH=\HHH_1\oplus \HHH_1$ for some Hilbert space $\HHH_1$, then
$\B\cong\B(\HHH_1)$.

Since $P_1\in\F_3$, any operator $T$ belonging to $\Alg(\F_3)$
must be upper triangular. The following lemma follows from the
invariance of $P_2$ and $P_3$ under $T$. The computation is
straight forward.

\begin{lemma}
With notation given above, $\left(
                  \begin{array}{cc}
                    T_{1} & T_{2} \\
                    0 & T_{3} \\
                  \end{array}
                \right)\in\Alg(\F_3)$, where $
T_1, T_2, T_3\in\B$, if and only if \begin{align*}
\sqrt{I-H_1}T_2\sqrt{I-H_1} &= \sqrt{H_1}T_3
\sqrt{I-H_1}-\sqrt{I-H_1}T_1\sqrt{H_1};\cr
 \sqrt{I-H_2}T_2V^*\sqrt{I-H_2}
&= \sqrt{H_2}VT_3 V^*\sqrt{I-H_2}-\sqrt{I-H_2}T_1\sqrt{H_2}.
\end{align*}
\end{lemma}

Using unbounded operators affiliated with $\LLL_{G_3}$, one can
construct many finite rank operators in $\Alg(\F_3)$. Unbounded
operators affiliated with a finite von Neumann form an algebra
\cite{KR}. Any finitely many unbounded operators have a common dense domain.
Let $\xi$ and $\eta$ be vectors in the common domain of
$\sqrt{H_1(I-H_1)^{-1}}$, $\sqrt{H_2(I-H_2)^{-1}}V$ and the
adjoint $(\sqrt{H_2(I-H_2)^{-1}}V)^*$. We shall use $x\otimes y$
to denote the rank one operator defined by $x\otimes
y(z)= \langle z,x  \rangle y$, for any $z\in\HHH$ with $x$ and $y$ arbitrarily
given. Now let $T_1=\xi\otimes (\sqrt{H_1(I-H_1)^{-1}}
-\sqrt{H_2(I-H_2)^{-1}}V)\eta$, $T_3=((\sqrt{H_1(I-H_1)^{-1}}
-\sqrt{H_2(I-H_2)^{-1}}V)^*\xi)\otimes\eta$ and $T_2=
\sqrt{H_1(I-H_1)^{-1}}T_3 -T_1\sqrt{H_1(I-H_1)^{-1}}$ (determined
by Lemma 1). Then $T=\left(
         \begin{array}{cc}
           T_{1} & T_{2} \\
           0 & T_{3} \\
         \end{array}
       \right)
       \in\Alg(\F_3)$ is a finite rank operator (at most rank
4). This shows that $\Alg(\F_3)$ contains many finite rank (and
thus compact) operators. In fact, Lemma 3 below will show that
$\Alg(\F_3)$ contains ``almost'' a copy of $\B(\HHH)$.

The following is a technical result that will be used frequently.
The result might be well known. We only sketch a proof here.

\begin{lemma}
 Suppose $U$ is a Haar
unitary element in a factor $\M$ of type II$_1$ and $A$ is an
element in (or an unbounded operator affiliated with) $\M$ such
that $A$ and $U$ are free with each other. Then any nonzero scalar
$\lambda$ can not be a point spectrum of $AU$.
\end{lemma}

\begin{proof}
Suppose $\lambda\in\C$ is a nonzero
point spectrum of $AU$. By symmetry and freeness of $A$ and
$\omega U$, we know that $\lambda$ must be a point spectrum for
$A(\omega U)$ for any $\omega\in\C$ with $|\omega|=1$. This
implies that $\omega^{-1}\lambda$ is a point spectrum of $AU$.
Suppose $P_\beta$ is the spectral projection of $AU$ supported at
$\beta\in\C$. Then $P_\lambda$ is equivalent to
$P_{\omega\lambda}$ for any $|\omega|=1$. A direct computation
shows that if $\lambda\neq\lambda_j$ for $j=1,2,\ldots, n$, then
$P_\lambda\wedge(P_{\lambda_1}\vee \cdots \vee P_{\lambda_n})=0$.
From finiteness of $\M$, it is easy to conclude that $P_\lambda=0$
when $\lambda\neq0$.
\end{proof}

In the following, we shall describe all elements in
$\Lat(\Alg(\F_3))$. Unbounded operators will be used in our
computation. All unbounded operators affiliated with the factor
$\LLL_{G_3}=M_2(\M)$ form an algebra. From function calculus, many
unbounded operators we encounter in this paper can be viewed as
(positive) functions defined on $(0,1)$ with respect to Lebesgue
measure. When $H$ ($=H_1$ or $H_2$) is identified with
$\cos^2\frac{\pi}2\theta$. Then $I-H$, $\sqrt{H(I-H)}$,
$\sqrt{H^{-1}}$, $\sqrt{(I-H)^{-1}}$, etc., can be viewed as
trigonometric functions and they are all determined by any one of
them. Lemma 2 also tells that many linear combinations of free
(non selfadjoint) operators such as $\sqrt{H_{1}(I-H_{1})^{-1}} -
\sqrt{H_{2}(I-H_{2})^{-1}}V=\sqrt{H_{1}(I-H_{1})^{-1}}
(I-\sqrt{H_{1}^{-1}(I-H_{1})}\sqrt{H_{2}(I-H_{2})^{-1}}V) $ are
invertible (with unbounded inverses).

When $S, T$ are unbounded operators affiliated with $\LLL_{G_3}$ or a
finite von Neumann algebra and $X\in\B(\HHH)$, then $SXT$ is an
unbounded operator that can be viewed as the weak-operator limit of
bounded operators of the form $SE_\epsilon XF_\epsilon T$ for
projections $E_\epsilon$ and $F_\epsilon$ in $\LLL_{G_3}$ (or the
finite von Neumann algebra) so that $SE_\epsilon$ and $F_\epsilon T$
are bounded and $E_\epsilon$, $F_\epsilon$ have strong operator
limit $I$ (as $\epsilon\to 0$). Thus, for any operator $X$ in a
weak-operator dense subalgebra $\cup_\epsilon E_\epsilon \B(\HHH)
F_\epsilon$ of $\B(\HHH)$, the operator $SXT$ is a bounded operator.

Using unbounded operators, we may restate the above lemma in the
following form.

\begin{lemma}
With $H_1,H_2,
V\in\M\subset\LLL_{G_3}$ given above, let $S=
\sqrt{H_{1}(I-H_{1})^{-1}} - \sqrt{H_{2}(I-H_{2})^{-1}}V$ be an
unbounded operator affiliated with $\M$. If $ T\in\Alg(\F_3)$,
then there is an $A \in \B$ such that
\begin{align*}
T =\left(
     \begin{array}{cc}
     A   & \sqrt{H_{1}(I-H_{1})^{-1}}S^{-1}AS - A\sqrt{H_{1}(I-H_{1})^{-1}} \\
     0   & S^{-1}AS \\
     \end{array}
   \right) \mbox{.}
\end{align*}
Conversely, if $A\in\B$ such that $\sqrt{H_{1}(I-H_{1})^{-1}}S^{-1}AS- A\sqrt{H_{1}(I-H_{1})^{-1}}$
and $S^{-1}AS$ are bounded operators, then the above $T$ belongs to $\Alg(\F_3)$.
\end{lemma}

The above lemma shows that $\Alg(\F_3)$ is quite large, in
particular, $\Alg(\F_3)\cap\LLL_{G_3}$ is infinite dimensional. The
following result follows easily from the above lemma and shows
that all nontrivial projections in $\Lat(\Alg(\F_3))$ have trace
$\frac12$.

\begin{corollary}
For any $Q \in
\Lat(\Alg(\F_3)) \backslash \{0, I, P_1\}$, we have that $Q \wedge
P_{1} = 0$, $Q \vee P_{1} = I$, and  $\tau(Q) = \frac{1}{2}$.
\end{corollary}

\begin{proof}
For any $Q \in \Lat(\Alg(\F_3))$,
$Q\wedge P_1\in\Lat(\Alg(\F_3))$. Thus $Q\wedge P_1$ is invariant
under all $T$ given in Lemma 3 and thus the $A$ in $(1,1)$ entry
of $T$. This implies that $Q\wedge P_1=P_1$ or $0$. Similarly we
can show that $Q\vee P_1=I$ or $P_1$.
\end{proof}

This corollary actually shows that for any distinct projections
$Q_{1}, Q_{2}$ in $\Lat(\Alg(\F_3)) \setminus \{0,I\}$, $Q_{1}
\wedge Q_{2} = 0$, and $Q_{1} \vee Q_{2} = I$.

For any $X\in\B(\HHH)$ or an unbounded operator $X$ affiliated with
a (finite) von Neumann algebra, we shall use $\supp(X)$ to denote
the support of $X$, i.e., the range projection of $X^*X$. When
$\supp(X)=\supp(X^*)=I$, $X$ has an (unbounded) inverse.

\begin{theorem}
For any projection $Q$
in $\Lat(\Alg(\F_3)) \setminus \{0, I, P_{1}\}$, there are $K$ and
$U$ in $\M$ such that
\begin{align*}
Q = \left(
     \begin{array}{cc}
      K & \sqrt{K(I-K)}U \\
      U^{*}\sqrt{K(I-K)} & U^{*}(I-K)U \\
  \end{array}
\right) ,
\end{align*}
where $\sqrt{K(I-K)^{-1}}$ (or $K$) and $U$ are determined by the
polar decomposition of $(1+a)\sqrt{H_{1}(I-H_{1})^{-1}}-
a\sqrt{H_{2}(I-H_{2})^{-1}}V =aS + \sqrt{H_{1}(I-H_{1})^{-1}}=
\sqrt{K(I-K)^{-1}} U$ for some $a \in \C$. Moreover for any given
$a$ in $\C$, the polar decomposition determines $U$ and $K$
uniquely which give rise to a projection $Q$ (in the above form)
in $\Lat(\Alg(\F_3))$.
\end{theorem}

\begin{proof}
Suppose $Q$ is given
 in the theorem. From Corollary 2, we know that
 $\supp(I-K) =I$. From Lemma 3,
for any  $A\in\B$ (the commutant of $M_2(\C)$ in $\B(l^2(G_3))$,
$\B\cong\B(\HHH)$ for some Hilbert space $\HHH$) such that $S^{-1}AS$
and $\sqrt{H_{1}(I-H_{1})^{-1}}S^{-1}AS -
A\sqrt{H_{1}(I-H_{1})^{-1}}$ are bounded, then
\begin{align*}
T =\left(
     \begin{array}{cc}
       A & \sqrt{H_{1}(I-H_{1})^{-1}}S^{-1}AS - A\sqrt{H_{1}(I-H_{1})^{-1}} \\
       0 & S^{-1}AS \\
     \end{array}
   \right) \in \Alg(\F_3),
\end{align*}
here $S=\sqrt{H_{1}(I-H_{1})^{-1}} - \sqrt{H_{2}(I-H_{2})^{-1}}V$.
Thus $(I-Q)TQ=0$. This implies that
 %%%%%i.e.,
%%%$$\eqalign{
%%%& \left(\matrix{ I-K & -\sqrt{K(I-K)}U  \cr -U^{*}\sqrt{K(I-K)} &
%%U^{*}KU }\right)\cdot \cr &\left(\matrix{A &
%%%\sqrt{H_{1}(I-H_{1})^{-1}}S^{-1}AS - A\sqrt{H_{1}(I-H_{1})^{-1}}
%%\cr 0 & S^{-1}AS } \right)\left(\matrix{ K & \sqrt{K(I-K)}U \cr
%%U^{*}\sqrt{K(I-K)} & U^{*}(I-K)U }\right)\cr &=0.}
%%%$$\bye
\begin{align*}
(I-&K)A\sqrt{K(I-K)}U +
(I-K)\left[\sqrt{H_{1}(I-H_{1})^{-1}}S^{-1}AS \right.\cr & -
\left.A\sqrt{H_{1}(I-H_{1})^{-1}}\right]U^{*}(I-K)U \\ &-
\sqrt{K(I-K)}US^{-1}ASU^{*}(I-K)U = 0.
\end{align*}
Since $\supp(I-K) = I$, $I-K$ is invertible. We have
\begin{align*}
&\sqrt{I-K}A\left[\sqrt{K} -
\sqrt{H_{1}(I-H_{1})^{-1}}U^{*}\sqrt{I-K}\right]\cr &=
\left[\sqrt{K}U -
\sqrt{I-K}\sqrt{H_{1}(I-H_{1})^{-1}}\right]S^{-1}ASU^{*}\sqrt{I-K}.
\end{align*}
This gives us
\begin{align*}
&A[\sqrt{K} -
\sqrt{H_{1}(I-H_{1})^{-1}}U^{*}\sqrt{I-K}](SU^{*}\sqrt{I-K})^{-1}\cr
&= \sqrt{(I-K)^{-1}}[\sqrt{K}U -
\sqrt{I-K}\sqrt{H_{1}(I-H_{1})^{-1}}]S^{-1}A.
\end{align*}
The above equation holds for all $A$ in a weak-operator dense
subalgebra of $\B$ $(\cong \B(\HHH_1))$. Thus there is an $a \in \C$
such that
\begin{align*}
a\sqrt{I-K} &= [\sqrt{K}U -
            \sqrt{I-K}\sqrt{H_{1}(I-H_{1})^{-1}}]S^{-1},\cr
aSU^*\sqrt{I-K}&= \sqrt{K} -
\sqrt{H_{1}(I-H_{1})^{-1}}U^{*}\sqrt{I-K}.
\end{align*}
This implies that
\begin{align*}
\sqrt{K(I-K)^{-1}}U = aS+ \sqrt{H_{1}(I-H_{1})^{-1}}.
\end{align*}
Conversely, when $K$ and $U$ are given by this equation, all above
equations hold. From Lemma 3 one checks easily that $Q$ given in
the theorem lies in $\Lat(\Alg(\F_3))$.
\end{proof}

\section{ $\Alg(\F_3)$ is a Kadison-Singer algebra}

In this section, we shall prove that $\Lat(\Alg(\F_3))$
is a Kadison-Singer lattice which implies that $\Alg(\F_3)$ is a
Kadison-Singer algebra.

\begin{lemma}
For any two distinct
projections $Q_{1}$, $Q_{2}$ in $\Lat(\Alg(\F_3)) \setminus \{0,
I, P_{1}\}$, we have that $\Alg(\{P_{1}, Q_{1}, Q_{2}\}) =
\Alg(\F_3)$.
\end{lemma}

\begin{proof}
By Theorem 3, we may assume that,
for $i=1,2$,
\begin{equation}\label{(*)}
\begin{split}
&Q_{i} =\left(
          \begin{array}{cc}
            K_{i} & \sqrt{K_{i}(I-K_{i})}U_{i} \\
            U_{i}^{*}\sqrt{K_{i}(I-K_{i})} & U_{i}^{*}(I-K_{i})U_{i} \\
          \end{array}
        \right) \mbox{ and } \\
&\sqrt{K_{i}(I-K_{i})^{-1}} U_{i} =
(1+a_i)\sqrt{H_{1}(I-H_{1})^{-1}} \\ &-
a_i\sqrt{H_{2}(I-H_{2})^{-1}}V \\
&=a_{i}S +
\sqrt{H_{1}(I-H_{1})^{-1}}.
\end{split}
\end{equation}
where $a_{1}, a_{2}\in\C$  and $a_{1} \neq a_{2}$. Then we have
$\sqrt{K_{1}(I-K_{1})^{-1}}U_{1} - \sqrt{K_{2}(I-K_{2})^{-1}}U_{2}
= (a_{1} - a_{2})S$. Replacing $S$ by $(a_1-a_2)S$ in Lemma 3,
we know that $\Alg(\{P_{1}, Q_{1}, Q_{2}\}) =
\Alg(\F_3)$.
\end{proof}

\begin{lemma}
For any three distinct
projections $Q_1, Q_2$ and $Q_3$ in $\Lat(\Alg(\F_3)) \setminus
\{0, I \}$, we always have that $P_{1} \in \Lat(\Alg(\{Q_{1},
Q_{2}, Q_{3}\}))$.
\end{lemma}

\begin{proof}
We may assume that $Q_{i} \in
\Lat(\Alg(\F_3))\setminus \{0, I, P_{1}\}$ and $Q_i$ is given by
$[1]$ in the proof of Lemma 4, for $i=1,2,3$ and $a_{1}, a_{2},
a_3$ are distinct scalars.

To prove this lemma, we only need to show that if
\begin{align*}
A =\left(
     \begin{array}{cc}
       A_{11} & A_{12} \\
       A_{21} & A_{22} \\
     \end{array}
   \right)\in
\Alg(\{Q_{1}, Q_{2}, Q_{3}\}),
\end{align*}
then $A_{21} = 0$.
Now assume that the above $A$ belongs to $\Alg(\{Q_1,Q_2, Q_3\})$.
Then $ (I-Q_{i})AQ_{i} = 0$, for $i=1,2,3$. The $(1,2)$ entries of
the equation $AQ_i=Q_iAQ_i$ give us that
\begin{align*}
&(I-K_i)( A_{11}\sqrt{K_i(I-K_i)}U_i+ A_{12}U_i^*(I-K_i)U_i)\cr &=
\sqrt{K_i(I-K_i)}U_i
(A_{21}\sqrt{K_i(I-K_i)}U_i+A_{22}U_i^*(I-K_i)U_i).
\end{align*}
This implies that
\begin{align*}
A_{11}&\sqrt{K_{i}(I-K_{i})^{-1}}U_{i} + A_{12} \\ &=
\sqrt{K_{i}(I-K_{i})^{-1}}U_{i}[A_{21}\sqrt{K_{i}(I-K_{i})^{-1}}U_{i}
+ A_{22}].
\end{align*}
Thus, for $i,j=1,2,3$ and from $\sqrt{K_{i}(I-K_{i})^{-1}} U_{i}
=a_{i}S + \sqrt{H_{1}(I-H_{1})^{-1}}$, we have
\begin{align*}
\sqrt{K_{i}(I  -K_{i})^{-1}} & U_{i}A_{21}
\sqrt{K_{i}(I-K_{i})^{-1}}U_{i} \\ &- \sqrt{K_{j}(I-K_{j})^{-1}}
U_{j}A_{21}\sqrt{K_{j}(I-K_{j})^{-1}}U_{j}\cr &= (a_{i}
-a_{j})[A_{11}S - SA_{22}].
\end{align*}
From this, we conclude that
\begin{align*}
&(a_{1}-a_{2})\sqrt{K_{3}(I-K_{3})^{-1}}U_{3}A_{21}
\sqrt{K_{3}(I-K_{3})^{-1}}U_{3}+\cr
&(a_{2}-a_{3})\sqrt{K_{1}(I-K_{1})^{-1}}U_{1}A_{21}
\sqrt{K_{1}(I-K_{1})^{-1}}U_{1} +\cr
&(a_{3}-a_{1})\sqrt{K_{2}(I-K_{2})^{-1}}U_{2}A_{21}
\sqrt{K_{2}(I-K_{2})^{-1}}U_{2} = 0.
\end{align*}
Using the relation $\sqrt{K_{i}(I-K_{i})^{-1}}U_{i} = a_{i}S +
\sqrt{H_{1}(I-H_{1})^{-1}}$ again, we easily get
\begin{align*}
&[a_{3}^{2}(a_{1} - a_{2}) + a_{1}^{2}(a_{2} - a_{3}) +
a_{2}^{2}(a_{3} - a_{1})]SA_{21}S \cr &+[a_{3}(a_{1} - a_{2}) +
a_{1}(a_{2} - a_{3}) + a_{2}(a_{3} -
a_{1})]\\
&\times [SA_{21}\sqrt{H_{1}(I-H_{1})^{-1}}+
\sqrt{H_{1}(I-H_{1})^{-1}}A_{21}S] \cr &+[(a_{1} - a_{2}) + (a_{2}
- a_{3}) \\ & \qquad + (a_{3} -
a_{1})]\sqrt{H_{1}(I-H_{1})^{-1}}A_{21}\sqrt{H_{1}(I-H_{1})^{-1}}
= 0,
\end{align*}
which implies that
\begin{align*}
(a_{1} - a_{2})(a_{1} - a_{3})(a_{2} - a_{3})SA_{21}S = 0.
\end{align*}
This gives us that $A_{21} = 0$ and the lemma follows.
\end{proof}

Now the follow theorem follows easily from our lemmas
and Theorem 3.

\begin{theorem}
With the above notation,
we have that $\Alg(\F_3)$ is a KS-algebra and $\Lat(\Alg(\F_3))$
is determined by the following: $P\in\Lat(\Alg(\F_3))$ and $P\neq
0, I, P_1$, if and only if
\begin{align*}
P=\left(
    \begin{array}{cc}
      K & \sqrt{K(I-K)}U \\
       U^{*}\sqrt{K(I-K)} & U^{*}(I-K)U \\
    \end{array}
  \right) ,
\end{align*}
where $K$ and $U$ are uniquely determined by the following polar
decomposition with any $a\in\C$: $(a+1)\sqrt{H_1(I-H_1)^{-1}} -a
\sqrt{H_2(I-H_2)^{-1}}V=\sqrt{K(I-K)^{-1}}U$. As a consequence, we
have $\tau(P)=\frac12$ and as $a$ tends to $\infty$, the
projection $P$ converges strongly to $P_1$. Thus
$\Lat(\Alg(\F_3))\setminus \{0,I\}$ is homeomorphic to $S^2$.
\end{theorem}

\begin{proof}
We only need to show the minimality
of $\Lat(\Alg(\F_3))$. Clearly for any sublattice $\LLL_1$
containing only two projections $Q_1,Q_2$ in $\Lat(\Alg(\F_3))$,
$\LLL_1$ can not generate the type $\II_1$ factor $\LLL_{G_3}$. Lemma
4 shows that any reflexive sublattice of $\Lat(\Alg(\F_3))$
containing more than two nontrivial projections must agree with
$\Lat(\Alg(\F_3))$. Thus $\Lat(\Alg(\F_3))$ is a Kadison-Singer
lattice.
\end{proof}

Although the above theorem is stated for $\F_3$, similar results
hold when $P_1, P_2$ and $P_3$ are not assumed to be free. There
are many other factors which can be generated by three
projections. Let $\M$ be a factor of type $\II_1$ with trace
$\tau$ and $M_2(\C)\subseteq\M$ and $M_2(\C)'\cap\M=\NNN$, a
subfactor of $\M$ with generators $H_1$, $H_2$ and $V$, where
$0\le H_1, H_2\le I$ and $V$ is a unitary operator. Suppose
$\HHH=L^2(\M,\tau)$. Let $P_1=\left(
     \begin{array}{cc}
       1 & 0 \\
        0 & 0  \\
     \end{array}\right) \in
M_2(\C)$ be a projection in $\M$. Furthermore, we assume that
\begin{align*}
P_2&=\left(
     \begin{array}{cc}
       H_1 & \sqrt{H_1(I-H_1)} \\
        \sqrt{H_1(I-H_1)} & I-H_1 \\
     \end{array}\right) \\
P_3&=\left(
     \begin{array}{cc}
       H_2 & \sqrt{H_2(I-H_2)}V \\
        V^*\sqrt{H_2(I-H_2)} & I-H_2  \\
     \end{array}\right)
\end{align*}
are projections in $\M$ such that  $P_2\vee P_3=I$ and $P_2\wedge
P_3=0$, and $\sqrt{H_1(I-H_1)^{-1}}$, $\sqrt{H_2(I-H_2)^{-1}}$ and
$\sqrt{H_1(I-H_1)^{-1}} -\sqrt{H_2(I-H_2)^{-1}}V$ are bounded
invertible operators. All above conditions are satisfied when
$H_1$, $H_2$ are free positive elements with disjoint spectra in
$(0,1)$ ($V=I$), and $\sqrt{H_1(I-H_1)^{-1}}$ and
$\sqrt{H_2(I-H_2)^{-1}}$ have disjoint spectra.

\begin{theorem}
With the above notation
and assumptions, we have that $\Alg(\{P_1, P_2,P_3\})$ is a
Kadison-Singer algebra and $\Lat(\Alg(\{P_1, P_2,P_3\}))$ is
determined by the following: $P\in\Lat(\Alg(\{P_1, P_2,P_3\}))$
and $P\neq 0, I, P_1$, if and only if
\begin{align*}
P= \left(
     \begin{array}{cc}
       L & \sqrt{L(I-L)}U \\
       U^{*}\sqrt{L(I-L)} & U^{*}(I-L)U \\
     \end{array}\right)
\end{align*}
where $L$ and $U$ are uniquely determined by the following polar
decomposition with any $a\in\C$: $(a+1)\sqrt{H_1(I-H_1)^{-1}} -a
\sqrt{H_2(I-H_2)^{-1}}V=\sqrt{L(I-L)^{-1}}U$. As a consequence, we
have $\tau(P)=\frac12$.
\end{theorem}

The proof of this theorem will be the same as that of Theorem 4.
Moreover, as $a\to\infty$ in Theorem 4, the projection $P\to P_1$
in strong-operator topology. Thus $\Lat(\Alg(\{P_1, P_2, P_3\}))
\setminus\{0,I\}$ is homeomorphic to the one point compactification
of $\C$, i.e., homeomorphic to $S^2$.

When a lattice contains four or more projections in a von Neumann
algebra, the situation is not clear.  Even for $\F_4$ (the lattice
generated by four free projections), we know from our above result
that $\Lat(\Alg(\F_4))$ contains several copies of $S^2$. But we
do not have a complete characterization of this lattice. The
following theorem shows that $\Alg(\F_\infty)$ contains no nonzero
compact operators. We believe that $\Alg(\F_n)$, when $n\ge4$,
does not contain any compact operators.

\begin{theorem}
Let $\F_\infty$ be the
lattice generated by countably infinitely many free projections of
trace $\frac12$ in $\L_{G_\infty}$, where $G_\infty$ is the free
product of countably infinitely many copies of $\Z_2$. Then
$\Alg(\F_\infty)$ does not contain any nonzero compact operators.
\end{theorem}

\begin{proof}
Let $U_j$, $j=1,2,\ldots$,  be the
standard generators of $\LLL_{G_n}$ and $P_j=\frac{I-U_j}2$ the free
projections in $\F_\infty$. Suppose $T\in\Alg(\F_\infty)$ is a
compact operator. From $(I-P_j)TP_j=0$, we know that
$(I+U_j)T(I-U_j)=0$ for $j=1,2,\ldots$. For any $x,y$ in
$l^2(G_\infty)$, we have
\begin{align*}
 \langle (I+U_j)T(I-U_j)x,y  \rangle= \langle T(I-U_j)x, (I+U_j)y  \rangle=0.
\end{align*}
Since $U_j$ has weak-operator limit zero, as $j\to\infty$, and $T$
compact, we have that $ \langle U_jx,y  \rangle \to 0$, $ \langle Tx, U_jy  \rangle \to0$ and
$TU_jx\to0$ (as $j\to\infty$). This implies that $ \langle Tx,y  \rangle=0$ and
thus $T=0$.
\end{proof}

\section{Reduced lattices}

Suppose $\LLL$ is a lattice of projections in a finite von
Neumann algebra $\M$ with a faithful normal trace $\tau$. Two
projections $P$ and $Q$ in $\LLL$ are said to be {\it connected} if,
for any $\epsilon>0$, there are elements $P_1, P_2,\ldots, P_n$ in
$\LLL$ such that $P_1=P$, $P_n=Q$, $|\tau(P_j-P_{j+1})|<\epsilon$,
and either $P_j\le P_{j+1}$ or $P_j\ge P_{j+1}$, for $j=1,\ldots,
n-1$. Define the connected component $O(P)$ of $P$ to be the set
of all projections in $\LLL$ that are connected with $P$. Let
$\Gamma_0(\LLL)$ be the set of all connected components in $\LLL$. We
shall see that $\Gamma_0(\LLL)$ carries an induced lattice structure
from $\LLL$ and we call $\Gamma_0(\LLL)$ the {\it reduced lattice} of
$\LLL$. It is clear that if $\LLL$ is a continuous nest, then
$\Gamma_0(\LLL)$ contains only one point. A basic fact on connected
components is given in the following.

\begin{proposition}
Suppose $\LLL$ is a
lattice of projections in a finite von Neumann algebra $\M$,
$P,Q\in\LLL$. If $O(P)\neq O(Q)$, then $O(P)\cap O(Q)=\emptyset$.
\end{proposition}

The proof of this proposition follows easily from the definition.
If $O(P)$ and $O(Q)$ are two elements in $\Gamma_0(\LLL)$, then, for
any $Q_1\in O(Q)$, it is easy to see that $O(P\vee Q)=O(P\vee
Q_1)$. Thus $O(P\vee Q)$ depends on the components $O(P)$ and
$O(Q)$, not the choices of $P$ and $Q$ in the components. We
define $O(P)\vee O(Q)=O(P\vee Q)$. Similarly we define that
$O(P)\wedge O(Q)=O(P\wedge Q)$. It is easy to shows that
$\Gamma_0(\LLL)$ is a lattice. The following theorem is immediate.

\begin{theorem}
Suppose $\LLL$ is a
lattice of projections in a finite von Neumann algebra $\M$ with a
(faithful normal) trace $\tau$. If $\tau(\LLL)$ contains only
finitely many trace values, then $\Gamma_0(\LLL)$ is the same as
$\LLL$ (i.e., every connected component contains only one element in
$\LLL$).
\end{theorem}

From this theorem, we know that
$\Gamma_0(\Lat(\Alg(\F_3)))=\Lat(\Alg(\F_3))$. Suppose $\LLL$ is a
Kadison-Singer lattice and $\LLL$ generates a finite von Neumann
algebra. If $\Gamma_0(\LLL)$ contains only one point, then we call
$\LLL$ {\it contractible}. Thus continuous nests are contractible. In
a forthcoming paper, we shall construct other contractible
Kadison-Singer lattices and also show that manifolds of higher
dimensions can appear as the reduced lattices of Kadison-Singer
lattices.

Reduced lattices can not contain continuous nests. The following
theorem shows that all possible trace values can appear in a
reduced lattice.

\begin{theorem}
Suppose $\LLL^{(n)}$ is
the lattice given in Section 4 in \cite{GY}. Then
$\Gamma_0(\LLL^{(n)})=\LLL^{(n)}$.
\end{theorem}

The proof is a direct computation. We omit the details here.
%%-- text of paper here --

%% == end of paper:

%% Optional Materials and Methods Section
%% The Materials and Methods section header will be added automatically.

%% Enter any subheads and the Materials and Methods text below.
%\begin{materials}
% Materials text
%\end{materials}


%% Optional Appendix or Appendices
%% \appendix Appendix text...
%% or, for appendix with title, use square brackets:
%% \appendix[Appendix Title]

\begin{acknowledgments}
Research supported in part by President Fund of Academy of
Mathematics and Systems Science, Chinese Academy of Sciences
\end{acknowledgments}

%% PNAS does not support submission of supporting .tex files such as BibTeX.
%% Instead all references must be included in the article .tex document.
%% If you currently use BibTeX, your bibliography is formed because the
%% command \verb+\bibliography{}+ brings the <filename>.bbl file into your
%% .tex document. To conform to PNAS requirements, copy the reference listings
%% from your .bbl file and add them to the article .tex file, using the
%% bibliography environment described above.

%%  Contact pnas@nas.edu if you need assistance with your
%%  bibliography.

% Sample bibliography item in PNAS format:
%% \bibitem{in-text reference} comma-separated author names up to 5,
%% for more than 5 authors use first author last name et al. (year published)
%% article title  {\it Journal Name} volume #: start page-end page.
%% ie,
% \bibitem{Neuhaus} Neuhaus J-M, Sitcher L, Meins F, Jr, Boller T (1991)
% A short C-terminal sequence is necessary and sufficient for the
% targeting of chitinases to the plant vacuole.
% {\it Proc Natl Acad Sci USA} 88:10362-10366.


%% Enter the largest bibliography number in the facing curly brackets
%% following \begin{thebibliography}

\begin{thebibliography}{21}

\bibitem{GY} L. Ge and W. Yuan, {\em Type I C*-algebrasKadison-Singer algebras,
I---hyperfinite case,} to appear, 2009

\bibitem{KS} R. Kadison and I. Singer, {\em Triangular operator
algebras. Fundamentals and hyperreducible theory,} Amer. J. Math.,
{\bf 82 } (1960),  227--259.

\bibitem{GS}  L. Ge and J. Shen, {\em On the generator problem of
von Neumann algebras,} Proc. of ICCM 2004 (Hong Kong), Intern.
Press, Boston.

\bibitem{H}  P. Halmos, {\em Reflexive lattices of subspaces,}
J. London Math. Soc. {\bf 4}  (1971),  257--263.

\bibitem{Ha} K. Harrison, {\em On lattices of invariant
subspaces,} Doctoral Thesis, Monash University, Melbourne, 1970.

%%\ref[Ka]  R. Kadison,
%%     {\eightsl On the orthogonalization of operator
%%     representations,} Amer. J. Math.
%%     {\eightbf 78}  (1955),  600--621.

\bibitem{VDN} D. Voiculescu, K. Dykema and A. Nica, {\em ``Free
Random Variables,''} CRM Monograph Series, vol. 1, 1992.

\bibitem{KR} R.\ Kadison and J.\ Ringrose, {\em ``Fundamentals of
the Operator Algebras,''} vols. I and II, Academic Press, Orlando,
1983 and 1986.

\end{thebibliography}

\end{article}
%%%%%%%%%%%%%%%%%%%%%%%%%%%%%%%%%%%%%%%%%%%%%%%%%%%%%%%%%%%%%%%%

%% Adding Figure and Table References
%% Be sure to add figures and tables after \end{article}
%% and before \end{document}

%% For figures, put the caption below the illustration.
%%
%% \begin{figure}
%% \caption{Almost Sharp Front}\label{afoto}
%% \end{figure}

%% For Tables, put caption above table
%%
%% Table caption should start with a capital letter, continue with lower case
%% and not have a period at the end
%% Using @{\vrule height ?? depth ?? width0pt} in the tabular preamble will
%% keep that much space between every line in the table.

%% \begin{table}
%% \caption{Repeat length of longer allele by age of onset class}
%% \begin{tabular}{@{\vrule height 10.5pt depth4pt  width0pt}lrcccc}
%% table text
%% \end{tabular}
%% \end{table}

%% For two column figures and tables, use the following:

%% \begin{figure*}
%% \caption{Almost Sharp Front}\label{afoto}
%% \end{figure*}

%% \begin{table*}
%% \caption{Repeat length of longer allele by age of onset class}
%% \begin{tabular}{ccc}
%% table text
%% \end{tabular}
%% \end{table*}

\end{document}
