\section{Hyperfinite Kadison-Singer factors}

\noindent In this section, we shall construct some hyperfinite
Kadison-Singer factors. We begin with a UHF C*-algebra $\A_n$ (see
\cite{G}) obtained by taking the completion (with respect to operator
norm) of $\otimes_{1}^\infty M_n(\C)$, denoted by $\AAA$ (or
equivalently, $\AAA=\cup_{j=1}^\infty M_{n^j}(\C)$). We denote by
$M_n^{(k)}(\C)$ the $k$th copy of $M_n(\C)$ in $\AAA$ (or $\A_n$)
and $E_{ij}^{(k)}$, $i,j=1,\ldots,n$, the standard matrix unit
system for $M_n^{(k)}(\C)$, for $k=1,2,\ldots$. Then we may write
$\AAA=M_n^{(1)}(\C)\otimes M_n^{(2)}(\C)\otimes\cdots$.  Let
$\NNN_m=M_n^{(1)}(\C)\otimes M_n^{(2)}(\C)\otimes\cdots\otimes
M_n^{(m)}(\C)$ ($\cong M_{n^m}(\C)$). Then
$\AAA=\cup_{m=1}^\infty\NNN_m$. Now, we construct inductively a
family of projections in $\NNN_m$.

When $m=1$, define $P_{1j}=\sum_{i=1}^j E_{ii}^{(1)}$, for
$j=1,\ldots, n-1$, and $P_{1n}=\frac1n\sum_{s,t=1}^n
E_{st}^{(1)}$. Suppose for $k= m-1$, $P_{kj}\in\NNN_k$ are
defined, for $j=1,\ldots, n$. Now we define
\begin{align}
P_{mj}&=P_{m-1,n-1}+ (I-P_{m-1,n-1})\sum_{i=1}^j E_{ii}^{(m)},
 j=1,\ldots, n-1, \\
P_{mn}&=P_{m-1, n-1} +(I-P_{m-1,n-1})\left(\frac{1}{n} \sum_{s,t=1}^n
E_{st}^{(m)}\right) .
\end{align}
Denote by $\LLL_{m}$ the lattice generated by $\{ P_{kj}: 1\leq k\leq m, 1\leq j \leq n \}$ and $\LLL_\infty=\cup_m\LLL_m$, the lattice generated by $\{P_{kj}: k\geq 1, 1\leq j\leq n\}$. We can easily show
inductively that $\NNN_m$ is generated by $\LLL_m$ (as a
finite-dimensional von Neumann algebra).

Let $\rho_n$ be a faithful state on $M_n(\C)$. We extend $\rho_n$
to a state on $\A_n$, denoted by $\rho$, i.e.,
$\rho=\rho_n\otimes\rho_n\otimes\cdots$. Let $\HHH$ be the Hilbert
space obtained by GNS construction on $(\A_n,\rho)$. It is well
known (see \cite{P}) that the weak-operator closure of $\A_n$ in
$\B(\HHH)$ is a hyperfinite factor $\RRR$ (when $\rho$ is a trace,
the factor $\RRR$ is type II$_1$). Then $\LLL_m$ and $\LLL_\infty$
become lattices of projections in $\RRR$.

\begin{theorem} 
With the
above notation, we have that $\Alg(\LLL_\infty)$ is a Kadison-Singer
factor containing the hyperfinite factor $\RRR'$ as its
diagonal.
\end{theorem}


Our above defined hyperfinite KS-factor depends on $n$ $(\geq 2)$
appeared in the UHF algebra construction. We shall see in Section
4 that, when $\rho$ is a trace, for different $n$, the
Kadison-Singer algebras constructed above are not unitarily
equivalent.

To prove Theorem 3, we need some lemmas.

\begin{lemma} 
With $\LLL_1\subset\NNN_1$ defined above and $E_{ij}^{(1)}$,
$i,j=1,\ldots,n$, the matrix units for $\NNN_1$, we have
\begin{align*}
\Alg(\LLL_1)&=\{ T\in\B(\HHH):  \quad E_{ii}^{(1)}TE_{jj}^{(1)}=0,
\qquad 1\leq j < i\leq n; \\
& \sum_{j=1}^n E_{11}^{(1)} TE_{j1}^{(1)}=
\sum_{j=2}^n E_{12}^{(1)}T
E_{j1}^{(1)}=\cdots=E_{1n}^{(1)}TE_{n1}^{(1)}\}.
\end{align*}
\end{lemma}

\begin{proof}
 Let $T$ be an element in
$\Alg(\LLL_1)$. Since $P_{1j}=\sum_{i=1}^j E_{ii}^{(1)}\in\LLL_1$ for
$j=1,\ldots,n-1$, we know that $E_{ii}^{(1)}TE_{jj}^{(1)}=0, 1\leq
j<i\leq n$. From $TP_{1n}^{(1)}=P_{1n}^{(1)}TP_{1n}^{(1)}$, we have
\begin{align*}
nT\sum_{i,j=1}^n E_{ij}^{(1)}=(\sum_{i,j=1}^n E_{ij}^{(1)} )T(
\sum_{i,j=1}^n E_{ij}^{(1)}) .
\end{align*}
Multiplying the above equation by $E_{1l}^{(1)}$ on left and
$E_{11}^{(1)}$ on right, we have
\begin{align*}
nE_{1l}^{(1)}T\sum_{i=1}^nE_{i1}^{(1)}&=n\sum_{i=1}^n E_{1l}^{(1)}
TE_{i1}^{(1)}\\
&=(\sum_{j=1}^nE_{1j}^{(1)})T (\sum_{i=1}^n
E_{i1}^{(1)})=\sum_{i,j=1}^n E_{1i}^{(1)} T E_{j1}^{(1)} .
\end{align*}
The right hand side is independent of $l$. By letting $l=1,\ldots,
n$ and applying $E_{ii}^{(1)}TE_{jj}^{(1)}=0$ when $1\leq j<i\leq
n$, we have that $\sum_{i=1}^n E_{11}^{(1)} TE_{i1}^{(1)}=
\sum_{i=2}^n E_{12}^{(1)}T
E_{i1}^{(1)}=\cdots=E_{1n}^{(1)}TE_{n1}^{(1)}$. It is easy to
check that when $T$ satisfies those identities in the lemma, $T$
must be an element in $\Alg(\LLL_1)$.
\end{proof}

In terms of matrix representations of elements in $\Alg(\LLL_1)$
with respect to matrix units in $\NNN_1$, we know from Lemma 2
that such an element $T$ is upper triangular. Moreover, one can
arbitrarily choose the strictly upper triangular part of $T$ and
use equations
\begin{align*}
\sum_{j=1}^n E_{11}^{(1)} TE_{j1}^{(1)}=
\sum_{j=2}^n E_{12}^{(1)}T
E_{j1}^{(1)}=\cdots=E_{1n}^{(1)}TE_{n1}^{(1)}
\end{align*}
to determine the diagonal entries of $T$ so that $T\in\Alg(\LLL_1)$. 
%%Notice that $\Alg(\L_\infty)=\cap_{j=1}^\infty\Alg(\L_j)$. Thus
%%our goal is to understand all elements in each $\Alg(\L_j)$.

\begin{lemma} For any $T$ in
$\Alg(\LLL_1)$, there are $T_1$ in $\Alg(\LLL_1)\cap \LLL_1'$ and $T_2$
in $\Alg(\LLL_\infty)$ $(\subseteq \Alg(\LLL_1))$ such that
$T=T_1+T_2$. In particular, when $E_{nn}^{(1)}TE_{nn}^{(1)}=0$,
$T=T_2\in\Alg(\LLL_\infty)$.
\end{lemma}

\begin{proof}
Suppose $T\in\Alg(\LLL_1)$ and let
\begin{align}
T_1=\sum_{i=1}^n E_{in}^{(1)}T E_{ni}^{(1)},\qquad T_2=T-T_1.
\end{align}
It is easy to check that $E_{ii}^{(1)}T_1 E_{jj}^{(1)}=0$ when
$i\neq j$ and, by Lemma 2, $T_1\in\Alg(\LLL_1)$. Moreover, for all
$l,k$, $ E_{lk}^{(1)}T_1=E_{lk}^{(1)}\sum_{i=1}^nE_{in}^{(1)}T
E_{ni}^{(1)} =E_{ln}^{(1)}T E_{nk}^{(1)}=T_1E_{lk}^{(1)}$. This
implies that $T_1\in\LLL_1'$ ($=\NNN_1'$).

Clearly $T_2\in\Alg(\LLL_1)$. Thus $T_2P_{1k}=P_{1k}T_2P_{1k}$, for
$k=1,\ldots, n$. We need to show that $T_2P_{jk}=P_{jk}T_2P_{jk}$,
for $j\geq 2$ and $k=1,\ldots, n$. By the definition of $P_{jk}$ in
(1), we know that $I-P_{jk}\leq E_{nn}^{(1)}$ for $j\geq 2$.  Now,
from $T_2\in\Alg(\LLL_1)$, we have
\begin{align*}
E_{nn}^{(1)}T_2&=E_{nn}^{(1)}\sum_{1\le l\le k\le n} E_{ll}^{(1)}
T_2 E_{kk}^{(1)}\\
&=E_{nn}^{(1)}T_2E_{nn}^{(1)}
=E_{nn}^{(1)}(T-T_1)E_{nn}^{(1)}=0.
\end{align*}
This implies that $0=(I-P_{jk})T_2=(I-P_{jk})T_2P_{jk}$. Thus we
have $T_2\in\Alg(\LLL_\infty)$.
\end{proof}

\begin{lemma}
If $T\in\Alg(\LLL_m)$ and
$(I-P_{m,n-1})T=0$ for some $m\geq1$, then $T\in\Alg(\LLL_\infty)$.
\end{lemma}

When $m=1$, the proof is given above. For a general $m$, the
argument is similar. We omit its details here. From the
construction of $P_{mk}$'s, we know that the differences between
elements in $\Alg(\LLL_m)$ and those in $\Alg(\LLL_{m+1})$ only occur
within $I-P_{m,n-1}$ $(=E_{nn}^{(1)}\otimes\cdots\otimes
E_{nn}^{(m)})$. Thus we have the following lemma.

\begin{lemma}
If $T\in\Alg(\LLL_m)$, then
$T\in\Alg(\LLL_{m+1})$ if and only if, for $j=1,\ldots,n$, the
projections $(I-P_{m,n-1})P_{m+1,j}(I-P_{m,n-1})$ are invariant
under $(I-P_{m,n-1})T(I-P_{m,n-1})$.
\end{lemma}

Inductively, we can easily prove the following lemma which
generalizes Lemma 3.

\begin{lemma}
If $T\in\Alg(\LLL_m)$, then
there are $T_1,\ldots, T_{m+1}$ in $\Alg(\LLL_m)$ such that
$T=T_1+\cdots +T_{m+1}$, where
$T_i\in\NNN_{i-1}'\cap\Alg(\LLL_\infty)$, $(I-P_{i,n-1})T_i=0$ for
$i=1,\ldots, m$ (here we let $\NNN_0=\C I$), and
$T_{m+1}\in\NNN_m'\cap\Alg(\LLL_m)$.
\end{lemma}

%%The following lemma is the key to prove the maximality of
%%$\Alg(\LLL_\infty)$.
%%
%%%\vskip6pt

\begin{lemma}
Suppose $T$ is an element
in $\B(\HHH)$ and $\AAA$ is the algebra generated by $T$ and
$\Alg(\LLL_\infty)$. If $\AAA\cap\AAA^*=\Alg(\LLL_\infty)
\cap\Alg(\LLL_\infty)^* =\RRR'$, then $T\in\Alg(\LLL_1)$.
\end{lemma}

\begin{proof}
Suppose $T\in\AAA$ is given. From the
comments preceding Lemma 3 and by taking a difference from an
element in $\Alg(\LLL_\infty)$, we may assume that, with respect to
matrix units $E_{ij}^{(1)}$ in $\NNN_1$, $T$ is lower triangular,
i.e., $E_{ii}^{(1)}TE_{jj}^{(1)}=0$ for $i<j$. Now we want to show
that $T$ is diagonal. If the strictly lower triangular entries of
$T$ are not all zero, then let $i_0$ be the largest integer such
that $E_{i_0i_0}^{(1)}TE_{jj}^{(1)}\neq0$ for some $j<i_0$. Among
all such $j$, let $j_0$ be the largest. Then we have that
$E_{ii}^{(1)}TE_{jj}^{(1)}=0$ if $i>j$ and $i>i_0$; or
$i=i_0>j>j_0$. It is easy to check (from Lemma 4) that
$E_{j_0,i_0-1}^{(1)}-E_{j_0i_0}^{(1)}\in\Alg(\LLL_\infty)$. Then $
T(E_{j_0,i_0-1}^{(1)}-E_{j_0i_0}^{(1)})  \in\AAA$. Define $T_1
=T(E_{j_0,i_0-1}^{(1)}-E_{j_0i_0}^{(1)})
 $. Then
\begin{align*}
T_1 &= \sum_{n\ge k \geq l\ge1} E_{kk}^{(1)}TE_{ll}^{(1)}( E_{j_0,
i_0 - 1}^{(1)} - E_{j_{0}i_{0}}^{(1)})\\
 &= \sum_{i_{0} \geq k \geq j_{0}}
E_{kk}^{(1)}T( E_{j_0, i_0 - 1}^{(1)} - E_{j_{0}i_{0}}^{(1)} ).
\end{align*}
Let $T_2= E_{i_0 i_0}^{(1)}T(E_{j_0, i_0 - 1}^{(1)} -
E_{j_{0}i_{0}}^{(1)})$, and 
\begin{align*}
T_{3} =\sum_{i_0 > k \geq j_{0}}
E_{kk}^{(1)}T(E_{j_0, i_0 - 1}^{(1)} - E_{j_{0}i_{0}}^{(1)}).
\end{align*}
Then $T_1=T_2+T_3$. From Lemma 4 again, $T_3\in\Alg(\LLL_\infty)$.
This implies that $T_2\in\AAA$.

Let $E_{i_0i_{0}}^{(1)}T E_{j_0i_{0}}^{(1)}=HV$ be the polar
decomposition (in $\B(\HHH)$), where $H$ is positive and $V$ a
partial isometry. From our assumption that
$E_{i_0i_0}^{(1)}TE_{j_0j_0}^{(1)}\neq 0$, we have $H\neq 0$,
$E_{i_0i_0}^{(1)} H=H E_{i_0i_0}^{(1)}=H$ and $E_{i_0i_0}^{(1)}
V=V E_{i_0i_0}^{(1)}=V$. Then $T_2 = HV E_{i_{0}, i_{0}-1}^{(1)} -
HV$. Define
\begin{align*}
T_{4} &= - E_{i_{0}-1,i_{0}}^{(1)}V^{*} E_{i_{0}, i_{0}-1}^{(1)} +
E_{i_{0}-1,i_{0}}^{(1)}V^{*}E_{i_{0} i_{0}}^{(1)},\cr
 T_{5} &= E_{i_{0}-1,i_{0}}^{(1)}HE_{i_{0}, i_{0} - 1}^{(1)} -
E_{i_{0}-1,i_{0}}^{(1)}HE_{i_{0} i_{0}}^{(1)}.
\end{align*}
It is easy to check, from Lemma 3, that $T_{4}, T_{5}
\in\Alg(\LLL_\infty)$. Let
\begin{align*}
T_{6} &= T_{2}T_{4} + T_{5} \\
&=(  HV E_{i_{0}, i_{0}-1}^{(1)} - HV)(- E_{i_{0}-1,i_{0}}^{(1)}V^{*} E_{i_{0}, i_{0}-1}^{(1)} 
\cr & +
E_{i_{0}-1,i_{0}}^{(1)}V^{*}E_{i_{0} i_{0}}^{(1)})
+ E_{i_{0}-1,i_{0}}^{(1)}HE_{i_{0}, i_{0} - 1}^{(1)} -
E_{i_{0}-1,i_{0}}^{(1)}HE_{i_{0} i_{0}}^{(1)}      \cr
    &= -HE_{i_{0}, i_{0} - 1}^{(1)} + HE_{i_{0}i_{0}}^{(1)} \\
    & +
E_{i_{0}-1,i_{0}}^{(1)}HE_{i_{0}, i_{0} - 1}^{(1)} -
E_{i_{0}-1,i_{0}}^{(1)}HE_{i_{0} i_{0}}^{(1)}.\cr 
&= -HE_{i_{0},
i_{0} - 1}^{(1)} + H + E_{i_{0}-1,i_{0}}^{(1)}HE_{i_{0}, i_{0} -
1}^{(1)} - E_{i_{0}-1,i_{0}}^{(1)}H.
\end{align*}
Clearly $T_6\in\AAA$ and $T_{6}^{*} = T_{6}$. But $T_6$ is not upper
triangular. Thus $T_6 \notin \Alg(\LLL_\infty)$. This implies that
$\AAA\cap\AAA^*\neq \Alg(\LLL_\infty)\cap\Alg(\LLL_\infty)^*$. This
contradiction shows that $T$ must be diagonal. Thus we have that
$T=\sum_{j=1}^n E_{jj}^{(1)}TE_{jj}^{(1)}$. Now we show that
$E_{11}^{(1)}TE_{11}^{(1)}=E_{1j}^{(1)}TE_{j1}^{(1)}$ for
$j=1,\ldots,n$.

Assume that there is an $i$ such that $E_{11}^{(1)}TE_{11}^{(1)}
\neq E_{1i}^{(1)}TE_{i1}^{(1)}$. Define
\begin{align*}
T_7= (E_{11}^{(1)} - E_{1i}^{(1)})T = E_{11}^{(1)}TE_{11}^{(1)} -
E_{1i}^{(1)}TE_{ii}^{(1)}.
\end{align*}
 Then similar to the construction of $T_1$, we see that $T_7\in\AAA$. Again
write $T_{8} = - E_{1i}^{(1)}TE_{i1}^{(1)} +
E_{1i}^{(1)}TE_{ii}^{(1)}$. One checks (by Lemma 3) that $T_{8}
\in \Alg(\LLL_\infty)$. Then
\begin{align*}
0 \neq T_{7} + T_{8} = E_{11}^{(1)}TE_{11}^{(1)} -
E_{1i}^{(1)}TE_{i1}^{(1)} \in {\A}.
\end{align*}
Set $T_7+T_{8}=V'H'$, the polar decomposition with $V'$ a partial
isometry. One easily checks that ${V'}^{*} - {V'}^{*}E_{12}^{(1)}
\in \Alg(\LLL_\infty)$. Then $ ({V'}^{*} -
{V'}^{*}E_{12}^{(1)})(T_7+T_{8}) = H'  \in {\A}$. Since $H'$ is
selfadjoint, $H'\in\A\cap\A^*$. But $E_{11}^{(1)}H'E_{11}^{(1)} \neq
0$ (with $E_{22}^{(1)}H'E_{22}^{(1)} = \cdots =
E_{nn}^{(1)}H'E_{nn}^{(1)}=0$).
 Thus $H'\notin\NNN_1'$
$(\supseteq\Alg(\LLL_\infty)\cap\Alg(\LLL_\infty)^*)$. This implies that
$\AAA\cap\AAA^*\neq \Alg(\LLL_\infty)\cap\Alg(\LLL_\infty)^*$. This
contradiction shows that  $E_{11}^{(1)}TE_{11}^{(1)} = \cdots =
E_{1n}^{(1)}TE_{n1}^{(1)}$. Therefore $T \in {\LLL}_{1}'\subseteq\Alg(\LLL_1)$.
\end{proof}

Now we restate our Theorem 3 in a slightly stronger form.

\begin{theorem}
Suppose $\AAA$ is a
subalgebra of $\B(\HHH)$ such that $\Alg(\LLL_\infty)\subseteq\AAA$ and
$\AAA\cap\AAA^*=\Alg(\LLL_\infty)\cap\Alg(\LLL_\infty)^*$. Then
$\AAA=\Alg(\LLL_\infty)$.
\end{theorem}

\begin{proof}
Without the loss of generality, we
may assume that $\AAA$ is generated by $T$ and $\Alg(\LLL_\infty)$.
From the above lemma, we have that $T\in\Alg(\LLL_1)$. Suppose
$T\in\Alg(\LLL_m)$ but $T\notin\Alg(\LLL_{m+1})$. From Lemma 6, we
write $T=S+T'$ such that $S\in\Alg(\LLL_\infty)$ and
$T'\in\NNN_m'\cap\Alg(\LLL_m)$. When we restrict all operators to
the commutant of $\NNN_m$ and working with matrix units
$E_{ij}^{(m+1)}$, similar computation as in the proof of Lemma 7
will show that $T'\in\Alg(\LLL_{m+1})$. This contradiction shows
that $T\in\cap_{m=1}^\infty\Alg(\LLL_m)=\Alg(\LLL_\infty)$.
\end{proof}


In the above theorem, we did not assume the closedness of $\AAA$
under any topology. Thus $\Alg(\LLL_\infty)$ has an algebraic
maximality property. Next section, we will show that
$\Lat(\Alg(\LLL_\infty))$ is the strong-operator closure of
$\LLL_\infty$. %%%Other Kadison-Singer lattices are also given.


\vskip20pt















