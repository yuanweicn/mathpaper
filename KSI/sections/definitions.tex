\section{ Definitions}

\noindent For basic theory on operator algebras, we refer to \cite{KR}.
We recall the definitions of some well known classes of non
selfadjoint operator algebras. For details on triangular algebras,
we refer to \cite{KS}. For others, we refer to \cite{RR}.

Suppose $\HHH$ is a separable Hilbert space and $\B(\HHH)$ the algebra
of all bounded linear operators on $\HHH$.  Let $\M$ be a von
Neumann subalgebra of $\B(\HHH)$. A {\it triangular (operator)
algebra} is a subalgebra $\TT$ of $\M$ such that $ \TT \cap \TT^*
= \A$, a maximal abelian selfadjoint subalgebra (masa) of $\M$.
One of the interesting cases is when  $\M=\B(\HHH)$.

Let $\PPP$ be a set of (orthogonal) projections in $\B(\HHH)$. Define
$\Alg(\PPP)=\{ T \in\B(\HHH): TP=PTP, \ {\mathrm for\ all}\ P\in\PPP\}$.
Then $\Alg(\PPP)$ is a weak-operator closed subalgebra of $\B(\HHH)$.
Similarly, for a subset $\mathcal S$ of $\B(\HHH)$, define $\Lat (\mathcal
S)=\{P\in\B(\HHH): P\ {\mathrm a \ projection},\ TP=PTP, \ {\mathrm for\
all}\ T\in\mathcal S\}$. Then $\Lat(\SSS)$ is a strong-operator closed
lattice of projections. A subalgebra $\B$ of $\B(\HHH)$ is called a
{\it reflexive (operator) algebra} if $\B=\Alg(\Lat (\B))$.
Similarly, a lattice $\LLL$ of projections in $\B(\HHH)$ is called a
{\it reflexive lattice (of projections)} if $\LLL=\Lat(\Alg(\LLL))$. A
{\it nest} is a totally ordered reflexive lattice. If $\LLL$ is a
nest, then $\Alg(\LLL)$ is called a {\it nest algebra}. Nest
algebras are generalizations of (hyperreducible) ``maximal
triangular'' algebras introduced by Kadison and Singer in \cite{KS}.
Kadison and Singer also show that nest algebras are the only
maximal triangular reflexive algebras (with a commutative lattice
of invariant projections). Motivated by this, we give the
following definition:

\vskip6pt

\begin{definition}
A subalgebra $\A$ of $\B(\HHH)$
is called a {\sl Kadison-Singer (operator) algebra} (or {\it
KS-algebra}) if $\A$ is reflexive and maximal with respect to the
{\it diagonal subalgebra} $\A\cap \A^*$ of $\A$, in the sense that
if there is another reflexive subalgebra $\frak B$ of $\B(\HHH)$
such that $\A\subseteq\frak B$ and $\frak B\cap\frak B^*=\A\cap
\A^*$, then $\A=\frak B$. When the diagonal of a KS-algebra is a
factor, we call the KS-algebra a {\it KS-factor} or a {\it
Kadison-Singer factor}. A lattice $\LLL$ of projections in $\B(\HHH)$
is called a {\it Kadison-Singer lattice} (or {\sl KS-lattice}) if
$\LLL$ is a minimal reflexive lattice that generates the von Neumann
algebra $\LLL''$, or equivalently $\LLL$ is reflexive and $\Alg(\LLL)$
is a Kadison-Singer algebra.
\end{definition}

\vskip6pt

Clearly nest algebras are KS-algebras. Since a nest generates an
abelian von Neumann algebra, we may view nest algebras as ``type
I'' KS-algebras and general KS-algebras as ``quantized'' nest
algebras. The maximality condition for a KS-algebra requires that
the associated lattice is ``reflexive and minimal'' in the sense
that there is no smaller reflexive sublattice that generates the
commutant of the diagonal algebra. We believe that the following
statement is true:

\begin{conjecture}
If $\A$ is a KS-algebra in $\B(\HHH)$ and $P\in\Lat(\A)$ ($\neq 0,I$), then
$I-P\notin\Lat(\A)$, i.e., a KS-algebra has no nontrivial reducing
invariant subspaces.
\end{conjecture}

The following lemma is an immediate consequence of the above
definition.

\vskip6pt

\begin{lemma}
Suppose $\A$ is a Kadison-Singer algebra in $\B(\HHH)$ and $\M$ is the commutant of $\A\cap\A^*$ in $\B(\HHH)$. Then $\Lat(\A)\subseteq\M$ and generates
$\M$ as a von Neumann algebra.
\end{lemma}

%\noindent{\it Proof.}\HHHskip8pt Since $\A\cap\A^*$ is a von Neumann
%algebra and $\Lat(\A\cap\A^*)\subseteq\M$, we have
%$\Lat(\A)\subseteq\M$. Let $\NNN$ be the von Neumann algebra
%generated by $\Lat(\A)$. Then $\NNN$ is a subalgebra of $\M$,
%which implies that $\M'\subseteq\NNN'$. It is clear that
%$\NNN'\subseteq \Alg(\Lat(\A))=\A$ and is selfadjoint. Thus
%$\NNN'\subseteq \A\cap\A^*=\M'$. Now $\NNN'=\M'$, which implies
%that $\NNN=\M$.
%\endproof

%\vskip6pt

%Suppose $\LLL=\{0,I\}$ is the trivial lattice. Then
%$\A=\Alg(\LLL)=\B(\HHH)$ is a KS-algebra and is also a factor of type
%%%%%%%%%I.
When $\A$ is a KS-algebra and $\A\cap\A^*$ is a factor of type I,
II or III, then $\A$ is called a KS-factor of the same type. In
the same way, we can further classify KS-factors into type II$_1$,
II$_\infty$, etc., similar to usual factors. A KS-algebra $\A$ is
said to be in a {\sl standard form}, or a {\it standard}
KS-algebra, if the diagonal $\A\cap\A^*$ of $\A$ is in a standard
form, i.e., $\A\cap\A^*$ has a cyclic and separating vector in
$\HHH$. In this case, the von Neumann algebra generated by
$\Lat(\A)$ (or the core, see \cite{KS}) is also in a standard form.

In the present article, one of our main goals is to give some
nontrivial examples of KS-algebras, in particular, KS-factors of
type II and III. The following theorem shows that all type II and
type III KS-algebras are truly non selfadjoint algebras.

\begin{theorem}
If $\A$ is a KS-algebra of type II or type III in $\B(\HHH)$, then $\A$ is not
selfadjoint.
\end{theorem}

\begin{proof}
Assume on the contrary that $\A$ is
selfadjoint. From our assumption we know that $\A'$ contains a
$2\times 2$ matrix subalgebra $\M_2$. Let $E_{ij}$, $i,j=1,2$, be
a matrix unit system for $\M_2$. Then one can construct a
reflexive lattice $\LLL$ generated by all projections in the
relative commutant of $\M_2$ in $\A'$ and two non commuting
projections $E_{11}$ and $\frac12\sum_{i,j}E_{ij}$ in $\M_2$. It
is easy to see that $\LLL$ generates $\A'$ as a von Neumann algebra.
One easily checks that $\Alg(\LLL)$ is non selfadjoint but
reflexive. Moreover its diagonal is equal to the commutant of
$\LLL$, which agrees with $\A$. This contradicts to the assumption
that $\A$ is a KS-algebra.
\end{proof}

\vskip6pt

Similar argument shows that any nontrivial standard KS-algebra,
even in the case of type I, is not selfadjoint. Standard
KS-algebras can be viewed as {\it maximal} upper triangular
algebras with a von Neumann algebra as its diagonal. \vskip6pt

\begin{definition}
Two Kadison-Singer
algebras are said to be {\sl isomorphic} if there is a norm
preserving (algebraic) isomorphism between the two algebras. Two
KS-algebras are called {\it unitarily equivalent} if there is a
unitary operator between the underlying Hilbert spaces that
induces an isomorphism between the KS-algebras. 
\end{definition}

It is easy to see that an isomorphism between two Kadison-Singer
algebras induces a * isomorphism between the diagonal subalgebras.

For lattices of projections on a Hilbert space, the definition of
an isomorphism is subtle. We consider a simple example where a
lattice $\LLL_0$ contains two free projections of trace $\frac12$
and $0, I$ in a type II$_1$ factor. As a lattice (with respect to
union, intersection and ordering), it is isomorphic to the lattice
generated by two rank-one projections on a two-dimensional
euclidean space. We shall call such an isomorphism (which
preserves only the lattice structure) an {\it algebraic (lattice)
isomorphism}. An isomorphism between two lattices, in this paper,
is an isomorphism that also induces a
* isomorphism between the von Neumann algebras they generate. To
avoid confusion, sometimes we call such isomorphisms {\it spatial
isomorphisms} between two lattices of projections.