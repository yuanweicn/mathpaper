\section{Introduction}

In \cite{KS}, Kadison and Singer initiate the study of
non-self-adjoint algebras of bounded operators on Hilbert spaces.
They introduce a class of algebras they call {\it triangular
operator algebras.}  An algebra $\TT$ is triangular (relative to a
factor $\M$) when $\TT\cap\TT^*$ is a maximal abelian (self-adjoint)
algebra in the factor $\M$.  (See section 2 for definitions.)  When
the factor is the algebra of all $n\times n$ complex matrices, this
condition guarantees that there is a unitary matrix $U$ such that
the mapping $A\to UAU^*$ transforms $\TT$ onto a subalgebra of the
upper triangular matrices.

Beginning with \cite{KS}, the theory of non-self-adjoint operator
algebras has undergone a vigorous development parallel to, but not
nearly as explosive as, that of the self-adjoint theory, the C*- and
von Neumann algebra theories.  Of course, the self-adjoint theory
began with the 1929-30 von Neumann article \cite{vN} --- well before the
1960 \cite{KS} article appeared.  Surprisingly, to the present authors,
and apparently to Kadison and Singer as well (from private
conversations), this parallel development has not produced the
synergistic interactions we would have expected from subjects that
are so closely and naturally related, and so likely to benefit from
cross connections with one another.

Considerable effort has gone into the study of triangular operator
algebras (see, for example, \cite{MSS}) and \cite{Ho}) and another class of
non-self-adjoint operator algebras, the {\it reflexive algebras\/}
(see, for example, \cite{H2}, \cite{Hd}, \cite{RR}, and \cite{La}).  Many definitive and
interesting results are obtained during the course of these
investigations.  For the most part, these more detailed results rely
on relations to compact, or even finite-rank, operators.  This
direction is taken in the seminal article \cite{KS}, as well.  In Section
3.2 of \cite{KS}, a detailed and complete classification is given for an
important class of (maximal) triangular algebras; but much depends
on the analysis of those $\TT$ for which (the ``diagonal")
$\TT\cap\TT^*$ is generated by one-dimensional projections.  On the
other hand, the emphasis of C*- and von Neumann algebra theory is on
those algebras where compact operators are (almost) absent.

One of our main goals in this article is to recapture the synergy
that should exist between the powerful techniques that have
developed in selfadjoint-operator-algebra theory and those of the
non-selfadjoint theory by conjoining the two theories.  We do this
by embodying those theories in a single class of algebras.  For
this, we mimic the defining relation for the triangular algebra
removing the commutativity assumption on the diagonal subalgebra
$\TT\cap\TT^*$ of $\TT$ and imposing suitable maximality and
reflexivity conditions on $\TT$ (compare Definition 2.1).  Our
particular focus is the case where the diagonal algebra is a factor.


The new non self-adjoint operator algebras will combine triangularity,
reflexivity and von Neumann algebra properties in their structure.
These algebras will be called {\it Kadison-Singer algebras} or {\it
KS-algebras} for simplicity. They are reflexive and maximal
triangular with respect to their ``diagonal subalgebras.''
Kadison-Singer factors (or KS-factors) are those with factors as
their diagonal algebras. These are highly noncommutative and non
selfadjoint operator algebras. In standard form, where the diagonal
and its commutant share a cyclic vector, such ``standard''
Kadison-Singer algebras have a large selfadjoint part. Many
selfadjoint features are preserved in them and concepts can be
borrowed directly from the theory of von Neumann algebras. In fact,
a more direct connection of Kadison-Singer algebras and von Neumann
algebras is through the lattice of invariant projections of a
KS-algebra. The lattice is reflexive and ``minimally generating'' in
the sense that it generates the commutant of the diagonal as a von
Neumann algebra. Most factors are generated by three projections
(see \cite{GS}). One of our main results shows that the reflexive algebra
which leaves three generating projections of a factor invariant is
often a Kadison-Singer algebra, which agrees with the fact that
three projections are ``minimally generating'' for a factor.
Moreover Kadison-Singer algebras associated with three projections
contain compact operators, and the reflexive lattice generated by
the three projections is often homeomorphic to the two-dimensional
sphere. We believe that the reflexive algebra given by four or more
free projections is a Kadison-Singer algebra and does not contain
any nonzero compact operators. Indeed, we shall show that the
reflexive algebra associated with infinitely many free projections
contains no nonzero compact operators. Through the study of minimal
reflexive lattice generators of a von Neumann algebra, we may better
understand the generator problem for von Neumann algebras and hence
the isomorphism problem for free group factors. These are some of
the deepest, most diffcult, and longest standing problems in von
Neumann algebra theory. The techniques we use are closely related to
those of the theory of selfadjoint operator algebras, especially
some of the recently developed theory of free probability \cite{VDN}. For
some of the important results and approaches in non selfadjoint
theory, we refer to \cite{R}, \cite{Ar}, \cite{L}, \cite{La2}, \cite{MSS}, 
\cite{DKP} and many
references in \cite{Da} and \cite{RR}.

There are four sections in this paper, the first in a series. In
Section 2, we give the definition of Kadison-Singer algebras (as
well as corresponding Kadison-Singer lattices) and a basic
classification according to their diagonals. In Section 3, we
construct Kadison-Singer factors with hyperfinite factors as their
diagonals. In Section 4, we describe the corresponding
Kadison-Singer lattices in detail. A new lattice invariant is
introduced to distinguish these lattices.

We hope that new examples and constructions of non selfadjoint
algebras will lead to new insights for some puzzling, old questions
in operator theory (see \cite{Ka} and \cite{RR}).