\section{Kadison-Singer lattices}

\vskip10pt

\noindent It is hard to determine whether a given lattice is a
Kadison-Singer lattice. The only known class is the family of nests
\cite{KS}. Some finite distributive lattices (see \cite{H2} and \cite{Ha}) are
Kadison-Singer lattices if they have a minimal generating property.
In this section, we will show that the strong-operator closure of
$\LLL_\infty$ defined in Section 3 is a Kadison-Singer lattice. For
simplicity of description, we shall assume that the state $\rho$ on
$\AAA_n$ is a trace, now denoted by $\tau$. Let $\RRR$ be the
hyperfinite II$_1$ factor generated by $\LLL_\infty$ (or $\AAA_n$). The
commutant $\RRR'$ of $\RRR$ is the diagonal subalgebra of
$\Alg(\LLL_\infty)$.

\begin{theorem}
Let
$\LLL^{(n)}$ be the strong-operator closure of $\LLL_\infty$, $\HHH$ the
Hilbert space obtained by GNS construction on $(\AAA_n, \tau)$. Then
we have that $\LLL^{(n)}=\Lat(\Alg(\LLL_\infty))$. For any
$r\in(0,1)$, if there are $a,l\in\N$ such that $r=\frac{a}{n^l}$,
then there are two distinct projections in $\LLL^{(n)}$ with trace
value $r$; otherwise there is only one projection in $\LLL^{(n)}$
with trace $r$. 
\end{theorem}

To understand the lattice structure of $\LLL^{(n)}$, we first
analyze the lattice properties of $\LLL_m$. From the definition of
$P_{1j}$, $j=1,\ldots,n$, the generators of $\LLL_1$, we know that
$\LLL_1$ consists of a nest $\{0, P_{11},\ldots, P_{1,n-1}, I\}$ in
$\NNN_1$ $(\cong M_n(\C))$ on the diagonal and a minimal
projection $P_{1n}$. It is easy to see that $P_{1n}\wedge
P_{1j}=0$ for $1\leq j\le n-1$, and their unions give rise to
another nest $\{0, P_{1n}, P_{1n}\vee P_{11}, \ldots, P_{1n}\vee
P_{1,n-1}=I\}$ in $\NNN_1$. The lattice $\LLL_1$ is the union of
these two nests. For any $1\leq k\leq n-1$, there are two distinct
projections in $\LLL_1$ such that they have the same trace $\frac{
k}{n}$. This pattern of double nests appears in $\LLL_2$ between any
two trace values $\frac{k}{n}$ and $\frac{k+1}{n}$, $0\leq k\leq n-1$. To
describe all these projections, we need more notation. For
$k=1,2,\ldots$, define
\begin{align*}
E_{i}^{(k)} &= \sum_{l=1}^{i} E_{ll}^{(k)}  \quad i=1,\ldots,
  n;\quad \\
F_{i}^{(k)} &= \frac{1}{i}\sum_{n\ge l,m > n-i} E_{lm}^{(k)}, \quad
i =1,\ldots,n.
\end{align*}
Since $E_{ij}^{(k)}$ and $E_{i'j'}^{(k')}$ are tensorial relations
for $k\neq k'$, we have $E_{i}^{(k)}$ and $F_{i}^{(k)}$ are
projections in $\NNN_{k-1}' \cap \NNN_k\ ( \NNN_{0} = \C I )$.
Also $E_{i}^{(k)}F_{j}^{(k)} = 0$ when $j \leq n-i$. We shall use
$\tau$ to denote the unique trace on both $\RRR$ and $\RRR'$,
$\tau(E_{i}^{(k)}) = \frac{i}{n}$, $\tau(F_{i}^{(k)}) =
\frac{1}{n}$.

\vskip6pt

\begin{lemma}
Suppose $P \in
\Lat(\Alg(\LLL_\infty))$ and $P \neq 0, I$. Then $P = E_{i}^{(1)} +
F_{n-i}^{(1)}Q$, for some  $i \in \{0,1,\ldots, n-1 \}$, $Q \in
{\NNN}_{1}'$, and also $Q \in \Lat(\NNN_{1}' \cap
\Alg(\LLL_\infty))$. When $P$ is given in this form,
$P\in\Lat(\Alg(\LLL_\infty))$.
\end{lemma}

\vskip6pt

\begin{proof}
First we show that if $P =
E_{i}^{(1)} + F_{n-i}^{(1)}Q$ with $Q$ described in the lemma,
then $P \in \Lat(\Alg(\LLL_\infty))$.

For any $T\in\Alg(\LLL_\infty)$, let $T=T_1+T_2$, given by (3) in
the proof of Lemma 3. Then it is easy to check that $T_{1} \in
\Alg(\LLL_\infty)$, $(I - P_{1,n-1})T_{1} = 0$ and $T_{2} \in
\NNN_1' \cap \Alg(\LLL_\infty)$. Since
$E_{i}^{(1)}=P_{1i}\in\LLL_1\subseteq\LLL_\infty$, we have $(I -
E_{i}^{(1)})T_jE_{i}^{(1)} = 0$, for $j\in\{1,2\}$. One can check
directly that $F_{n-i}^{(1)}(I-E_{i}^{(1)}) =
F_{n-i}^{(1)}=(I-E_i^{(1)})F_{n-i}^{(1)}$. Thus
$F_{n-i}^{(1)}TE_i^{(1)}=0$. Since $Q$ commutes with $\NNN_1$, we
have
\begin{align*}
(I-P)TP &= (I - E_{i}^{(1)} - F_{n-i}^{(1)}Q)T(E_{i}^{(1)} +
F_{n-i}^{(1)}Q)
 \cr &= (I-E_{i}^{(1)})TF_{n-i}^{(1)}Q -
F_{n-i}^{(1)}QTE_{i}^{(1)}  \cr & \qquad- F_{n-i}^{(1)}QTF_{n-i}^{(1)}Q
\cr
 & =(I-E_{i}^{(1)})TF_{n-i}^{(1)}Q -
        QF_{n-i}^{(1)}(I-E_{i}^{(1)})TF_{n-i}^{(1)}Q.
\end{align*}
 The above equations hold when $T$ is replaced by $T_1$ or $T_2$. From
our assumptions that $Q \in \Lat(\NNN_{1}' \cap
\Alg(\LLL_\infty))$, $E_i^{(1)}, F_{n-i}^{(1)}\in\NNN_1$ and $T_2\in
\NNN_1'\cap\Alg(\LLL_\infty)$, we have
\begin{align*}
(I-P)T_{2}P
&=((I-E_i^{(1)})F_{n-i}^{(1)}-QF_{n-i}^{(1)}(I-E_i^{(1)}))T_2Q \\
&=
F_{n-1}^{(1)}(I - Q)T_{2}Q = 0.
\end{align*}
Next we show that
$(I-E_i^{(1)})T_1F_{n-i}^{(1)} = 0$, which implies that
$(I-P)T_1P=0$. Note that
\begin{align*}
(n-i)(I-E_{i}^{(1)})T_1 F_{n-i}^{(1)} &= (\sum_{j= i+1}^{n}
E_{jj}^{(1)})T_{1}(\sum_{l,m = i+1}^{n} E_{lm}^{(1)})\\
&=\sum_{j,m =
i+1}^{n}\sum_{l = j}^{n} E_{jj}^{(1)}T_{1}E_{lm}^{(1)}.
\end{align*}
By Lemma 3 and $(I - P_{n-1}^{(1)})T_{1} = E_{nn}^{(1)}T_{1} =
0$, we have $\sum_{l= j}^{n} E_{jj}^{(1)}T_{1}E_{lm}^{(1)} = 0$.
So $(n-i)(I-E_{i}^{(1)})T_{1}F_{n-i}^{(1)} = 0$. Thus $(I-P)TP=0$
which implies that $P \in \Lat(\Alg(\LLL_\infty))$.

Now for any $P \in \Lat(\Alg(\LLL_\infty))$, let $i_0$, $1\le i_0\le
n$, be the smallest integer such that
$E_{i_{0}i_{0}}^{(1)}PE_{i_{0}i_{0}}^{(1)} \neq E_{i_{0}
i_{0}}^{(1)}$. Then $E_{i i}^{(1)}PE_{ii}^{(1)} = E_{ii}^{(1)}$
for $1\leq i \leq i_{0}-1$ and $P = E_{i_{0}-1}^{(1)} + P_{1}$,
where $P_{1}$ is a projection and $E_{i}^{(1)}P_{1} = 0$ for $i\leq
i_0-1$. First we assume that $i_{0} \leq n-1$. For any $A \in
\B(\HHH)$ and $i_1\geq i_0+1$, define $A_{i_{1}} = E_{i_{0}
i_{0}}^{(1)}A(E_{i_{0} i_{0}}^{(1)} - E_{i_{0}i_{1}}^{(1)})$. Then
$A_{i_1} \in \Alg(\LLL_\infty)$. Since $P \in
\Lat(\Alg(\LLL_\infty))$, we have
\begin{align*}
 0 &= (I - E_{i_{0}-1}^{(1)} - P_{1})A_{i_{1}}(E_{i_{0}-1}^{(1)} +
 P_{1})\\
   &= (I - P_{1})E_{i_{0}i_{0}}^{(1)}A(E_{i_{0}i_{0}}^{(1)} -
E_{i_{0} i_{1}}^{(1)})P_{1}.
\end{align*}
From
$E_{i_{0}i_{0}}^{(1)}(I - P_{1})E_{i_{0} i_{0}}^{(1)} \neq
0$, the above equation implies that $E_{i_{0} i_{0}}^{(1)}P_{1} =
E_{i_{0}i_{1}}^{(1)}P_{1}$, for all $i_1\ge i_0$. So, multiplying
by $P_1E_{i_0i_0}^{(1)}=P_1E_{ji_0}^{(1)}$ (the adjoint of the
above equation) on the right hand side, we have $E_{i_{0}
i_{0}}^{(1)}P_{1}E_{i_{0}i_{0}}^{(1)} = E_{i_{0}
i_1}^{(1)}P_{1}E_{ji_{0}}^{(1)}$ for all $i_1, j \geq i_{0}$. This
implies that $P_{1} = F_{n - i_{0}+1}^{(1)}Q$, where $Q$ is a
projection in $\NNN_{1}'$. If $i_{0} = n$, then  $P_{1}$ can be
written as $F_{1}^{(1)}Q$ for $Q \in \NNN_{1}'$. From $P \in
\Lat(\Alg(\LLL_\infty))$, it is easy to see that $Q \in
\Lat(\NNN_{1}' \cap \Alg(\LLL_\infty))$.
\end{proof}

\begin{lemma}
Suppose $P \in
\Lat(\Alg(\LLL_\infty))$. Then there exist $Q \in \NNN_{k}' \cap
\Lat({\NNN}_{k}' \cap \Alg(\LLL_\infty))$ and integers $a_k$ such
that
\begin{align*}
P = E_{a_{1}}^{(1)} + F_{n-a_{1}}^{(1)}E_{a_{2}}^{(2)} + \cdots
+(\prod_{i = 1}^{k-1}F_{n-a_{i}}^{(i)})E_{a_{k}}^{(k)} + (\prod_{i
= 1}^{k}F_{n-a_{i}}^{(i)})Q,
\end{align*}
where $0 \leq a_{i} \leq n-1$ and $\tau(P) =
\sum_{i=1}^{k}\frac{a_{i}}{n^{i}} + \frac{\tau(Q)}{n^{k}}$ which
is a real number lies in the closed interval
$\left[\sum_{i=1}^{k}\frac{a_{i}}{n^{i}},\
\sum_{i=1}^{k}\frac{a_{i}}{n^{i}} +
\frac{1}{n^{k}}\right]\subseteq[0,1]$.
\end{lemma}

The above lemma follows easily from induction. The details are
similar to the proof of Lemma 8.

\begin{proof}[Proof of Theorem 5]
To describe an
arbitrary projection $P$ in $\Lat(\Alg(\LLL_\infty))$ in more
details, we need to know the trace value of $P$. First when
$\tau(P) = \sum_{i=1}^{k}\frac{a_{i}}{n^{i}}$, where $0\leq a_i\leq
n-1$ and $a_{k} \neq 0$, then there are two cases, either
\begin{align*}
P &= \sum_{j = 1}^{k} (\prod_{i =
1}^{j-1}F_{n-a_{i}}^{(i)})E_{a_{j}}^{(j)} ,\qquad  ({\rm let}\
\prod_{i = 1}^{0}F_{n-a_{i}}^{(i)} = I)\quad {\rm or}\cr
 P & = \sum_{j
= 1}^{k-1} (\prod_{i = 1}^{j-1}F_{n-a_{i}}^{(i)})E_{a_{j}}^{(j)} +
(\prod_{i = 1}^{k-1}F_{n-a_{i}}^{(i)})E_{a_{k}-1}^{(k)} \\
& \qquad \qquad +
(\prod_{i = 1}^{k-1}F_{n-a_{i}}^{(i)})F_{n-a_{k}+1}^{(k)}.
\end{align*}
Note that the above two projections correspond to the case when
$Q=0$ for the decomposition
$\tau(P)=\sum_{i=1}^{k}\frac{a_{i}}{n^{i}}$, or respectively $Q=I$
for $\tau(P)=\sum_{i=1}^{k-1}
\frac{a_{i}}{n^{i}}+\frac{a_k-1}{n^k}+ \frac1{n^k}$ in Lemma 9.
So $P \in \Lat(\Alg(\LLL_\infty))$. Thus for any $r =
\frac{a}{n^{l}}$ for some integer $l
> 0$ and any integer $a$ such that $0 < a < n^{l}$, there are
exactly two projections in $\Lat(\Alg(\LLL_\infty))$ with trace $r$.

Secondly, when $r \in (0,1)$ and $r\neq \frac{a}{n^{l}}$ for any
positive integer $l$ and any integer $a$ with $0 < a < n^{l}$, we
shall show that there is a unique $P$ in $\Lat(\Alg(\LLL_\infty))$
with trace $r$. For the given $r$, there is a unique expansion $r
= \sum_{k=1}^{\infty}\frac{a_{k}}{n^{k}}$, where $a_k$ is an
integer with $0 \leq a_{k} \leq n-1$, there are infinitely many
non zero $a_k$'s and infinitely many $a_k\neq n-1$. (This is
because repeating $n-1$ as coefficients from certain place on will
result $r$ being $\frac{a}{n^l}$, e.g., $0.09999\cdots=0.1$ when
$n=10$.) In fact, Lemma 9 gives the existence and uniqueness of
such a projection:
\begin{align*}
P=E_{a_{1}}^{(1)} + F_{n-a_{1}}^{(1)}E_{a_{2}}^{(2)} +
         F_{n-a_{1}}^{(1)}F_{n-a_{2}}^{(2)}E_{a_{3}}^{(3)} + \cdots.
\end{align*}

It is not hard to see that $P$ is the strong-operator limit of
finite sums. The finite sums
\begin{align*}
Q_{k} =  \sum_{j = 1}^{k} (\prod_{i =
1}^{j-1}F_{n-a_{i}}^{(i)})E_{a_{j}}^{(j)}\in\LLL_\infty ,  \quad  k
= 1 , 2,\ldots,
\end{align*}
 $Q_{1} < Q_{2} < \cdots < Q_{k} < \cdots<P$ and $\lim_{k
\to\infty} \tau(Q_{k}) = r=\tau(P)$.
\end{proof}

The following theorem gives us infinitely many non isomorphic
Kadison-Singer lattices.

\begin{theorem}
For $n\neq k$,
$\LLL^{(n)}$ and $\LLL^{(k)}$ are not algebraically isomorphic as
lattices.
\end{theorem}

\begin{proof}
For any $n\geq 2$, first we observe
from Lemma 9 that if $E \in \LLL^{(n)}$ is a minimal projection,
then $E= (\prod_{i = 1}^{m}F_{n}^{(i)})E_{1}^{(m+1)}$ for $m =0,
1, 2, \ldots$.

Suppose that $P \in \LLL^{(n)}$ is given as in Lemma 9,
\begin{align*}
P = &E_{a_{1}}^{(1)} + F_{n-a_{1}}^{(1)}E_{a_{2}}^{(2)} + \cdots
+(\prod_{i = 1}^{m}F_{n-a_{i}}^{(i)})E_{a_{m+1}}^{(m+1)} \\
& \qquad \qquad+ 
(\prod_{i = 1}^{m}F_{n-a_{i}}^{(i)})F_{n-a_{m+1}}^{(m+1)}Q,
\end{align*}
where $0 \leq a_{i} \leq n-1$ and $Q \in \NNN_{m+1}' \cap
\Lat({\NNN}_{m+1}' \cap \Alg(\LLL^{(n)}))$. We shall show that $P
\wedge (\prod_{i = 1}^{m}F_{n}^{(i)})E_{1}^{(m+1)} = 0$ for some
$m\geq 0$ if and only if $a_{m+1} = 0$.

First if $a_{m+1} > 0$, it is easy to check that the minimal
projection $(\prod_{i = 1}^{m}F_{n}^{(i)})E_{1}^{(m+1)} \leq P$.
Conversely, if $a_{m+1} = 0$, then
\begin{align*}
P = &E_{a_{1}}^{(1)} + F_{n-a_{1}}^{(1)}E_{a_{2}}^{(2)} + \cdots +
(\prod_{i=1}^{m-1}F_{n-a_{i}}^{(i)})E_{a_{m}}^{(m)}\\
& \qquad \qquad +(\prod_{i=1}^{m}F_{n-a_{i}}^{(i)})F_{n}^{(m+1)}Q.
\end{align*}
Let $\xi \in P(\HHH) \wedge (\prod_{i =
1}^{m}F_{n}^{(i)})E_{1}^{(m+1)}(\HHH)$, and $E =
(I-E_{n-1}^{(1)})(I-E_{n-1}^{(2)})\cdots(I-E_{n-1}^{(m+1)})$. We
have $E\xi = E(\prod_{i = 1}^{m}F_{n}^{(i)})E_{1}^{(m+1)}\xi = 0$.
But
\begin{align*}
0 &= PEP\xi =(\prod_{i=1}^{m}F_{n-a_{i}}^{(i)})F_{n}^{(m+1)}
E(\prod_{i=1}^{m}F_{n-a_{i}}^{(i)})F_{n}^{(m+1)}Q\xi\cr
&=(\prod_{i=1}^{m}F_{n-a_{i}}^{(i)}(I-E_{n-1}^{(i)})
F_{n-a_{i}}^{(i)})F_{n}^{(m+1)}(I-E_{n-1}^{m+1})
F_{n}^{(m+1)}Q\xi\cr
&=\frac{1}{n}(\prod_{i=1}^{m}\frac{1}{n-a_{i}})
(\prod_{i=1}^{m}F_{n-a_{i}}^{(i)})F_{n}^{(m+1)}Q\xi.
\end{align*}
This shows that $\xi \in (\prod_{i =
1}^{m}F_{n}^{(i)})E_{1}^{(m+1)}(\HHH) \wedge P_{1}(\HHH)$, here $P_{1}
= P - (\prod_{i=1}^{m}F_{n-a_{i}}^{(i)})F_{n}^{(m+1)}Q$. Let
$\widetilde{E} =
(I-E_{n-1}^{(1)})(I-E_{n-1}^{(2)})\cdots(I-E_{n-1}^{(m)})$. Then
$\widetilde{E}\xi = \widetilde{E}P_{1}\xi = 0$ and
\begin{align*}
0 &= (\prod_{i =
1}^{m}F_{n}^{(i)})E_{1}^{(m+1)}\widetilde{E}(\prod_{i =
1}^{m}F_{n}^{(i)})E_{1}^{(m+1)}\xi \\
&=(\prod_{i =
1}^{m}F_{n}^{(i)}(I-E_{n}^{(i)})F_{n}^{(i)})E_{1}^{(m+1)}\xi\cr
  &= \frac{1}{n^{m}}(\prod_{i = 1}^{m}F_{n}^{(i)})E_{1}^{(m+1)}\xi
  = \frac{1}{n^{m}}\xi.
\end{align*}
This shows that $\xi = 0$ and thus $P \wedge (\prod_{i =
1}^{m}F_{n}^{(i)})E_{1}^{(m+1)} = 0$.

For any $\SSS \subset \LLL^{(n)}$, we define
\begin{align*}
\ZZZ(\SSS) = \{P \in
\LLL^{(n)} :\  P \wedge Q = 0, \ {\rm for\ all\ } Q \in \SSS \}.
\end{align*}
From
the above, we know that, for any minimal projection $(\prod_{i =
1}^{m}F_{n}^{(i)}) E_{1}^{(m+1)}$,
\begin{align*}
\ZZZ(\{(\prod_{i = 1}^{m}F_{n}^{(i)})& E_{1}^{(m+1)} \})\cr =
\{E_{a_{1}}^{(1)} &+ F_{n-a_{1}}^{(1)}E_{a_{2}}^{(2)} + \cdots +
(\prod_{i=1}^{m-1}F_{n-a_{i}}^{(i)})E_{a_{m}}^{(m)}\\
& \qquad \qquad +(\prod_{i=1}^{m}F_{n-a_{i}}^{(i)})F_{n}^{(m+1)}Q: \cr &  0 \leq
a_{i} \leq n-1, Q \in {\NNN}_{m+1}' \cap Lat({\NNN}_{m+1}' \cap
\Alg(\LLL^{(n)}))\}.
\end{align*}
Now it is not hard to show that $\ZZZ(\ZZZ(\{(\prod_{i =
1}^{m}F_{n}^{(i)})E_{1}^{(m+1)} \})) = \{ (\prod_{i =
1}^{m}F_{n}^{(i)})E_{k}^{(m+1)}:\  k = 0, 1, \ldots, n-1 \}$. The
number of elements in this set is an invariant of
$\LLL^{(n)}$.
\end{proof}


















