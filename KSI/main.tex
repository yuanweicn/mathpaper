%% PNAStmpl.tex
%% Template file to use for PNAS articles prepared in LaTeX
%% Version: Apr 14, 2008


%%%%%%%%%%%%%%%%%%%%%%%%%%%%%%
%% BASIC CLASS FILE 
%% PNAStwo for two column articles is called by default.
%% Uncomment PNASone for single column articles. One column class
%% and style files are available upon request from pnas@nas.edu.
%% (uncomment means get rid of the '%' in front of the command)
%\documentclass{pnasone}
\documentclass{pnastwo}

%%%%%%%%%%%%%%%%%%%%%%%%%%%%%%
%% Changing position of text on physical page:
%% Since not all printers position
%% the printed page in the same place on the physical page,
%% you can change the position yourself here, if you need to:

% \advance\voffset -.5in % Minus dimension will raise the printed page on the 
                         %  physical page; positive dimension will lower it.

%% You may set the dimension to the size that you need.

%%%%%%%%%%%%%%%%%%%%%%%%%%%%%%
%% OPTIONAL GRAPHICS STYLE FILE

%% Requires graphics style file (graphicx.sty), used for inserting
%% .eps files into LaTeX articles.
%% Note that inclusion of .eps files is for your reference only;
%% when submitting to PNAS please submit figures separately.

%% Type into the square brackets the name of the driver program 
%% that you are using. If you don't know, try dvips, which is the
%% most common PC driver, or textures for the Mac. These are the options:

% [dvips], [xdvi], [dvipdf], [dvipdfm], [dvipdfmx], [pdftex], [dvipsone],
% [dviwindo], [emtex], [dviwin], [pctexps], [pctexwin], [pctexhp], [pctex32],
% [truetex], [tcidvi], [vtex], [oztex], [textures], [xetex]

\usepackage[pdftex]{graphicx}

%%%%%%%%%%%%%%%%%%%%%%%%%%%%%%
%% OPTIONAL POSTSCRIPT FONT FILES

%% PostScript font files: You may need to edit the PNASoneF.sty
%% or PNAStwoF.sty file to make the font names match those on your system. 
%% Alternatively, you can leave the font style file commands commented out
%% and typeset your article using the default Computer Modern 
%% fonts (recommended). If accepted, your article will be typeset
%% at PNAS using PostScript fonts.


% Choose PNASoneF for one column; PNAStwoF for two column:
%\usepackage{PNASoneF}
%\usepackage{PNAStwoF}

%%%%%%%%%%%%%%%%%%%%%%%%%%%%%%
%% ADDITIONAL OPTIONAL STYLE FILES

%% The AMS math files are commonly used to gain access to useful features
%% like extended math fonts and math commands.

\usepackage{amssymb,amsfonts,amsmath}

%%%%%%%%%%%%%%%%%%%%%%%%%%%%%%
%% OPTIONAL MACRO FILES
%% Insert self-defined macros here.
%% \newcommand definitions are recommended; \def definitions are supported

%\newcommand{\mfrac}[2]{\frac{\displaystyle #1}{\displaystyle #2}}
%\def\s{\sigma}

\newenvironment{proof}[1][Proof]{\begin{trivlist}
\item[\hskip \labelsep {\bfseries #1}]}{\end{trivlist}}

\newcommand{\A}{\mathcal A}
\newcommand{\AAA}{\mathfrak A}
\newcommand{\B}{\mathcal B}
\newcommand{\CCC}{\mathcal C}
\newcommand{\DDD}{\mathcal D}
\newcommand{\F}{\mathcal F}
\newcommand{\G}{\mathcal G}
\newcommand{\HHH}{\mathcal H} %for Hilbert space
\newcommand{\LLL}{\mathcal L} % for lattice
\newcommand{\PPP}{\mathcal P}
\newcommand{\M}{\mathcal M}
\newcommand{\NNN}{\mathcal N} %for nest
\newcommand{\RRR}{\mathcal R}
\newcommand{\SSS}{\mathcal S}
\newcommand{\W}{\mathcal W}
\newcommand{\ZZZ}{\mathcal Z}
\newcommand{\supp}{\mathop{\mathrm supp}}
\newcommand{\TT}{\mathcal T}


\newcommand{\e}[2][]{e^{#1}_{#2}} %for matrix unit \e[upper index]{lower index}

\newcommand{\Lat}{\mathrm Lat}
\newcommand{\Alg}{\mathrm Alg}
\newcommand{\tensor}{\mathop{\bar \otimes}}
\newcommand{\tr}{\tau}

\newcommand{\C}{\mathbb C} %for complex number
\newcommand{\R}{\mathbb R}  %for real number
\newcommand{\Z}{\mathbb Z} %for integer
\newcommand{\N}{\mathbb N} % for nature number


%%%%%%%%%%%%%%%%%%%%%%%%%%%%%%
%% Don't type in anything in the following section:
%%%%%%%%%%%%
%% For PNAS Only:
\contributor{Submitted to Proceedings
of the National Academy of Sciences of the United States of America}
\url{www.pnas.org/cgi/doi/10.1073/pnas.0709640104}
\copyrightyear{2008}
\issuedate{Issue Date}
\volume{Volume}
\issuenumber{Issue Number}
%%%%%%%%%%%%

\begin{document}

%%%%%%%%%%%%%%%%%%%%%%%%%%%%%%


%% For titles, only capitalize the first letter
%% \title{Almost sharp fronts for the surface quasi-geostrophic equation}

\title{Kadison-Singer Algebras, I \\
        -----Hyperfinite Case}


%% Enter authors via the \author command.  
%% Use \affil to define affiliations.
%% (Leave no spaces between author name and \affil command)

%% Note that the \thanks{} command has been disabled in favor of
%% a generic, reserved space for PNAS publication footnotes.

%% \author{<author name>
%% \affil{<number>}{<Institution>}} One number for each institution.
%% The same number should be used for authors that
%% are affiliated with the same institution, after the first time
%% only the number is needed, ie, \affil{number}{text}, \affil{number}{}
%% Then, before last author ...
%% \and
%% \author{<author name>
%% \affil{<number>}{}}

%% For example, assuming Garcia and Sonnery are both affiliated with
%% Universidad de Murcia:
%% \author{Roberta Graff\affil{1}{University of Cambridge, Cambridge,
%% United Kingdom},
%% Javier de Ruiz Garcia\affil{2}{Universidad de Murcia, Bioquimica y Biologia
%% Molecular, Murcia, Spain}, \and Franklin Sonnery\affil{2}{}}

\author{Liming Ge\affil{1}{ University of New Hampshire, Durham,USA} 
\affil{2}{Chinese Academy of Sciences, Beijing, China},
Wei Yuan\affil{2}{}}

\contributor{Submitted to Proceedings of the National Academy of Sciences
of the United States of America}

%% The \maketitle command is necessary to build the title page.
\maketitle

%%%%%%%%%%%%%%%%%%%%%%%%%%%%%%%%%%%%%%%%%%%%%%%%%%%%%%%%%%%%%%%%
\begin{article}

\begin{abstract} 
This is abstract
\end{abstract}


%% When adding keywords, separate each term with a straight line: |
\keywords{von Neumann algebra | Kadison-Singer algebra | Kadison-Singer lattice | Reflexive }

%% Optional for entering abbreviations, separate the abbreviation from
%% its definition with a comma, separate each pair with a semicolon:
%% for example:
%% \abbreviations{SAM, self-assembled monolayer; OTS,
%% octadecyltrichlorosilane}

% \abbreviations{}

%% The first letter of the article should be drop cap: \dropcap{}
%\dropcap{I}n this article we study the evolution of ''almost-sharp'' fronts

%% Enter the text of your article beginning here and ending before
%% \begin{acknowledgements}
%% Section head commands for your reference:
%% \section{}
%% \subsection{}
%% \subsubsection{}

\section{Introduction}

In \cite{KS}, Kadison and Singer initiate the study of
non-self-adjoint algebras of bounded operators on Hilbert spaces.
They introduce a class of algebras they call {\it triangular
operator algebras.}  An algebra $\TT$ is triangular (relative to a
factor $\M$) when $\TT\cap\TT^*$ is a maximal abelian (self-adjoint)
algebra in the factor $\M$.  (See section 2 for definitions.)  When
the factor is the algebra of all $n\times n$ complex matrices, this
condition guarantees that there is a unitary matrix $U$ such that
the mapping $A\to UAU^*$ transforms $\TT$ onto a subalgebra of the
upper triangular matrices.

Beginning with \cite{KS}, the theory of non-self-adjoint operator
algebras has undergone a vigorous development parallel to, but not
nearly as explosive as, that of the self-adjoint theory, the C*- and
von Neumann algebra theories.  Of course, the self-adjoint theory
began with the 1929-30 von Neumann article \cite{vN} --- well before the
1960 \cite{KS} article appeared.  Surprisingly, to the present authors,
and apparently to Kadison and Singer as well (from private
conversations), this parallel development has not produced the
synergistic interactions we would have expected from subjects that
are so closely and naturally related, and so likely to benefit from
cross connections with one another.

Considerable effort has gone into the study of triangular operator
algebras (see, for example, \cite{MSS}) and \cite{Ho}) and another class of
non-self-adjoint operator algebras, the {\it reflexive algebras\/}
(see, for example, \cite{H2}, \cite{Hd}, \cite{RR}, and \cite{La}).  Many definitive and
interesting results are obtained during the course of these
investigations.  For the most part, these more detailed results rely
on relations to compact, or even finite-rank, operators.  This
direction is taken in the seminal article \cite{KS}, as well.  In Section
3.2 of \cite{KS}, a detailed and complete classification is given for an
important class of (maximal) triangular algebras; but much depends
on the analysis of those $\TT$ for which (the ``diagonal")
$\TT\cap\TT^*$ is generated by one-dimensional projections.  On the
other hand, the emphasis of C*- and von Neumann algebra theory is on
those algebras where compact operators are (almost) absent.

One of our main goals in this article is to recapture the synergy
that should exist between the powerful techniques that have
developed in selfadjoint-operator-algebra theory and those of the
non-selfadjoint theory by conjoining the two theories.  We do this
by embodying those theories in a single class of algebras.  For
this, we mimic the defining relation for the triangular algebra
removing the commutativity assumption on the diagonal subalgebra
$\TT\cap\TT^*$ of $\TT$ and imposing suitable maximality and
reflexivity conditions on $\TT$ (compare Definition 2.1).  Our
particular focus is the case where the diagonal algebra is a factor.


The new non self-adjoint operator algebras will combine triangularity,
reflexivity and von Neumann algebra properties in their structure.
These algebras will be called {\it Kadison-Singer algebras} or {\it
KS-algebras} for simplicity. They are reflexive and maximal
triangular with respect to their ``diagonal subalgebras.''
Kadison-Singer factors (or KS-factors) are those with factors as
their diagonal algebras. These are highly noncommutative and non
selfadjoint operator algebras. In standard form, where the diagonal
and its commutant share a cyclic vector, such ``standard''
Kadison-Singer algebras have a large selfadjoint part. Many
selfadjoint features are preserved in them and concepts can be
borrowed directly from the theory of von Neumann algebras. In fact,
a more direct connection of Kadison-Singer algebras and von Neumann
algebras is through the lattice of invariant projections of a
KS-algebra. The lattice is reflexive and ``minimally generating'' in
the sense that it generates the commutant of the diagonal as a von
Neumann algebra. Most factors are generated by three projections
(see \cite{GS}). One of our main results shows that the reflexive algebra
which leaves three generating projections of a factor invariant is
often a Kadison-Singer algebra, which agrees with the fact that
three projections are ``minimally generating'' for a factor.
Moreover Kadison-Singer algebras associated with three projections
contain compact operators, and the reflexive lattice generated by
the three projections is often homeomorphic to the two-dimensional
sphere. We believe that the reflexive algebra given by four or more
free projections is a Kadison-Singer algebra and does not contain
any nonzero compact operators. Indeed, we shall show that the
reflexive algebra associated with infinitely many free projections
contains no nonzero compact operators. Through the study of minimal
reflexive lattice generators of a von Neumann algebra, we may better
understand the generator problem for von Neumann algebras and hence
the isomorphism problem for free group factors. These are some of
the deepest, most diffcult, and longest standing problems in von
Neumann algebra theory. The techniques we use are closely related to
those of the theory of selfadjoint operator algebras, especially
some of the recently developed theory of free probability \cite{VDN}. For
some of the important results and approaches in non selfadjoint
theory, we refer to \cite{R}, \cite{Ar}, \cite{L}, \cite{La2}, \cite{MSS}, 
\cite{DKP} and many
references in \cite{Da} and \cite{RR}.

There are four sections in this paper, the first in a series. In
Section 2, we give the definition of Kadison-Singer algebras (as
well as corresponding Kadison-Singer lattices) and a basic
classification according to their diagonals. In Section 3, we
construct Kadison-Singer factors with hyperfinite factors as their
diagonals. In Section 4, we describe the corresponding
Kadison-Singer lattices in detail. A new lattice invariant is
introduced to distinguish these lattices.

We hope that new examples and constructions of non selfadjoint
algebras will lead to new insights for some puzzling, old questions
in operator theory (see \cite{Ka} and \cite{RR}).
\section{ Definitions}

\noindent For basic theory on operator algebras, we refer to \cite{KR}.
We recall the definitions of some well known classes of non
selfadjoint operator algebras. For details on triangular algebras,
we refer to \cite{KS}. For others, we refer to \cite{RR}.

Suppose $\HHH$ is a separable Hilbert space and $\B(\HHH)$ the algebra
of all bounded linear operators on $\HHH$.  Let $\M$ be a von
Neumann subalgebra of $\B(\HHH)$. A {\it triangular (operator)
algebra} is a subalgebra $\TT$ of $\M$ such that $ \TT \cap \TT^*
= \A$, a maximal abelian selfadjoint subalgebra (masa) of $\M$.
One of the interesting cases is when  $\M=\B(\HHH)$.

Let $\PPP$ be a set of (orthogonal) projections in $\B(\HHH)$. Define
$\Alg(\PPP)=\{ T \in\B(\HHH): TP=PTP, \ {\mathrm for\ all}\ P\in\PPP\}$.
Then $\Alg(\PPP)$ is a weak-operator closed subalgebra of $\B(\HHH)$.
Similarly, for a subset $\mathcal S$ of $\B(\HHH)$, define $\Lat (\mathcal
S)=\{P\in\B(\HHH): P\ {\mathrm a \ projection},\ TP=PTP, \ {\mathrm for\
all}\ T\in\mathcal S\}$. Then $\Lat(\SSS)$ is a strong-operator closed
lattice of projections. A subalgebra $\B$ of $\B(\HHH)$ is called a
{\it reflexive (operator) algebra} if $\B=\Alg(\Lat (\B))$.
Similarly, a lattice $\LLL$ of projections in $\B(\HHH)$ is called a
{\it reflexive lattice (of projections)} if $\LLL=\Lat(\Alg(\LLL))$. A
{\it nest} is a totally ordered reflexive lattice. If $\LLL$ is a
nest, then $\Alg(\LLL)$ is called a {\it nest algebra}. Nest
algebras are generalizations of (hyperreducible) ``maximal
triangular'' algebras introduced by Kadison and Singer in \cite{KS}.
Kadison and Singer also show that nest algebras are the only
maximal triangular reflexive algebras (with a commutative lattice
of invariant projections). Motivated by this, we give the
following definition:

\vskip6pt

\begin{definition}
A subalgebra $\A$ of $\B(\HHH)$
is called a {\sl Kadison-Singer (operator) algebra} (or {\it
KS-algebra}) if $\A$ is reflexive and maximal with respect to the
{\it diagonal subalgebra} $\A\cap \A^*$ of $\A$, in the sense that
if there is another reflexive subalgebra $\frak B$ of $\B(\HHH)$
such that $\A\subseteq\frak B$ and $\frak B\cap\frak B^*=\A\cap
\A^*$, then $\A=\frak B$. When the diagonal of a KS-algebra is a
factor, we call the KS-algebra a {\it KS-factor} or a {\it
Kadison-Singer factor}. A lattice $\LLL$ of projections in $\B(\HHH)$
is called a {\it Kadison-Singer lattice} (or {\sl KS-lattice}) if
$\LLL$ is a minimal reflexive lattice that generates the von Neumann
algebra $\LLL''$, or equivalently $\LLL$ is reflexive and $\Alg(\LLL)$
is a Kadison-Singer algebra.
\end{definition}

\vskip6pt

Clearly nest algebras are KS-algebras. Since a nest generates an
abelian von Neumann algebra, we may view nest algebras as ``type
I'' KS-algebras and general KS-algebras as ``quantized'' nest
algebras. The maximality condition for a KS-algebra requires that
the associated lattice is ``reflexive and minimal'' in the sense
that there is no smaller reflexive sublattice that generates the
commutant of the diagonal algebra. We believe that the following
statement is true:

\begin{conjecture}
If $\A$ is a KS-algebra in $\B(\HHH)$ and $P\in\Lat(\A)$ ($\neq 0,I$), then
$I-P\notin\Lat(\A)$, i.e., a KS-algebra has no nontrivial reducing
invariant subspaces.
\end{conjecture}

The following lemma is an immediate consequence of the above
definition.

\vskip6pt

\begin{lemma}
Suppose $\A$ is a Kadison-Singer algebra in $\B(\HHH)$ and $\M$ is the commutant of $\A\cap\A^*$ in $\B(\HHH)$. Then $\Lat(\A)\subseteq\M$ and generates
$\M$ as a von Neumann algebra.
\end{lemma}

%\noindent{\it Proof.}\HHHskip8pt Since $\A\cap\A^*$ is a von Neumann
%algebra and $\Lat(\A\cap\A^*)\subseteq\M$, we have
%$\Lat(\A)\subseteq\M$. Let $\NNN$ be the von Neumann algebra
%generated by $\Lat(\A)$. Then $\NNN$ is a subalgebra of $\M$,
%which implies that $\M'\subseteq\NNN'$. It is clear that
%$\NNN'\subseteq \Alg(\Lat(\A))=\A$ and is selfadjoint. Thus
%$\NNN'\subseteq \A\cap\A^*=\M'$. Now $\NNN'=\M'$, which implies
%that $\NNN=\M$.
%\endproof

%\vskip6pt

%Suppose $\LLL=\{0,I\}$ is the trivial lattice. Then
%$\A=\Alg(\LLL)=\B(\HHH)$ is a KS-algebra and is also a factor of type
%%%%%%%%%I.
When $\A$ is a KS-algebra and $\A\cap\A^*$ is a factor of type I,
II or III, then $\A$ is called a KS-factor of the same type. In
the same way, we can further classify KS-factors into type II$_1$,
II$_\infty$, etc., similar to usual factors. A KS-algebra $\A$ is
said to be in a {\sl standard form}, or a {\it standard}
KS-algebra, if the diagonal $\A\cap\A^*$ of $\A$ is in a standard
form, i.e., $\A\cap\A^*$ has a cyclic and separating vector in
$\HHH$. In this case, the von Neumann algebra generated by
$\Lat(\A)$ (or the core, see \cite{KS}) is also in a standard form.

In the present article, one of our main goals is to give some
nontrivial examples of KS-algebras, in particular, KS-factors of
type II and III. The following theorem shows that all type II and
type III KS-algebras are truly non selfadjoint algebras.

\begin{theorem}
If $\A$ is a KS-algebra of type II or type III in $\B(\HHH)$, then $\A$ is not
selfadjoint.
\end{theorem}

\begin{proof}
Assume on the contrary that $\A$ is
selfadjoint. From our assumption we know that $\A'$ contains a
$2\times 2$ matrix subalgebra $\M_2$. Let $E_{ij}$, $i,j=1,2$, be
a matrix unit system for $\M_2$. Then one can construct a
reflexive lattice $\LLL$ generated by all projections in the
relative commutant of $\M_2$ in $\A'$ and two non commuting
projections $E_{11}$ and $\frac12\sum_{i,j}E_{ij}$ in $\M_2$. It
is easy to see that $\LLL$ generates $\A'$ as a von Neumann algebra.
One easily checks that $\Alg(\LLL)$ is non selfadjoint but
reflexive. Moreover its diagonal is equal to the commutant of
$\LLL$, which agrees with $\A$. This contradicts to the assumption
that $\A$ is a KS-algebra.
\end{proof}

\vskip6pt

Similar argument shows that any nontrivial standard KS-algebra,
even in the case of type I, is not selfadjoint. Standard
KS-algebras can be viewed as {\it maximal} upper triangular
algebras with a von Neumann algebra as its diagonal. \vskip6pt

\begin{definition}
Two Kadison-Singer
algebras are said to be {\sl isomorphic} if there is a norm
preserving (algebraic) isomorphism between the two algebras. Two
KS-algebras are called {\it unitarily equivalent} if there is a
unitary operator between the underlying Hilbert spaces that
induces an isomorphism between the KS-algebras. 
\end{definition}

It is easy to see that an isomorphism between two Kadison-Singer
algebras induces a * isomorphism between the diagonal subalgebras.

For lattices of projections on a Hilbert space, the definition of
an isomorphism is subtle. We consider a simple example where a
lattice $\LLL_0$ contains two free projections of trace $\frac12$
and $0, I$ in a type II$_1$ factor. As a lattice (with respect to
union, intersection and ordering), it is isomorphic to the lattice
generated by two rank-one projections on a two-dimensional
euclidean space. We shall call such an isomorphism (which
preserves only the lattice structure) an {\it algebraic (lattice)
isomorphism}. An isomorphism between two lattices, in this paper,
is an isomorphism that also induces a
* isomorphism between the von Neumann algebras they generate. To
avoid confusion, sometimes we call such isomorphisms {\it spatial
isomorphisms} between two lattices of projections.
\section{Hyperfinite Kadison-Singer factors}

\noindent In this section, we shall construct some hyperfinite
Kadison-Singer factors. We begin with a UHF C*-algebra $\A_n$ (see
\cite{G}) obtained by taking the completion (with respect to operator
norm) of $\otimes_{1}^\infty M_n(\C)$, denoted by $\AAA$ (or
equivalently, $\AAA=\cup_{j=1}^\infty M_{n^j}(\C)$). We denote by
$M_n^{(k)}(\C)$ the $k$th copy of $M_n(\C)$ in $\AAA$ (or $\A_n$)
and $E_{ij}^{(k)}$, $i,j=1,\ldots,n$, the standard matrix unit
system for $M_n^{(k)}(\C)$, for $k=1,2,\ldots$. Then we may write
$\AAA=M_n^{(1)}(\C)\otimes M_n^{(2)}(\C)\otimes\cdots$.  Let
$\NNN_m=M_n^{(1)}(\C)\otimes M_n^{(2)}(\C)\otimes\cdots\otimes
M_n^{(m)}(\C)$ ($\cong M_{n^m}(\C)$). Then
$\AAA=\cup_{m=1}^\infty\NNN_m$. Now, we construct inductively a
family of projections in $\NNN_m$.

When $m=1$, define $P_{1j}=\sum_{i=1}^j E_{ii}^{(1)}$, for
$j=1,\ldots, n-1$, and $P_{1n}=\frac1n\sum_{s,t=1}^n
E_{st}^{(1)}$. Suppose for $k= m-1$, $P_{kj}\in\NNN_k$ are
defined, for $j=1,\ldots, n$. Now we define
\begin{align}
P_{mj}&=P_{m-1,n-1}+ (I-P_{m-1,n-1})\sum_{i=1}^j E_{ii}^{(m)},
 j=1,\ldots, n-1, \\
P_{mn}&=P_{m-1, n-1} +(I-P_{m-1,n-1})\left(\frac{1}{n} \sum_{s,t=1}^n
E_{st}^{(m)}\right) .
\end{align}
Denote by $\LLL_{m}$ the lattice generated by $\{ P_{kj}: 1\leq k\leq m, 1\leq j \leq n \}$ and $\LLL_\infty=\cup_m\LLL_m$, the lattice generated by $\{P_{kj}: k\geq 1, 1\leq j\leq n\}$. We can easily show
inductively that $\NNN_m$ is generated by $\LLL_m$ (as a
finite-dimensional von Neumann algebra).

Let $\rho_n$ be a faithful state on $M_n(\C)$. We extend $\rho_n$
to a state on $\A_n$, denoted by $\rho$, i.e.,
$\rho=\rho_n\otimes\rho_n\otimes\cdots$. Let $\HHH$ be the Hilbert
space obtained by GNS construction on $(\A_n,\rho)$. It is well
known (see \cite{P}) that the weak-operator closure of $\A_n$ in
$\B(\HHH)$ is a hyperfinite factor $\RRR$ (when $\rho$ is a trace,
the factor $\RRR$ is type II$_1$). Then $\LLL_m$ and $\LLL_\infty$
become lattices of projections in $\RRR$.

\begin{theorem} 
With the
above notation, we have that $\Alg(\LLL_\infty)$ is a Kadison-Singer
factor containing the hyperfinite factor $\RRR'$ as its
diagonal.
\end{theorem}


Our above defined hyperfinite KS-factor depends on $n$ $(\geq 2)$
appeared in the UHF algebra construction. We shall see in Section
4 that, when $\rho$ is a trace, for different $n$, the
Kadison-Singer algebras constructed above are not unitarily
equivalent.

To prove Theorem 3, we need some lemmas.

\begin{lemma} 
With $\LLL_1\subset\NNN_1$ defined above and $E_{ij}^{(1)}$,
$i,j=1,\ldots,n$, the matrix units for $\NNN_1$, we have
\begin{align*}
\Alg(\LLL_1)&=\{ T\in\B(\HHH):  \quad E_{ii}^{(1)}TE_{jj}^{(1)}=0,
\qquad 1\leq j < i\leq n; \\
& \sum_{j=1}^n E_{11}^{(1)} TE_{j1}^{(1)}=
\sum_{j=2}^n E_{12}^{(1)}T
E_{j1}^{(1)}=\cdots=E_{1n}^{(1)}TE_{n1}^{(1)}\}.
\end{align*}
\end{lemma}

\begin{proof}
 Let $T$ be an element in
$\Alg(\LLL_1)$. Since $P_{1j}=\sum_{i=1}^j E_{ii}^{(1)}\in\LLL_1$ for
$j=1,\ldots,n-1$, we know that $E_{ii}^{(1)}TE_{jj}^{(1)}=0, 1\leq
j<i\leq n$. From $TP_{1n}^{(1)}=P_{1n}^{(1)}TP_{1n}^{(1)}$, we have
\begin{align*}
nT\sum_{i,j=1}^n E_{ij}^{(1)}=(\sum_{i,j=1}^n E_{ij}^{(1)} )T(
\sum_{i,j=1}^n E_{ij}^{(1)}) .
\end{align*}
Multiplying the above equation by $E_{1l}^{(1)}$ on left and
$E_{11}^{(1)}$ on right, we have
\begin{align*}
nE_{1l}^{(1)}T\sum_{i=1}^nE_{i1}^{(1)}&=n\sum_{i=1}^n E_{1l}^{(1)}
TE_{i1}^{(1)}\\
&=(\sum_{j=1}^nE_{1j}^{(1)})T (\sum_{i=1}^n
E_{i1}^{(1)})=\sum_{i,j=1}^n E_{1i}^{(1)} T E_{j1}^{(1)} .
\end{align*}
The right hand side is independent of $l$. By letting $l=1,\ldots,
n$ and applying $E_{ii}^{(1)}TE_{jj}^{(1)}=0$ when $1\leq j<i\leq
n$, we have that $\sum_{i=1}^n E_{11}^{(1)} TE_{i1}^{(1)}=
\sum_{i=2}^n E_{12}^{(1)}T
E_{i1}^{(1)}=\cdots=E_{1n}^{(1)}TE_{n1}^{(1)}$. It is easy to
check that when $T$ satisfies those identities in the lemma, $T$
must be an element in $\Alg(\LLL_1)$.
\end{proof}

In terms of matrix representations of elements in $\Alg(\LLL_1)$
with respect to matrix units in $\NNN_1$, we know from Lemma 2
that such an element $T$ is upper triangular. Moreover, one can
arbitrarily choose the strictly upper triangular part of $T$ and
use equations
\begin{align*}
\sum_{j=1}^n E_{11}^{(1)} TE_{j1}^{(1)}=
\sum_{j=2}^n E_{12}^{(1)}T
E_{j1}^{(1)}=\cdots=E_{1n}^{(1)}TE_{n1}^{(1)}
\end{align*}
to determine the diagonal entries of $T$ so that $T\in\Alg(\LLL_1)$. 
%%Notice that $\Alg(\L_\infty)=\cap_{j=1}^\infty\Alg(\L_j)$. Thus
%%our goal is to understand all elements in each $\Alg(\L_j)$.

\begin{lemma} For any $T$ in
$\Alg(\LLL_1)$, there are $T_1$ in $\Alg(\LLL_1)\cap \LLL_1'$ and $T_2$
in $\Alg(\LLL_\infty)$ $(\subseteq \Alg(\LLL_1))$ such that
$T=T_1+T_2$. In particular, when $E_{nn}^{(1)}TE_{nn}^{(1)}=0$,
$T=T_2\in\Alg(\LLL_\infty)$.
\end{lemma}

\begin{proof}
Suppose $T\in\Alg(\LLL_1)$ and let
\begin{align}
T_1=\sum_{i=1}^n E_{in}^{(1)}T E_{ni}^{(1)},\qquad T_2=T-T_1.
\end{align}
It is easy to check that $E_{ii}^{(1)}T_1 E_{jj}^{(1)}=0$ when
$i\neq j$ and, by Lemma 2, $T_1\in\Alg(\LLL_1)$. Moreover, for all
$l,k$, $ E_{lk}^{(1)}T_1=E_{lk}^{(1)}\sum_{i=1}^nE_{in}^{(1)}T
E_{ni}^{(1)} =E_{ln}^{(1)}T E_{nk}^{(1)}=T_1E_{lk}^{(1)}$. This
implies that $T_1\in\LLL_1'$ ($=\NNN_1'$).

Clearly $T_2\in\Alg(\LLL_1)$. Thus $T_2P_{1k}=P_{1k}T_2P_{1k}$, for
$k=1,\ldots, n$. We need to show that $T_2P_{jk}=P_{jk}T_2P_{jk}$,
for $j\geq 2$ and $k=1,\ldots, n$. By the definition of $P_{jk}$ in
(1), we know that $I-P_{jk}\leq E_{nn}^{(1)}$ for $j\geq 2$.  Now,
from $T_2\in\Alg(\LLL_1)$, we have
\begin{align*}
E_{nn}^{(1)}T_2&=E_{nn}^{(1)}\sum_{1\le l\le k\le n} E_{ll}^{(1)}
T_2 E_{kk}^{(1)}\\
&=E_{nn}^{(1)}T_2E_{nn}^{(1)}
=E_{nn}^{(1)}(T-T_1)E_{nn}^{(1)}=0.
\end{align*}
This implies that $0=(I-P_{jk})T_2=(I-P_{jk})T_2P_{jk}$. Thus we
have $T_2\in\Alg(\LLL_\infty)$.
\end{proof}

\begin{lemma}
If $T\in\Alg(\LLL_m)$ and
$(I-P_{m,n-1})T=0$ for some $m\geq1$, then $T\in\Alg(\LLL_\infty)$.
\end{lemma}

When $m=1$, the proof is given above. For a general $m$, the
argument is similar. We omit its details here. From the
construction of $P_{mk}$'s, we know that the differences between
elements in $\Alg(\LLL_m)$ and those in $\Alg(\LLL_{m+1})$ only occur
within $I-P_{m,n-1}$ $(=E_{nn}^{(1)}\otimes\cdots\otimes
E_{nn}^{(m)})$. Thus we have the following lemma.

\begin{lemma}
If $T\in\Alg(\LLL_m)$, then
$T\in\Alg(\LLL_{m+1})$ if and only if, for $j=1,\ldots,n$, the
projections $(I-P_{m,n-1})P_{m+1,j}(I-P_{m,n-1})$ are invariant
under $(I-P_{m,n-1})T(I-P_{m,n-1})$.
\end{lemma}

Inductively, we can easily prove the following lemma which
generalizes Lemma 3.

\begin{lemma}
If $T\in\Alg(\LLL_m)$, then
there are $T_1,\ldots, T_{m+1}$ in $\Alg(\LLL_m)$ such that
$T=T_1+\cdots +T_{m+1}$, where
$T_i\in\NNN_{i-1}'\cap\Alg(\LLL_\infty)$, $(I-P_{i,n-1})T_i=0$ for
$i=1,\ldots, m$ (here we let $\NNN_0=\C I$), and
$T_{m+1}\in\NNN_m'\cap\Alg(\LLL_m)$.
\end{lemma}

%%The following lemma is the key to prove the maximality of
%%$\Alg(\LLL_\infty)$.
%%
%%%\vskip6pt

\begin{lemma}
Suppose $T$ is an element
in $\B(\HHH)$ and $\AAA$ is the algebra generated by $T$ and
$\Alg(\LLL_\infty)$. If $\AAA\cap\AAA^*=\Alg(\LLL_\infty)
\cap\Alg(\LLL_\infty)^* =\RRR'$, then $T\in\Alg(\LLL_1)$.
\end{lemma}

\begin{proof}
Suppose $T\in\AAA$ is given. From the
comments preceding Lemma 3 and by taking a difference from an
element in $\Alg(\LLL_\infty)$, we may assume that, with respect to
matrix units $E_{ij}^{(1)}$ in $\NNN_1$, $T$ is lower triangular,
i.e., $E_{ii}^{(1)}TE_{jj}^{(1)}=0$ for $i<j$. Now we want to show
that $T$ is diagonal. If the strictly lower triangular entries of
$T$ are not all zero, then let $i_0$ be the largest integer such
that $E_{i_0i_0}^{(1)}TE_{jj}^{(1)}\neq0$ for some $j<i_0$. Among
all such $j$, let $j_0$ be the largest. Then we have that
$E_{ii}^{(1)}TE_{jj}^{(1)}=0$ if $i>j$ and $i>i_0$; or
$i=i_0>j>j_0$. It is easy to check (from Lemma 4) that
$E_{j_0,i_0-1}^{(1)}-E_{j_0i_0}^{(1)}\in\Alg(\LLL_\infty)$. Then $
T(E_{j_0,i_0-1}^{(1)}-E_{j_0i_0}^{(1)})  \in\AAA$. Define $T_1
=T(E_{j_0,i_0-1}^{(1)}-E_{j_0i_0}^{(1)})
 $. Then
\begin{align*}
T_1 &= \sum_{n\ge k \geq l\ge1} E_{kk}^{(1)}TE_{ll}^{(1)}( E_{j_0,
i_0 - 1}^{(1)} - E_{j_{0}i_{0}}^{(1)})\\
 &= \sum_{i_{0} \geq k \geq j_{0}}
E_{kk}^{(1)}T( E_{j_0, i_0 - 1}^{(1)} - E_{j_{0}i_{0}}^{(1)} ).
\end{align*}
Let $T_2= E_{i_0 i_0}^{(1)}T(E_{j_0, i_0 - 1}^{(1)} -
E_{j_{0}i_{0}}^{(1)})$, and 
\begin{align*}
T_{3} =\sum_{i_0 > k \geq j_{0}}
E_{kk}^{(1)}T(E_{j_0, i_0 - 1}^{(1)} - E_{j_{0}i_{0}}^{(1)}).
\end{align*}
Then $T_1=T_2+T_3$. From Lemma 4 again, $T_3\in\Alg(\LLL_\infty)$.
This implies that $T_2\in\AAA$.

Let $E_{i_0i_{0}}^{(1)}T E_{j_0i_{0}}^{(1)}=HV$ be the polar
decomposition (in $\B(\HHH)$), where $H$ is positive and $V$ a
partial isometry. From our assumption that
$E_{i_0i_0}^{(1)}TE_{j_0j_0}^{(1)}\neq 0$, we have $H\neq 0$,
$E_{i_0i_0}^{(1)} H=H E_{i_0i_0}^{(1)}=H$ and $E_{i_0i_0}^{(1)}
V=V E_{i_0i_0}^{(1)}=V$. Then $T_2 = HV E_{i_{0}, i_{0}-1}^{(1)} -
HV$. Define
\begin{align*}
T_{4} &= - E_{i_{0}-1,i_{0}}^{(1)}V^{*} E_{i_{0}, i_{0}-1}^{(1)} +
E_{i_{0}-1,i_{0}}^{(1)}V^{*}E_{i_{0} i_{0}}^{(1)},\cr
 T_{5} &= E_{i_{0}-1,i_{0}}^{(1)}HE_{i_{0}, i_{0} - 1}^{(1)} -
E_{i_{0}-1,i_{0}}^{(1)}HE_{i_{0} i_{0}}^{(1)}.
\end{align*}
It is easy to check, from Lemma 3, that $T_{4}, T_{5}
\in\Alg(\LLL_\infty)$. Let
\begin{align*}
T_{6} &= T_{2}T_{4} + T_{5} \\
&=(  HV E_{i_{0}, i_{0}-1}^{(1)} - HV)(- E_{i_{0}-1,i_{0}}^{(1)}V^{*} E_{i_{0}, i_{0}-1}^{(1)} 
\cr & +
E_{i_{0}-1,i_{0}}^{(1)}V^{*}E_{i_{0} i_{0}}^{(1)})
+ E_{i_{0}-1,i_{0}}^{(1)}HE_{i_{0}, i_{0} - 1}^{(1)} -
E_{i_{0}-1,i_{0}}^{(1)}HE_{i_{0} i_{0}}^{(1)}      \cr
    &= -HE_{i_{0}, i_{0} - 1}^{(1)} + HE_{i_{0}i_{0}}^{(1)} \\
    & +
E_{i_{0}-1,i_{0}}^{(1)}HE_{i_{0}, i_{0} - 1}^{(1)} -
E_{i_{0}-1,i_{0}}^{(1)}HE_{i_{0} i_{0}}^{(1)}.\cr 
&= -HE_{i_{0},
i_{0} - 1}^{(1)} + H + E_{i_{0}-1,i_{0}}^{(1)}HE_{i_{0}, i_{0} -
1}^{(1)} - E_{i_{0}-1,i_{0}}^{(1)}H.
\end{align*}
Clearly $T_6\in\AAA$ and $T_{6}^{*} = T_{6}$. But $T_6$ is not upper
triangular. Thus $T_6 \notin \Alg(\LLL_\infty)$. This implies that
$\AAA\cap\AAA^*\neq \Alg(\LLL_\infty)\cap\Alg(\LLL_\infty)^*$. This
contradiction shows that $T$ must be diagonal. Thus we have that
$T=\sum_{j=1}^n E_{jj}^{(1)}TE_{jj}^{(1)}$. Now we show that
$E_{11}^{(1)}TE_{11}^{(1)}=E_{1j}^{(1)}TE_{j1}^{(1)}$ for
$j=1,\ldots,n$.

Assume that there is an $i$ such that $E_{11}^{(1)}TE_{11}^{(1)}
\neq E_{1i}^{(1)}TE_{i1}^{(1)}$. Define
\begin{align*}
T_7= (E_{11}^{(1)} - E_{1i}^{(1)})T = E_{11}^{(1)}TE_{11}^{(1)} -
E_{1i}^{(1)}TE_{ii}^{(1)}.
\end{align*}
 Then similar to the construction of $T_1$, we see that $T_7\in\AAA$. Again
write $T_{8} = - E_{1i}^{(1)}TE_{i1}^{(1)} +
E_{1i}^{(1)}TE_{ii}^{(1)}$. One checks (by Lemma 3) that $T_{8}
\in \Alg(\LLL_\infty)$. Then
\begin{align*}
0 \neq T_{7} + T_{8} = E_{11}^{(1)}TE_{11}^{(1)} -
E_{1i}^{(1)}TE_{i1}^{(1)} \in {\A}.
\end{align*}
Set $T_7+T_{8}=V'H'$, the polar decomposition with $V'$ a partial
isometry. One easily checks that ${V'}^{*} - {V'}^{*}E_{12}^{(1)}
\in \Alg(\LLL_\infty)$. Then $ ({V'}^{*} -
{V'}^{*}E_{12}^{(1)})(T_7+T_{8}) = H'  \in {\A}$. Since $H'$ is
selfadjoint, $H'\in\A\cap\A^*$. But $E_{11}^{(1)}H'E_{11}^{(1)} \neq
0$ (with $E_{22}^{(1)}H'E_{22}^{(1)} = \cdots =
E_{nn}^{(1)}H'E_{nn}^{(1)}=0$).
 Thus $H'\notin\NNN_1'$
$(\supseteq\Alg(\LLL_\infty)\cap\Alg(\LLL_\infty)^*)$. This implies that
$\AAA\cap\AAA^*\neq \Alg(\LLL_\infty)\cap\Alg(\LLL_\infty)^*$. This
contradiction shows that  $E_{11}^{(1)}TE_{11}^{(1)} = \cdots =
E_{1n}^{(1)}TE_{n1}^{(1)}$. Therefore $T \in {\LLL}_{1}'\subseteq\Alg(\LLL_1)$.
\end{proof}

Now we restate our Theorem 3 in a slightly stronger form.

\begin{theorem}
Suppose $\AAA$ is a
subalgebra of $\B(\HHH)$ such that $\Alg(\LLL_\infty)\subseteq\AAA$ and
$\AAA\cap\AAA^*=\Alg(\LLL_\infty)\cap\Alg(\LLL_\infty)^*$. Then
$\AAA=\Alg(\LLL_\infty)$.
\end{theorem}

\begin{proof}
Without the loss of generality, we
may assume that $\AAA$ is generated by $T$ and $\Alg(\LLL_\infty)$.
From the above lemma, we have that $T\in\Alg(\LLL_1)$. Suppose
$T\in\Alg(\LLL_m)$ but $T\notin\Alg(\LLL_{m+1})$. From Lemma 6, we
write $T=S+T'$ such that $S\in\Alg(\LLL_\infty)$ and
$T'\in\NNN_m'\cap\Alg(\LLL_m)$. When we restrict all operators to
the commutant of $\NNN_m$ and working with matrix units
$E_{ij}^{(m+1)}$, similar computation as in the proof of Lemma 7
will show that $T'\in\Alg(\LLL_{m+1})$. This contradiction shows
that $T\in\cap_{m=1}^\infty\Alg(\LLL_m)=\Alg(\LLL_\infty)$.
\end{proof}


In the above theorem, we did not assume the closedness of $\AAA$
under any topology. Thus $\Alg(\LLL_\infty)$ has an algebraic
maximality property. Next section, we will show that
$\Lat(\Alg(\LLL_\infty))$ is the strong-operator closure of
$\LLL_\infty$. %%%Other Kadison-Singer lattices are also given.


\vskip20pt
















\section{Kadison-Singer lattices}

\vskip10pt

\noindent It is hard to determine whether a given lattice is a
Kadison-Singer lattice. The only known class is the family of nests
\cite{KS}. Some finite distributive lattices (see \cite{H2} and \cite{Ha}) are
Kadison-Singer lattices if they have a minimal generating property.
In this section, we will show that the strong-operator closure of
$\LLL_\infty$ defined in Section 3 is a Kadison-Singer lattice. For
simplicity of description, we shall assume that the state $\rho$ on
$\AAA_n$ is a trace, now denoted by $\tau$. Let $\RRR$ be the
hyperfinite II$_1$ factor generated by $\LLL_\infty$ (or $\AAA_n$). The
commutant $\RRR'$ of $\RRR$ is the diagonal subalgebra of
$\Alg(\LLL_\infty)$.

\begin{theorem}
Let
$\LLL^{(n)}$ be the strong-operator closure of $\LLL_\infty$, $\HHH$ the
Hilbert space obtained by GNS construction on $(\AAA_n, \tau)$. Then
we have that $\LLL^{(n)}=\Lat(\Alg(\LLL_\infty))$. For any
$r\in(0,1)$, if there are $a,l\in\N$ such that $r=\frac{a}{n^l}$,
then there are two distinct projections in $\LLL^{(n)}$ with trace
value $r$; otherwise there is only one projection in $\LLL^{(n)}$
with trace $r$. 
\end{theorem}

To understand the lattice structure of $\LLL^{(n)}$, we first
analyze the lattice properties of $\LLL_m$. From the definition of
$P_{1j}$, $j=1,\ldots,n$, the generators of $\LLL_1$, we know that
$\LLL_1$ consists of a nest $\{0, P_{11},\ldots, P_{1,n-1}, I\}$ in
$\NNN_1$ $(\cong M_n(\C))$ on the diagonal and a minimal
projection $P_{1n}$. It is easy to see that $P_{1n}\wedge
P_{1j}=0$ for $1\leq j\le n-1$, and their unions give rise to
another nest $\{0, P_{1n}, P_{1n}\vee P_{11}, \ldots, P_{1n}\vee
P_{1,n-1}=I\}$ in $\NNN_1$. The lattice $\LLL_1$ is the union of
these two nests. For any $1\leq k\leq n-1$, there are two distinct
projections in $\LLL_1$ such that they have the same trace $\frac{
k}{n}$. This pattern of double nests appears in $\LLL_2$ between any
two trace values $\frac{k}{n}$ and $\frac{k+1}{n}$, $0\leq k\leq n-1$. To
describe all these projections, we need more notation. For
$k=1,2,\ldots$, define
\begin{align*}
E_{i}^{(k)} &= \sum_{l=1}^{i} E_{ll}^{(k)}  \quad i=1,\ldots,
  n;\quad \\
F_{i}^{(k)} &= \frac{1}{i}\sum_{n\ge l,m > n-i} E_{lm}^{(k)}, \quad
i =1,\ldots,n.
\end{align*}
Since $E_{ij}^{(k)}$ and $E_{i'j'}^{(k')}$ are tensorial relations
for $k\neq k'$, we have $E_{i}^{(k)}$ and $F_{i}^{(k)}$ are
projections in $\NNN_{k-1}' \cap \NNN_k\ ( \NNN_{0} = \C I )$.
Also $E_{i}^{(k)}F_{j}^{(k)} = 0$ when $j \leq n-i$. We shall use
$\tau$ to denote the unique trace on both $\RRR$ and $\RRR'$,
$\tau(E_{i}^{(k)}) = \frac{i}{n}$, $\tau(F_{i}^{(k)}) =
\frac{1}{n}$.

\vskip6pt

\begin{lemma}
Suppose $P \in
\Lat(\Alg(\LLL_\infty))$ and $P \neq 0, I$. Then $P = E_{i}^{(1)} +
F_{n-i}^{(1)}Q$, for some  $i \in \{0,1,\ldots, n-1 \}$, $Q \in
{\NNN}_{1}'$, and also $Q \in \Lat(\NNN_{1}' \cap
\Alg(\LLL_\infty))$. When $P$ is given in this form,
$P\in\Lat(\Alg(\LLL_\infty))$.
\end{lemma}

\vskip6pt

\begin{proof}
First we show that if $P =
E_{i}^{(1)} + F_{n-i}^{(1)}Q$ with $Q$ described in the lemma,
then $P \in \Lat(\Alg(\LLL_\infty))$.

For any $T\in\Alg(\LLL_\infty)$, let $T=T_1+T_2$, given by (3) in
the proof of Lemma 3. Then it is easy to check that $T_{1} \in
\Alg(\LLL_\infty)$, $(I - P_{1,n-1})T_{1} = 0$ and $T_{2} \in
\NNN_1' \cap \Alg(\LLL_\infty)$. Since
$E_{i}^{(1)}=P_{1i}\in\LLL_1\subseteq\LLL_\infty$, we have $(I -
E_{i}^{(1)})T_jE_{i}^{(1)} = 0$, for $j\in\{1,2\}$. One can check
directly that $F_{n-i}^{(1)}(I-E_{i}^{(1)}) =
F_{n-i}^{(1)}=(I-E_i^{(1)})F_{n-i}^{(1)}$. Thus
$F_{n-i}^{(1)}TE_i^{(1)}=0$. Since $Q$ commutes with $\NNN_1$, we
have
\begin{align*}
(I-P)TP &= (I - E_{i}^{(1)} - F_{n-i}^{(1)}Q)T(E_{i}^{(1)} +
F_{n-i}^{(1)}Q)
 \cr &= (I-E_{i}^{(1)})TF_{n-i}^{(1)}Q -
F_{n-i}^{(1)}QTE_{i}^{(1)}  \cr & \qquad- F_{n-i}^{(1)}QTF_{n-i}^{(1)}Q
\cr
 & =(I-E_{i}^{(1)})TF_{n-i}^{(1)}Q -
        QF_{n-i}^{(1)}(I-E_{i}^{(1)})TF_{n-i}^{(1)}Q.
\end{align*}
 The above equations hold when $T$ is replaced by $T_1$ or $T_2$. From
our assumptions that $Q \in \Lat(\NNN_{1}' \cap
\Alg(\LLL_\infty))$, $E_i^{(1)}, F_{n-i}^{(1)}\in\NNN_1$ and $T_2\in
\NNN_1'\cap\Alg(\LLL_\infty)$, we have
\begin{align*}
(I-P)T_{2}P
&=((I-E_i^{(1)})F_{n-i}^{(1)}-QF_{n-i}^{(1)}(I-E_i^{(1)}))T_2Q \\
&=
F_{n-1}^{(1)}(I - Q)T_{2}Q = 0.
\end{align*}
Next we show that
$(I-E_i^{(1)})T_1F_{n-i}^{(1)} = 0$, which implies that
$(I-P)T_1P=0$. Note that
\begin{align*}
(n-i)(I-E_{i}^{(1)})T_1 F_{n-i}^{(1)} &= (\sum_{j= i+1}^{n}
E_{jj}^{(1)})T_{1}(\sum_{l,m = i+1}^{n} E_{lm}^{(1)})\\
&=\sum_{j,m =
i+1}^{n}\sum_{l = j}^{n} E_{jj}^{(1)}T_{1}E_{lm}^{(1)}.
\end{align*}
By Lemma 3 and $(I - P_{n-1}^{(1)})T_{1} = E_{nn}^{(1)}T_{1} =
0$, we have $\sum_{l= j}^{n} E_{jj}^{(1)}T_{1}E_{lm}^{(1)} = 0$.
So $(n-i)(I-E_{i}^{(1)})T_{1}F_{n-i}^{(1)} = 0$. Thus $(I-P)TP=0$
which implies that $P \in \Lat(\Alg(\LLL_\infty))$.

Now for any $P \in \Lat(\Alg(\LLL_\infty))$, let $i_0$, $1\le i_0\le
n$, be the smallest integer such that
$E_{i_{0}i_{0}}^{(1)}PE_{i_{0}i_{0}}^{(1)} \neq E_{i_{0}
i_{0}}^{(1)}$. Then $E_{i i}^{(1)}PE_{ii}^{(1)} = E_{ii}^{(1)}$
for $1\leq i \leq i_{0}-1$ and $P = E_{i_{0}-1}^{(1)} + P_{1}$,
where $P_{1}$ is a projection and $E_{i}^{(1)}P_{1} = 0$ for $i\leq
i_0-1$. First we assume that $i_{0} \leq n-1$. For any $A \in
\B(\HHH)$ and $i_1\geq i_0+1$, define $A_{i_{1}} = E_{i_{0}
i_{0}}^{(1)}A(E_{i_{0} i_{0}}^{(1)} - E_{i_{0}i_{1}}^{(1)})$. Then
$A_{i_1} \in \Alg(\LLL_\infty)$. Since $P \in
\Lat(\Alg(\LLL_\infty))$, we have
\begin{align*}
 0 &= (I - E_{i_{0}-1}^{(1)} - P_{1})A_{i_{1}}(E_{i_{0}-1}^{(1)} +
 P_{1})\\
   &= (I - P_{1})E_{i_{0}i_{0}}^{(1)}A(E_{i_{0}i_{0}}^{(1)} -
E_{i_{0} i_{1}}^{(1)})P_{1}.
\end{align*}
From
$E_{i_{0}i_{0}}^{(1)}(I - P_{1})E_{i_{0} i_{0}}^{(1)} \neq
0$, the above equation implies that $E_{i_{0} i_{0}}^{(1)}P_{1} =
E_{i_{0}i_{1}}^{(1)}P_{1}$, for all $i_1\ge i_0$. So, multiplying
by $P_1E_{i_0i_0}^{(1)}=P_1E_{ji_0}^{(1)}$ (the adjoint of the
above equation) on the right hand side, we have $E_{i_{0}
i_{0}}^{(1)}P_{1}E_{i_{0}i_{0}}^{(1)} = E_{i_{0}
i_1}^{(1)}P_{1}E_{ji_{0}}^{(1)}$ for all $i_1, j \geq i_{0}$. This
implies that $P_{1} = F_{n - i_{0}+1}^{(1)}Q$, where $Q$ is a
projection in $\NNN_{1}'$. If $i_{0} = n$, then  $P_{1}$ can be
written as $F_{1}^{(1)}Q$ for $Q \in \NNN_{1}'$. From $P \in
\Lat(\Alg(\LLL_\infty))$, it is easy to see that $Q \in
\Lat(\NNN_{1}' \cap \Alg(\LLL_\infty))$.
\end{proof}

\begin{lemma}
Suppose $P \in
\Lat(\Alg(\LLL_\infty))$. Then there exist $Q \in \NNN_{k}' \cap
\Lat({\NNN}_{k}' \cap \Alg(\LLL_\infty))$ and integers $a_k$ such
that
\begin{align*}
P = E_{a_{1}}^{(1)} + F_{n-a_{1}}^{(1)}E_{a_{2}}^{(2)} + \cdots
+(\prod_{i = 1}^{k-1}F_{n-a_{i}}^{(i)})E_{a_{k}}^{(k)} + (\prod_{i
= 1}^{k}F_{n-a_{i}}^{(i)})Q,
\end{align*}
where $0 \leq a_{i} \leq n-1$ and $\tau(P) =
\sum_{i=1}^{k}\frac{a_{i}}{n^{i}} + \frac{\tau(Q)}{n^{k}}$ which
is a real number lies in the closed interval
$\left[\sum_{i=1}^{k}\frac{a_{i}}{n^{i}},\
\sum_{i=1}^{k}\frac{a_{i}}{n^{i}} +
\frac{1}{n^{k}}\right]\subseteq[0,1]$.
\end{lemma}

The above lemma follows easily from induction. The details are
similar to the proof of Lemma 8.

\begin{proof}[Proof of Theorem 5]
To describe an
arbitrary projection $P$ in $\Lat(\Alg(\LLL_\infty))$ in more
details, we need to know the trace value of $P$. First when
$\tau(P) = \sum_{i=1}^{k}\frac{a_{i}}{n^{i}}$, where $0\leq a_i\leq
n-1$ and $a_{k} \neq 0$, then there are two cases, either
\begin{align*}
P &= \sum_{j = 1}^{k} (\prod_{i =
1}^{j-1}F_{n-a_{i}}^{(i)})E_{a_{j}}^{(j)} ,\qquad  ({\rm let}\
\prod_{i = 1}^{0}F_{n-a_{i}}^{(i)} = I)\quad {\rm or}\cr
 P & = \sum_{j
= 1}^{k-1} (\prod_{i = 1}^{j-1}F_{n-a_{i}}^{(i)})E_{a_{j}}^{(j)} +
(\prod_{i = 1}^{k-1}F_{n-a_{i}}^{(i)})E_{a_{k}-1}^{(k)} \\
& \qquad \qquad +
(\prod_{i = 1}^{k-1}F_{n-a_{i}}^{(i)})F_{n-a_{k}+1}^{(k)}.
\end{align*}
Note that the above two projections correspond to the case when
$Q=0$ for the decomposition
$\tau(P)=\sum_{i=1}^{k}\frac{a_{i}}{n^{i}}$, or respectively $Q=I$
for $\tau(P)=\sum_{i=1}^{k-1}
\frac{a_{i}}{n^{i}}+\frac{a_k-1}{n^k}+ \frac1{n^k}$ in Lemma 9.
So $P \in \Lat(\Alg(\LLL_\infty))$. Thus for any $r =
\frac{a}{n^{l}}$ for some integer $l
> 0$ and any integer $a$ such that $0 < a < n^{l}$, there are
exactly two projections in $\Lat(\Alg(\LLL_\infty))$ with trace $r$.

Secondly, when $r \in (0,1)$ and $r\neq \frac{a}{n^{l}}$ for any
positive integer $l$ and any integer $a$ with $0 < a < n^{l}$, we
shall show that there is a unique $P$ in $\Lat(\Alg(\LLL_\infty))$
with trace $r$. For the given $r$, there is a unique expansion $r
= \sum_{k=1}^{\infty}\frac{a_{k}}{n^{k}}$, where $a_k$ is an
integer with $0 \leq a_{k} \leq n-1$, there are infinitely many
non zero $a_k$'s and infinitely many $a_k\neq n-1$. (This is
because repeating $n-1$ as coefficients from certain place on will
result $r$ being $\frac{a}{n^l}$, e.g., $0.09999\cdots=0.1$ when
$n=10$.) In fact, Lemma 9 gives the existence and uniqueness of
such a projection:
\begin{align*}
P=E_{a_{1}}^{(1)} + F_{n-a_{1}}^{(1)}E_{a_{2}}^{(2)} +
         F_{n-a_{1}}^{(1)}F_{n-a_{2}}^{(2)}E_{a_{3}}^{(3)} + \cdots.
\end{align*}

It is not hard to see that $P$ is the strong-operator limit of
finite sums. The finite sums
\begin{align*}
Q_{k} =  \sum_{j = 1}^{k} (\prod_{i =
1}^{j-1}F_{n-a_{i}}^{(i)})E_{a_{j}}^{(j)}\in\LLL_\infty ,  \quad  k
= 1 , 2,\ldots,
\end{align*}
 $Q_{1} < Q_{2} < \cdots < Q_{k} < \cdots<P$ and $\lim_{k
\to\infty} \tau(Q_{k}) = r=\tau(P)$.
\end{proof}

The following theorem gives us infinitely many non isomorphic
Kadison-Singer lattices.

\begin{theorem}
For $n\neq k$,
$\LLL^{(n)}$ and $\LLL^{(k)}$ are not algebraically isomorphic as
lattices.
\end{theorem}

\begin{proof}
For any $n\geq 2$, first we observe
from Lemma 9 that if $E \in \LLL^{(n)}$ is a minimal projection,
then $E= (\prod_{i = 1}^{m}F_{n}^{(i)})E_{1}^{(m+1)}$ for $m =0,
1, 2, \ldots$.

Suppose that $P \in \LLL^{(n)}$ is given as in Lemma 9,
\begin{align*}
P = &E_{a_{1}}^{(1)} + F_{n-a_{1}}^{(1)}E_{a_{2}}^{(2)} + \cdots
+(\prod_{i = 1}^{m}F_{n-a_{i}}^{(i)})E_{a_{m+1}}^{(m+1)} \\
& \qquad \qquad+ 
(\prod_{i = 1}^{m}F_{n-a_{i}}^{(i)})F_{n-a_{m+1}}^{(m+1)}Q,
\end{align*}
where $0 \leq a_{i} \leq n-1$ and $Q \in \NNN_{m+1}' \cap
\Lat({\NNN}_{m+1}' \cap \Alg(\LLL^{(n)}))$. We shall show that $P
\wedge (\prod_{i = 1}^{m}F_{n}^{(i)})E_{1}^{(m+1)} = 0$ for some
$m\geq 0$ if and only if $a_{m+1} = 0$.

First if $a_{m+1} > 0$, it is easy to check that the minimal
projection $(\prod_{i = 1}^{m}F_{n}^{(i)})E_{1}^{(m+1)} \leq P$.
Conversely, if $a_{m+1} = 0$, then
\begin{align*}
P = &E_{a_{1}}^{(1)} + F_{n-a_{1}}^{(1)}E_{a_{2}}^{(2)} + \cdots +
(\prod_{i=1}^{m-1}F_{n-a_{i}}^{(i)})E_{a_{m}}^{(m)}\\
& \qquad \qquad +(\prod_{i=1}^{m}F_{n-a_{i}}^{(i)})F_{n}^{(m+1)}Q.
\end{align*}
Let $\xi \in P(\HHH) \wedge (\prod_{i =
1}^{m}F_{n}^{(i)})E_{1}^{(m+1)}(\HHH)$, and $E =
(I-E_{n-1}^{(1)})(I-E_{n-1}^{(2)})\cdots(I-E_{n-1}^{(m+1)})$. We
have $E\xi = E(\prod_{i = 1}^{m}F_{n}^{(i)})E_{1}^{(m+1)}\xi = 0$.
But
\begin{align*}
0 &= PEP\xi =(\prod_{i=1}^{m}F_{n-a_{i}}^{(i)})F_{n}^{(m+1)}
E(\prod_{i=1}^{m}F_{n-a_{i}}^{(i)})F_{n}^{(m+1)}Q\xi\cr
&=(\prod_{i=1}^{m}F_{n-a_{i}}^{(i)}(I-E_{n-1}^{(i)})
F_{n-a_{i}}^{(i)})F_{n}^{(m+1)}(I-E_{n-1}^{m+1})
F_{n}^{(m+1)}Q\xi\cr
&=\frac{1}{n}(\prod_{i=1}^{m}\frac{1}{n-a_{i}})
(\prod_{i=1}^{m}F_{n-a_{i}}^{(i)})F_{n}^{(m+1)}Q\xi.
\end{align*}
This shows that $\xi \in (\prod_{i =
1}^{m}F_{n}^{(i)})E_{1}^{(m+1)}(\HHH) \wedge P_{1}(\HHH)$, here $P_{1}
= P - (\prod_{i=1}^{m}F_{n-a_{i}}^{(i)})F_{n}^{(m+1)}Q$. Let
$\widetilde{E} =
(I-E_{n-1}^{(1)})(I-E_{n-1}^{(2)})\cdots(I-E_{n-1}^{(m)})$. Then
$\widetilde{E}\xi = \widetilde{E}P_{1}\xi = 0$ and
\begin{align*}
0 &= (\prod_{i =
1}^{m}F_{n}^{(i)})E_{1}^{(m+1)}\widetilde{E}(\prod_{i =
1}^{m}F_{n}^{(i)})E_{1}^{(m+1)}\xi \\
&=(\prod_{i =
1}^{m}F_{n}^{(i)}(I-E_{n}^{(i)})F_{n}^{(i)})E_{1}^{(m+1)}\xi\cr
  &= \frac{1}{n^{m}}(\prod_{i = 1}^{m}F_{n}^{(i)})E_{1}^{(m+1)}\xi
  = \frac{1}{n^{m}}\xi.
\end{align*}
This shows that $\xi = 0$ and thus $P \wedge (\prod_{i =
1}^{m}F_{n}^{(i)})E_{1}^{(m+1)} = 0$.

For any $\SSS \subset \LLL^{(n)}$, we define
\begin{align*}
\ZZZ(\SSS) = \{P \in
\LLL^{(n)} :\  P \wedge Q = 0, \ {\rm for\ all\ } Q \in \SSS \}.
\end{align*}
From
the above, we know that, for any minimal projection $(\prod_{i =
1}^{m}F_{n}^{(i)}) E_{1}^{(m+1)}$,
\begin{align*}
\ZZZ(\{(\prod_{i = 1}^{m}F_{n}^{(i)})& E_{1}^{(m+1)} \})\cr =
\{E_{a_{1}}^{(1)} &+ F_{n-a_{1}}^{(1)}E_{a_{2}}^{(2)} + \cdots +
(\prod_{i=1}^{m-1}F_{n-a_{i}}^{(i)})E_{a_{m}}^{(m)}\\
& \qquad \qquad +(\prod_{i=1}^{m}F_{n-a_{i}}^{(i)})F_{n}^{(m+1)}Q: \cr &  0 \leq
a_{i} \leq n-1, Q \in {\NNN}_{m+1}' \cap Lat({\NNN}_{m+1}' \cap
\Alg(\LLL^{(n)}))\}.
\end{align*}
Now it is not hard to show that $\ZZZ(\ZZZ(\{(\prod_{i =
1}^{m}F_{n}^{(i)})E_{1}^{(m+1)} \})) = \{ (\prod_{i =
1}^{m}F_{n}^{(i)})E_{k}^{(m+1)}:\  k = 0, 1, \ldots, n-1 \}$. The
number of elements in this set is an invariant of
$\LLL^{(n)}$.
\end{proof}




















%%-- text of paper here --

%% == end of paper:

%% Optional Materials and Methods Section
%% The Materials and Methods section header will be added automatically.

%% Enter any subheads and the Materials and Methods text below.
%\begin{materials}
% Materials text
%\end{materials}


%% Optional Appendix or Appendices
%% \appendix Appendix text...
%% or, for appendix with title, use square brackets:
%% \appendix[Appendix Title]

\begin{acknowledgments}
Research supported in part by Chinese Academy of Sciences and President Fund of 
Academy of Mathematics and Systems Science, Chinese Academy of Sciences
\end{acknowledgments}

%% PNAS does not support submission of supporting .tex files such as BibTeX.
%% Instead all references must be included in the article .tex document. 
%% If you currently use BibTeX, your bibliography is formed because the 
%% command \verb+\bibliography{}+ brings the <filename>.bbl file into your
%% .tex document. To conform to PNAS requirements, copy the reference listings
%% from your .bbl file and add them to the article .tex file, using the
%% bibliography environment described above.  

%%  Contact pnas@nas.edu if you need assistance with your
%%  bibliography.

% Sample bibliography item in PNAS format:
%% \bibitem{in-text reference} comma-separated author names up to 5,
%% for more than 5 authors use first author last name et al. (year published)
%% article title  {\it Journal Name} volume #: start page-end page.
%% ie,
% \bibitem{Neuhaus} Neuhaus J-M, Sitcher L, Meins F, Jr, Boller T (1991) 
% A short C-terminal sequence is necessary and sufficient for the
% targeting of chitinases to the plant vacuole. 
% {\it Proc Natl Acad Sci USA} 88:10362-10366.


%% Enter the largest bibliography number in the facing curly brackets
%% following \begin{thebibliography}

\begin{thebibliography}{21}

\bibitem{KS} R. Kadison and I. Singer, {\em Triangular operator
algebras. Fundamentals and hyperreducible theory}, Amer. J. Math.,
    {\bf 82 } (1960),  227--259.

\bibitem{vN} J. von Neumann, {\em Zur Algebra der
Functionaloperationen und Theorie der normalen Operatoren}, Math.
Ann. {\bf 102} (1930), 370--427.

\bibitem{MSS} P. S. Muhly, K. Saito and B. Solel, {\em Coordinates
for triangular operator algebras,} Ann. of Math., {\bf 127}
(1988), 245--278.

\bibitem{Ho} A. Hopenwasser, {\em Completely isometric maps and
triangular operator algebras}, Proc. London Math. Soc. {\bf 25}
(1972), 96--114.

\bibitem{H2} P. Halmos, {\em Reflexive lattices of subspaces,} J.
London Math. Soc.  {\bf 4}  (1971),  257--263.

\bibitem{Hd} D. Hadwin, {\em A general view of reflexivity}, Trans.
Amer. Math. Soc. {\bf 344} (1994), 325--360.


\bibitem{RR} H. Radjavi and P. Rosenthal, {\em ``Invariant
Subspaces''}, Springer--Verlag, Berlin, 1973.

\bibitem{La} D. Larson, {\em Reflexivity, algebraic reflexivity and
linear interpolation}, Amer. J. Math. {\bf 110} (1998),
283--299.

\bibitem{GS} L. Ge and J. Shen, {\em On the generator problem of
von Neumann algebras,} AMS/IP Studies in Adv. Math. {\bf 42}
(2008), 257--275.

\bibitem{VDN} D. Voiculescu, K. Dykema and A. Nica, {\em ``Free
Random Variables,''}     CRM Monograph Series, vol. 1, 1992.

\bibitem{R} J. Ringrose, {\em On some algebras of operators,} Proc.
London Math. Soc. (3) {\bf 15} (1965), 61--83.

\bibitem{Ar} W. Arveson, {\em Operator algebras and invariant
subspaces,}  Ann. of Math. {\bf 100} (1974), 433--532.

\bibitem{L} E. C. Lance, {\em Cohomology and perturbations of nest
algebras,} Proc. London Math. Soc. {\bf 43} (1981), 334--356.

\bibitem{La2} D. Larson, {\em Similarity of nest algebras,} Ann. of
Math. {\bf 121} (1988), 409--427.


\bibitem{DKP} K. Davidson, E. Katsoulis and D. Pitts, {\em The
structure of free semigroup Algebras,}  J. Reine Ang. Math.
(Crelle's Journal), {\bf 533} (2001), 99--125.

\bibitem{Da}  K. Davidson, {\em ``Nest Algebras,''} Research Notes
in Math. vol. 191, Pitman, Boston-Londodn Melbourne, 1988.

\bibitem{Ka}  R. Kadison,
     {\em On the orthogonalization of operator
     representations,} Amer. J. Math.
     {\bf 78}  (1955),  600--621.
     
\bibitem{KR} R.\ Kadison and J.\ Ringrose, {\em ``Fundamentals of
the Operator Algebras,''} vols. I and II, Academic Press, Orlando,
1983 and 1986.

\bibitem{G} J. Glimm, {\em Type I C*-algebras,} Ann. Math.
{\bf 73} (1961), 572--611.

\bibitem{P} R. Powers,  {\em Representations of Uniformly
Hyperfinite Algebras and Their Associated von Neumann Rings,} Ann.
Math. {\bf 86} (1967), 138--171.

\bibitem{Ha} K. Harrison, {\em On lattices of invariant subspaces,}
Doctoral Thesis, Monash University, Melbourne, 1970.

\end{thebibliography}

\end{article}
%%%%%%%%%%%%%%%%%%%%%%%%%%%%%%%%%%%%%%%%%%%%%%%%%%%%%%%%%%%%%%%%

%% Adding Figure and Table References
%% Be sure to add figures and tables after \end{article}
%% and before \end{document}

%% For figures, put the caption below the illustration.
%%
%% \begin{figure}
%% \caption{Almost Sharp Front}\label{afoto}
%% \end{figure}

%% For Tables, put caption above table
%%
%% Table caption should start with a capital letter, continue with lower case
%% and not have a period at the end
%% Using @{\vrule height ?? depth ?? width0pt} in the tabular preamble will
%% keep that much space between every line in the table.

%% \begin{table}
%% \caption{Repeat length of longer allele by age of onset class}
%% \begin{tabular}{@{\vrule height 10.5pt depth4pt  width0pt}lrcccc}
%% table text
%% \end{tabular}
%% \end{table}

%% For two column figures and tables, use the following:

%% \begin{figure*}
%% \caption{Almost Sharp Front}\label{afoto}
%% \end{figure*}

%% \begin{table*}
%% \caption{Repeat length of longer allele by age of onset class}
%% \begin{tabular}{ccc}
%% table text
%% \end{tabular}
%% \end{table*}

\end{document}

