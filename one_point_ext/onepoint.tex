\documentclass[12pt]{article}
\usepackage{mathrsfs}
\usepackage{amssymb}
\usepackage{amssymb, amsmath, graphicx}
\usepackage{graphicx}



\newtheorem{theorem}{Theorem}
\newtheorem{acknowledgement}[theorem]{Acknowledgement}
\newtheorem{algorithm}[theorem]{Algorithm}
\newtheorem{axiom}[theorem]{Axiom}
\newtheorem{case}[theorem]{Case}
\newtheorem{claim}[theorem]{Claim}
\newtheorem{conclusion}[theorem]{Conclusion}
\newtheorem{condition}[theorem]{Condition}
\newtheorem{conjecture}[theorem]{Conjecture}
\newtheorem{corollary}[theorem]{Corollary}
\newtheorem{criterion}[theorem]{Criterion}
\newtheorem{definition}[theorem]{Definition}
\newtheorem{example}[theorem]{Example}
\newtheorem{exercise}[theorem]{Exercise}
\newtheorem{lemma}[theorem]{Lemma}
\newtheorem{notation}[theorem]{Notation}
\newtheorem{problem}[theorem]{Problem}
\newtheorem{proposition}[theorem]{Proposition}
\newtheorem{remark}[theorem]{Remark}
\newtheorem{solution}[theorem]{Solution}
\newtheorem{summary}[theorem]{Summary}
\newenvironment{proof}[1][Proof]{\textbf{#1.} }{\ \rule{0.5em}{0.5em}}

\setcounter{page}{1}
 \setlength{\topmargin}{-0.1in}
 \setlength{\textheight}{8.5in}
 \setlength{\textwidth}{6.4in}
 \setlength{\oddsidemargin}{0.0in}
\setlength{\evensidemargin}{0.10in}


\def\H{{\mathscr{H} }}\def\L{{\mathscr{L}}}
\def\D{{\mathscr{D} }}
\def\M{{\mathscr{ M}}}
\def\NN{{\mathscr{ N}}}
\def\F{{\mathscr{F}}}
\def\P{{\mathscr{P}}}

\def\N{{\mathbb{N}}}
\def\I{{\mathbb{I}}}
\def\Z{{\mathbb{Z}}}\def\C{{\mathbb{C}}}
\def\Lat{{\mathcal Lat}}\def\Alg{{\mathcal Alg}}


\newcommand{\AAA}{\mathfrak A}
\newcommand{\BBB}{\mathcal B}
\newcommand{\CCC}{\mathcal C}
\newcommand{\HHH}{\mathscr H} %for Hilbert space
\newcommand{\LLL}{\mathscr L} % for lattice
\newcommand{\MMM}{\mathcal M}
\newcommand{\NNN}{\mathcal N} %for nest
\newcommand{\SSS}{\mathcal S}


\newcommand{\PP}[1]{ P_{#1}} %for projections
\newcommand{\QQ}[1]{ Q_{#1}}

\newcommand{\e}[2][]{e^{#1}_{#2}} %for matrix unit \e[upper index]{lower index}

\def\Lat{\mathcal Lat}
\def\Alg{\mathcal Alg}
\def\tensor{\mathop{\bar \otimes}}
\def\tr{\tau}

\def\C{\mathbb C} %for complex number
\def\R{\mathbb R}  %for real number
\def\Z{\mathbb Z} %for integer
\def\N{\mathbb N} % for nature number

%%%%%%%%%%%%%%%%%%
%short cuts
%%%%%%%%%%%%%%%%%%



%%%%%%%%%%%%%%%%%%
%For proof
%\begin{proof}
%                 \qed
%\end{proof}
%
% a set \ {b...} use \backslash
%%%%%%%%%%%%%%%%%%%


%Setup End----------------------------------------------------------------------------------------

\begin{document}
\title{\textbf{On Some Examples of Kadison-Singer
Algebras  } }
\author{\textbf{Liguang Wang\thanks{Liguang Wang was supported by the NSF of China (No.
10626031) and the
Scientific Research Fund of the Shandong Provincial Education Department (No. J08LI15). }}\\
{\small School of Mathematical Sciences, Qufu Normal University}\\
{\small Qufu, Shandong, 273165, China} \\
{\textbf{Wei Yuan\thanks{Wei Yuan was supported by Chinese Academy of Sciences. } }}\\
{\small Academy of Mathematics and System Sciences}\\
{\small Chinese Academy of Sciences, Beijing 100080, China} \\
{\small E-mail: wangliguang0510@163.com, yuanwei.cn@gmail.com}}
\date{}
\maketitle \baselineskip=14.2pt \vskip0.1cm

\begin{quote}
\textbf{Abstract} {In this paper, we first show that the algebra
corresponding to the one point extension of a maximal nest on a
Hilbert space is a Kadison-Singer algebra with diagonal equal to $\C
I$. Three examples are also given. Let $\NN$ be a von Neumann algebra in $\BBB(\H)$ with a normal faithful tracial state $\tau$. Suppose  $\{P_1, P_2, \cdots, P_n\}$ is a nest with $ P_1\leq  P_2\leq \cdots \leq  P_n=I$ and $\tau(P_k)=\frac{k}{n}$,  $Q$ is a projection with  $\tau(Q)=\frac{1}{n}$. If $Q$ is free from $\{P_1, P_2, \cdots, P_n\}$ with respect to $\tau$, then the lattice generated by $\{P_1, P_2, \cdots, P_n\}$ and $Q$ is also a Kadison-Singer lattice and $\Alg(\L)$ is a Kadison-Singer algebra.}

\textbf{Keywords} {Kadison-Singer algebra; Kadison-singer Lattice;
Nest; Free. }

{\small} {\small \textbf{2000 MR Subject Classification} 46L10,
46L50}
\end{quote}

\section{Introduction}

Bounded linear operators on an infinite-dimensional separable
Hilbert space are generalizations of finite matrices on Euclidean
spaces. Non selfadjoint operator algebras were introduced to study
questions concerning the structural properties of operators. For
example, whether every bounded linear operator has a (nontrivial)
invariant closed subspace is a longstanding open question in
operator theory (see \cite{RR} and \cite{Ar}). Considerable effort
has been made to the study of two classes of non selfadjoint
operator algebras: triangular algebras (introduced in \cite{KS}) and
reflexive algebras (see \cite{R}). Nest algebras belong to the
intersection of these two (see \cite{Da}). Many people have tried to
extend the theory of selfadjoint operator algebras
(C$^{\ast}$-algebras and von Neumann algebras) as well as its
techniques to non selfadjoint operator algebras. One of such
attempts is to use an arbitrary factor to replace the algebra of all
bounded linear operators (a factor of type I$_\infty$). But it has
not been a fruitful attempt. Many of the definitive results on non
selfadjoint algebras (or single operators) rely on some relations to
compact operators (or finite-rank operators).\newline



Suppose $\H$ is a separable Hilbert space and $\BBB(\H)$ is the
algebra of all bounded linear operators on $\H$. Let $\P$ be a set
of (orthogonal) projections in $\BBB(\H)$. Define $$\Alg(\P)=\{ T
\in \BBB(\H): TP=PTP, \ {\rm for\ all}\ P\in\P\}.$$ Then $\Alg(\P)$
is a weak-operator closed subalgebra of $\BBB(\H)$. Similarly, for a
subset $\mathscr{S}$ of $\BBB(\H)$, define $$\Lat (\mathscr
S)=\{P\in \BBB(\H): P\ {\rm is\ a \ projection},\ TP=PTP, \ {\rm
for\ all}\ T\in\mathscr{S}\}.$$ Then $\Lat(\mathscr S)$ is a
strong-operator closed lattice of projections. A subalgebra
$\mathscr{B}$ of $\BBB(\H)$ is said to be a reflexive (operator)
algebra if $\mathscr{B}=\Alg(\Lat(\mathscr B))$. Similarly, a
lattice $\L$ of projections in $\BBB(\H)$ is called a reflexive
lattice (of projections) if $\L=\Lat(\Alg(\L))$. A nest is a totally
ordered reflexive lattice. If $\L$ is a nest, then $\Alg(\L)$ is
called a nest algebra. Nest algebras are generalizations of
(hyperreducible) ``maximal triangular'' algebras introduced by
Kadison and Singer in \cite{KS}. Kadison and Singer also showed that
nest algebras are the only maximal triangular reflexive algebras
(with a commutative lattice of invariant projections). Motivated by
this, Liming Ge and Wei Yuan introduced the following definition in
\cite{GY1}.\newline

\noindent{\textbf{Definition}}\quad A subalgebra $\mathscr{A}$ of
$\BBB(\H)$ is called a Kadison-Singer (operator) algebra (or
KS-algebra) if $\mathscr{A}$ is reflexive and maximal with respect
to the diagonal subalgebra $\mathscr{A}\cap \mathscr{A}^*$ of
$\mathscr{A}$, in the sense that if there is another reflexive
subalgebra $\mathscr{B}$ of $\BBB(\H)$ such that
$\mathscr{A}\subseteq\mathscr{B}$ and
$\mathscr{B}\cap\mathscr{B}^*=\mathscr{A}\cap\mathscr{A}^*$, then
$\mathscr{A}=\mathscr{B}$. When the diagonal of a KS-algebra
$\mathscr{A}$ is a factor, we say $\mathscr{A}$ is a Kadison-Singer
factor (or KS-factor). A lattice $\L$ of projections in $\BBB(\H)$
is called a Kadison-Singer lattice (or KS-lattice) if $\L$ is a
minimal reflexive lattice that generates the von Neumann algebra
$\L''$, or equivalently $\L$ is reflexive and $\Alg(\L)$ is a
Kadison-Singer algebra.\newline

In \cite{GY1}, Liming Ge and Wei Yuan constructed examples of
Kadison-Singer algebras with  hyperfinite factors as diagonal. Let
$G_n$ be the free product of $\Z_2$ with itself $n$ times, for
$n\ge2$ or $n=\infty$. Let $\L_{G_n}$ be the group von Neumann
algebra associated to the group $G_n$ (\cite{KR}). If
$U_1,\ldots,U_n$ are canonical generators for $\L_{G_n}$
corresponding to the generators of $G_n$ with $U_j^2=I$. Then
$\frac{I-U_j}2$ ($j=1, \cdots, n$) are projections of trace
$\frac12$. Let $\mathscr{F}_n$ be the lattice consisting of these
$n$ free projections with trace $\frac{1}{2}$ and $0, I$. Then
$\F_n$ is a minimal lattice which generates $\L_{G_n}$ as a von
Neumann algebra. In \cite{GY2}, Ge and Yuan showed that $\Alg(\F_n)$
($n\leq3$) is a Kadison-Singer algebra and
$\Lat(\Alg(\F_3))\setminus \{0,I\}$ is homeomorphic to the
two-dimensional sphere $\mathbb{S}^2$.\newline

In this paper we will give some examples of Kadison-Singer algebras
with trivial diagonal. This paper contains three sections. In
Section 2, we give our main results. We show that the one point
extension of a maximal nest on a separable infinite dimensional
Hilbert space is a Kadison-singer lattice and
 the corresponding algebra is a Kadison-Singer algebra. We also give three examples coming from the one point
extension of an $\N$-ordered maximal nest, a $\Z$-ordered maximal
nest and a continuous maximal nest respectively. 
Let $\NN$ be a von Neumann algebra in $\BBB(\H)$ with a normal faithful tracial state $\tau$. Suppose  $\{P_1, P_2, \cdots, P_n\}$ is a nest with $ P_1\leq  P_2\leq \cdots \leq  P_n=I$ and $\tau(P_k)=\frac{k}{n}$,  $Q$ is a projection with  $\tau(Q)=\frac{1}{n}$. In
Section 3, we show that if $Q$ is free from $\{P_1, P_2, \cdots, P_n\}$ with respect to $\tau$, then the lattice generated by $\{P_1, P_2, \cdots, P_n\}$ and $Q$ is also a Kadison-Singer lattice and $\Alg(\L)$ is a Kadison-Singer algebra.

\section{One point extension of a maximal nest}

Let $\HHH$ be a separable Hilbert space with $dim\HHH = \infty$.
Let $\NNN$ be a nest in $\BBB(\HHH)$ such that $\AAA =\NNN '$ is a
maximal abelian self-adjoint subalgebra of $\BBB(\HHH)$. Suppose
$\xi \in \HHH$ is a separating vector for $\AAA$. Let $P_{\xi}$ be
the orthogonal projection from $\HHH$ onto the one dimensional
subspace of $\HHH$ generated by $\xi$.

\begin{lemma} The von Neumann algebra generated by $\NNN$ and
$P_{\xi}$ is $\BBB(\HHH)$, i.e.,  $\{ \NNN, \PP\xi \}'' =
\BBB(\HHH)$.
\end{lemma}

\noindent\begin{proof} Let $Q$ be a projection in $\{\NNN, \PP\xi
\}' \subset \NNN' = \AAA$. Then $Q\PP\xi = \PP\xi Q$ and therefore
$$ Q \xi = \PP\xi Q\xi=\langle  Q\xi, \xi \rangle \xi.
$$
Since $\xi$ is a separating vector for $\AAA$, we have $Q = \langle
Q\xi, \xi\rangle I$. Since $Q$ is a projection, $Q=0$ or $Q=I$.
Hence $Q\xi=0$ or $Q\xi=\xi$. Thus $\{\NNN, \PP\xi \}'=\mathbb{C}I$
and the result follows.
\end{proof}\newline

Since $\xi$ is a separating vector for $\AAA$, the map $\Phi(Q) =
\langle Q\xi, \xi\rangle$ is  an order preserving homeomorphism of
$(\NNN, <)$ onto a compact subset $\SSS$ of $[0, 1]$. From now on,
for any $t \in \SSS$, we denote $\QQ t = \Phi^{-1}(t)$ the inverse
image of $t$ under $\Phi$. In other words, $\QQ t \in \NNN$ and
$\langle \QQ t \xi , \xi \rangle = t$. By the definition, we have
$\QQ {t_1}< \QQ {t_2}$ for $t_1 < t_2 \in \SSS$ and $\QQ 0 = 0$,
$\QQ 1 = I$. Also we have $\QQ {t_1} = \QQ {t_2} $ if and only if
$t_1 = t_2$.


\begin{lemma} Let $\LLL = \{ \QQ {t} , \QQ {t} \vee \PP {\xi} : t\in \SSS\}$.
Then $\LLL $ is a lattice.
\end{lemma}

\noindent\begin{proof} It suffices to show
\begin{align*}
\QQ {t_1} \wedge  (\QQ {t_2} \vee \PP {\xi}) = \QQ {min(t_1, t_2)},
\end{align*}
for $t_1$ and $t_2$ in $\SSS\backslash  \{1\}$. It is clear that the
above equality holds when $t_1 \leq t_2$. It remains to consider the
case when $t_1
> t_2$. Note that $$\QQ {t_2} \leq \QQ {t_1} \wedge  (\QQ {t_2} \vee \PP
{\xi}) = P.$$ If $P \neq \QQ {t_2}$, then $\QQ {t_2} \vee \PP {\xi}
= P \leq \QQ {t_1}$. It follows that $\xi \in \QQ {t_1}(\HHH)$ and
$\QQ{t_1}\xi=\xi$. Since $\xi$ is separating for $\AAA$, we have
$\QQ {t_1}=I$. This contradicts with the assumption that $t_1\neq 1$
and ends the proof.
\end{proof}\newline

Next we will show that $\LLL$ is a reflexive lattice, i.e., $\LLL =
\Lat(\Alg(\LLL))$.

\begin{lemma}
Suppose $E \in \Lat(\Alg(\LLL))$. If there exist a unit vector
$\zeta$ in $E(\HHH)$ and a $t\in \SSS\backslash \{1\}$ such that
 $(I - \QQ {t}) \zeta \notin span\{ (I
- \QQ {t}) \xi \}$, then $\QQ {t} \leq E$.
\end{lemma}

\noindent\begin{proof} Let
\begin{align*}
\eta = (I - \QQ {t}) \zeta - \left\langle\zeta, \frac{(I - \QQ
{t})\xi}{\|(I - \QQ {t}) \xi \|}\right\rangle\frac{(I - \QQ
{t})\xi}{\|(I - \QQ {t}) \xi \|}.
\end{align*} Then $\eta\neq 0$. For any $\beta \in \QQ {t} (\HHH)$,
we consider the map $T_{\beta}$ from $\HHH$ into $\HHH$ defined by
\begin{align*}
T_{\beta} (\alpha)=\langle \alpha, \eta\rangle \beta, \hskip12pt
\alpha\in \HHH.
\end{align*}
Then $T_{\beta}$ belongs to $\Alg(\LLL)$. Indeed, for any $s \in
\SSS$, $ s \leq t$, if $\alpha \in \QQ {s}( \HHH)$, then
\begin{align*}
T_{\beta}( \alpha) = \langle \alpha, \eta\rangle \beta = \langle(I-
\QQ {t}) \alpha, \eta\rangle \beta = 0.
\end{align*} This shows that $T_{\beta}$ leaves $\QQ {s}$ ($s\in \SSS$, $s\leq t$) invariant.
And obviously $T_{\beta}$ leaves $\PP {\xi}(\HHH)$ and $\QQ
{s}(\HHH) (s
> t)$ invariant. Thus
\begin{align*}
T_{\beta}\zeta = \left(\|(I- \QQ {t})\zeta \|^{2} - |\langle \zeta,
\frac{(I - \QQ {t})\xi}{\|(I - \QQ {t}) \xi \|}\rangle
|^{2}\right)\beta \in E (\HHH).
\end{align*} It follows that $\beta\in E(\HHH)$.
Since $\beta$ is arbitrary in $\QQ {t}(\HHH)$, we have $\QQ {t} \leq
E$.
\end{proof}

\begin{lemma}
Let $\zeta \in \HHH$. If there exist $t_1$ and $ t_2 $ in
$\SSS\backslash \{0, 1\}$ such that
\begin{align*}
(I - \QQ {t_i}) \zeta = a_{t_i}(I - \QQ {t_i})\xi,\hskip4pt  i = 1,
2,
\end{align*}
then $a_{t_1}=a_{t_2}$.
\end{lemma}

\noindent\begin{proof} Without loss of generality, we may assume
that $a_{t_1} = 0$ (from replacing $\zeta$ by $ \zeta-a_{t_1}\xi$).
Then we have
$$
      (I - \QQ {t_1})\zeta = 0  \hskip6pt \mbox{and }\hskip6pt
      (I - \QQ {t_2})\zeta = a_{t_2}(I - \QQ {t_2})\xi.
  $$This is equivalent to $$
      \QQ {t_1}\zeta = \zeta  \hskip6pt\mbox{and}\hskip6pt
      \QQ {t_2}(\zeta - a_{t_2}\xi) = \zeta - a_{t_2}\xi.    $$
It follows from Lemma 2 that $\zeta \in \QQ {t_1} \wedge (\QQ {t_2}
\vee \PP {\xi}) = \QQ {min(t_1, t_2)}$. This implies that
$a_{t_2}\QQ {t_2}\xi = a_{t_2} \xi$. If $a_{t_2} \neq 0$, then $\xi
\in \QQ {t_2}(\HHH)$. Since $\xi$ is a separating vector of $\AAA$,
we have $\QQ{t_2}=I$. This contradicts with the fact that $t_2\neq
1$. This completes the proof.
\end{proof}\newline

The following Corollary is an immediate consequence of the above
lemma.

\begin{corollary}
Let $\zeta \in \HHH$ and $\{t_i\}_{i=1}^{\infty} \subset \SSS
\setminus \{ 0 \}$ be a sequence of numbers with $\lim_{i} t_i = 0$.
If, for any $t_i$, there is a complex number $a_{t_i}$ such that $(I
- \QQ {t_i})\zeta = a_{t_i}(I - \QQ {t_i}) \xi$, then there is a
complex number $a$ such that $\zeta = a\xi$.
\end{corollary}

\begin{lemma}
If $E \in \Lat(\Alg(\LLL)) \setminus \{0 \}$ and $E \neq \PP {\xi}$,
then there exists a $t \in \SSS\backslash \{0\}$ such that $\QQ {t}
\leq E$.
\end{lemma}

\noindent\begin{proof} Since $E \neq \PP {\xi}$, there exists a
$\zeta \in E(\HHH)$ such that $ \zeta \notin span \{\xi \}$. If
there is a $t \in \SSS \setminus \{0, 1 \}$ such that
\begin{align*}
(I - \QQ {t})\zeta \notin span \{(I - \QQ {t})\xi \},
\end{align*}
then we have $\QQ {t} \leq E$ by Lemma 3. Otherwise we know that for
any $t \in \SSS \setminus \{0, 1\}$, $(I - \QQ {t})\zeta \in span
\{(I - \QQ {t})\xi \}$. Because $\zeta \notin span\{ \xi \}$, there
is an $\epsilon > 0$ such that $(0, \epsilon) \cap \SSS = \emptyset$
by Corollary 5. Let
\begin{align*}
t_{0} = inf \{ t \hskip4pt |\hskip4pt (0, t) \cap \SSS \neq
\emptyset, t \in \SSS \}.
\end{align*}
Since $\SSS$ is compact, we have $t_{0} \in \SSS$. Thus $\QQ {t_0}$
is a minimal projection. Denote by $e$ the unit vector which spans
$\QQ {t_0} (\HHH)$. Let
\begin{align*}
\beta = \zeta - \langle \zeta , \xi\rangle \xi.
\end{align*}
Then the linear operator $T_e$ on $\HHH$ defined by
\begin{align*}
T_{e}(\alpha) = \langle \alpha, \beta \rangle e, \hskip12pt  \alpha
\in \HHH,
\end{align*}
is in $\Alg(\LLL)$. This implies that $e \in E(\HHH)$ and $\QQ {t_0}
\leq E$.
\end{proof}

\begin{theorem}
$\LLL$ is a reflexive lattice.
\end{theorem}

\noindent\begin{proof} Suppose $E \in \Lat(\Alg(\LLL)) \setminus
\{0, I \}$. When $E \neq \PP {\xi}$, let
\begin{align*}
t_0 = sup \{ t \in \SSS | \QQ {t} \leq E \}.
\end{align*}
It follows from Lemma 6 that $t_0
> 0$. If $E = \QQ {t_0}$, then the result follows. Now assume $E \neq \QQ
{t_0}$. Given $ \zeta \in E(\HHH)$ and $(I - \QQ {t_0})\zeta \neq
0$. By Lemma 3, we have for any $t\in \SSS$, $ t > t_0$,
\begin{align*}
(I - \QQ {t})\zeta \in span \{ (I - \QQ {t})\xi \}.
\end{align*}
If $(I - \QQ {t_0})\zeta \notin span \{ (I - \QQ {t_0})\xi \}$, then
we have $(t_0, 1) \cap \SSS \neq \emptyset$. Let
\begin{align*}
t_1 = inf \{ t \in \SSS | t > t_0\}.
\end{align*}
It is not hard to show that $1 > t_1 > t_0$, $t_1 \in \SSS$ and
$(t_0, t_1) \cap \SSS = \emptyset$. Let $e$ be the unit vector that
spans $(\QQ {t_1} - \QQ {t_0})\HHH$ and
\begin{align*}
\beta = (I - \QQ {t_0}) \zeta - \left\langle\zeta, \frac{(I - \QQ
{t_0})\xi}{\|(I - \QQ {t_0}) \xi \|} \right\rangle \frac{(I - \QQ
{t_0})\xi}{\|(I - \QQ {t_0}) \xi \|}.
\end{align*}
Then the operator $T_e$ on $\HHH$ defined by
\begin{align*}
T_{e} ( \alpha)=\langle \alpha, \beta \rangle e,\hskip12pt \alpha
\in \HHH,
\end{align*}
is in $\Alg(\LLL)$ and $e \in E(\HHH)$. This means that $\QQ {t_1}
\leq E$ and we get a contradiction. So we have for any $ \zeta \in E
(\HHH)$,
\begin{align*}
(I - \QQ {t_0})\zeta = a(I - \QQ {t_0}) \xi
\end{align*}
for some $a \in \C$. Note that there is a vector $\zeta$ which makes
$a \neq 0$ (otherwise $E = \QQ {t_0}$). Hence $(I - \QQ {t_0}) \xi
\in E(\HHH)$. Since $\QQ {t_0} \xi \in E (\HHH)$, we have $\xi \in E
(\HHH)$ and $\zeta - a\xi \in \QQ {t_0} (\HHH)$. Therefore we have
$E = \QQ {t_0} \vee \PP {\xi}$.
\end{proof}\newline

Next we will show that $\LLL$ is a KS-lattice. Let $\LLL_{0}$ be a
reflexive sub-lattice of $\LLL$.

\begin{lemma}
If $\{ \LLL_{0} \}'' = \BBB(\HHH)$, then $\PP {\xi} \in \LLL_{0}$.
\end{lemma}

\noindent\begin{proof} Suppose $\PP {\xi} \notin \LLL_{0}$. Let
\begin{align*}
t_0 = \inf \{t \in \SSS | \QQ {t} \vee \PP {\xi }  \in \LLL_0 \}.
\end{align*}
Since $\PP {\xi} \notin \LLL_0$, $t_0 > 0$. Also since $\{ \LLL_{0}
\}'' = \BBB(\HHH)$, there must be a $t \in \SSS \setminus \{1 \}$
such that $\QQ {t} \vee \PP {\xi }  \in \LLL_0$. Hence $t_0 < 1$.
Since $\SSS$ is compact, we have $t_0 \in \SSS$. Finally it is easy
to see that $\QQ {t_0} \in \LLL_{0}'$. This is a contradiction and
the result follows. \end{proof}\newline

If there is no non-trival projection $\QQ {t}( \neq 0, I)$ in
$\LLL_{0}$, then $\{\LLL_{0} \}''$ will be abelian. Hence
\begin{align*} \SSS_0 = \{t \in \SSS | \QQ {t} \in \LLL_0 \} \neq
\{0, 1\}.
\end{align*}
By the completeness of $\LLL_0$, we have that $\SSS_0$ is a closed
(and hence compact) subset of $\SSS$.

\begin{lemma}
Suppose $\{ \LLL_{0} \}'' = \BBB(\HHH)$. If $t \in \SSS \setminus
\SSS_0$, then $(t, 1) \cap \SSS_0 \neq \emptyset$.
\end{lemma}

\noindent\begin{proof} Assume that $(t, 1) \cap \SSS_0 = \emptyset$.
Then $dim(I-\QQ {t})(\HHH) \geq 1$. Let
\begin{align*}
t_0 = \sup \{ s \in \SSS_0 | s < t \} < t.
\end{align*}
Then $dim(I-\QQ {t_0})\HHH \geq 2$ and $(t_0, 1) \cap \SSS_0 =
\emptyset$. Now it is easy to see that $I \neq \QQ {t_0} \vee \PP
{\xi } \in \LLL_{0}' $. This is a contradiction and the result
follows.
\end{proof}

\begin{theorem}
If $\LLL_0$ generates $\BBB(\HHH)$, then $\LLL_0 = \LLL$.
\end{theorem}

\noindent\begin{proof} Suppose there is a $t$ in $\SSS \setminus
\SSS_0$. By the lemma above there is a $t_0$ in $(t, 1) \cap
\SSS_0$. Let
\begin{align*}
t_1 &= \sup \{ s \in \SSS_0 | s < t \} < t, \\
t_2 &= \inf \{s \in \SSS_0 | s > t \} \in (t, 1).
\end{align*}
Note that for any $s \in (t_1, t_2) \cap \SSS$, $\QQ {s} \vee \PP
{\xi} \notin \LLL_0$ (otherwise, $\QQ {s} = \QQ {t_0} \wedge (\QQ
{s} \vee \PP {\xi}) \in \LLL_0$ which implies that $s \in \SSS_0$,
but this contradicts with the fact that $t_1 < s < t_2$.) Since $dim
(\QQ {t_2} - \QQ {t_1})\HHH \geq 2$, we can choose a subprojection
$F$ of $(\QQ {t_2} - \QQ {t_1})$ such that $F(\QQ {t_2} - \QQ
{t_1})\xi = 0$. Then $F \in \LLL_{0}'$ and we get a contradiction.
Thus $\SSS_0 = \SSS$. Since $\PP {\xi} \in \LLL_0$, we have $\LLL_0
= \LLL$.
\end{proof}\newline

The following theorem is our main result which follows from above
results.

\begin{theorem}
$\LLL$ is a Kadison-Singer lattice and $\Alg(\LLL)$ is a
Kadison-Singer algebra.
\end{theorem}



In the following, we will give some examples of
Kadison-Singer lattices and Kadison-Singer algebras coming from one point extension of a maximal nest. 
Our first examples comes from the one point extension of a
$\mathbb{Z}$-ordered maximal nest.

\noindent\begin{example} Suppose $\mathscr{H}$ is a separable
infinite dimensional Hilbert space with an orthogonal bases $\{e_i:
i\in \mathbb{N}\}$. For each $i\in \mathbb{N}$, let $P_n$ be the
orthogonal projection of $\mathscr{H}$ onto the linear subspace of
$\H$ generated by $\{e_1, e_2, \cdots, e_n\}$. Let
$\xi=\sum_{n=1}^{\infty}a_ne_n\in\H$ be a vector with all $a_n\neq
0$. Without loss of generality, we can assume that $a_1=1$. Let
$P_{\xi}$ be the orthogonal projection of $\mathscr{H}$ onto the one
dimensional subspace generated by $\mathbb{C}\xi$. Let $0$ and $I$
be the zero operator and identity operator on $\mathscr{H}$
respectively. Since $P_n\wedge P_{\xi}=0$ for any $n\in \N$,
$$\mathscr{L}=\{0, I, P_n, P_{\xi}, P_n\vee P_{\xi}: n\in
\mathbb{N}\}$$ is the lattice generated by $\{P_n: n\in
\mathbb{N}\}$ and $P_{\xi}$. It follows from section 2 that
$\mathscr{L}$ is a Kadison-Singer lattice and $Alg(\mathscr{L})$ is
a Kadison-Singer algebra with diagonal equal to $\mathbb{C}I$..
\end{example}

The second example concerns the one point extension of a
$\mathbb{Z}$-ordered type nest.

\noindent\begin{example} Suppose $\H$ is an infinite dimensional
Hilbert space with orthogonal basis $\{e_n: n\in \mathbb{Z}\}$. For
each $n\in \mathbb{Z}$, let $P_n$ be the orthogonal projection of
$\H$ onto the closed subspace spanned by $\{e_k: k\in \Z, k\leq
n\}$. Note that $lim_{n\rightarrow -\infty}P_n=0$ and
$lim_{n\rightarrow \infty}P_n=I$ in the strong operator topology.
Given a vector $\xi=\sum_{-\infty}^{\infty}a_ke_k\in \H$ with
$a_k\neq 0$ for all $k\in \Z$. Let $P_{\xi}$ be the orthogonal
projection of $\H$ onto the closed subspace of $\H$ spanned by
$\xi$. Let $\L$ be the lattice generated by $P_{n}$ ($n\in \Z$) and
$P_{\xi}$. Then since $P_n\wedge P_{\xi}=0$ for all $n\in \Z$,
$$\L=\{P_n, P_{\xi}, P_{n}\vee P_{\xi}, 0, I: n\in \Z\}.$$ From
results obtained in section 2, $\mathscr{L}$ is a Kadison-Singer
lattice and $Alg(\mathscr{L})$ is a Kadison-Singer algebra with
diagonal equal to $\mathbb{C}I$..
\end{example}

Our last examples is about the continuous nest which is of order
$[0, 1]$.

\noindent\begin{example} Let $\H=L^2[0,1]$ equipped with the inner
product: $$\langle f, g\rangle=\int_{[0, 1]}\overline{g(x)}f(x)dx.$$
For every $f\in L^{\infty}[0, 1]$, define $M_f$ on $\H$ by
$$(M_{f}(g))(x)=f(x)g(x), \forall g\in L^2[0, 1], x\in [0, 1].$$ Then $M_f$ is a
bounded linear operator on $\H$. Let $$\D=\{M_f: f\in L^{\infty}[0,
1]\}.$$ Then $\D$ is a maximal abelian subalgebra of $B(\H)$ (for
the proof of these facts, see \cite{KR}).

For any $t\in [0, 1]$, let $\chi_{[0, t]}$ be the characteristic
function of $[0, t]$ and $P_t=M_{\chi_{[0, t]}}$ be the orthogonal
projection on $L^2[0, 1]$ defined by $P_t(g)=\chi_{[0, t]}g$.
Suppose that $\xi\in \H$ is a measurable function that is nonzero
almost everywhere (we may suppose that $\xi(t)\neq 0$ for all $t\in
[0, 1]$). Let $P_{\xi}$ be the orthogonal projection of $\H$ onto
the one dimensional closed subspace $\mathbb{C}\xi$. Let
$\mathscr{L}$ be the lattice generated by $\{P_t: t\in [0, 1]\}$ and
$P_{\xi}$. Then by results in section 2, we see that $\L$ is a
Kadison-Singer lattice and $Alg(\L)$ is a Kadison-Singer algebra
with diagonal equal to $\mathbb{C}I$.
\end{example}






\section{The lattice generated by a finite nest and a projection free with it}

First we recall some definitions from free probability theory (\cite{VDN}).   Suppose $\NN$ is finite von Neumann
algebra $\NN$ and a faithful normal tracial state $\tau$.   Elements in $\NN$ are
called random variables. Let $(\mathscr{N}, \tau)$ be a $W^{\ast}$-probability space and $\mathcal{I}$ be a fixed index set. A family of unital
subalgebras $\{\mathscr{N}_i: i\in \mathcal I\}$ of $\mathscr{N}$
are free with respect to the trace $\tau$ if $\tau(A_1A_2\cdots
A_n)=0$ whenever $A_j\in \mathscr{N}_{i_j}$ ($i_j\in \mathcal{I}$), $i_1\neq i_2\neq \cdots
\neq i_n$ and $\tau(A_j)=0$ for $1\leq j\leq n$ and every $n$ in
$\mathbb{N}$. The random variables $X_1, X_2, \cdots, X_n$ in $\NN$
are free with respect to $\tau$ if the von Neumann subalgebras
$\mathscr{A}_i$ of $\NN$ generated by $X_i$, respectively, are
free.\newline

Let $n$ be a positive integer such that $n\geq 2$. Suppose $\{P_1, P_2, \cdots, P_n=I\}$ is a family of projections in a finite von Neumann algebra $\NN$($\subset B(\H)$, where $\H$ is a Hilbert space) with faithful tracial state $\tau$ such that $$P_1\leq P_2\leq \cdots\leq P_n$$ and $\tau(P_k)=\frac{k}{n}$ for $k=1, 2, \cdots, n$. Let $Q$ be a projection in $\NN$ free with $\{P_1, P_2, \cdots, P_n\}$ and $\tau(Q)=\frac{1}{n}$. \newline

K. Dykema \cite{D2} and F. Radulescu \cite{R} introduced, independently, the interpolated
free group factors $\mathscr{L}(F_t)$, for $t>1$. These factors can be obtained
from the free group factors by suitable compression with projections.
The following lemma is an easy consequence of Theorem 2.3 in \cite{D1}.\newline

\begin{lemma} The von Neumann algebra $\M$ generated by $\{P_1, P_2, \cdots, P_n\}$ and $Q$ is
$$\{P_1, P_2, \cdots, P_n\}^{''}\ast \{Q\}^{''}\cong \mathscr{L}(F_{1+\frac{1}{n}-\frac{2}{n^2}}).$$
\end{lemma}


\begin{lemma} For any $i\in \{1, 2, \cdots, n-1\}$, we have $P_i\wedge Q=0$.\end{lemma}

\noindent\begin{proof} It suffice to show that $P_1\wedge Q=0$. It is not difficult to  that the $\tau((PQP)^n)\rightarrow 0$ as $n\rightarrow \infty$. Since $(PQP)^n\rightarrow P\wedge Q$ in weak operator topology, we have $\tau(P\wedge Q)=0$ which shows that $P\wedge Q=0$. This completes the proof. \end{proof}\newline

\begin{lemma} Let $$\L=\{0, P_i, Q, P_i\vee Q: i=1, 2, \cdots, n\}.$$ Then $\L$ is the reflexive lattice   generated by $\{P_1, P_2, \cdots, P_n\}$ and $Q$.\end{lemma}

\noindent\begin{proof}  Note that $P_i\wedge Q=0$ for  $i=1, 2, \cdots, n-1$. It is easy to see that  $\L$ is a lattice generated by $\{P_1, P_2, \cdots, P_n\}$ and $Q$. To show $\L$ is distributive, it suffice to show that
$\L$ is distributive. Note we have the following equalities.

(1) $(P_i\vee Q)\wedge P_j=P_{min\{i, j\}}$. We may assume $i\leq j$. Then $P_i\leq P_j$ and $P_i\leq P_i\vee Q$, thus $(P_i\vee Q)\wedge P_j\geq P_{min\{i, j\}}$. But they have same trace (use $\tau(E)+\tau(F)=\tau(E\vee F)+\tau(E\wedge F))$. Hence $(P_i\vee Q)\wedge P_j=P_{min\{i, j\}}$.

(2) For $i, j\in \{1, 2, \cdots, n-1\}$, we have $(P_i\vee P_j)\wedge Q=0$. 

Use this two formulas it is easy to show that $\L$ is distributive. It follows from \cite{Ha} that $\L$  is a reflexive lattice.
\end{proof}\newline

\begin{lemma} $\L$ is the minimal lattice that generate the von Neumann algebra $\M$. \end{lemma}

\noindent\begin{proof}  Let $\L_0$ be a reflexive sublattice of $\L$ such that ${\L_0}^{''}=\M=\L^{''}$. We want to show that $\L_0=\L$.
First we show that $Q\in \L_0$. Indeed if $Q\notin \L_0$, then $P_1\in {\L_0}^{'}=\L^{'}$ which contradicts with the fact that $PQ\neq QP$. Thus $Q\in \L_0$.
Now we show $\forall i\in \{1, 2, \cdots, n-1\}$, we have $P_i\in \L_0$.
Let $k$ be the minimal number in $\{1, 2, \cdots, n-1\}$ such that $P_1, \cdots, P_{k-1}\in \L_0$, $P_k\notin \L_0$.
Then we must have $P_{k}\wedge Q\notin \L_0$ (otherwise suppose $P_k\wedge Q\in \L_0$. If $P_j\notin \L_0$ for all $j\neq k+1$, $j\neq n$, then $P_{k}\wedge Q\in {\L_0}^{'}=\L^{'}$ which is a contradiction). Hence $\L_{0}^{''}$ is a von Neumann algebra containing in $\mathscr{L}(F_t)$ with $t=1+\frac{n-4}{n^2}$ and $\L_{0}^{''}\neq \L^{''}$. This is again a contradiction and shows that $\{P_1, P_2, \cdots, P_n-1\}\subset \L_0$.
This shows that $\L_0=\L$ and completes the proof. \end{proof}\newline

It follows from above Lemmas that we have the following result.\newline


\begin{theorem} $\L$ is a Kadison-Singer lattice and $\Alg\L$ is a Kadison-Singer algebra.\end{theorem}




\noindent\textbf{Acknowledgments} The authors want to thank
Professor  Liming Ge and Chengjun Hou for their encouragement and useful discussions.
\newline


\begin{thebibliography}{\sl 110}

\baselineskip=16pt \parskip=4pt

\bibitem{RR} H. Radjavi and P. Rosenthal, {Invariant Subspaces},
Springer--Verlag, Berlin, 1973.

\bibitem{Ar} W. Arveson, {
Operator algebras and invariant subspaces,} Ann. of Math. {100}
(1974), 433--532.

\bibitem{KS} R. Kadison and I. Singer, { Triangular operator algebras.
Fundamentals and hyperreducible theory}, Amer. J. Math., {82 }
(1960), 227--259.

\bibitem{R} J. Ringrose, { On some algebras of operators,} Proc. London
Math. Soc. (3) {15} (1965), 61--83.

\bibitem{Da}  K. Davidson, {Nest Algebras,} Research Notes in Math.
vol. 191, Pitman, Boston-London-Melbourne, 1988.

\bibitem{GY1} L. Ge and W. Yuan, { Kadison-Singer algebras,
I---hyperfinite case,} PNAS, (5)107(2010), 1838-1843.

\bibitem{KR} R.\ Kadison and J.\ Ringrose, { Fundamentals of the
Operator Algebras}, vols. I and II, Academic Press, Orlando, 1983
and 1986.

\bibitem{GY2} L. Ge and W. Yuan, {Kadison-Singer algebras,
II,} PNAS , (11)107(2010), 4840-4844.

\bibitem{VDN} D. Voiculescu, K. Dykema and A. Nica, \emph{Free
Random Variable}, CRM Monograph Series, Vol. 1, A. M.S. Providence,
RI, 1992.

\bibitem{D1} K. Dykema, Free products of hyperfinite von Neumann algebras and free dimension
Duke Math. J. 69 (1993), 97-119.

\bibitem{D2} K. Dykema, Interpolated free group factors
Pacific J. Math. 163 (1994), 123-135.

\bibitem{R} F. Radulescu, Random matrices, Amalgamated free products and subfactors of the von
Neumann algebra of a free group, Invent. Math. 115 (1994), 347-389.

\bibitem{Ha}
K.Harrison, On lattices of invariant subspaces. Doctoral Thesis, 1970.

\end{thebibliography}

\end{document}
