\documentclass[12pt]{article}
\usepackage{mathrsfs}
\usepackage{amssymb}

%%%%%%%%%%%%%%%%%%%%%%%%%%%%%%%%%%%%%%%%%%%%%%%%%%%%%%%%%%%%%%%%%%%%%%%%%%%%%%%%%%%%%%%%%%%%%%%%%%%%
\usepackage{graphicx}
\usepackage{amsmath}

%TCIDATA{OutputFilter=LATEX.DLL}
%TCIDATA{Created=Tue Nov 12 09:44:21 2002}
%TCIDATA{LastRevised=Wednesday, April 02, 2003 18:37:24}
%TCIDATA{<META NAME="GraphicsSave" CONTENT="32">}
%TCIDATA{<META NAME="DocumentShell" CONTENT="General\Blank Document">}
%TCIDATA{Language=American English}
%TCIDATA{CSTFile=LaTeX article (bright).cst}

\newtheorem{theorem}{Theorem}
\newtheorem{acknowledgement}[theorem]{Acknowledgement}
\newtheorem{algorithm}[theorem]{Algorithm}
\newtheorem{axiom}[theorem]{Axiom}
\newtheorem{case}[theorem]{Case}
\newtheorem{claim}[theorem]{Claim}
\newtheorem{conclusion}[theorem]{Conclusion}
\newtheorem{condition}[theorem]{Condition}
\newtheorem{conjecture}[theorem]{Conjecture}
\newtheorem{corollary}[theorem]{Corollary}
\newtheorem{criterion}[theorem]{Criterion}
\newtheorem{definition}[theorem]{Definition}
\newtheorem{example}[theorem]{Example}
\newtheorem{exercise}[theorem]{Exercise}
\newtheorem{lemma}[theorem]{Lemma}
\newtheorem{notation}[theorem]{Notation}
\newtheorem{problem}[theorem]{Problem}
\newtheorem{proposition}[theorem]{Proposition}
\newtheorem{remark}[theorem]{Remark}
\newtheorem{solution}[theorem]{Solution}
\newtheorem{summary}[theorem]{Summary}
\newenvironment{proof}[1][Proof]{\textbf{#1.} }{\ \rule{0.5em}{0.5em}}
 \setcounter{page}{1}
 \setlength{\topmargin}{-0.2in}
 \setlength{\textheight}{8.5in}
 \setlength{\textwidth}{6.3in}
 \setlength{\oddsidemargin}{0.2in}
\begin{document}


\title{\textbf{Central sequence algebras of the tensor product of $\mathrm{II}_{1}$ facotrs}}



\author{\textbf{Wenming Wu\thanks{Wenming Wu was partially supported
by Natural Science
Foundation Project of CQ CSTC (No. CSTC, 2010BB9318) and Chongqing Normal University(No.10XLZ001).}} \\
{\small College of Mathematical Sciences}\\
{\small Chongqing Normal University, Chongqing, 400047, China}\\
\textbf{Wei Yuan}\\
{\small Academy of Mathematics and Systems Science}\\
{\small Chinese Academy of Science, Beijing, 100084, China}\\
{\small E-mail: wuwm@amss.ac.cn, yuanwei.cn@gmail.com} }
\date{}
\maketitle
\def\M{{\mathscr{M}}}
\baselineskip=12.5pt \vskip0.1cm
\begin{quote}
\textbf{Abstract:} { Let $\mathcal{M}$ and $\mathcal{N}$ be two type
$\mathrm{II}_{1}$ factors and $\omega$ a free ultrafilter for the
natural number $\mathbb{N}$. It is shown that, if the central
sequence algebra $\mathcal{N}_{\omega}$ is abelian and there is a
non-atomic abelian subalgebra $\mathcal{A}$ in $\mathcal{M}$ such
that
$(\mathcal{M}\otimes\mathcal{N})_{\omega}\subseteq(\mathcal{A}\otimes\mathcal{N})^{\omega}$,
then $(\mathcal{M}\otimes\mathcal{N})_{\omega}$ is abelian. It is
also given two specify factors of type $\mathrm{II}_{1}$ which
satisfy the above-mentioned conditions.}

\textbf{Keywords:} {Type II$_1 $ factor; Central sequence algebras;
Tensor product; Crossed product}

{\small \textbf{2000 MR Subject Classification:} 46L10}
\end{quote}


\section{Introduction and Preliminaries}

J.von Neumann introduced one kind of rings, which now are called von
Neumann algebras, consisting of some bounded linear operators acting
on some Hilbert space. These algebras are strong-operator closed
self-adjoint subalgebras of the algebra of all bounded linear
operators acting on a Hilbert space. Factors are von Neumann
algebras whose center consists of the scalar multiples of the
identity operator. Every von Neumann algebras is the direct sum or
direct integral of factors. Thus factors are the building blocks for
all von Neumann algebras. Murray and von Neumann classified factors
by means of a relative dimension function. Finite factors are those
for which this dimension function has a bounded range. Finite
factors whose dimension function has an infinite range are referred
as the factors of type $\mathrm{II}_{1}$. For $\mathrm{II}_{1}$
factors, this dimension function gives rise to a unique normalized
faithful tracial state.

By using of the ideas of central sequences, Murray and von Neumann
showed that there are two non-isomorphic factors of type
$\mathrm{II}_{1}$. Central sequences in a finite factor form an
algebra. The central sequence algebra for a factor of type
$\mathrm{II}_{1}$ can be viewed as the relative commutant of the
algebra in the ultrapower of the algebra constructed from a given
free ultrafilter on the natural number $\mathbb{N}$. Dixmier
{\cite{[Dix]}} showed that the central sequence algebra of a factor
of type $\mathrm{II}_{1}$ is either trivial (the scale multiples of
the identity) or non-atomic. Indeedly the central sequence algebra
for a factor of type $\mathrm{II}_{1}$ is non-trivial if and only if
the origin factor has the property $\Gamma$. Furthermore McDuff
{\cite{[MD]}} showed that if the central sequence algebra of a
$\mathrm{II}_{1}$ factor is non-commutative, then it is a von
Neumann algebra of type $\mathrm{II}_{1}$. She also showed that the
central sequence algebra of a $\mathrm{II}_{1}$ factor is
noncommutative if and only if the origin factor is isomorphic to the
tensor product of the hyperfinite $\mathrm{II}_{1}$ factor with
itself.

The factor with trivial central sequence algebra, which is
equivalent to that it has not the property $\Gamma$, is referred to
as the full factor. A.Connes {\cite{[Co]}} shown that the tensor
product von Neumann algebra of two $\mathrm{II}_{1}$ factors is full
if and only if both the two factors are full. Thus the central
sequence algebra of the tensor product of two $\mathrm{II}_{1}$
factors with trivial sequence algebra is still full. Of course, if
one of the central sequence algebras of two $\mathrm{II}_{1}$
factors is non-commutative (i.e.one of two factors is McDuff
factor), then the central sequence algebra of the tensor product of
them is non-commutative. However, given two type $\mathrm{II}_{1}$
factors with abelian and nontrivial central sequence algebra, it is
not known whether the central sequence algebra of the tensor product
of these factors is still abelian and nontrivial. Recently, Fang, Ge
and Li {\cite{[FGL]}} have gotten that the central sequence algebra
of the injective $\mathrm{II}_{1}$ factor is prime, then the central
sequence algebra of the tensor product of two factors may not be the
tensor product of the central sequence algebras of the factors.
Nowadays the above-mentioned question is still open.

In this paper, we partly answer the previous problem in positive. We
in fact restrict our attention to the factors of type
$\mathrm{II}_{1}$ whose central sequence algebras satisfy some
particular conditions. Let $\mathcal{M}$ and $\mathcal{N}$ be two
$\mathrm{II}_{1}$ factors with separable predual. If the central
sequence algebra of one of the two factors, say $\mathcal{N}$, is
abelian and the central sequence algebra of their tensor product is
contained in the ultrapower of the tensor product of $\mathcal{N}$
and a non-atomic abelian von Neumann subalgebra of $\mathcal{M}$,
then the central sequence algebra of the two factors is abelian. As
a corollary, under the previous conditions, we can see that the
tensor product of two $\mathrm{II}_{1}$ factors, which have the
property $\Gamma$ and are not McDuff factors, also has the property
$\Gamma$ and are not McDuff factor.



This paper is organized as follows. In the rest part of this
section, we firstly recall the basic concepts and properties of the
central sequence algebras. In the second section, we get the main
result of this paper. In the last section, we give two specify
$\mathrm{II}_{1}$ factors which satisfy the above-mentioned
conditions.

Now let's recall the basic facts of the central sequence in
$\mathrm{II}_{1}$ factors. Let $\mathcal{M}$ be a $\mathrm{II}_{1}$
factor with separable predual and $\tau$ the unique faithful normal
tracial state on $\mathcal{M}$. For any $X\in\mathcal{M}$,
$\|X\|_{2}=\tau(X^{*}X)^{1/2}$ is the trace norm induced by $\tau$.
Suppose that $(T_{n})$ is a bounded sequence of elements of
$\mathcal{M}$. Then $(T_{n})$ is said to be a {\it central sequence}
in $\mathcal{M}$ if
$$\|[T_{n},A]||_{2}=\|T_{n}A-AT_{n}\|_{2}\rightarrow0$$
as $n\rightarrow\infty$ for any $A\in\mathcal{M}$. Note that any
subsequence of a central sequence is still a central sequence. The
central sequence $(T_{n})$ is called {\it non-trivial} if
$\|T_{n}-\tau(T_{n})I||_{2}\nrightarrow0$ as $n\rightarrow\infty$.

Suppose that $\omega$ is a free ultrafilter of the natural number
$\mathbb{N}$. Let $l^{\infty}(\mathcal{M})$ be the set of all
bounded sequence in $\mathcal{M}$, and let
$$I_{\omega}=\{(T_{n})\in l^{\infty}(\mathcal{M}):\lim_{\omega}\|T_{n}\|_{2}=0\}.$$
Then $I_{\omega}$ is a maximal ideal in $l^{\infty}(\mathcal{M})$.
Thus $l^{\infty}(\mathcal{M})/I_{\omega}$ forms a $\mathrm{II}_{1}$
factor {\cite{[Sa]}} which is said to be the ultrapoqer of
$\mathcal{M}$ and denoted by $\mathcal{M}^{\omega}$. There is a
canonical imbedding of $\mathcal{M}$ into $\mathcal{M}^{\omega}$.
Then under the meaning of imbedding, for any element $(T_{n})$ in
the relative commutant $\mathcal{M}'\cap\mathcal{M}^{\omega}$, there
is a subsequence of $(T_{n})$ which is a central sequence. Thus we
say that the relative commutant
$\mathcal{M}'\cap\mathcal{M}^{\omega}$ is the {\it central sequence
algebra} of $\mathcal{M}$ and denoted by $\mathcal{M}_{\omega}$.



\section{Main result}

Suppose that $\mathcal{M}$ and $\mathcal{N}$ are factors of type
$\mathrm{II}_{1}$ acting on the separable Hilbert spaces
$\mathcal{H}$ and $\mathcal{K}$ respectively. Let $\tau$ be the
faithful normal tracial states on $\mathcal{M}$  and $\mathcal{N}$
for the convenience. Let $\omega$ be a free ultrafilter on
$\mathbb{N}$ and $\mathcal{M}_{\omega}$ and $\mathcal{N}_{\omega}$
be the central sequence algebras of $\mathcal{M}$ and $\mathcal{N}$
respectively. \vskip6pt

\noindent{\bf Theorem 2.1.}  With notations as above. If
$\mathcal{N}_{\omega}$ is abelian and there is a maxiaml abelian von
Neumann subalgebra $\mathcal{A}$ of $\mathcal{M}$ such that
$$(\mathcal{M}\otimes\mathcal{N})_{\omega}\subset(\mathcal{A}\otimes\mathcal{N})^{\omega},$$
then $(\mathcal{M}\otimes\mathcal{N})_{\omega}$ is abelian.

\vskip6pt As McDuff showed that the central sequence algebra of a
factor of type $\mathrm{II}_{1}$ is abelian if and only if any two
central sequences in it are commutating {\cite{[MD]}}, thus we just
need to show that any two central sequences in
$\mathcal{M}\otimes\mathcal{N}$ are commutating.

Now we firstly discuss the tensor product
$\mathcal{A}\otimes\mathcal{N}$. As $\mathcal{A}$ is a maximal
abelian von Neumann subalgebra of $\mathcal{M}$, then $\mathcal{A}$
is non-atomic. Thus $\mathcal{A}$ is unitary equivalent to
$\mathcal{A}_{0}\otimes\mathbb{C}I$ where
$\mathcal{H}\cong\mathcal{H}_{1}\otimes\mathcal{H}_{2}$ and
$\mathcal{A}_{0}$ is a maximal non-atomic abelian von Neumann
subalgebra of $\mathcal{B}(\mathcal{H}_{1})$ {\cite{[KR]}}. Then
there is a unitary operator $W$ from $\mathcal{H}_{1}$ onto the
Hilbert space $L^{2}([0,1],\mu)$ such that
$\varphi(\mathcal{A}_{0})$ is identified with the multiplication
algebra $L^{\infty}([0,1],\mu)$, where $\varphi$ is the mapping
defined as $A\rightarrow WAW^{*}$ for any $A\in\mathcal{A}_{0}$ and
$\mu$ is the Lebesgue measure on the interval $[0,1]$. Furthermore,
we have $\tau(A)=\int_{[0,1]}\varphi(A)(x)d\mu(x)$ for any
$A\in\mathcal{A}_{0}$. For the convenience, we identify
$\mathcal{H}_{1}$ with $L^{2}([0,1],\mu)$, $\mathcal{A}$ with
$L^{\infty}([0,1],\mu)$ and $A$ with $\varphi(A)$. Thus
$\mathcal{H}\otimes\mathcal{K}=\mathcal{H}_{2}\otimes
L^{2}([0,1],\mu)\otimes\mathcal{K}$.

Note that the Hilbert space $L^{2}([0,1],\mu)\otimes\mathcal{K}$ is
identified with $L^{2}([0,1],\mathcal{K})$ where the inner product
is given by {\cite{[Ta]}}
$$\langle\xi,\eta\rangle=\int_{[0,1]}\langle\xi(x),\eta(x)\rangle d\mu(x), \xi,\eta\in L^{2}([0,1],\mathcal{K}).$$
Furthermore, let $\eta\in\mathcal{K}$ and $E$ be a measurable subset
of $[0,1]$ with respect to the Lebesgue measure $\mu$. Then
$\chi_{E}\eta$ is a vector of $L^{2}([0,1],\mu)\otimes\mathcal{K}$
where $\chi_{E}$ is the characteristic function of $E$. Each element
in the linear span of all $\chi_{E}\eta$ is referred to as the {\it
simple functions} with values in $\mathcal{K}$ on the interval
$[0,1]$.




Similar to the elementary real analysis, we have the following
technique results.\vskip6pt

\noindent{\bf Lemma 2.2.} For any $\xi\in L^{2}([0,1],\mathcal{K})$,
there is a sequence of simple functions
$\{\xi_{n}(=\sum_{k}\chi_{A_{n,k}}\eta_{n,k})\}$ such that
$\xi_{n}(x)\rightarrow\xi(x)$ for almost every $x\in [0,1]$ and
$\xi_{n}\rightarrow\xi$ under the norm of $L^{2}([0,1],\mathcal{K})$
as  $n\rightarrow\infty$. \vskip6pt

\noindent{\it Proof.} For any $\xi\in L^{2}([0,1],\mathcal{K})$,
there is a sequence $(\sum_{k}f_{n,k}\otimes\eta_{n,k})_{n}$ of the
linear combinations of simple tensors, where $f_{n,k}\in
L^{2}([0,1],\mu)$ and $\eta_{n,k}\in\mathcal{K}$, such that
$$\sum_{k}f_{n,k}\otimes\eta_{n,k}\rightarrow\xi$$
in $L^{2}([0,1],\mathcal{K})$ as $n\rightarrow\infty$. It is
equivalent to
$$\int_{[0,1]}||\xi(x)-(\sum_{k}f_{n,k}\otimes\eta_{n,k})(x)||^{2}d\mu(x)\rightarrow 0$$ as
$n\rightarrow\infty$. Thus there is a subsequence of
$(\sum_{k}f_{n,k}\otimes\eta_{n,k})_{n}$ which is converges to $\xi$
for almost every $x\in [0,1]$. So we just need to show that the
result holds for any simple tensor in $L^{2}([0,1],\mathcal{K})$.

Suppose that $f\otimes\eta\in
L^{2}([0,1],\mu)\otimes\mathcal{K}(=L^{2}([0,1],\mathcal{K}))$, then
there is a sequence of simple functions $(f_{n})\in
L^{2}([0,1],\mu)$ such that $f_{n}\rightarrow f$ in
$L^{2}([0,1],\mu)$ and $f_{n}(x)\rightarrow f(x)$ for almost every
$x\in [0,1]$. Thus the sequence $(f_{n}\otimes\eta)(=f_{n}\eta)$ is
convergent to $f\otimes\eta$ and $f_{n}(x)\eta\rightarrow f(x)\eta$
for almost every $x\in [0,1]$.\endproof

\vskip4pt

Now we discuss the action of the von Neumann algebra
$L^{\infty}([0,1])\otimes\mathcal{B}(\mathcal{K})$ on
$L^{2}([0,1],\mathcal{K})$.

\vskip6pt

\noindent{\bf Lemma 2.3.} Let $T\in\mathcal{B}(\mathcal{K})$ and
$f\in L^{\infty}([0,1])$. For any $\xi\in
L^{2}([0,1],\mu)\otimes\mathcal{K}$, we have $$((I\otimes
T)\xi)(x)=T\xi(x), \  \ ((f\otimes I)\xi)(x)=f(x)\xi(x)$$ for almost
every $x\in [0,1]$.

\vskip6pt

\noindent{\it Proof.} Let
$(\xi_{n}(=\sum_{k}\chi_{E_{n,k}}\eta_{n,k}))$ be a sequence of
simple functions in $L^{2}([0,1],\mathcal{K})$ such that
$\xi_{n}(x)\rightarrow\xi(x)$ for almost every $x\in [0,1]$ and
$\xi_{n}\rightarrow\xi$ in $L^{2}([0,1],\mathcal{K})$.

Then we have $(I\otimes T)\xi_{n}\rightarrow (I\otimes T)\xi$. Thus
there is a subsequence $(\xi_{n_{j}})$ of $(\xi_{n})$ such that
$((I\otimes T)\xi_{n_{j}})(x)\rightarrow ((I\otimes T)\xi)(x)$ for
almost every $x\in [0,1]$. For the convenience, we use the sequence
$(\xi_{n})$ to denote the subsequence. Hence, we have
\begin{align*}
((I\otimes
T)\xi_{n})(x)=\sum_{k}\chi_{E_{n,k}}(x)T\eta_{n,k}\rightarrow((I\otimes
T)\xi)(x)
\end{align*}
for almost every $x\in [0,1]$.

At the same time, as $\xi_{n}(x)\rightarrow\xi(x)$ for almost every
$x\in [0,1]$, thus we have
$$T\xi_{n}(x)=T(\sum_{k}\chi_{E_{n,k}}\otimes\eta_{n,k})(x)=\sum_{k}\chi_{E_{n,k}}(x)T\eta_{n,k}\rightarrow T\xi(x)$$
as $n\rightarrow\infty$ almost everywhere. Therefore, we get
$$((I\otimes T)\xi)(x)=T\xi(x)$$
almost everywhere.

Similar to the above argument, we also have $((f\otimes I)\xi)(x)=f(x)\xi(x)$ almost everywhere.
\endproof

According to the above results, it is easy to know that $f\otimes T$
is a decomposable operator for any $f\in L^{\infty}([0,1],\mu)$ and
any $T\in\mathcal{B}(\mathcal{K})$. As
$L^{\infty}([0,1],\mu)\otimes\mathcal{N}$ commutes with the diagonal
algebra $L^{\infty}([0,1])\otimes \mathbb{C}I$, thus
$L^{\infty}([0,1],\mu)\otimes\mathcal{N}$ is a von Neumann algebra
consisting of decomposable operators. Furthermore, we have the
following conclusion.

\vskip6pt

\noindent{\bf Lemma 2.4.} For any $T\in
L^{\infty}([0,1],\mu)\otimes\mathcal{N}$, we have
$T(x)\in\mathcal{N}$ almost everywhere. \vskip6pt

\noindent{\it Proof.} As the Hilbert space $\mathcal{K}$ is
separable, then there exists a strong operator topology dense
sequence $\{A_{n}'\}$ of operators in the unit ball
$(\mathcal{N}')_{1}$ of the commutant of $\mathcal{N}$. Then for any
$\xi\in L^{2}([0,1],\mu)\otimes\mathcal{K}$, as $T$ is decomposable,
we have
$$||T(I\otimes A_{n}')\xi-(I\otimes A_{n}')T\xi||^{2}=\int_{[0,1]}||T(x)A_{n}'\xi(x)-A_{n}'T(x)\xi(x)||^{2}d\mu(x)=0.$$

Thus $\|T(x)A_{n}'\xi(x)-A_{n}'T(x)\xi(x)\|=0$ almost everywhere for any $n=1,2,\cdots$ and
$\xi\in L^{2}([0,1],\mu)\otimes\mathcal{K}$. In particular, pick a
dense sequence $\{\eta_{k}:k\in\mathbb{N}\}$ in $\mathcal{K}$, then we have
$$\|T(x)A_{n}'\eta_{k}-A_{n}'T(x)\eta_{k}||=0$$
almost everywhere for any $n$ and $k$. Thus we get $\|T(x)A_{n}'-A_{n}'T(x)\|=0$ almost everywhere.
Hence we have $T(x)\in\mathcal{N}$ almost everywhere.
\endproof
\vskip6pt

\noindent{\bf Remark 2.5.} Note that
$L^{\infty}([0,1],\mu)\otimes\mathcal{N}$ is a von Neumann
subalgebra of $\mathcal{M}\otimes\mathcal{N}$ up to some
isomorphism. For any $T\in L^{\infty}([0,1],\mu)\otimes\mathcal{N}$,
we have
\begin{align*}\|T\|_{2}&=\tau(T^{*}T)^{\frac{1}{2}}=(\int_{[0,1]}\tau(T(x)^{*}T(x))d\mu(x))^{\frac{1}{2}}\\
&=(\int_{[0,1]}\|T(x)\|^{2}_{2}d\mu(x))^{\frac{1}{2}}.
\end{align*}

Now according to the result of McDuff, to get theorem 1, we just
need to show that any two central sequences of
$\mathcal{M}\otimes\mathcal{N}$ are commutating. In the lemma that
follows, we prove a technique result about the central sequences of
$\mathcal{M}\otimes\mathcal{N}$ under the condition
$(\mathcal{M}\otimes\mathcal{N})\subset(\mathcal{A}\otimes\mathcal{N})^{\omega}$.

\vskip6pt

\noindent{\bf Lemma 2.6.} Suppose that
$(T_{n})\in(L^{\infty}([0,1],\mu)\otimes\mathcal{N})^{\omega}$ is a
central sequence of $\mathcal{M}\otimes\mathcal{N}$. There is a
subsequence $(T_{n_{k}})$ of $(T_{n})$ such that
$(T_{n_{k}}(x))\in\mathcal{N}_{\omega}$ almost everywhere. \vskip6pt

\noindent{\it Proof.} Without loss of generality, we can suppose
that $\|T_{n}\|=1$. As $(T_{n})$ is a central sequence, then for any
$A\in\mathcal{N}$, we have
\begin{align}\|T_{n}(I\otimes A)-(I\otimes A)T_{n}\|_{2}^{2}=\int_{[0,1]}\|T_{n}(x)A-AT_{n}(x)\|^{2}_{2}d\mu(x)\rightarrow0
\end{align}
as $n\rightarrow\infty$. Since $\mathcal{N}$ has separable predual,
we can pick a $\|\cdot\|_{2}-$dense sequence $\{A_{k}\}$ in the unit ball $\mathcal{N}_{1}$ of $\mathcal{N}$.

For the operator $A_{1}$, by the equation (1), there is a
subsequence $(T_{n}^{(1)})$ of $(T_{n})$ such that
$$\|T_{n}^{(1)}(x)A_{1}-A_{1}T_{n}^{(1)}(x)\|^{2}_{2}\rightarrow0$$
as $n\rightarrow\infty$ almost everywhere. Note that (1) also holds
for the subsequence $(T_{n}^{1})$. Then for the operator $A_{2}$,
 there exists a subsequence $(T_{n}^{(2)})$ of $(T_{n}^{(1)})$ such that
$$\|T_{n}^{(2)}(x)A_{i}-A_{i}T_{n}^{(2)}(x)\|^{2}_{2}\rightarrow0$$
as $n\rightarrow\infty$ almost everywhere for $i=1,2$.

By induction on the positive integer $k$, we can find a subsequence
$(T_{n}^{(k+1)})$ of $(T_{n}^{(k)})$ such that
$$\|T_{n}^{(k+1)}(x)A_{i}-A_{i}T_{n}^{(k+1)}(x)\|^{2}_{2}\rightarrow0$$
as $n\rightarrow\infty$ almost everywhere for $i=1,2,\cdots,k+1$. Thus we can construct a sequence $\{\{T_{n}^{(k)}\}:k=1,2,\cdots\}$
of subsequences of $\{T_{n}\}$
such that
$$\|T_{n}^{(k)}(x)A_{i}-A_{i}T_{n}^{(k)}(x)\|^{2}_{2}\rightarrow0$$
as $n\rightarrow\infty$ almost everywhere for $1\leq i\leq k$.

Now we define $R_{n}=T_{n}^{(n)}$ for any $n\in\mathbb{N}$. Then $\{R_{n}\}$ is a subsequence of
$\{T_{n}\}$ such that for any $A_{k},k=1,2,\cdots$, we have
$$\|R_{n}(x)A_{k}-A_{k}R_{n}(x)\|^{2}_{2}\rightarrow0$$
as $n\rightarrow\infty$ almost everywhere.

Note that the sequence $(R_{n}(x))\in\mathcal{N}^{\omega}$ almost everywhere since $\{R_{n}\}$ is a subsequence of $\{T_{n}\}$ and all
$T_{n},n=1,2,\cdots$ are decomposable operators. Thus $(R_{n}(x))\in\mathcal{N}_{\omega}$ almost everywhere.
\endproof

Note that any subsequence of a central sequence of
$\mathcal{M}\otimes\mathcal{N}$ is still a central sequence. Now we
can show that the theorem 2.1. holds. \vskip6pt



\noindent{\it Proof of theorem 1.} Let $(T_{n})$ and $(S_{n})$ be
two central sequences of $\mathcal{M}\otimes\mathcal{N}$. Then we
have
$$(T_{n}),\  \ (S_{n})\in(L^{\infty}([0,1],\mu)\otimes\mathcal{N})^{\omega}.$$
Let $a_{n}=\|T_{n}S_{n}-S_{n}T_{n}\|_{2}$. Then $0\leq a_{n}\leq
2\|(T_{n})\|\|(S_{n})\|$, thus $(a_{n})$ is a bounded real number
sequence. \vskip4pt

\noindent{\bf Claim.} For any subsequence $(a_{n}^{(1)})$ of
$(a_{n})$, there is a subsequence $(a_{n}^{(2)})$ of $(a_{n}^{(1)})$
such that $a_{n}^{(2)}\rightarrow0$ as $n\rightarrow\infty$.
 \vskip4pt

In fact, for the subsequence $(a_{n}^{(1)})$, let $(T_{n}^{(1)})$
and $(S_{n}^{(1)})$ be the corresponding subsequences of $(T_{n})$
and $(S_{n})$ such that
$$a_{n}^{(1)}=\|T_{n}^{(1)}S_{n}^{(1)}-S_{n}^{(1)}T_{n}^{(1)}\|_{2}.$$
Then according to the above lemmas, there is a subsequence
$(T_{n}^{(2)})$ of $(T_{n}^{(1)})$ such that
$(T_{n}^{(2)}(x))\in\mathcal{N}_{\omega}$ almost everywhere. As the
corresponding subsequence $(S_{n}^{(2)})$ of $(S_{n}^{1})$ still is
a central sequence and
$(S_{n}^{(2)})\in(L^{\infty}([0,1],\mu)\otimes\mathcal{N})^{\omega}$,
thus there is a subsequence $(S_{n}^{(3)})$ of $(S_{n}^{(2)})$ such
that $(S_{n}^{(3)}(x))\in\mathcal{N}_{\omega}$ almost everywhere.

Up to a measurable subset of $[0,1]$ with measure zero, we define a
sequence of essential bounded measurable functions $(f_{n})$ as
following
$$f_{n}(x)=\|T_{n}^{(3)}(x)S_{n}^{(3)}(x)-S_{n}^{(3)}(x)T_{n}^{(3)}(x)\|_{2},\
\ x\in[0,1].$$ Then $\|f_{n}(x)\|\leq 2\|(T_{n})\|\|(S_{n})\|$
almost everywhere. Since $(S_{n}^{(3)}(x))\in\mathcal{N}_{\omega}$
and $(T_{n}^{(3)}(x))\in\mathcal{N}_{\omega}$ almost everywhere and
$\mathcal{N}_{\omega}$ is abelian, then $f_{n}(x)\rightarrow
0(n\rightarrow\infty)$ almost everywhere. Hence by the Lesbegue
dominated converge theorem, we have
\begin{align*}
\lim_{n\rightarrow\infty}a_{n}^{(3)}&=\lim_{n\rightarrow\infty}(\int_{[0,1]} |f_{n}(x)|^{2}d\mu(x))^{\frac{1}{2}}\\
&=(\int_{[0,1]}\lim_{n\rightarrow\infty}f_{n}(x)^{2}d\mu(x))^{\frac{1}{2}}=0.
\end{align*}
Thus the claim holds.

Now as for any subsequence $(a_{n}^{(1)})$ of $(a_{n})$, there is a
subsequence $(a_{n}^{(2)})$ of $(a_{n}^{(1)})$ such that
$a_{n}^{(2)}\rightarrow0$, thus we have $a_{n}\rightarrow 0$. Hence
$\|T_{n}S_{n}-S_{n}T_{n}\|_{2}\rightarrow 0$ as
$n\rightarrow\infty$. Thus we have shown that
$(\mathcal{M}\otimes\mathcal{N})_{\omega}$ is abelian.\endproof

\vskip6pt

\noindent{\bf Remark 2.7.} According to the above argument, the
result also holds under the further assumption which $\mathcal{A}$
is a non-atomic abelian von Neumann subalgebra of $\mathcal{M}$.
Furthermore, under the condition of the theorem 2.1., if the
$\mathrm{II}_{1}$ factors $\mathcal{M}$ and $\mathcal{N}$ has the
property $\Gamma$ and is not McDuff factor, then the tensor product
$\mathcal{M}\otimes\mathcal{N}$ also has the property $\Gamma$ and
is not McDuff factor.

\section{Specified examples}

In the previous section, we have shown that, under some conditions,
the central sequence algebra
$(\mathcal{M}\otimes\mathcal{N})_{\omega}$ is abelian. However,
there is a natural question if there are two factors of type
$\mathrm{II}_{1}$ satisfying the conditions in the theorem 2.1. In
this section, by using of the properties of the free group and the
crossed product of von Neumann algebras, we will give specified
examples of $\mathrm{II}_{1}$ factors which satisfy those
conditions.\vskip6pt

\noindent{\bf Example 3.1.} Let $F_{2}$ be the free group on two
generators $a$ and $b$. Suppose that $\mathcal{L}_{F_{2}}$ is the
group factor generated by the operators $L_{a}$ and $L_{b}$ where
$g\rightarrow L_{g}$ is the left regular representation of $F_{2}$
on the Hilbert space $l^{2}(F_{2})$. The shift operator $U$ acting
on the Hilbert space $l^{2}(\mathbb{Z})$ is defined as following
$$Ue_{n}=e_{n+1}, \   \ n \in\mathbb{Z}$$
where $e_{n}$ is the function on $\mathbb{Z}$ such that
$e_{n}(m)=\delta_{n,m}$ for any $m\in\mathbb{Z}$. Let $E_{n}$ be the
orthogonal projection from $l^{2}(\mathbb{Z})$ onto the subspace
$\mathbb{C}e_{n}$. It is well known that $U$ is a Haar unitary with
respect to the vector state $\omega_{e_{0}}$. Let
$\mathcal{L}_{\mathbb{Z}}$ be the von Neumann algebra generated by
the operator $U$.

Now we define a unitary representation of $F_{2}$ on
$l^{2}(\mathbb{Z})$ as following
$$V_{a}(e_{n})=e_{n},\   \ V_{b}(e_{n})=e^{-in\theta}e_{n},\   \ n\in\mathbb{Z},$$
where $\frac{\theta}{2\pi}\in [0,1)$ is an irrational number.
It is easy to check that $V_{b}E_{n}=e^{-in\theta}E_{n}$ for any $n\in\mathbb{Z}$ and $V_{b}UV_{b}^{-1}=e^{-i\theta}U$.

By using of the above unitary representation of $F_{2}$, we define
an implemented action $\alpha$ of $F_{2}$ on
$\mathcal{L}_{\mathbb{Z}}$ as following
$$\alpha_{g}(A)=V_{g}AV_{g}^{*},\   \ g\in F_{2},\  \ A\in\mathcal{L}_{\mathbb{Z}}.$$
The {\it crossed product}
$\mathcal{L}_{\mathbb{Z}}\rtimes_{\alpha}F_{2}$ of
$\mathcal{L}_{\mathbb{Z}}$ under the action $\alpha$ of $F_{2}$ is
the von Neumann subalgebra of $\mathcal{B}(l^{2}(\mathbb{Z})\otimes
l^{2}(F_{2}))$ generated by the operators
$$A\otimes I, \   \ V_{g}\otimes L_{g},\   \ A\in\mathcal{L}_{\mathbb{Z}},\   \ g\in F_{2}.$$

Note that $\mathcal{L}_{\mathbb{Z}}\rtimes_{\alpha}F_{2}= \{U\otimes
I,I\otimes L_{a}, \sum_{n}e^{-in\theta}E_{n}\otimes L_{b}\}''$.
Furthermore, any element in
$\mathcal{L}_{\mathbb{Z}}\rtimes_{\alpha}F_{2}$ has the form
$\sum_{g\in F_{2}}x_{g}V_{g}\otimes L_{g}$ with $x_{g}\in\{U\otimes
I\}''$.

Recalling the following definitions. A (finite) von Neumann algebra
$\mathcal{M}$ is {\it solid} if for every diffuse von Neumann
subalgebra $\mathcal{A}$, the relative commutant
$\mathcal{A}'\cap\mathcal{M}$ is injective. A (finite) von Neumann
algebra is {\it semi-solid} if the relative commutant of any type
$\mathrm{II}_{1}$ von Neumann subalgebra is injective. A maximal
abelian von Neumann subalgebra $\mathcal{A}$ of $\mathcal{M}$ is
{\it singular} if the normalizer $\mathcal{N}(\mathcal{A})=\{W:W$ is
a unitary in $\mathcal{M}$ such that
$W\mathcal{A}W^{*}=\mathcal{A}\}$ is contained in $\mathcal{A}$.

Now we discuss the crossed product
$\mathcal{L}_{\mathbb{Z}}\rtimes_{\alpha}F_{2}$.\vskip6pt

\noindent{\bf Proposition 3.2.} The crossed product
$\mathcal{L}_{\mathbb{Z}}\rtimes_{\alpha}F_{2}$ have the following
properties:\vskip4pt

(1) $\mathcal{L}_{\mathbb{Z}}\rtimes_{\alpha}F_{2}$ is a factor of
type $\mathrm{II}_{1}$, \vskip4pt

(2) The crossed product
$\mathcal{L}_{\mathbb{Z}}\rtimes_{\alpha}F_{2}$ is semisolid but not
solid,

\vskip4pt

(3) Let $\mathcal{A}=\{U\otimes I, V_{a}\otimes L_{a}\}''$, then
$\mathcal{A}$ is a singular maximal abelian von Neumann subalgebra
of $\mathcal{L}_{\mathbb{Z}}\rtimes_{\alpha}F_{2}$.\vskip6pt

\noindent{\it Proof.} Let $\tau_{0}(\sum_{g\in
F_{2}}x_{g}V_{g}\otimes L_{g})\triangleq \omega_{e_{0}}(x_{e})$.
Then it is easy to shown that $\tau_{0}$ is a tracial state of
$\mathcal{L}_{\mathbb{Z}}\rtimes_{\alpha}F_{2}$. Thus the crossed
product $\mathcal{L}_{\mathbb{Z}}\rtimes_{\alpha}F_{2}$ is a finite
von Neumann algebra.

Let $\sum_{g\in F_{2}}x_{g}V_{g}\otimes L_{g}$ be an element in the
center of $\mathcal{L}_{\mathbb{Z}}\rtimes_{\alpha}F_{2}$. Then
according to the following equation
$$(V_{a}\otimes L_{a})(\sum_{g\in
F_{2}}x_{g}V_{g}\otimes L_{g})(V_{a}\otimes L_{a})^{*}=\sum_{g\in
F_{2}}x_{a^{-1}ga}V_{g}\otimes L_{g}=\sum_{g\in
F_{2}}x_{g}V_{g}\otimes L_{g},$$ we have $x_{g}=x_{a^{-1}ga}$ for
any $g\in F_{2}$. Hence if there is $b^{\pm1}$ in the reduced form
of $g$, then $x_{g}=0$. So we can assume that the central element
has the form $\sum_{n\in\mathbb{Z}} x_{a^{n}}V_{a^{n}}\otimes
L_{a^{n}}$.

Now by the equation $$(V_{b}\otimes L_{b})(\sum_{n\in
\mathbb{Z}}x_{a^{n}}V_{a^{n}}\otimes L_{a^{n}})(V_{b}\otimes
L_{b})^{*}=\sum_{n\in\mathbb{Z}}x_{a^{n}}V_{a^{n}}\otimes
L_{a^{n}},$$ we furthermore have $x_{g}=0$ for all $g\in
F_{2}\setminus\{e\}$. However, as $(V_{b}\otimes
L_{b})'\cap(\mathcal{L}_{\mathbb{Z}}\otimes I)$ is trivial, thus the
center of $\mathcal{L}_{\mathbb{Z}}\rtimes_{\alpha}F_{2}$ is also
trivial. Hence the claim (1) holds.

As the free group $F_{2}$ is a word-hyperbolic group and its action
$\alpha$ on $\mathcal{L}_{\mathbb{Z}}$ is trace-preserving, thus the
crossed product $\mathcal{L}_{\mathbb{Z}}\rtimes_{\alpha}F_{2}$ is
semisolid {\cite{[Oz]}}. However, since the operators $V_{a}\otimes
L_{a}$ and $V_{bab^{-1}}\otimes L_{bab^{-1}}$ generate a type
$\mathrm{II}_{1}$ factor which is isomorphic to the free group
factor $\mathcal{L}_{F_{2}}$ and
$$V_{a}\otimes
L_{a}, V_{bab^{-1}}\otimes
L_{bab^{-1}}\in(\mathcal{L}_{\mathbb{Z}}\otimes
I)'\cap(\mathcal{L}_{\mathbb{Z}}\rtimes_{\alpha}F_{2}),$$ thus the
crossed product $\mathcal{L}_{\mathbb{Z}}\rtimes_{\alpha}F_{2}$ is
not solid. Hence the claim (2) holds.


It is easy to check that the operator $V_{a}$ commutates with the
operator $U$, so $\mathcal{A}$ is an abelian von Neumann subalgebra
of $\mathcal{L}_{\mathbb{Z}}\rtimes_{\alpha}F_{2}$. Note that in the
proof of the claim (1), we have shown that
$\mathcal{A}'\cap(\mathcal{L}_{\mathbb{Z}}\rtimes_{\alpha}F_{2})\subset\mathcal{A}$.
Thus $\mathcal{A}$ is maximal abelian.

Let $\mathcal{A}_{0}$ be the abelian von Neumann subalgebra of
$\mathcal{A}$ which is generated by the operator $V_{a}\otimes
L_{a}$. As $V_{a}\otimes L_{a}$ is a Haar unitary, then
$\mathcal{A}_{0}$ is diffuse. For any element $g\in F_{2}$ whose
reduced form contains the letter $b$, it is an exercise to check
that $(V_{g}\otimes L_{g})\mathcal{A}_{0}(V_{g}\otimes
L_{g})^{*}\perp\mathcal{A}$. Thus $V_{g}\otimes L_{g}$ is orthogonal
to the algebra $\mathcal{N}(\mathcal{A})''$ {\cite{[Po]}}. Hence
$\mathcal{N}(\mathcal{A})''\subset\mathcal{A}$. Thus the claim (3)
holds.
\endproof

\vskip6pt

\noindent{\bf Remark 3.3.} Recalling that a type $\mathrm{II}_{1}$
factor $\mathcal{M}$ is {\it prime} if
$\mathcal{M}\neq\mathcal{M}_{1}\overline{\otimes}\mathcal{M}_{2}$
for any type $\mathrm{II}_{1}$ factors $\mathcal{M}_{1}$ and
$\mathcal{M}_{2}$. As the crossed product
$\mathcal{L}_{\mathbb{Z}}\rtimes_{\alpha}F_{2}$ is semisolid, thus
it is a prime $\mathrm{II}_{1}$ factor.

Let $\omega$ be a free ultrafilter on $\mathbb{N}$. Then we have the
following conclusion. \vskip6pt



\noindent{\bf Proposition 3.4.}  The central sequence algebra
$(\mathcal{L}_{\mathbb{Z}}\rtimes_{\alpha}F_{2})_{\omega}$ is
abelian and nontrivial. \vskip6pt

\noindent{\it Proof.} Suppose that
$(t_{n})\in(\mathcal{L}_{\mathbb{Z}}\rtimes_{\alpha}F_{2})^{\omega}$
is a nontrivial central sequence. Without loss of generality, we can
assume that
$$\|t_{n}\|=1, \  \ \tau(t_{n})=0, \  \ t_{n}=\sum_{g\in
F_{2}}x_{g}^{(n)}V_{g}\otimes L_{g}$$ with
$x_{g}^{(n)}\in\mathcal{L}_{\mathbb{Z}}\otimes I$.

Let $S=\{g\in F_{2}:$ the reduced form of $g$ beginning with $b^{\pm
1}\}$. If $\|t_{n}|_{S}\|_{2}\nrightarrow0$, then there is a
subsequence $(t_{n_{k}})$ of $(t_{n})$ such that
$$\lim_{k\rightarrow\infty}\sum_{g\in S}\|x_{g}^{(n_{k})}\|_{2}^{2}=c$$
where $c\in\mathbb{R}$ is a positive constant. Since any subsequence
of a central sequence is also a central sequence, thus we can assume
that
 $\sum_{g\in S}\|x_{g}^{(n)}\|_{2}^{2}\rightarrow c.$
Then there exists an positive integer $n_{0}$ such that
$$\|t_{n}|_{S}\|_{2}^{2}=\sum_{g\in
S}\|x_{g}^{(n)}\|_{2}^{2}>\frac{c}{2}$$
for any $n>n_{0}$.

Let $\delta=\frac{c}{8}$ and $N$ be a positive integer such that
$\frac{cN}{4}>1$. As $(t_{n})$ is a central sequence, then for
$a,a^{2},\cdots,a^{N}$, there is a $n_{1}\in\mathbb{N}(\geqslant
n_{0})$ such that we have
$$\|(V_{a}\otimes L_{a})^{i}t_{n}(V_{a}\otimes L_{a})^{-i}-t_{n}\|_{2}<\delta,\     \ i=1,2,\cdots,N,$$
for any $n>n_{1}$.

Then by the following
equations
$$\|((V_{a}\otimes L_{a})^{i}t_{n}(V_{a}\otimes L_{a})^{-i})|_{S}\|^{2}_{2}= \sum_{g\in
a^{-i}Sa^{i}}\|x_{g}^{(n)}\|^{2}_{2}=\|t_{n}|_{a^{-i}Sa^{i}}\|_{2}^{2},$$
we have
$$|\|t_{n}|_{a^{-i}Sa^{i}}\|^{2}_{2}-\|t_{n}|_{S}\|^{2}_{2}|\leqslant
2\|(a^{i}t_{n}a^{-i})|_{S}-t_{n}|_{S}\|_{2}<2\delta.$$

Therefore we get
$$\|t_{n}|_{a^{-i}Sa^{i}}\|^{2}_{2}>\|t_{n}|_{S}\|^{2}_{2}-2\delta, \  \
i=1,2,\cdots,N.$$

As the sets $a^{-i}Sa^{i}(i=1,2,\cdots,N)$ are pairwise disjoint,
then we have
$$1\geqslant\|t_{n}\|^{2}_{2}\geqslant\sum_{i=1}^{N}\|t_{n}|_{a^{-i}Sa^{i}}\|_{2}^{2}\geqslant
N(\|t_{n}|_{S}\|^{2}_{2}-2\delta)\geqslant\frac{cN}{4}>1$$ which is
impossible. Hence $\|t_{n}|_{S}\|_{2}\rightarrow 0$. Substituting
the $(t_{n})$ by $(t_{n}-(t_{n}|_{S}))$, then we can further assume
that all $t_{n}$ vanish at the set $S$.


Now let $S_{0}=\{g\in F_{2}$: the reduced form of $g$ beginning with
$a^{\pm1}\}$. Then for any $0\neq i\in\mathbb{Z}$,
 we have $b^{i}S_{0}b^{-i}\cap S_{0}=\emptyset$. As $(t_{n})$ vanishes at the set $S$,
 thus $t_{n}|_{b^{i}S_{0}b^{-i}}=0$ and then we have
$$\|(V_{b}\otimes L_{b})^{i}t_{n}(V_{b}\otimes L_{b})^{-i}-t_{n}\|^{2}_{2}
\geqslant\|t_{n}|_{S_{0}}\|_{2}^{2}.$$ As $(t_{n})$ is a central
sequence, thus we have $\|t_{n}|_{S_{0}}\|_{2}\rightarrow0$ as
$n\rightarrow\infty$.

According to the above arguments, now we can assume that the support
$\{g\in F_{2}:x^{n}(g)\neq 0\}$ of $t_{n}$ is concentrating on the
unit element $e$ of $F_{2}$.
 Then the central sequence $(t_{n})$ satisfy the condition $t_{n}\in\mathcal{L}_{\mathbb{Z}}\otimes I$.
 Thus $(\mathcal{L}_{\mathbb{Z}}\rtimes_{\alpha}F_{2})_{\omega}$ is abelian as $\mathcal{L}_{\mathbb{Z}}$ is abelian.

Finally, since $\frac{\theta}{2\pi}\in[0,1)$ is an irrational
number, then there is a sequence $(n_{k})$ of natural numbers such
that $\frac{n_{k}\theta}{2\pi}-[\frac{n_{k}\theta}{2\pi}]\rightarrow
0$ as $k\rightarrow\infty$. For the given sequence $(n_{k})$, let
$t_{k}=U^{n_{k}}\otimes I$. Then $\|t_{k}\|=1,\tau(t_{k})=0$ and
$\|t_{k}\|_{2}=1$. It is obvious that $(t_{k})$ commutates with $U$
and $a$. Furthermore, as
$$\|(V_{b}\otimes L_{b})t_{k}(V_{b}\otimes L_{b})^{-1}-t_{k}\|_{2}=\|(e^{-in_{k}\theta}-1)U^{n_{k}}\|_{2}\rightarrow0,\   \ k\rightarrow \infty,$$
thus $(t_{k})$ is a nontrivial central sequence of
$\mathcal{L}_{\mathbb{Z}}\rtimes_{\alpha}F_{2}$. \endproof \vskip6pt

Note that in the proof of proposition 3.4., we have shown that the
central sequence algebra
$(\mathcal{L}_{\mathbb{Z}}\rtimes_{\alpha}F_{2})_{\omega}$ is
contained in the ultrapower of $\mathcal{L}_{\mathbb{Z}}\otimes I$.
Thus we get the following consequence.\vskip6pt

\noindent{\bf Corollary 3.5.} Let
$\mathcal{M}=\mathcal{L}_{\mathbb{Z}}\rtimes_{\alpha}F_{2}$ and
$\mathcal{A}=\mathcal{L}_{\mathbb{Z}}\otimes I$. Then we have
$$(\mathcal{M}\otimes\mathcal{M})_{\omega}\subset(\mathcal{A}\otimes\mathcal{M})^{\omega}.$$
\vskip6pt

In fact, similar to the arguments in the proof of the proposition
3.4, we can show that the central sequence algebra
$(\mathcal{M}\otimes\mathcal{M})_{\omega}$ is contained in
$(\mathcal{A}\otimes\mathcal{A})^{\omega}$. Hence
$(\mathcal{M}\otimes\mathcal{M})_{\omega}$ is abelian. Of course,
$(\mathcal{M}\otimes\mathcal{M})_{\omega}$ is also nontrivial. Hence
we have given a specified example of $\mathrm{II}_{1}$ factor
$\mathcal{M}$, which has the property $\Gamma$ and is not McDuff
factor, such that the tensor product $\mathcal{M}\otimes\mathcal{M}$
also have the property $\Gamma$ and is not McDuff factor.\vskip6pt

\noindent{\bf Acknowledgements} \vskip6pt

We would like to thank Hou Chenjun and Wang Liguang for many helpful
discussions. In particular, we would like to thank Ge Liming from
whom we have learned a lot. This research was carried out while the
first author was visiting the Academy of Mathematics and Systems
Science of CAS and Morningside Center of Mathematics.




\begin{thebibliography}{1}


\bibitem{[Co]} A.Connes, {\em{Classification of injective factors}},
Ann. of Math., 104(1976), 73-115.


\bibitem{[Dix]} J.Dixmier, {\em{Quelques propri\'{e}t\'{e}s des suites centrals dans les facteurs de type $\mathrm{II}_{1}$}},
(French) Invent.Math., 7(1969), 215-225.

\bibitem{[FGL]} J.Fang, L. Ge and W. Li, {\em{Central sequence algebras of von Neumann
algebras}}, Taiwanese J.Math., 1(2006), 187-200.


\bibitem{[KR]} R.V.Kadison and J.R.Ringrose, {\em{Fundamentals of theory of operator
algebras $\mathrm{II}$: advanced theory}}, Academic Press, Orlando,
1986.

\bibitem{[MD]} D. McDuff, {\em{Central sequences and the hyperfinite
factors}}, Proc. London Math. Soc., 21(1970), 443-461.

\bibitem{[Oz]} N.Ozawa, {\em{A Kurosh-type theorem for type $\mathrm{II}_{1}$
factors}}, Inter.Math.Res.Notices, Volume 2006, Article ID 97560,
1-21.

\bibitem{[Po]} S.Popa, {\em{Orthogonal pairs of *-subalgebras in finite von Neumann
algebras}}, J.Operator Theory, 9(1983), 253-268.

\bibitem{[Sa]} S.Sakai, {\em{The theory of W*-algebras}}, Lecture
notes, Yale University, 1962.

\bibitem{[Ta]} M.Takesaki, {\em{Theory of operator algebras
$\mathrm{I}$}}, Springer-Verlag, Berlin, 2002.












\end{thebibliography}


\end{document}
