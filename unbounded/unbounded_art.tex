\documentclass[a4paper,10pt]{amsart}

\usepackage[protrusion=true,expansion=true]{microtype} 
\usepackage{fancyhdr}
\usepackage[utf8]{inputenc}
\usepackage{graphicx} 
\usepackage{wrapfig} 

\usepackage{mathpazo}
\usepackage[T1]{fontenc}
\usepackage{amsmath}
\usepackage{amssymb}
\usepackage{hyperref}
\usepackage{cleveref}
\usepackage{comment}
\usepackage{color}

\newtheorem{example}{Example}[section]
\newtheorem{theorem}{Theorem}[section]
\newtheorem{proposition}{Proposition}[section]
\newtheorem{corollary}{Corollary}[section]
\newtheorem{definition}{Definition}[section]
\newtheorem{lemma}{Lemma}[section]
\newtheorem{remark}{Remark}[section]
\newtheorem{question}{Question}[section]

\crefname{lemma}{Lemma}{lemmas}
\crefname{remark}{Remark}{remark}
\crefname{corollary}{Corollary}{corollary}
\crefname{theorem}{Theorem}{theorem}
\crefname{example}{Example}{example}
\crefname{definition}{Definition}{definition}

\newcommand{\AAA}{\mathfrak A}
\newcommand{\BBB}{\mathcal B}
\newcommand{\CCC}{\mathcal C}
\newcommand{\HHH}{\mathcal H} %for Hilbert space
\newcommand{\LLL}{\mathcal L} % for lattice
\newcommand{\MMM}{\mathcal M}


\newcommand{\Lat}{\mathcal Lat}
\newcommand{\Alg}{\mathcal Alg}
\newcommand{\tr}{\tau}
\newcommand{\C}{\mathbb C} %for complex number
\newcommand{\R}{\mathbb R}  %for real number
\newcommand{\Z}{\mathbb Z} %for integer
\newcommand{\N}{\mathbb N} % for nature number
% self defined vars
\newcommand{\titleinfo}{A note on operators commuted with a 
    unbounded operator afflicated with II$_1$ factors}
\newcommand{\authorinfo}{Wenming Wu and Wei Yuan} 

\linespread{1.05}
\pagestyle{fancyplain}
\fancyhf{}
%\fancyhf[HLE,HRO]{\titleinfo}
\fancyhf[HRE,HLO]{\authorinfo}
\fancyhf[FC]{\thepage}

\begin{document}

\title{\LARGE\textbf{\titleinfo}} 
\author{\large\textsc{\authorinfo}} 
\address{AMSS}  
\email{}

\date{}

%\renewcommand{\abstractname}{Summary} 
\begin{abstract}
Enter abstract here
\end{abstract}

% Keywords
\subjclass[2010]{Primary 47L75; Secondary 15A30}
\keywords{von Neumann algebras, unbounded operators}
\thanks{}
\maketitle


\section*{Introduction}
Introduction here!

\section{Unbounded Fuglede-Putnam Theorem}
The celebrated Fuglede-Putnam theorem in its classical form is as follows:
\begin{theorem}[Fuglede-Putnam Theorem] \label{fthm0}
   If $T$ is a bounded operator and if $M$ and $N$ are normal operators, then
   \begin{align*}
       TN \subset MT \Rightarrow TN^{*} \subset M^{*}T.
   \end{align*}
\end{theorem}

Fuglede \cite{F} proved the foregoing theorem in the case $N=M$ in 1950 and Putnam 
extended it to the form given above one year later \cite{CRP}. In 1958, Rosenblum
published a very elegant proof of the theorem in the case that $M$ and $N$ are
bounded. We will prove the following version of Fuglede-Putnam theorem.

\begin{theorem}\label{fthm1}
    Let $\AAA$ be a finite von Neumann algebra,
    and $T$ a closed operator affiliated with $\AAA$. 
    If $N$ is a normal operators in $\AAA$ and $NT = TN$, then
    $N^{*}T = TN^{*}$.
\end{theorem}

In order to prove this theorem, we will first prove the following lemma.
\begin{lemma} \label{flam1}
   Suppose that $\AAA$ is a finite von Neumann algebra. 
   Let $N$ be a normal operator in $\AAA$ and $P$ a projection in $\AAA$.
   If $(I-P)NP = 0$, then $PN = NP$.
\end{lemma}

\begin{proof}
   Let $\tau$ be a faithful trace on $\AAA$ and  
   \begin{align*}
      P = \begin{pmatrix}
          I & 0\\
          0 & 0
      \end{pmatrix} 
      \qquad 
      N = \begin{pmatrix}
          N_1 & N_2 \\
          0 & N_3
      \end{pmatrix}. 
   \end{align*}
   Since $\tau(PN^{*}NP) = \tau(PNN^{*}P)$, 
   $\tau|_{P\AAA P}(N_{1}^{*}N_1) = \tau|_{P\AAA P}(N_{1}N_{1}^{*} +
   N_2 N_{2}^{*})$. Therefore, $\tau(N_2 N_{2}^{*}) = 0$ and $
   N_2 = 0$.
\end{proof}


\begin{lemma} \label{flam2}
    Let $\AAA$ be a finite von Neumann algebra,
    and $H$ a closed positive operator affiliated with $\AAA$. 
    If $M$ and $N$ are normal operators in $\AAA$ and $NH = HM$,
    then $N^{*}H = HM^{*}$, $NH= HN$ and $MH=HM$. 
    Furthermore, if $Ker(H) = \{0\}$, then
    $M=N$.
\end{lemma}

\begin{proof}
    If $ker(H) \neq \{0\}$, then it is not hard to see that 
    $(I-E_{0})ME_{0} = 0$, where $E_0$ is the orthonormal projection 
    onto $ker(H)$. 
    By \cref{flam1}, $E_{0}M = ME_{0}$. 
    Since $M^{*}H = HN^{*}$, similar argument shows
    that $E_{0}N^{*} = N^{*}E_{0}$.
    
    Therefore we assume $ker(H) = \{0\}$.
    Let $\tau$ be a faithful trace on $\AAA$.
    Let $\{ E_{\lambda} \}$ be the resolution of identity in $\AAA$ such that
    \begin{align*}
        H = \int_{0}^{\infty} \lambda d E_{\lambda}. 
    \end{align*}
Fix a $\lambda > 0$, let
\begin{align*}
    P = E_{\lambda} = \begin{pmatrix}
          I & 0\\
          0 & 0
      \end{pmatrix}, \quad 
   H = \begin{pmatrix}
       H_1 & 0\\
       0 & H_2
   \end{pmatrix}, \quad  
   N = \begin{pmatrix}
       N_{11} & N_{12}\\ 
       N_{21} & N_{22} \\
   \end{pmatrix}, \quad  
   M = \begin{pmatrix}
       M_{11} & M_{12}\\ 
       M_{21} & M_{22} \\
   \end{pmatrix},
\end{align*}
where $H_1 = HE_{\lambda}$ and $H_2 = H(I-E_{\lambda})$.
$NH = HM$ implies that
\begin{align*}
   \begin{pmatrix}
       H_{1}^{-1}N_{11}H_{1} & H_{1}^{-1}N_{12}H_{2}\\
       H_{2}^{-1}N_{21}H_{1} & H_{2}^{-1}N_{22}H_{2}\\
   \end{pmatrix} = 
   \begin{pmatrix}
       M_{11} & M_{12}\\ 
       M_{21} & M_{22} \\
   \end{pmatrix}
\end{align*}
Since $M$ is normal, we have $\tau(PM^{*}MP) = \tau(PMM^{*}P)$, 
\begin{align*}
    &\tau \left ( \begin{pmatrix}
           H_{1}N_{11}^{*}H_{1}^{-2}N_{11}H_{1} + 
           H_{1}N_{21}^{*}H_{2}^{-2}N_{21}H_{1} & 0 \\
           0 & 0
       \end{pmatrix} \right )\\
       = &\tau \left (\begin{pmatrix}
       H_{1}^{-1}N_{11}H_{1}^{2}N_{11}^{*}H_{1}^{-1}
       + H_{1}^{-1}N_{12}H_{2}^{2}N_{12}^{*}H_{1}^{-1} & 0\\
       0 & 0
   \end{pmatrix} \right). 
\end{align*}
Note that 
\begin{align*}
    \tau|_{P\AAA P}(H_{1}N_{11}^{*}H_{1}^{-2}N_{11}H_{1}) 
    = \tau|_{P\AAA P}(H_{1}^{-1}N_{11}H_{1}^{2}N_{11}^{*}H_{1}^{-1}).
\end{align*}
We have
\begin{align*}
    \tau|_{P\AAA P}(H_{1}N_{21}^{*}H_{2}^{-2}N_{21}H_{1}) 
    = \tau|_{P\AAA P}(H_{1}^{-1}N_{12}H_{2}^{2}N_{12}^{*}H_{1}^{-1}). 
\end{align*}
Since $\| H_{1} \| \leq \lambda$ and $\|H_{2}^{-1}\| \leq \frac{1}{\lambda}$, 
we have
\begin{align*}
    \tau \left (
    \begin{pmatrix}
        H_{1}N_{21}^{*}H_{2}^{-2}N_{21}H_{1} & 0\\
        0 & 0
    \end{pmatrix} \right)
    &\leq  \frac{1}{\lambda^{2}}
    \tau \left (
    \begin{pmatrix}
        H_{1}N_{21}^{*}N_{21}H_{1} & 0 \\
        0 & 0
    \end{pmatrix} \right) \\
    & = \frac{1}{\lambda^{2}} \tau \left (
         \begin{pmatrix}
             0 & H_1 N_{21}^{*}\\
             0 & 0
         \end{pmatrix}
         \begin{pmatrix}
             0 & 0\\
             N_{21}H_1 & 0
         \end{pmatrix}
    \right ) \\
    & = \frac{1}{\lambda^{2}} \tau \left (
        \begin{pmatrix}
            0 & 0\\
            0 & N_{21}H_{1}^{2}N_{21}^{*}
         \end{pmatrix}
    \right ) \\
    & \leq \tau \left (
        \begin{pmatrix}
            0 & 0\\
            0 & N_{21}N_{21}^{*}
         \end{pmatrix}
    \right )
    = \tau \left (
        \begin{pmatrix}
            N_{21}^{*}N_{21} & 0\\
            0 & 0 
         \end{pmatrix}
    \right )
\end{align*}
Let $Q = E_{\beta} - E_{\lambda}$ where $\beta > \lambda$.
\begin{align*}
    &\tau \left (
    \begin{pmatrix}
        H_{1}^{-1}N_{12}H_{2}^{2}N_{12}^{*}H_{1}^{-1} & 0\\
        0 & 0
    \end{pmatrix} \right) \\
    &\geq \beta^{2}
     \tau \left (     
     \begin{pmatrix}
         H_{1}^{-1}N_{12}(I-Q)N_{12}^{*}H_{1}^{-1} & 0\\
        0 & 0
    \end{pmatrix} \right )
    + \lambda^{2} \tau \left (
     \begin{pmatrix}
         H_{1}^{-1}N_{12}QN_{12}^{*}H_{1}^{-1} & 0\\
        0 & 0
    \end{pmatrix} \right)\\
    &= \beta^{2} \tau \left ( 
    \begin{pmatrix}
        0 & 0\\
        0 & (I-Q)N_{12}^{*}H_{1}^{-2}N_{12}(I-Q)
    \end{pmatrix} \right )
    + \lambda^{2} \tau \left (
    \begin{pmatrix}
        0 & 0\\
        0 & QN_{12}^{*}H_{1}^{-2}N_{12}Q
    \end{pmatrix} \right ) \\
    & \geq \frac{\beta^{2}}{\lambda^2}\tau \left (
    \begin{pmatrix}
        0 & 0\\
        0 & (I-Q)N_{12}^{*}N_{12}(I-Q)
    \end{pmatrix} \right )
    + \tau \left (
    \begin{pmatrix}
        0 & 0\\
        0 & QN_{12}^{*}N_{12}Q
    \end{pmatrix} \right ) \\ 
    & = \frac{\beta^{2}}{\lambda^2}\tau \left (
    \begin{pmatrix}
        N_{12}(I-Q)N_{12}^{*} & 0\\
        0 & 0
    \end{pmatrix} \right )
    + \tau \left (
    \begin{pmatrix}
        N_{12}QN_{12}^{*} & 0\\
        0 & 
    \end{pmatrix} \right )
\end{align*}
Since $N^{*}N = NN^{*}$, we have
\begin{align*}
    \tau \left(
    \begin{pmatrix} 
        N_{12}N_{12}^{*} & 0 \\
        0 & 0 \\
    \end{pmatrix}
    \right )= 
    \tau \left (
    \begin{pmatrix} 
        N_{21}^{*}N_{21} & 0 \\
        0 & 0
    \end{pmatrix}    
    \right ).
\end{align*}
Therefore
\begin{align*}
    \frac{\beta^{2}}{\lambda^2}\tau \left (
    \begin{pmatrix}
        N_{12}(I-Q)N_{12}^{*} & 0\\
        0 & 0
    \end{pmatrix} \right )
    + \tau \left (
    \begin{pmatrix}
        N_{12}QN_{12}^{*} & 0\\
        0 & 
    \end{pmatrix} \right ) \leq
    \tau \left(
    \begin{pmatrix} 
        N_{12}N_{12}^{*} & 0 \\
        0 & 0 \\
    \end{pmatrix}
    \right ).
\end{align*}
Thus
\begin{align*}
    \frac{\beta^{2}}{\lambda^2}\tau \left (
    \begin{pmatrix}
        N_{12}(I-Q)N_{12}^{*} & 0\\
        0 & 0
    \end{pmatrix} \right ) \leq
    \tau \left(
    \begin{pmatrix} 
        N_{12}(I-Q)N_{12}^{*} & 0 \\
        0 & 0 \\
    \end{pmatrix}
    \right ).
\end{align*}
This implies that $N_{12}(I-Q)N_{12}^{*} = 0$. Since $E_{\lambda} = 
\wedge_{\alpha > \lambda} E_{\alpha}$, we have $N_{12}N^{*}_{12} = 0$.
By \cref{flam1}, we have $E_{\lambda}N = NE_{\lambda}$.
Since $\lambda$ is arbitrary, $NH = HN$. By the hypothesis, $(N-M)H = 0$.
Thus, $N=M$ since $ker(H)= \{0\}$.
\end{proof}

Now we can prove \cref{fthm1}.

\begin{proof}[Proof of \cref{fthm1}]
    Let $T = UH$ be the polar decomposition of $T$ (
    Since $\AAA$ is a finite von Neumann algebra, we could
    also assume that $U$ is a unitary). $NT = TN$ is equivalent to  
   $U^{*}NUH = HN$. Let $M = U^{*}NU$. We have $MH = HN$. By \cref{flam2},
   $M^{*}H = HN^{*}$, and $N^{*}T = T N^{*}$. 
\end{proof}

\begin{comment}
\begin{theorem} 
    Let $\AAA$ be a finite von Neumann algebra with a faithful trace $\tau$,
    and $T$ is a closed operator affiliated with $\AAA$ and $\tau(T^{*}T) < \infty$. 
    If $M$ and $N$ are normal operators in $\AAA$ and $MT = TN$, then
    $M^{*}T = TN^{*}$.
\end{theorem}

\begin{proof} 
    We could assume that $\AAA$ acing on $L^2(\AAA, \tau)$ and $\zeta$ is a
    trace vector. By the fact $\tau(T^{*}T) < \infty$, we have 
    $\| T\zeta \|_2 < \infty$ and $\zeta \in \mathfrak{D}(T)$. By
    Theorem 7.2.15 of \cite{KR}, there exists a * anti-isomorphism $\phi$ of $\AAA$ 
    onto $\AAA'$ such that $A\zeta = \phi(A)\zeta$ for any $A \in \AAA$. 
    Therefore, $\AAA \zeta = \AAA' \zeta \subset \mathfrak{D}(T)$.
    By induction, $MT = TN$ implies that $M^{k}T=TN^{k} $ for all $k \in \N$.
    Thus for any $z \in \C$, for any $A'$ in $\AAA'$, we have
    \begin{align}\label{feq2}
        e^{\Bar{z}M}T = Te^{\Bar{z}N} \mbox{ and }
        e^{\Bar{z}M}T A' \zeta = Te^{\Bar{z}N} A' \zeta. 
    \end{align}
    Let $F(z) = e^{zM^{*}}Te^{-zN^{*}}A' \zeta$. By \cref{feq1},  
    \begin{align*}
        F(z) = A' e^{zM^{*}}e^{-\Bar{z}M}Te^{\Bar{z}N}e^{-zN^{*}}\zeta
             = A' e^{zM^{*}-\Bar{z}M}Te^{\Bar{z}N-zN^{*}}\zeta
             = A' e^{zM^{*}-\Bar{z}M}e^{\Bar{z}\phi(N)-z\phi(N)^{*}}T\zeta.
    \end{align*}
    Note that $e^{zM^{*}-\Bar{z}M}$ is a unitary in $\AAA$ and 
    $\phi(e^{\Bar{z}N-zN^{*}})$ is a unitary in $\AAA'$. Hence 
    $\|F(z)\|_2 \leq \|A'\|\|T\zeta\|_2$ for any $z \in \C$. Since
    $F(z) = e^{zM^{*}}A' e^{-z\phi(N)^{*}}T\zeta$ is analytic, $F(z)$ is 
    constant and equal to $F(0) = T A' \zeta$. Therefore, 
    $0 = F'(z) = M^{*}TA' \zeta - A' \phi(N)^{*}T\zeta = 
    M^{*}TA' \zeta - TN^{*} A'\zeta$ and $M^{*}T = TN^{*}$. 
\end{proof}
\end{comment}

\begin{corollary}\label{fcor1}
    Let $\AAA$ be a finite von Neumann algebra,
    and $T$ a closed operator affiliated with $\AAA$. 
    If $N$ is a normal operators in $\AAA$ and $NT = TN$, then
    $AT = TA$ for each $A$ in the von Neumann algebra generated by $N$.
\end{corollary}
\begin{proof}
    By \cref{fthm1} and Lemma 5.6.13 in \cite{KR}, we have the result.
\end{proof}

\begin{corollary}\label{fcor2}
    Let $\AAA$ be a finite von Neumann algebra,
    and $T$ a closed operator affiliated with $\AAA$. 
    If $N$ is a normal operator affiliated with $\AAA$ and $NT = TN$, then
    $N^{*}T = TN^{*}$.
\end{corollary}

\begin{proof}
   Let $E_n$ be the spectral projections for $N$ corresponding to the set 
   $\{ z : |z| \leq n \}$ for each positive integer $n$. 
   Thus $\{ E_{n} \}$ is a increasing
   sequence of projections with strong-operator limit $I$. By $NT = TN$, we have
   \begin{align*}
      (E_{n}NE_{n})(E_nTE_{n}) = (E_{n}TE_{n})(E_{n}NE_{n}).
   \end{align*}  
   By \cref{fthm1},
   \begin{align*}
       E_{n}N^{*}TE_{n} = (E_{n}N^{*}E_{n})(E_nTE_{n}) 
       = (E_{n}TE_{n})(E_{n}N^{*}E_{n})=E_{n}TN^{*}E_{n}.
   \end{align*} 
   Note that $E_{n} \leq E_{m}$ if $n \leq m$. 
   Multiply $E_n$ from right on both side of the equation 
   $E_{m}N^{*}TE_{m} = E_{m}TN^{*}E_{m}$, we have
   $E_{m}N^{*}TE_{n} = E_{m}TN^{*}E_{n}$.
   Let $m$ tend to $\infty$,  we get $N^{*}TE_{n} = TN^{*}E_{n}$. 
   Thus, $E_{n}T^{*}N = E_{n}NT^{*}$. Let $n$ tend to $\infty$ we have
   $T^{*}N = NT^{*}$ and $N^{*}T = TN^{*}$. 
\end{proof}

\begin{corollary}\label{fcor3}
    Let $\AAA$ be a finite von Neumann algebra,
    and $T$ a closed operator affiliated with $\AAA$. 
    If $N$ and $M$ are normal operators affiliated with $\AAA$ and $MT = TN$, then
    $M^{*}T = TN^{*}$.
\end{corollary}

\begin{proof}
    Note that
    \begin{align*}
      \AAA \otimes M_{s}(\C) = \{
        \begin{pmatrix}
         A_{11} & A_{12} \\
         A_{21} & A_{22}
    \end{pmatrix} : A_{ij} \in \AAA \}  
    \end{align*}
    is also a finite von Neumann algebra. 
    Consider
    \begin{align*}
        N_{1} = \begin{pmatrix}
            N & 0\\
            0 & M
        \end{pmatrix}
        \mbox{ and } 
        T_{1} = \begin{pmatrix}
            0 & 0\\
            T & 0\\
        \end{pmatrix}.
    \end{align*}
    Note that $N_1$ is normal, $N_{1}T_{1} = T_{1}N_{1}$. 
    By \cref{fcor2}, we have
    $N_{1}^{*}T_{1} = T_{1}N_{1}^{*}$. Comparing the $(2,1)$ entry then
    gives $M^{*}T = TN^{*}$.
\end{proof}

\begin{corollary}\label{fcor4}
    Let $\AAA$ be a finite von Neumann algebra with a faithful trace $\tau$,
    and $T$ is a closed operator affiliated with $\AAA$. 
    If $N$ and $M$ are  normal operators in $\AAA$ and $MT = TN$, then
    $f(M)T = Tf(N)$ for any measurable function $f$.
\end{corollary}

Let $T$ be a closed operator. The numerical range of $T$, denoted by $W(T)$,
is defined as
\begin{align*}
    W(T) = \{\langle T\xi, \xi \rangle: \xi \in \mathfrak{D}(T), \|\xi \|_2 = 1 \}. 
\end{align*}

In \cite{Em}, Embry proved the following theorem.
\begin{theorem}
   Let $N$ and $M$ be two commuting bounded normal operators and $T$ a bounded
   operator such that $0 \notin W(T)$. If $MT = TN$, then $N = M$. 
\end{theorem}

By \cref{fcor4} and an argument parallel to that used in \cite{Em}, we have 
the following fact.

\begin{corollary}\label{fcor5}
    Let $\AAA$ be a finite von Neumann algebra,
    and $T$ a closed operator affiliated with $\AAA$. 
    If $N$ and $M$ are two commuting normal operators in 
    $\AAA$, $MT = TN$ and $0 \notin W(T)$, 
    then $N = M$.
\end{corollary}

The following result is well known.

\begin{lemma} \label{flam3}
    Let $\AAA$ be a separable II$_1$ factor. There exist two
    maximal abelian subalgebras $\mathfrak{M}_1$, $\mathfrak{M}_2$ such
    that $\mathfrak{M}_1 \cap \mathfrak{M}_2 =  \C I$.
\end{lemma}

\begin{proof}
    By Corollary 4.1 in \cite{P}, there is a hyperfinite subfactor $\mathcal{R}$
    such that $\mathcal{R}' \cap \AAA = \C I$. Let 
    $\widetilde{\mathfrak{M}_1}$ and $\widetilde{\mathfrak{M}_2}$ 
    be two orthognal maximal abelian
    subalgebras generate $\mathcal{R}$. 
    There exist two maximal abelian subalgebras 
    $\mathfrak{M}_1$ and $\mathfrak{M}_2$ of $\AAA$ contains
    $\widetilde{\mathfrak{M}_1}$ and $\widetilde{\mathfrak{M}_2}$
    respectively. If $T \in \mathfrak{M}_1 \cap \mathfrak{M}_2$,
    then $T$ in $\AAA$ commute with all elements in 
    $\widetilde{\mathfrak{M}_1}$ and $\widetilde{\mathfrak{M}_2}$.
    Hence $T$ commutes with $\mathcal{R}$ and $T$ is a scalar.
\end{proof}

\begin{corollary}\label{fcor6}
   If $\AAA$ be a separable II$_1$ factor, then there exists 
   a closed operator $T$ affiliated with $\AAA$ such that 
   $NT \neq TN$ for any nontrivial normal operator affiliated with 
   $\AAA$.
\end{corollary}
\begin{proof}
    By \cref{flam3}, there exist two maximal abelian
    subalgebras $\mathfrak{M}_1$ and $\mathfrak{M}_2$ of $\AAA$ such that 
    $\mathfrak{M}_1 \cap \mathfrak{M}_2 = 
    \C I$. Let $T = H_1 + iH_2$ where $H_1$ and $H_2$ are two positive 
    invertible (the inverse is a bouned positive operator in $\AAA$) operators
    generate $\mathfrak{M}_1$ and $\mathfrak{M}_2$ respective. 
    Suppose $N \eta \AAA$ is a nontrivial
    normal operator and $NT = TN$. By \cref{fcor2}, 
    $N^{*}T = TN^{*}$. Hence $NT^{*} = T^{*}N$. 
    This implise that $TH_1 = H_1 T$ and $TH_2 = TH_2$.
    If $T$ is a unitary, then $T$ is in $\mathfrak{M}_1 \cap \mathfrak{M}_2 = \C I$.
    If $T$ is not a unitary, then $(I+T^{*}T)^{-1}$ is in
    $\mathfrak{M}_1 \cap \mathfrak{M}_2 = \C I$. Thus $T$ must be a scalar. 
\end{proof}

\section{Examples}
Throughout this section $(X, \mathcal{B}, \mu)$ will denote a 
probability space. 
For simplicity of notation, we will use $h$ to denote the 
multiplication operator associated with the measurable function $h$
whenever there is no danger of ambiguity.
Let $G$ be a group acts on
$X$ ergodically, i.e., if $X \in \mathcal{B}$ and $\mu(g(X)\setminus X) = 0$ for 
each $g$ in $G$, then either $m(X) = 0$ or $m(S \setminus X) = 0$, and 
preserves the measure.
This induces 
an automorphic representation of $G$ on $L^{\infty}(X)$: 
$\alpha_{g}(h)(x) = h(g^{-1}(x))$. The cross product 
$L^{\infty}(X) \rtimes_{\alpha} G$ is the von Neumann algebra, acting on 
$L^{2}(X, d\mu) \otimes l^{2}(G)$, generated by the operators
\begin{align*}
    \psi(h) = \sum_{g \in G}\alpha_{g}^{-1}(h) \otimes E_{g}, 
    \qquad L_g = I \otimes l_g \qquad (h \in L^{\infty}(X), g \in G),
\end{align*}
where $E_g$ is the projection onto the one dimensional subspace 
spanned by $e_{g}$ in $l^{2}(G)$. 

Since $G$ preserves the measure, $L^{\infty}(X) \rtimes_{\alpha} G$ is a II$_1$
factor. Let $s_1, s_2, \ldots, s_n$ be n elements in $G$ and 
$\{h_{s_i} \}_{i = 1}^{n}$ be a measurable functions on $X$ such that 
$\mu(\{x : h_{s_i}(x) = 0 \mbox{ or } \infty \}) = 0$.
It is easy to see that the operator
\begin{align*}
    T = \sum_{i = 1}^{n} \psi(h_{s_i})L_{s_i} =
    \sum_{i = 1}^{n} \sum_{g \in G} \alpha_{g}^{-1}(h_{s_i}) \otimes E_{g}l_{s_i}
\end{align*}
is affiliated with $L^{\infty}(X) \rtimes_{\alpha} G$. Let 
\begin{align*}
    A = \sum_{s} \sum_{g} \alpha_{g}^{-1}(f_{s}) \otimes E_{g}l_{s}
\end{align*}
be any operator in $L^{\infty}(X) \rtimes_{\alpha} G$. We have
\begin{align*}
\left\langle AT \xi \otimes e_l, \beta \otimes e_g \right \rangle &=
    \left\langle (\sum_{i=1}^{n} \sum_{g_2} \alpha_{g_2}^{-1}(h_{s_i}) \otimes
    E_{g_2}l_{s_i})\xi \otimes e_l, (\sum_{s} \sum_{g_1} \alpha_{g_1}^{-1}
    (\bar{f_{s}})
\otimes l_{s^{-1}}E_{g_1}) \beta \otimes e_g \right \rangle \\ 
&= \left\langle \sum_{i=1}^{n} \alpha_{s_{i}l}^{-1}(h_{s_i})\xi \otimes e_{s_{i}l}, 
\sum_{s} \alpha_{g}^{-1}(\bar{f_s})\beta \otimes e_{s^{-1}g} \right \rangle \\
&= \sum_{i=1}^{n}\left\langle \alpha_{s_{i}l}^{-1}(h_{s_i})\xi,
\alpha_{g}^{-1}(\overline{f_{gl^{-1}s_{i}^{-1}}})\beta \right \rangle \\
&= \left\langle \sum_{i=1}^{n}
\alpha_{g}^{-1}(f_{gl^{-1}s_{i}^{-1}})\alpha_{s_{i}l}^{-1}(h_{s_i})\xi, \beta \right
\rangle,
\end{align*}
and
\begin{align*}
\left\langle TA \xi \otimes e_l, \beta \otimes e_g \right \rangle &=
\left\langle 
(\sum_{s} \sum_{g_1} \alpha_{g_1}^{-1}(f_{s}) \otimes E_{g_1}L_{s}) \xi \otimes e_l,
(\sum_{i=1}^{n}\sum_{g_2} \alpha_{g_2}^{-1}(\overline{h_{s_i}}) \otimes
l_{s_{i}^{-1}}E_{g_2}) \beta \otimes e_{g} \right \rangle \\ 
&= \left \langle \sum_{s} \alpha_{sl}^{-1}(f_{s})\xi \otimes e_{sl},
\sum_{i=1}^{n}\alpha_{g}^{-1}(\overline{h_{s_i}})\beta \otimes e_{s_{i}^{-1}g} \right
\rangle\\ 
& = \left \langle
\sum_{i=1}^{n} \alpha_{g}^{-1}(h_{s_i})\alpha_{s_{i}^{-1}g}^{-1}(f_{s_{i}^{-1}gl^{-1}}) 
\xi,
\beta \right \rangle
\end{align*}
If $AT = TA$, then (let $gl^{-1} = s$)
\begin{align}\label{eq1}
    \sum_{i=1}^{n}\alpha_{g}^{-1}(f_{ss_{i}^{-1}})\alpha_{s_{i}s^{-1}g}^{-1}
    (h_{s_i}) =
    \sum_{i=1}^{n}\alpha_{g}^{-1}(h_{s_i})\alpha_{s^{-1}_{i}g}^{-1}(f_{s_{i}^{-1}s}).
\end{align}




\subsection{Hyperfinite case}

Let $G = \Z$, $s_1 = 1$, $g = n$, $s = m+1$. By \cref{eq1} we have
\begin{align*}
    \alpha_{-n}(f_{m})\alpha_{m-n}(h_{1}) =
\alpha_{-n}(h_{1})\alpha_{1-n}(f_{m}).
\end{align*}
Apply $\alpha_{n}$ to both side of the equation above, we have
\begin{align*}
f_{m}\alpha_{m}(h_{1}) =
h_{1}\alpha(f_{m}).
\end{align*}
Recall that $h_{i}$ is a measurable functions on $X$ such that 
$\mu(\{x : h_{s_i}(x) = 0 \mbox{ or } \infty \}) = 0$.
Therefore we have
\begin{align*}
\frac{\alpha(f_{m})}{f_{m}} = \frac{\alpha_{m}(h_{1})}{h_{1}}.
\end{align*}

Let
\begin{align*}
   k_{m} =  \begin{cases}
       h_{1}\alpha_{1}(h_{1})\cdots\alpha_{m-1}(h_{1}) &\mbox{ if } m > 0, \\
       1 &\mbox{ if } m = 0, \\
       \alpha_{-1}(\frac{1}{h_1})\alpha_{-2}(\frac{1}{h_1})
       \cdots\alpha_{m}(\frac{1}{h_1}) &\mbox{ if } m < 0.
   \end{cases} 
\end{align*}
We have
\begin{align*}
    \frac{\alpha(f_{m})}{f_{m}} = \frac{\alpha(k_{m})}{k_{m}} \mbox{ and } 
    \alpha(\frac{f_{m}}{k_{m}}) = \frac{f_{m}}{k_{m}}.
\end{align*}

By lemma 8.6.6 in \cite{KR}, there exist $c_{k}$, $k = 0, \pm 1, \pm 2, \ldots$, such
that $f_{m} = c_{m}k_{m}$, a.e..
It is not hard to choose $h_1$, e.g. let $X = S^{1}$ and 
$h_1 = \frac{e^{2\pi i \theta} + 1}
{e^{2\pi i \theta} -1}$, 
such that each $k_m$($m \neq 0)$ is an unbounded 
measurable function.

Recall that a Cartan subalgebra $\MMM$ in a II$_1$ factor $\AAA$ is a maximal
abelian *-subalgebra with normalizer $\mathcal{N}_{\AAA}(\MMM) =
\{ U \in \mathcal{U}(\AAA) : U^{*}\MMM  U = \MMM\}$ generating $\AAA$.

\begin{lemma} \label{hy_lam1}
    Let $\mathcal{R}$ be the hyperfinite von Neumann algebra. There exists 
    a closed operator $T = UH \eta \mathcal{R}$ such that 
    $T' \cap \mathcal{R} = \C I$, $T$ generates $\mathcal{R}$ i.e.
    $U$ and $H$ generate $\mathcal{R}$ and $H$ generates a Cartan subalgebra of 
    $\mathcal{R}$.
\end{lemma}

Since any two Cartan subalgebra of the hyperfinite II$_1$ factor are conjugate by
an automorphism of $\mathcal{R}$, we have the following result.

\begin{corollary} \label{hy_cor1}
   Let $\AAA$ be a separable II$_1$ factor. There exists a hyperfinite II$_1$ 
   subfactor $\MMM$ of $\AAA$ and a 
   $T \eta \MMM$ such that 
   $NT \neq TN$ for any nontrivial normal operator affiliated with $\AAA$.
\end{corollary}

\begin{proof}
    By Corollary 4.2 in \cite{P}, there is a hyperfinite II$_1$ subfactor $\MMM$ 
    such that $\MMM' \cap \AAA = \C I$. 
    Let $T = UH$ be the closed operator in \cref{hy_lam1}.
    If $N$ is a normal operator in $\AAA$ and $NT = TN$, then $N \in \MMM' \cap \AAA$. 
\end{proof}

\subsection{Non $\Gamma$ case}
Consider the example in \cite{WY}. Let $G = F_{2}$ be the free group 
generated by $a, b$. Assume that the subgroup generated by $a$ acts on 
$X$ ergodically and the subgroup generated by $b$ acts on $X$ trivially.

Let $T = \sum_{g} \alpha_{g}^{-1}(h_{a}) \otimes E_{g}L_{a}$.
If $A = \sum_{s} \sum_{g} \alpha_{g}(f_{s}) 
\otimes E_{g}L_{s}$
is in $L^{\infty}(X) \rtimes_{\alpha} G$ and 
commute with $T$, then by \cref{eq1} 
\begin{align*}
    \alpha_{g}^{-1}(f_{sa^{-1}})\alpha_{as^{-1}g}^{-1}(h_{a}) =
    \alpha_{g}^{-1}(h_{a})\alpha_{a^{-1}g}^{-1}(f_{a^{-1}s}), \qquad \forall g, s \in G.
\end{align*}

Let $s = sa$ and $g = a$. For simplicity
of notation, we will use $\alpha^{n}$ to denote
$\alpha_{a^{n}}$. Let $\rho$ be the group homomorphism from $F_2$ to $\Z$ 
such that $\rho(a) = 1$,
$\rho(b) = 0$. Let $\rho(s) = m$, we have
\begin{align*}
    \alpha^{-1}(f_{s})\alpha^{m-1}(h_{a}) =
    \alpha^{-1}(h_{a})f_{a^{-1}sa}, \qquad \forall s \in G \mbox{ and } m \in \Z.
\end{align*}
Therefore, $\frac{\alpha(f_{a^{-1}sa})}{f_s} = \frac{\alpha^{m}(h_a)}{h_a}$. 

Let
\begin{align*}
   k_{m} =  \begin{cases}
       h_{a}\alpha^{1}(h_{a})\cdots\alpha^{m-1}(h_{a}) &\mbox{ if } m > 0, \\
       1 &\mbox{ if } m = 0, \\
       \alpha^{-1}(\frac{1}{h_a})\alpha^{-2}(\frac{1}{h_a})
       \cdots\alpha^{m}(\frac{1}{h_a}) &\mbox{ if } m < 0.
   \end{cases} 
\end{align*}
For any $s \in G$, if $\rho(s)=m$, then 
$\frac{f_{a^{-1}sa}}{k_m} = \alpha^{-1}(\frac{f_s}{k_m})$.
Therefore $f_{a^{-n}sa^{n}} = \alpha^{-n}(\frac{f_s}{k_m})k_m$.

Let $m = 0$.
If $s$ contains $b^{\pm 1}$ in the reduced form, then
$f_{s} = 0$ since $\sum_{n \in Z} \| f_{a^{-n}sa^{n}} \|_{2}^{2} 
= \sum_{n \in Z} \|\alpha_{n}^{-1}(f_{s}) \|_{2}^{2}
    = \sum_{n \in Z} \| f_{s} \|_{2}^{2} \leq \infty$.

If $s$ contains $b^{\pm 1}$ in the reduced form and $m = \rho(s) \neq 0$, then
$\frac{f_s}{k_m} = 0$. Indeed, if $\frac{f_s}{k_m} = h \neq 0$, then 
there exists a measurable subset 
$A$ in $\mathcal{B}$, $\mu(A) > 0$, $h(x) > c$ for almost 
every $x \in A$ and 
$ |c\chi_{A}k_{m}| \geq \delta > 0$, $\delta > 0$. 
By Furstenberg's multiple recurrence theorem \cite{MT}[Theorem 7.4], we have
\begin{align*}
{\lim \inf}_{N \rightarrow \infty} \frac{1}{N} \sum^{N}_{n = 1} \mu(A \cap
\alpha^{-n}(A)) = \varepsilon > 0.
\end{align*}
This implies that there exists a subsequence $n_{i}$ such that $\mu(A \cap
\alpha^{-n_i}(A)) >= \varepsilon$. Therefore, we have
\begin{align}
\infty = \sum_{i}\delta^{2}\varepsilon \leq
\sum_{i} \int \|f_{a^{-n_{i}}sa^{n_{i}}}\|^{2} d\mu < \infty .
\end{align}
It is a contradiction. Hence, $\frac{f_s}{k_m} = 0$ if 
$s$ contains $b^{\pm 1}$ in the reduced form. 

\begin{remark}
Note that in the above proof, we do not need $0$ and $\infty$ to be the
cluster point of the range of $h_a$. In another word, this is just a 
tedious way to show that $A$ must be in the von Neumann subalgebra generated
by $L_a$ and $L^{\infty}(X)$. 
\end{remark}

Suppose that $T = KW$ is a unbounded operator affiliated with a II$_{1}$ 
factor such that $T' \cap \AAA = \C I$.
Here $K = \sqrt{TT^{*}}$ and $W$ is a unitary.


Let 
\begin{align*}
&P_1 = \begin{pmatrix}
I & 0\\
0 & 0 
\end{pmatrix} \qquad
P_2 = \begin{pmatrix}
0 & 0\\
0 & I 
\end{pmatrix}\qquad 
P_3 = \begin{pmatrix}
\frac{1}{2} & \frac{1}{2}\\
\frac{1}{2} & \frac{1}{2}
\end{pmatrix}\\
&P_4 = \begin{pmatrix}
H & \sqrt{H(I-H)}V\\
V^{*}\sqrt{H(I-H)} & V^{*}(I - H)V.
\end{pmatrix}
\end{align*}
We have 
\begin{align*}
Alg(\{ P_1, P_2, P_3, P_4 \}) = \{ 
\begin{pmatrix}
T & 0\\
0 & T
\end{pmatrix}
= 
\begin{pmatrix}
T & 0\\
0 & S^{-1}TS
\end{pmatrix} | T \in P_1\AAA P_1\},
\end{align*}
where $S = -\sqrt{H(I - H)^{-1}}V$.
Just make $V = -W$ and $H = \frac{K^{2}}{I + K^{2}}$, we get a 
transitive lattice contains 4 elements.

\section{General}


%-----------------------------------------------------------------------------------
%BIBLIOGRAPHY
%-----------------------------------------------------------------------------------
\begin{thebibliography}{9}
    \bibitem{MV1}
        F. Murray and J. von Neumann, 
        \em{On Rings of Operators}, 
        Ann. Mat. 37, 1936, 116–229. 

    \bibitem{KR} 
        R. Kadison and J. Ringrose,
        {\em Fundamentals of the theory of operator algebras I, II},
        Graduate Studies in Mathematics, 15-16, AMS, 1997. 

    \bibitem{MT}
    M. Einsiedler and T. Ward, 
    {\em Ergodic Theory with a view towards Number Theory}, 
    GTM259, Springer-Verlag 2011.

    \bibitem{F}
        B. Fuglede,
        {\em A commutativity theorem for normal operators},
        PNAS USA, 36(1), 1950, 35-40.

    \bibitem{CRP}
        C. R. Putnam,
        {\em On normal operators in Hilbert space},
        Amer. J. Math. 73(2), 1951, 357-362.

    \bibitem{MR}
        M. Rosenblum, 
        {\em On a theorem of Fuglede and Putnam}, 
        J. London Math. Soc., 33, 1958, 376–377.
    
    \bibitem{Em}
        M. R. Embry, 
        {\em Similarities involving normal operators on Hilbert space}, 
        Pacific. J. Math., 35(2), 1970, 331–336.

    \bibitem{M}
        M. Mortad,
        {\em Products and sums of bounded and unbounded normal operators: 
        Fuglede-Putnam versus Embry}. 
        (English) Rev. Roum. Math. Pures Appl., 56(3), 2011, 195-205

    \bibitem{P}
         S. Popa, 
         {\em On a problem of R.V. Kadison on maximal 
         abelian *-subalgebras in factors}, 
         Invent. Math. 65, 1981, 269–281.

    \bibitem{WY}
       W. Wu and W. Yuan, 
       {\em A remark on central sequence algebras
       of the tensor product of $\mathrm{II}_{1}$ factors}

\end{thebibliography}
%-----------------------------------------------------------------------------------

\end{document}
